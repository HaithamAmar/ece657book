% Shared chapter include list (print + reflowable ebook).
% Keep this file free of page-break/layout commands; those belong in the
% entrypoint (ece657_notes.tex vs ece657_ebook.tex).
%
% NOTE: Chapter numbering is by \section in this manuscript.

% Part dividers (unnumbered) are intentionally included here so both PDF and
% EPUB builds share the same high-level structure and TOC grouping.
\part*{Part I: Foundations and the ERM toolbox}
\addcontentsline{toc}{part}{Part I: Foundations and the ERM toolbox}
% Chapter 1
\section{About This Book}\label{chap:intro}

Intelligent systems are engineered artifacts that perceive, reason, and act under constraints. This chapter sets the shared vocabulary (historical context, core definitions, recurring design themes), and \Cref{fig:roadmap} shows how the strands connect and where to enter.

\begin{tcolorbox}[summarybox, title={Learning Outcomes}]
After this chapter, you should be able to:
\begin{itemize}
    \item Explain what this book means by an \emph{intelligent system} (and how that differs from an \emph{intelligent machine}).
    \item Place modern AI ideas in a brief historical context (logic, computation, and learning).
    \item Use the book's organizing lenses (system components, levels of intelligence) to interpret later chapters.
    \item Navigate the book structure and reading paths using the roadmap figure.
\end{itemize}
\end{tcolorbox}

Use this chapter as a standing reference: later tools reuse the same system vocabulary and the same habit of checking assumptions against constraints. The point is continuity, not memorization: each later chapter instantiates the same design loop with different model families.

\begin{tcolorbox}[summarybox, title={Design motif}]
We treat ``intelligence'' operationally: specify what a system represents, what actions it can take, and how it checks itself against objectives and constraints.
\end{tcolorbox}

\subsection{Historical Foundations of Intelligent Systems}
\label{sec:intro_historical_foundations_of_intelligent_systems}

A brief historical sketch helps place intelligent systems within a longer tradition that runs from early mechanical devices, through symbolic logic and computation, to modern machine learning.

\paragraph{Mechanical Automata and Scholastic Logic}

In the 12th--13th centuries, engineers such as Al-Jazari designed programmable water clocks and mechanical automata whose gears, cams, and valves executed fixed sequences of actions. Although these devices lacked learning or internal models, they embodied the idea that artifacts could sense (via floats and levers), transform signals mechanically, and act on their environment. In parallel, medieval scholars such as Ibn~S\={\i}n\={a} and Thomas Aquinas refined Aristotelian syllogistic logic, systematizing patterns of valid inference even though a fully symbolic notation did not yet exist.

\paragraph{The Mechanical Computer and Early Programming}

In the 19th century, Charles Babbage designed the mechanical computer now known as the \emph{Analytical Engine}. Ada Lovelace is often cited as one of the first programmers; her notes on the Analytical Engine include an algorithm for computing Bernoulli numbers and helped establish programming as a discipline.

An important (and still practical) lesson from this era is the ``garbage in, garbage out'' principle: if incorrect input is provided to a computational system, the output will also be incorrect. In modern terms, this is a reminder that data quality and validation are part of the intelligence pipeline, not an afterthought.

\paragraph{Mathematical Logic and Formal Reasoning}

The symbolic formalism used in modern AI emerged in the 19th and early 20th centuries. Works by George Boole (1847), Gottlob Frege (1879), Giuseppe Peano (1889), and later Bertrand Russell and Alfred North Whitehead (1910--1913) introduced algebraic and predicate-calculus notations that underpin automated reasoning. Formal inference rules such as:

\begin{align}
\text{If } A = B \text{ and } B = C, \text{ then } A = C.
    \label{eq:auto:lecture_1_intro:1}
\end{align}

This exemplifies the transitivity of equality---an example of a valid inference rule operating on equality relations---and provides a basis for reasoning systems that manipulate symbols according to formal rules.

\paragraph{The Turing Test and the Birth of AI}

The mid-20th century marked a pivotal moment with Alan Turing's proposal of the \emph{Turing Test} in 1950. This test was designed to assess a machine's ability to exhibit intelligent behavior indistinguishable from that of a human. The Turing Test shifted the focus from mechanical computation to the broader question of machine intelligence.

\paragraph{Early Machine Learning and Symbolic AI}

Following the Turing Test, research into machine learning and symbolic AI accelerated. In the 1950s, the perceptron model was introduced as an early neural network capable of binary classification. Around the same time, James Slagle developed an early influential AI program: a symbolic integration system capable of performing calculus operations symbolically rather than numerically. This line of work anticipated themes later formalized in decision procedures for elementary integration \citep{Risch1969} and demonstrated that machines could manipulate abstract symbols to solve problems, a core idea in symbolic AI.

\paragraph{Summary of Key Historical Milestones}

\begin{itemize}
    \item \textbf{12th--13th Centuries:} Mechanical automata (e.g., Al-Jazari) and scholastic refinements of syllogistic logic.
    \item \textbf{19th Century:} Charles Babbage's Analytical Engine and Ada Lovelace's pioneering programming notes; Boole and contemporaries formalize symbolic logic.
    \item \textbf{Early 20th Century:} Frege, Peano, Russell, and Whitehead develop predicate calculus and logicist foundations.
    \item \textbf{1950:} Alan Turing's Turing Test frames the question of machine intelligence.
    \item \textbf{1950s:} Development of early machine learning models (perceptrons) and symbolic AI programs (e.g., Slagle's integration system).
\end{itemize}

This historical arc sets the stage for contemporary intelligent systems: programmable artifacts whose behavior is grounded in formal models, implemented on digital hardware, and increasingly trained or tuned from data. The sections that follow make the working definitions and modeling assumptions explicit.

% Examination and assessment policy details removed from Chapter 1; see Appendix: Course Logistics if needed.

% \subsection{Course Recommendations}

% \paragraph{Concurrent Courses}
% \begin{itemize}
%     \item It is not necessary to take ECE657 concurrently with related courses such as ECE570.
%     \item Taking both simultaneously may lead to cognitive overload or confusion because ECE570 covers overlapping supervised-learning algorithms (e.g., SVMs, AdaBoost, kernel PCA) but uses a statistics-first notation---probabilistic modeling and statistical learning theory are introduced before systems considerations---and larger Kaggle-style projects, whereas ECE657 emphasizes systems thinking and hybrid intelligent architectures.
%     \item The chapters are designed to be self-contained, so scheduling can be adapted to your offering.
% \end{itemize}

% \paragraph{Independent Study}
% \begin{itemize}
%     \item Students are encouraged to engage in individual reading and exploration beyond the chapter materials.
%     \item This will help deepen understanding and prepare for assignments and exams.
% \end{itemize}

% A week-by-week topic plan may be provided separately; readers can review it to anticipate upcoming discussions when available.

\subsection{Defining Artificial Intelligence and Intelligent Systems}
\label{sec:intro_defining_artificial_intelligence_and_intelligent_systems}

Artificial Intelligence (AI) is often misunderstood as merely a collection of popular applications such as image recognition or voice detection. However, these are just subsets of a much broader field. Instead of defining AI by its famous applications, it is more accurate to view AI as a body of collective algorithms, research, and engineering practice aimed at enabling machines to perceive their environment, perform inference, and take purposeful actions.

\paragraph{Core Definition of AI}

Following the agent-centric view of \citet{RussellNorvig2021}, artificial intelligence studies
computational agents that map percepts to actions through algorithms operating over explicit
representations (state graphs, feature vectors, logical predicates, or probabilistic models)
subject to domain constraints (physical limits, safety rules, resource budgets). See also
\citet{PooleMackworth2017} for a complementary treatment focused on agent architectures.
Each model we study is evaluated on whether its assumptions support competent \emph{perception}
(information acquisition), \emph{reasoning and decision-making} (information processing), and
\emph{action} (environment intervention), where a \emph{percept} denotes the data received at a
decision epoch (a discrete sensing-and-decision instant; e.g., sensor readings, feature vectors, linguistic tokens) and an \emph{action} denotes the
command issued to the environment or downstream system.

Many model-based systems generate hypotheses and test them, yet the field also includes purely
reactive controllers (e.g., subsumption architectures in behavior-based robotics or PID loops)
that optimize behavior without explicit hypothesis testing.
Classic behavior-based robotics research \citep{Brooks1986,Arkin1998} treats such controllers as
intelligent agents. They satisfy the perception--action cycle even in the absence of symbolic
reasoning. We flag them as boundary cases: they remain control-theoretic constructs, yet they
highlight the continuum between classical control and adaptive AI systems. Throughout this book we discuss both deliberative reasoning (planning, inference, search) and
reflexive intelligence (engineered feedback loops that achieve goals without symbolic reasoning),
and we try to make clear which lens is being used in a given chapter.

For now, if we adopt a value-centric view of AI, we can characterize intelligent systems by the kinds of questions they help us answer. In practice, three capabilities dominate:
\begin{itemize}
    \item Explaining the past,
    \item Understanding the present, and
    \item Predicting the future.
\end{itemize}
Framed this way, the parallel with human intelligence becomes explicit: both artificial systems and humans are judged by how well they can reconstruct what has happened, make sense of what is happening, and anticipate what is likely to happen next. For example, humans use memory and narrative to explain past events, situational awareness to understand ongoing interactions, and mental models to predict likely outcomes. Analytic systems that perform root-cause analysis in power grids or credit-risk models in finance primarily \emph{explain the past}; monitoring systems such as anomaly detectors and online recommendation engines focus on \emph{understanding the present}; time-series forecasters and large language models that predict the next token or utterance instantiate the \emph{predicting the future} role. Modern AI architectures often blend these roles, but keeping the three questions in mind provides a useful lens for interpreting model behavior.

To connect this value-centric lens to concrete designs, we now make more precise what we mean by an intelligent system and how a design begins with a clearly stated problem and representation.

\subsection{Intelligent Systems}\label{par:intelligent-systems}

An \emph{intelligent system} is an artificial entity composed of both software and hardware components that:
\begin{itemize}
    \item Acquire, store, and apply knowledge,
    \item Perceive and interpret environmental data to maintain situational awareness,
    \item Make decisions and act based on incomplete or imperfect information.
\end{itemize}

In contrast, an \emph{intelligent machine} is usually a single embodied device (for example, a robot arm on a factory line) whose sensing, reasoning, and actuation are co-located. Intelligent systems can comprise multiple cooperating machines plus cloud services; intelligent machines are one concrete realization within that broader system-of-systems view.

This working definition is consistent with those used in cyber-physical systems literature and the
IEEE Standards Association's descriptions of intelligent agents, emphasizing perception, cognition,
and action as the three pillars of autonomy.%
\footnote{Compare with the IEEE Global Initiative on Ethics of Autonomous and Intelligent Systems, \emph{Ethically Aligned Design}, 1st ed., 2019.}

Here, ``knowledge'' encompasses encoded data sets, learned model parameters, rule bases, and
semantic ontologies that the system can query or update during operation. The hardware enables
interaction with the environment (e.g., sensors, actuators), while the software performs reasoning
and decision-making.

\subsubsection{From value-centric questions to concrete designs}
\label{sec:intro_from_value_centric_questions_to_concrete_designs_sub}

The three value-centric questions (``explain the past, understand the present, predict the future'') only become actionable once a designer fixes a problem statement, a representation, and the constraints under which the system operates. Rather than treating these as separate case studies, we fold them into a compact design checklist that we reuse whenever it helps structure a design discussion:
\begin{enumerate}
    \item \textbf{Problem definition.} State the task in operational terms. Example: ``Detect stop signs quickly enough to enable safe braking.'' The definition should tell us which of the three value-centric roles dominates (here: understanding the present, plus explaining why braking events occur).
    \item \textbf{Representation.} Decide how the world will be encoded numerically. Stop-sign detection uses camera images (matrices of intensities) plus metadata such as lane boundaries or GPS position; a financial recommender might rely on structured tabular data.
    \item \textbf{Objectives and constraints.} Specify the metric to optimize (e.g., minimize false negatives) and the hard constraints (minimum stopping distance, latency budgets, regulatory rules). Practical implementations refine these with regions of interest, masking, or sensor fusion (LiDAR + camera) so the classifier only runs where a stop sign could plausibly appear.
\end{enumerate}

These three ingredients determine what the intelligent system must sense, infer, and control. Once they are in place we can reason about the interacting components that implement the perception $\rightarrow$ reasoning $\rightarrow$ action loop.

\subsubsection{Components of AI Systems: Thinking, Perception, and Action}
\label{sec:intro_components_of_ai_systems_thinking_perception_and_action_sub}

AI systems can be decomposed into three interrelated components:

\begin{description}
\item[Perception:] How the system senses and interprets environmental data, extracting features or state estimates.
\item[Reasoning and Decision-Making:] How the system combines models and control policies with learned value functions to plan actions or react in real time.
\item[Action:] How the system executes decisions to affect the environment.
\end{description}

\paragraph{Example: Autonomous Vehicle}
Perception starts with sensors (for example, cameras) whose signals are converted into numeric arrays and state estimates. Reasoning uses those estimates to classify objects (stop signs, pedestrians) and predict near-future motion. Action closes the loop by issuing steering, acceleration, and braking commands that satisfy safety constraints.

\subsection{Case Study: AI-Enabled Camera as an Intelligent System}
\label{sec:intro_case_study_ai_enabled_camera_as_an_intelligent_system}

The design checklist above becomes concrete when we dissect a deployed system. Consider a networked camera that detects humans and escalates alarms in an industrial plant.

\begin{tcolorbox}[summarybox, title={Checklist instantiated for the camera system}]
\begin{description}[style=nextline]
    \item[Problem + value role] Detect humans entering restricted zones (understand the present) and log footage for later audits (explain the past).
    \item[Representation] Images are streamed as \(H\times W\times 3\) tensors; regions of interest and background models are maintained to suppress noise.
    \item[Objectives/constraints] Maintain $<200$\, ms end-to-end latency and $<1\%$ false negatives; respect privacy/retention policies.
    \item[Hardware (perception/action)] CMOS (complementary metal-oxide-semiconductor) sensor, on-board DSP (digital signal processor)/accelerator, motorized pan/tilt for re-targeting.
    \item[Software (reasoning)] A YOLO-style object detector (e.g., YOLOv8) fine-tuned on site-specific data, fused with Kalman filters for track smoothing and MPC (model predictive control) logic that commands the pan/tilt actuator or triggers alerts.
    \item[Integration] Edge inference handles immediate reactions; metadata is sent to a cloud analytics service that enriches logs and retrains models (predicting the future by anticipating recurrent intrusion times).
\end{description}
\end{tcolorbox}

This decomposition highlights the same three pillars---perception, reasoning, and action---while adding the operational nuance (latency budgets, privacy constraints) that graduate-level systems must address. Many later chapters discuss building blocks that could be used in such a pipeline: convolutional networks (\Cref{chap:cnn}) for visual detection, recurrent models (\Cref{chap:rnn}) for temporal smoothing and sequence prediction, supervised learning and calibration (\Crefrange{chap:supervised}{chap:logistic}) for reliable scoring and threshold selection, fuzzy controllers (\Crefrange{chap:fuzzysets}{chap:fuzzyinference}) for rule-based escalation policies, and evolutionary algorithms (\Cref{chap:evo}) for tuning design choices such as placement, thresholds, or hyperparameters.

An intelligent system is therefore not just the hardware or the software alone, but the \emph{system of components} working together to perceive, reason, and act under explicit objectives.

\subsection{Levels and Architectures of Intelligent Systems}
\label{sec:intro_levels_and_architectures_of_intelligent_systems}

Having introduced working definitions and concrete examples, we now summarize capabilities and architectural patterns that reappear across the book.

\paragraph{What Constitutes Intelligence in Systems?}

Intelligence in systems is often characterized by the perception--reasoning--action loop, augmented by learning and adaptation. A fuller capability checklist appears later in the Key Characteristics paragraph after the system vignettes.

These capabilities can be realized in various architectures, ranging from connectionist models (e.g., neural networks) to symbolic systems and hybrid approaches.

\paragraph{Levels of Intelligence (as an organizing lens)}

Intelligence is not necessarily binary (intelligent vs.\ non-intelligent); rather, deployed systems combine different degrees of reactivity, deliberation, and adaptation. For the purposes of this book we use a four-layer shorthand---reactive systems (level 1), deliberative planners (level 2), adaptive learners (level 3), and meta-cognitive agents that reason about their own policies (level 4)---as an informal organizing lens rather than a strict hierarchy. It is compatible with domain-specific taxonomies (e.g., SAE Levels~0--5 for automated driving). The closing Key Takeaways return to it with representative algorithms from later chapters.

\paragraph{Connectionist vs.\ agent-based/decentralized approaches}

Two broad paradigms in intelligent system design are:

\begin{itemize}
    \item \textbf{Connectionist Models:} Systems structured as interconnected processing units (e.g., neural networks) with defined input-output stages.
    \item \textbf{Agent-based or decentralized systems:} Collections of agents or modules that operate semi-independently, often with only local communication, such as swarm intelligence or evolutionary algorithms.
\end{itemize}

Both approaches have merits and limitations, and hybrid models often combine elements of each.

\paragraph{Example: Swarm Intelligence}

Swarm systems consist of multiple agents solving subproblems independently but collectively achieving a global objective. Each agent follows simple rules without a global world model, yet the emergent behavior can be intelligent. This contrasts with monolithic systems possessing explicit internal representations.

Swarm intelligence can be formalized via decentralized update laws of the form $\mathbf{x}_i(t+1) = f\big(\mathbf{x}_i(t),\{\mathbf{x}_j(t)\}_{j \in \mathcal{N}_i}\big)$, where each agent $i$ interacts only with its neighborhood $\mathcal{N}_i$. Similar update patterns appear again in \Cref{chap:evo}, but we do not focus on stability proofs in this book.

\paragraph{Examples of Input and Output Variables in Dynamic Systems}

To ground these ideas, consider input/output sketches from systems readers often encounter in labs or industry. Each pairs raw sensory cues with actuator or decision outputs.

\begin{itemize}
    \item \textbf{Autonomous quadrotor:}
    \begin{itemize}
        \item \emph{Inputs:} Inertial measurement unit (IMU) rates, barometer/altimeter, camera or LiDAR features, GPS fixes.
        \item \emph{Outputs:} Motor thrust commands and attitude setpoints that regulate yaw/pitch/roll and track waypoints.
    \end{itemize}

    \item \textbf{Smart microgrid:}
    \begin{itemize}
        \item \emph{Inputs:} Load forecasts, solar/wind availability, electricity prices, state-of-charge estimates for batteries.
        \item \emph{Outputs:} Dispatch setpoints (generator outputs, battery charge/discharge, demand-response signals) that balance stability, cost, and emissions.
    \end{itemize}

    \item \textbf{Building HVAC controller:}
    \begin{itemize}
        \item \emph{Inputs:} Zone temperature/CO\textsubscript{2}/humidity sensors, occupancy estimates, outdoor weather feeds.
        \item \emph{Outputs:} Fan speeds, damper positions, valve openings, heat-pump setpoints---tunable levers to meet comfort and energy targets.
    \end{itemize}

    \item \textbf{Robot-assisted surgery:}
    \begin{itemize}
        \item \emph{Inputs:} Endoscopic vision, force/torque sensing at instruments, surgeon console commands.
        \item \emph{Outputs:} Precise tool trajectories, force limits, and safety interlocks that respect tissue constraints.
    \end{itemize}
\end{itemize}

These vignettes echo a common pattern: intelligent systems fuse heterogeneous sensors to produce calibrated control or decision signals under safety, comfort, or performance constraints.

\begin{tcolorbox}[perspectivebox, title={Emotions as Utility Signals}]
From a design perspective, emotions can be viewed abstractly as changes in an agent's internal utility or value function: positive affect corresponds to utility gains, negative affect to losses, and social emotions to comparisons between agents' utilities. Artificial systems can mimic this by modulating learning rates, exploration pressure, or safety margins in response to internal ``frustration'' or ``satisfaction'' signals without presupposing rich phenomenology. This book treats this view strictly as a modeling device for embedding motivational signals into controllers; affective computing and cognitive science work with much richer state representations than we use here.
\end{tcolorbox}

\paragraph{Key Characteristics of Intelligent Systems}

Building on the examples above, we summarize the essential capabilities that characterize an intelligent system:

\begin{enumerate}
    \item \textbf{Sensory Perception:} The system must be able to receive and interpret inputs from its environment, which may be in various forms such as numerical data, images, sounds, or tactile signals.

    \item \textbf{Pattern Recognition and Learning:} The system should identify patterns within the input data, including hidden or subtle features, and improve its performance over time by learning from experience.

    \item \textbf{Knowledge Retention:} Acquired knowledge must be stored and utilized for future decision-making.

    \item \textbf{Inference from Incomplete Information:} The system should be capable of drawing conclusions and making decisions even when presented with partial or approximate data.

    \item \textbf{Adaptability:} It must handle unfamiliar or novel situations by generalizing from prior knowledge and adapting its behavior accordingly.

    \item \textbf{Inductive Reasoning:} The system should be able to generalize patterns from observed examples---i.e., infer general rules or hypotheses from specific data instances (e.g., learn a classifier from labeled data). This differs from applying pre-written conditional logic; induction discovers the rules, whereas conditional statements merely execute them.
\end{enumerate}

\paragraph{Intelligent Systems as Decision Makers}

At the core, intelligent systems perform a mapping from inputs to outputs, where the outputs represent decisions or actions influenced by the system's internal understanding or model of the environment. Formally, if we denote the input vector by $\mathbf{x} \in \mathcal{X}$ and the output vector by $\mathbf{y} \in \mathcal{Y}$, then an intelligent system implements a function
\begin{equation}
    \mathbf{y} = f(\mathbf{x}; \theta),
    \label{eq:intelligent_mapping}
\end{equation}
where $\theta$ represents internal parameters or knowledge that may evolve over time through learning. This abstraction is shared by the diverse examples seen so far: they all ingest data, transform it through a parameterized mapping, and emit decisions or control signals. In this book we primarily use the system-level language (mapping under constraints), and we use ``machine'' when embodiment and actuation are the point. Because the chapter alternates between these viewpoints, we briefly clarify terminology and the limits of the language we use.

\subsection{Intelligent Systems and Intelligent Machines}
\label{sec:intro_intelligent_systems_and_intelligent_machines}

\paragraph{Terminology Clarification}

\begin{itemize}
\item \textbf{Intelligent System:} A computational system (encompassing its hardware, software, and data interfaces) that perceives its environment, processes information, and acts autonomously or semi-autonomously (with limited human oversight or shared control).

    \item \textbf{Intelligent Machine:} A physical instantiation of an intelligent system, often embodied as a robot or automated device.
\end{itemize}

The terms are related but not identical; intelligent machines are a subset of intelligent systems, typically emphasizing the physical embodiment.

\paragraph{Behavior, Not Components}

The word \emph{intelligent} is inevitably anthropocentric: in practice, we judge intelligence through observed behavior and performance under constraints. Motors, sensors, and circuits are enabling components, but they are not ``intelligent'' on their own. Intelligence emerges from how the full system processes inputs, maintains state, and selects actions.

Intelligence is also not synonymous with optimality. Many deployed systems are approximate, noisy, or biased, yet they can still be meaningfully analyzed and improved as goal-directed agents. In this book, ``intelligent'' is therefore used in an engineering sense: a system that maps percepts to actions with a design intent, and that can be evaluated against explicit objectives and constraints.

\paragraph{Examples}

Robots developed by Boston Dynamics (e.g., quadrupeds) illustrate how feedback control, state estimation, and trajectory planning can produce behaviors that humans interpret as intelligent (balance recovery, robust locomotion, disturbance rejection) even though the system has no intrinsic understanding or feelings. Voice-activated assistants and robots provide a second common example: they appear intelligent because they can condition actions on language inputs, maintain limited context, and complete tasks that align with user intent.

\paragraph{Consciousness and Intelligence}

While machines can exhibit intelligent behaviors, the question of whether they possess consciousness or self-awareness remains open and is a subject of ongoing research and philosophical debate.

In this book we treat consciousness operationally:\\
we focus on meta-cognition (self\hyp{}monitoring of one's own decision process) rather than phenomenal awareness. This keeps the discussion tied to observable, designable behaviors rather than philosophical claims.

\begin{tcolorbox}[perspectivebox, title={Author's note: ``subject of its own thought''}]
When I say \emph{strong} machine intelligence, I mean something specific: the system can turn the lens inward. It does not just predict; it also keeps enough self-modeling machinery (confidence monitors, policy checks, explanation traces) to critique and revise \emph{its own} reasoning. This is the machine becoming the subject of its own thought.

The rest of the book uses a simpler four-layer taxonomy (reactive~$\rightarrow$ deliberative~$\rightarrow$ adaptive~$\rightarrow$ meta-cognitive). Keep this ``subject of its own thought'' lens in the background: it is why Level~4 systems demand extra care in design and governance.
\end{tcolorbox}

\subsection{Levels, Meta-cognition, and Safety}\label{subsec:levels}

This book uses levels of intelligence as an organizing lens rather than a formal taxonomy. The four levels introduced above (reactive, deliberative, adaptive, and meta-cognitive) are meant to clarify what a system can do, what it must represent, and what kinds of failures are plausible. For a working definition of AI and intelligent systems, see \Cref{par:intelligent-systems}.

\paragraph{Meta-cognition (Operational View)}

In this book, meta-cognition refers to a controller's ability to monitor, assess, and revise its own reasoning policies. In practice this can look like confidence monitors, audits of decision traces, and bounded self-correction loops rather than unconstrained self-modification.

\paragraph{Implications and Risks}

If a system can improve its own utility autonomously and rapidly, it may induce competitive dynamics in which improving one utility degrades another's. This occurs in multi-agent settings (competing organizations or robots) and in multi-objective optimization when safety objectives conflict with performance. These scenarios motivate conservative design and governance, especially as systems move from adaptive learning to self-monitoring and policy revision.

\paragraph{Designing Safe Intelligent Systems}

One practical mitigation is to require auditable decision traces, routine self-inspection/error analysis, and bounded backtracking/self-correction inside explicit, designer-defined interfaces.

\noindent Such systems can improve without uncontrolled self-modification: policy updates are gated by testable criteria, and any self-editing of code or reward functions proceeds only through approved interfaces.

\bigskip
\noindent\textbf{Reader's guide.} The remainder of this chapter is practical: who the book is for, how chapters fit together, and how to navigate recurring structure. Notation and reading conventions are collected in the front matter (see \emph{Notation and Conventions}).

\subsection{Audience, Prerequisites, and Scope}
\label{sec:intro_audience_prerequisites_and_scope}

This material has been rewritten to stand on its own as a book. It surveys the design and analysis of intelligent systems along two main strands:
\begin{itemize}
    \item data\hyp{}driven models for prediction and decision making (linear models, kernels, deep
    networks, sequence models, Transformers);
    \item soft\hyp{}computing and search methods (self\hyp{}organizing maps, fuzzy systems, evolutionary and
    genetic algorithms).
\end{itemize}
The emphasis is on breadth with enough mathematical depth that you can relate ideas across chapters rather than treating each technique in isolation.

The book also maintains a deliberate dual emphasis: representation learning with neural and kernelized models on one hand, and soft\hyp{}computing approaches (fuzzy systems and evolutionary optimization) on the other. This balance keeps robustness, interpretability, and optimization themes all in view rather than treating deep networks in isolation.

    The assumed background is undergraduate calculus and linear algebra (vectors, matrices, eigenvalues) and basic probability and statistics. No prior dedicated AI or machine\hyp{}learning course is assumed: key ideas such as losses, optimization, gradient descent, backpropagation, kernels, fuzzy operators, and evolutionary operators are introduced from first principles when they first appear. Familiarity with signals and systems, and with linear time-invariant (LTI) models in particular, is helpful for the sequence-modeling and control-oriented parts of the book; \Cref{app:linear_systems} (\emph{Linear Systems Primer}) provides a concise refresher.

\subsection{Roadmap and Reading Paths}
\label{sec:intro_roadmap_and_reading_paths}
\begin{figure}[!b]
    \centering
    \captionsetup{width=.65\linewidth}
    \resizebox{.9\linewidth}{!}{%
    \begin{tikzpicture}[>=Stealth, node distance=1.6cm and 1.8cm]
        \tikzstyle{block}=[draw, rounded corners, align=center, minimum width=2.6cm, minimum height=0.8cm, fill=gray!5]
        % Main chain
        \node[block] (reg) {Linear\\Regression};
        \node[block, right=of reg] (log) {Logistic\\Regression};
        \node[block, right=of log] (mlp) {MLP\\(Backprop)};
        \node[block, right=of mlp] (cnn) {CNN\\(Conv/Pool)};
        \node[block, right=of cnn] (rnn) {RNN\\(BPTT)};
        % Branch: SOM -> Fuzzy
        \node[block, below=1.2cm of mlp] (som) {SOM\\(Competitive)};
        \node[block, right=of som] (fuzzy) {Fuzzy Sets\\\& Inference};
        % Parallel: Optimization -> GA/GP
        \node[block, above=1.2cm of mlp] (opt) {Optimization\\(GD/Reg)};
        \node[block, right=of opt] (ga) {Evolutionary\\(GA/GP)};
        % Arrows main
        \draw[->] (reg) -- (log);
        \draw[->] (log) -- (mlp);
        \draw[->] (mlp) -- (cnn);
        \draw[->] (cnn) -- (rnn);
        % Arrows branches
        \draw[->] (mlp) -- (som);
        \draw[->] (som) -- (fuzzy);
        \draw[->] (opt) -- (ga);
        % Cross-links (light)
        \draw[->, gray!60] (mlp) to[bend left=15] (ga);
        \draw[->, gray!60] (fuzzy) to[bend right=20] (rnn);
    \end{tikzpicture}%
    }
    \captionsetup{justification=raggedright}
    \caption{Roadmap of the book strands (core supervised path; SOM/fuzzy; optimization/evolutionary). It also serves as a quick prerequisite map when you want to jump ahead and later backfill foundations.}
    \label{fig:roadmap}
\end{figure}

\Cref{fig:roadmap} summarizes the narrative arc of the book: a core supervised path (linear and logistic regression to MLPs to CNNs to RNNs), a branch through competitive learning and fuzzy inference for rule-based reasoning, and a parallel thread on optimization culminating in evolutionary computing. Early on, \Cref{chap:symbolic} provides a complementary symbolic-search perspective so we can contrast ``intelligence via transformations'' with ERM-based modeling. Chapters cross-reference one another so you can skim the path most relevant to your project and return for foundational refreshers as needed.

Readers arrive with different goals. The roadmap is intentionally a dependency graph rather than a single linear track; the following paths are common starting points:
\begin{enumerate}
    \item \textbf{ML-focused path:} \Crefrange{chap:supervised}{chap:logistic} $\rightarrow$ \Crefrange{chap:mlp}{chap:backprop} $\rightarrow$ \Crefrange{chap:cnn}{chap:nlp} $\rightarrow$ \Cref{chap:transformers}.
    \item \textbf{Control/systems path:} \Crefrange{chap:intro}{chap:supervised} $\rightarrow$ \Cref{chap:hopfield} $\rightarrow$ \Crefrange{chap:softcomp}{chap:fuzzyinference} $\rightarrow$ \Cref{chap:evo}.
    \item \textbf{Soft-computing path:} \Crefrange{chap:intro}{chap:supervised} $\rightarrow$ \Crefrange{chap:softcomp}{chap:fuzzyinference} with optional detours to \Cref{chap:mlp} and \Cref{chap:evo}.
\end{enumerate}

\subsection{Using and Navigating This Book}
\label{sec:intro_using_and_navigating_this_book}
\begin{itemize}\sloppy
    \item \textbf{Before each chapter:} skim the Learning Outcomes and check where it sits on the Roadmap.
    \item \textbf{While reading:} follow cross-referenced figures\slash equations and pause at the short checkpoints and worked examples.
    \item \textbf{After each chapter:} review the Summary and Common Pitfalls; revisit the pseudocode and attempt the Exercises.
    \item \textbf{Keep the front matter close:} the Notation and Conventions section defines symbols reused throughout the book; cross-references point back to it when conventions matter.
\end{itemize}

\noindent\textbf{Conventions and reading aids.} The front matter summarizes notation and common conventions, and it also includes a short guide to reading figures and recurring box styles.

\medskip
\begin{tcolorbox}[summarybox, title={Key takeaways}]
\textbf{Minimum viable mastery}
\begin{itemize}
    \item Intelligent systems integrate perception, decision, and action; we study both model\hyp{}based and control\hyp{}based realizations.
    \item The system--machine distinction is mostly about embodiment: intelligent machines are physical instances of intelligent systems.
    \item ``Levels'' are an organizing lens, and meta\hyp{}cognition is treated operationally (self\hyp{}monitoring and bounded self\hyp{}correction), not philosophically.
\end{itemize}
\medskip
\textbf{Common pitfalls}
\begin{itemize}
    \item Treating the roadmap as a single linear syllabus rather than a dependency graph of prerequisites.
    \item Confusing ``probability'' with ``decision'': many later failures come from thresholds and costs, not from modeling.
    \item Letting notation drift: keep \Cref{app:notation_collisions} close when symbols are reused in different chapters.
\end{itemize}
\end{tcolorbox}

\begin{tcolorbox}[summarybox, title={Exercises and lab ideas}]
\begin{itemize}
    \item Pick one engineered system you know (e.g., a recommender, a robot, a control loop) and identify its \emph{percepts}, \emph{internal representation}, and \emph{actions}.
    \item For one section of this chapter, rewrite the core definition in your own words and list one concrete failure mode the definition helps you anticipate.
    \item Choose a reading path using \Cref{fig:roadmap}, and write down which two chapters you will skim first (and why).
\end{itemize}
\medskip
\noindent\textbf{If you are skipping ahead.} Keep the operational vocabulary (representation, actions, objective/goal test, and audit/checks) and the roadmap (\Cref{fig:roadmap}) in mind; later chapters assume these as the organizing lens for both symbolic and data-driven tools.
\end{tcolorbox}

\paragraph{Where we head next.} \Cref{chap:symbolic} grounds this vocabulary in a compact case study: symbolic integration as transformation search. The example shows decomposition, safe rewrites, heuristic branching, backtracking, and residual checks in one place; the supervised chapters then reinterpret the same loop with losses, parameter updates, and validation audits. Keep this continuity in view: the implementation changes, but the engineering questions (state, action, objective, verification) stay the same through fuzzy reasoning and evolutionary optimization.

% Chapter 2
\section{Symbolic Integration and Problem-Solving Strategies}\label{chap:symbolic}
\graphicspath{{assets/lec2_part1/}}

\Cref{chap:intro} treated ``intelligence'' operationally: a system should represent a problem, choose actions, and verify or correct itself under constraints. This chapter makes that concrete with a deliberately small example---symbolic integration---where the system's actions are algebraic transformations. \Cref{fig:roadmap} situates this symbolic strand alongside the data-driven path.

We study symbolic problem solving through the lens of \emph{integration-by-transformation}: preserve meaning while rewriting an expression into a form that exposes a solution. The point is not the catalog of tricks in isolation, but the system-level pattern: explicit representations, meaning-preserving actions, heuristic branching with backtracking, and a clear goal test that certifies correctness.

It helps to separate transformations into two tiers: reliable moves that never change the antiderivative class (factoring constants, linear substitutions, polynomial division) and heuristic moves that may succeed only for certain structures. A practical policy is to apply every reliable step first, then branch judiciously through the heuristic catalog while keeping enough state to backtrack.

\begin{tcolorbox}[summarybox,title={Learning Outcomes}]
\begin{itemize}
    \item Decompose symbolic integration problems into safe vs.\ heuristic transformations and understand when each is appropriate.
    \item Trace the transformation-tree search (state save/restore, heuristics, termination) and connect it to broader notions of intelligent problem solving.
    \item Contrast symbolic transformation search with data-driven modeling pipelines and identify what each paradigm contributes.
\end{itemize}
\end{tcolorbox}

\begin{tcolorbox}[summarybox,title={Design motif}]
Meaning\hyp{}preserving moves plus a mechanical check: if \(F(x)\) is proposed as an antiderivative of \(f(x)\), then differentiating should recover the integrand on the declared domain, i.e., \(F'(x)=f(x)\) (equivalently, \(F'(x)-f(x)=0\)).
\end{tcolorbox}

\subsection{Context and Motivation}
\label{sec:symbolic_context_and_motivation}

Consider the task of solving an integral of the form
\[
\int f(x) \, dx,
\]
where \( f(x) \) may be a complicated function. Traditional approaches often rely on consulting integral tables or applying well-known formulas. For example, integrals such as
\[
\int \frac{1}{x} \, dx = \ln|x| + C.
\]
These are straightforward and can be solved by direct lookup or simple substitution.

However, many integrals encountered in practice do not match any entry in standard integral tables, nor do they succumb easily to elementary techniques. The integration-by-transformation view treats this as a search problem: propose meaning-preserving rewrites, backtrack when a branch gets harder, and verify candidates mechanically by differentiation.

\begin{tcolorbox}[summarybox,title={Aside: the Risch algorithm}]
The Risch algorithm \citep{Risch1969} decides whether an elementary antiderivative exists for a large class of integrands by reducing the problem to algebra over differential fields. Unfortunately its implementation is intricate, requires case explosions, and still leaves many useful nonelementary functions unresolved. Practical computer algebra systems therefore augment Risch-style decision procedures with heuristic transformations---the focus of this chapter---to keep runtimes bounded and to return human-readable answers when possible.
\end{tcolorbox}

\subsection{Problem Decomposition and Transformation}
\label{sec:symbolic_problem_decomposition_and_transformation}

A key insight in tackling complex integrals is to \emph{reduce} the problem into manageable subproblems. This involves applying \emph{transformations} to rewrite the integral into a form that is either directly solvable or closer to known forms.

\paragraph{Safe Transformations}

We define \emph{safe transformations} as invertible substitutions that allow back-substitution: if \(u=\phi(x)\) and \(F_u\) is an antiderivative of \(T[g](u)\), then \(F_u\!\circ\!\phi\) differentiates back to \(g(x)\). Safe transformations are algebraic substitutions or factorings that survive reversal. Examples include:

\begin{itemize}
  \item \textbf{Constant factor extraction:} If \(G\) is an antiderivative of \(g\), then \(a g\) has antiderivative \(a G\); differentiating confirms \(\frac{d}{dx}[a G(x)] = a g(x)\).
  \item \textbf{Linear substitution:} Let \(u = ax + b\) with \(a \neq 0\). Differentiating gives \(du = a\,dx\) and hence \(dx = \frac{du}{a}\); substituting shows that
  \[
    \int f(ax + b)\,dx = \frac{1}{a} \int f(u)\,du.
  \]
  This is the standard change-of-variables formula.
  \item \textbf{Polynomial division:} If \( p(x) \) and \( q(x) \) are polynomials with \(\deg p \geq \deg q\), then perform polynomial division:
  \[
  \frac{p(x)}{q(x)} = s(x) + \frac{r(x)}{q(x)},
  \]
  where \(\deg r < \deg q\). Linearity of integration lets us integrate \(s(x)\) term-by-term, while the proper fraction \(\frac{r(x)}{q(x)}\) can be addressed via partial fractions or further substitutions, yielding an equivalent antiderivative.
\end{itemize}

These transformations are \emph{safe} because they always preserve the integral's value and simplify the problem without introducing ambiguity. When a substitution transforms an integral over \(x \in [0,1]\) into \(\int_0^1 u^{b}(1-u)^{c}\,du\) with \(b,c > -1\), the resulting definite integral evaluates to a Beta function \(B(b+1,c+1)\); the Beta identity applies to that definite integral on \([0,1]\), and it is therefore customary to fall back on it when an elementary antiderivative is unavailable.

\paragraph{Example: Applying Safe Transformations}

Suppose we have an integral of the form
\[
\int a \cdot x^{b} (1 - x)^{c} \, dx,
\]
where \(a, b, c\) are constants. A safe transformation might be to factor out the constant \(a\) and then consider substitutions or binomial expansions that reduce the powers to known integrals (e.g., Beta-function evaluations when \(b\) and \(c\) are integers).

\subsection{Limitations of Safe Transformations}
\label{sec:symbolic_limitations_of_safe_transformations}

After exhaustively applying all safe transformations, we may still encounter integrals that do not match any known solvable form. At this point, the system must decide whether the problem is solvable by known methods or if alternative strategies are necessary.

\subsection{Heuristic Transformations}
\label{sec:symbolic_heuristic_transformations}

When safe transformations fail to yield a solution, we turn to \emph{heuristic transformations}, which are not guaranteed to succeed but often provide a path forward. These heuristics are based on experience, pattern recognition, and mathematical intuition.

\paragraph{Definition}

Heuristic transformations are problem-solving \emph{tricks} or \emph{strategies} that attempt to rewrite the integral into a solvable form by exploiting structural properties of the integrand. They may involve:

\begin{itemize}
  \item Trigonometric identities and substitutions, e.g., using relationships among \(\sin x\), \(\cos x\), \(\tan x\), \(\cot x\), \(\sec x\), and \(\csc x\).
  \item Algebraic manipulations that simplify complicated expressions.
  \item Variable substitutions that transform the integral into a standard form.
  \item Recognizing patterns such as functions of \(10x\) or other scaled arguments and applying appropriate scaling substitutions (e.g., if the integrand contains \(f(cx)\), introduce \(u = cx\) so that the scale factor is absorbed).
\end{itemize}

\paragraph{Example: Trigonometric Heuristics}

Consider an integral involving sine and cosine:
\[
\int \frac{\sin x}{\cos x} \, dx.
\]
Recognizing that \(\sin x / \cos x = \tan x\), we can rewrite the integral as
\[
\int \tan x \, dx = -\ln|\cos x| + C.
\]
which is a standard integral with the constant of integration explicitly noted.

Similarly, if the integrand involves expressions like \(\sin^2 x + \cos^2 x\), we can use the Pythagorean identity to simplify.

\paragraph{Heuristics as a Form of Intelligence}

The use of heuristic transformations reflects a form of mathematical intelligence: the ability to recognize patterns, apply non-obvious substitutions, and creatively manipulate expressions to reach a solution. Unlike safe transformations, heuristics may fail or lead to dead ends, but they expand the problem-solving repertoire beyond mechanical procedures.

\begin{tcolorbox}[summarybox,title={Absolute values and branches}]
\begin{itemize}
    \item \textbf{Square roots:} specify the sign/branch. If you drop \(|\cdot|\), restrict the substitution interval so the sign is fixed (e.g., \(\cos y \ge 0\) on \(y\in(-\pi/2,\pi/2)\) so \(\sqrt{1-\sin^2 y}=\cos y\)).
    \item \textbf{Logarithms:} default to \(\log|f(x)|\) unless you guarantee \(f\) keeps one sign on the declared domain.
    \item \textbf{Arctrig/hyperbolic inverses:} state principal values and any periodicity you rely on for back-substitution.
\end{itemize}
\end{tcolorbox}


\subsection{Summary of the Approach}
\label{sec:symbolic_summary_of_the_approach}

The overall strategy for symbolic integration can be summarized as follows:

\begin{enumerate}
  \item \textbf{Apply all safe transformations} to simplify the integral and attempt to match known solvable forms.
  \item \textbf{Re-evaluate the transformed integrand} to identify structural cues (symmetry, polynomial degree, trigonometric patterns).
  \item \textbf{Choose among multiple transformation paths} by comparing simple cost heuristics such as expression-tree depth, number of nonzero coefficients, or anticipated integration rules.
  \item \textbf{Fallback to heuristics and backtracking} when safe transformations stall, maintaining a stack of previous states to enable systematic exploration.
\end{enumerate}

\paragraph{Cost heuristic.} Score candidates by a triple: tree depth, number of nonlinear operators, and symbol count. The nonlinearity term counts transcendental nodes (e.g., trigonometric, exponential, logarithmic), while the symbol count tallies AST nodes excluding simple literals such as \(-1,0,1\). Prefer branches that reduce or preserve this triple, and break ties toward rules with known templates (e.g., partial fractions, reduction formulas).
% Part A worked example (continued)

\subsection{Heuristic Transformations: Revisiting the Integral with \texorpdfstring{\(1 - x^2\)}{1 - x squared}}
\label{sec:symbolic_heuristic_transformations_revisiting_the_integral_with_1_x_2_1_x_squared}

Recall the integral under consideration:
\begin{equation}
  \int \frac{4}{(1 - x^2)^{5/2}} \, dx.
  \label{eq:original_integral}
\end{equation}

For real-valued integration we restrict attention to \(|x| < 1\), ensuring the denominator \((1 - x^2)^{5/2}\) is well-defined and nonzero on the interval of interest.

When encountering expressions involving \(1 - x^2\), a classical heuristic substitution is:
\[
x = \sin y,
\]
which leverages the Pythagorean identity:
\[
1 - \sin^2 y = \cos^2 y.
\]

Applying this substitution transforms the integral into a trigonometric form that is often easier to handle.

\paragraph{Step 1: Substitution and Differential}

Set
\[
x = \sin y \quad \Longrightarrow \quad dx = \cos y \, dy.
\]
We take \(y = \arcsin x\) with \(y \in \left[-\frac{\pi}{2}, \frac{\pi}{2}\right]\) so that the substitution remains bijective on the domain \(|x| \le 1\), and note \(\cos y \ge 0\) on this interval so that \(\sqrt{1-\sin^2 y}=\cos y\) is consistent with the chosen branch.

Substituting into \eqref{eq:original_integral} and using \(dx = \cos y \, dy\) yields
\begin{align*}
\int \frac{4}{(1 - x^2)^{5/2}} \, dx &= \int \frac{4}{(1 - \sin^2 y)^{5/2}} \cos y \, dy \\
&= \int \frac{4\cos y}{(\cos^2 y)^{5/2}} \, dy \\
&= 4 \int \cos^{-4} y \, dy \\
&= 4 \int \sec^{4} y \, dy.
\end{align*}
The intermediate step \(\cos y \cdot \cos^{-5}y = \cos^{-4}y\) is made explicit so the exponent arithmetic is transparent.

Thus, the integral reduces to
\begin{equation}
  4 \int \sec^{4} y \, dy.
  \label{eq:sec4_integral}
\end{equation}

\paragraph{Step 2: Choosing the Next Transformation}

At this stage, two common safe transformations are available:

\begin{itemize}
  \item Express \(\sec^4 y\) in terms of \(\tan y\), using the identity \(\sec^2 y = 1 + \tan^2 y\), and then perform substitution \(u = \tan y\) with \(du = \sec^2 y \, dy\).
  \item Use reduction formulas for powers of secant directly, e.g.,
  \[
  \int \sec^{n} y \, dy = \frac{\sec^{n-2} y \tan y}{n-1} + \frac{n-2}{n-1} \int \sec^{n-2} y \, dy, \quad n > 1.
  \]
\end{itemize}
Standard reduction formulas provide a deterministic alternative if the substitution path is judged too costly.

The choice between these paths is nontrivial, especially for an automated system. Humans often pick the substitution \(u = \tan y\) intuitively because it simplifies the integral, but a machine requires a deterministic decision rule.

\paragraph{Step 3: Functional Composition and Path Selection}

To automate the choice, the system evaluates the \emph{functional composition} of the integral expressions along each path (e.g., measuring expression-tree depth, symbolic coefficient growth, or the number of distinct functions involved):

\begin{itemize}
  \item \textbf{Path 1:} Substitution \(u = \tan y\) reduces the integral to a polynomial in \(u\), which is straightforward to integrate.
  \item \textbf{Path 2:} Direct reduction of \(\sec^4 y\) may involve more complex recursive steps.
\end{itemize}

From a cost perspective, Path 1 is cheaper and more direct, so the system prioritizes it. However, if this path fails to yield a solution, the system must backtrack and attempt Path 2.

\paragraph{Two safe options from here}
\begin{itemize}
  \item \textbf{(a) Substitution \(u=\tan y\).} Since \(\sec^4 y\,dy = \sec^2 y\,(\sec^2 y\,dy)\) and \(\sec^2 y = 1 + \tan^2 y\), set \(u=\tan y\), \(du=\sec^2 y\,dy\):
  \[
  \begin{aligned}
  4 \int \sec^4 y\,dy
    &= 4 \int (1+u^2)\,du \\
    &= 4\left(u + \tfrac{u^3}{3}\right)+C \\
    &= 4\tan y + \tfrac{4}{3}\tan^3 y + C.
  \end{aligned}
  \]
  \item \textbf{(b) Reduction formula.} For even \(n>1\),
  \[
    \int \sec^{n} y \, dy = \frac{\sec^{n-2} y \tan y}{n-1} + \frac{n-2}{n-1} \int \sec^{n-2} y \, dy.
  \]
  Applying this with \(n=4\) gives \(\int \sec^{4} y \, dy = \tan y + \tfrac{1}{3}\tan^{3} y + C\), reproducing the same primitive without the \(u\)-substitution.
  \end{itemize}

\paragraph{Back-substitution and check}

Using \(\tan y = \dfrac{x}{\sqrt{1-x^2}}\), both paths yield:
\[
F(x) = 4 \left[ x (1 - x^2)^{-1/2} + \frac{x^3}{3(1 - x^2)^{3/2}} \right] + C.
\]
Differentiating term by term shows \(F'(x)=4(1-x^2)^{-5/2}\) on \(|x|<1\). Outside \((-1,1)\) the principal-branch integrand is complex-valued; a real continuation rewrites the integral as \(\int 4(x^2-1)^{-5/2}dx\) with \(x=\cosh t\).

\paragraph{Pattern rule}
For integrals of the form \(\int (1 - x^2)^{-k-1/2} dx\), the substitution \(x=\sin y\) reduces them to \(\int \sec^{2k} y\,dy\); apply the even-power reduction accordingly. This is why the \(1 - x^2\) pattern triggers the trigonometric branch.
% Part A worked example (continued)

\subsection{Example: Solving an Integral via Transformation Trees}
\label{sec:symbolic_example_solving_an_integral_via_transformation_trees}

The worked integral above shows how the search explored multiple branches (e.g., \(x=\sin y\) vs.\ \(x=\tanh u\)) while respecting the declared domain. Heuristic branches that do not fit the domain are pruned; competing safe branches (substitution vs.\ reduction) converge to the same antiderivative. The key insight is that solving integrals can be viewed as traversing a \emph{decision tree} of transformations, with goal tests supplied by residual checks and domain conditions.

\subsection{Transformation Trees and Search Strategies}
\label{sec:symbolic_transformation_trees_and_search_strategies}

\paragraph{Definition:} A \textbf{transformation tree} is a conceptual structure representing all possible sequences of transformations applied to an expression in an attempt to solve or simplify it.

\begin{itemize}
  \item Each node corresponds to a state of the expression.
  \item Edges correspond to transformations (safe or heuristic).
  \item Leaves correspond to either solved expressions or dead ends (no solution).
\end{itemize}

\Cref{fig:lec3_transform_tree} shows the actual tree explored for \(\int \frac{4}{(1-x^2)^{5/2}}\,dx\). Solid branches denote safe algebraic steps (guaranteed progress), while dashed branches illustrate heuristic substitutions that may fail and trigger backtracking. Computer algebra systems follow similar playbooks \citep{Bronstein2005,Risch1969}: a Risch-style decision core handles provably solvable cases, while a curated bank of heuristics (pattern rewrites, rational substitutions, special-function fallbacks) explores auxiliary branches with explicit depth/time budgets.

\begin{figure}[h]
        \centering
        \ifdefined\HCode
            % TeX4ht stability: avoid adjustbox in EPUB builds.
            \resizebox{\linewidth}{!}{%
                \begin{tikzpicture}[
                    node distance=1.2cm and 1.0cm,
                    base/.style={
                        rectangle,
                        rounded corners=3pt,
                        line width=1pt,
                        minimum width=38mm,
                        text width=36mm,
                        inner sep=3mm,
                        align=center,
                        font=\small\sffamily
                    },
                    safe/.style={
                        base,
                        draw=cbBlue!70!black,
                        fill=cbBlue!25,
                        label={[anchor=north east, font=\scriptsize\bfseries, text=cbBlue!70!black, inner sep=2pt, yshift=-1pt]north east:[S]}
                    },
                    heur/.style={
                        base,
                        draw=cbOrange!70!black,
                        fill=cbOrange!25,
                        dashed,
                        label={[anchor=north east, font=\scriptsize\bfseries, text=cbOrange!70!black, inner sep=2pt, yshift=-1pt]north east:[H]}
                    },
                    line/.style={
                        draw=gray!80,
                        line width=1pt,
                        rounded corners=4pt,
                        ->,
                        >=Stealth
                    }
                ]

                    % Level 0
                    \node[safe] (root) at (0,0) {\(\displaystyle \int \frac{4}{(1-x^2)^{5/2}}\,dx\)};

                    % Level 1
                    \node[safe] (factor) at (-4.2, -2.8) {Factor constants\\restrict \(|x|<1\)};
                    \node[heur] (letu)   at ( 4.2, -2.8) {Let \(u = 1-x^2\)};

                    % Level 2
                    \node[heur] (subs)   at (-6.5, -6.0) {Substitute \(x=\sin y\)};
                    \node[heur] (sinh)   at (-2.0, -6.0) {Try \(x=\tanh u\)\\(hyperbolic twin)};
                    \node[heur] (stall)  at ( 4.2, -6.0) {Heuristic stalls\\backtrack};

                    % Level 3
                    \node[safe] (dx)     at (-8.8, -9.0) {Use \(dx=\cos y\,dy\)};
                    \node[safe] (reduce) at (-4.2, -9.0) {Reduce to \(4\int \sec^4 y\,dy\)};

                    % Edges
                    \draw[line] (root.south) -- ++(0,-0.6) -| (factor.north);
                    \draw[line] (root.south) -- ++(0,-0.6) -| (letu.north);

                    \draw[line] (factor.south) -- ++(0,-0.6) -| (subs.north);
                    \draw[line] (factor.south) -- ++(0,-0.6) -| (sinh.north);

                    \draw[line] (letu) -- (stall);

                    \draw[line] (subs.south) -- ++(0,-0.6) -| (dx.north);
                    \draw[line] (subs.south) -- ++(0,-0.6) -| (reduce.north);

                    % Legend
                    \node[
                        draw=gray!30,
                        rounded corners,
                        fill=white,
                        inner sep=8pt,
                        minimum width=10cm,
                        minimum height=1.5cm
                    ] (legendBox) at (0, -11.0) {};

                    \node[safe, minimum width=2.5cm, text width=2cm, scale=0.8, label={}]
                        at ($(legendBox.west)+(2.0,0)$) {Safe move};

                    \node[heur, minimum width=2.5cm, text width=2cm, scale=0.8, label={}]
                        at ($(legendBox.west)+(5.5,0)$) {Heuristic};

                    \node[right, text=gray!80, font=\scriptsize\sffamily, align=left]
                        at ($(legendBox.west)+(7.0,0)$) {Labels \textbf{[S]} and \textbf{[H]}\\indicate strategy};
                \end{tikzpicture}%
            }
        \else
            \resizebox{\linewidth}{!}{%
                \begin{tikzpicture}[
                    node distance=1.2cm and 1.0cm,
                    base/.style={
                        rectangle,
                        rounded corners=3pt,
                        line width=1pt,
                        minimum width=38mm,
                        text width=36mm,
                        inner sep=3mm,
                        align=center,
                        font=\small\sffamily
                    },
                    safe/.style={
                        base,
                        draw=cbBlue!70!black,
                        fill=cbBlue!25,
                        label={[anchor=north east, font=\scriptsize\bfseries, text=cbBlue!70!black, inner sep=2pt, yshift=-1pt]north east:[S]}
                    },
                    heur/.style={
                        base,
                        draw=cbOrange!70!black,
                        fill=cbOrange!25,
                        dashed,
                        label={[anchor=north east, font=\scriptsize\bfseries, text=cbOrange!70!black, inner sep=2pt, yshift=-1pt]north east:[H]}
                    },
                    line/.style={
                        draw=gray!80,
                        line width=1pt,
                        rounded corners=4pt,
                        ->,
                        >=Stealth
                    }
                ]

                    % Level 0
                    \node[safe] (root) at (0,0) {\(\displaystyle \int \frac{4}{(1-x^2)^{5/2}}\,dx\)};

                    % Level 1
                    \node[safe] (factor) at (-4.2, -2.8) {Factor constants\\restrict \(|x|<1\)};
                    \node[heur] (letu)   at ( 4.2, -2.8) {Let \(u = 1-x^2\)};

                    % Level 2
                    \node[heur] (subs)   at (-6.5, -6.0) {Substitute \(x=\sin y\)};
                    \node[heur] (sinh)   at (-2.0, -6.0) {Try \(x=\tanh u\)\\(hyperbolic twin)};
                    \node[heur] (stall)  at ( 4.2, -6.0) {Heuristic stalls\\backtrack};

                    % Level 3
                    \node[safe] (dx)     at (-8.8, -9.0) {Use \(dx=\cos y\,dy\)};
                    \node[safe] (reduce) at (-4.2, -9.0) {Reduce to \(4\int \sec^4 y\,dy\)};

                    % Edges
                    \draw[line] (root.south) -- ++(0,-0.6) -| (factor.north);
                    \draw[line] (root.south) -- ++(0,-0.6) -| (letu.north);

                    \draw[line] (factor.south) -- ++(0,-0.6) -| (subs.north);
                    \draw[line] (factor.south) -- ++(0,-0.6) -| (sinh.north);

                    \draw[line] (letu) -- (stall);

                    \draw[line] (subs.south) -- ++(0,-0.6) -| (dx.north);
                    \draw[line] (subs.south) -- ++(0,-0.6) -| (reduce.north);

                    % Legend
                    \node[
                        draw=gray!30,
                        rounded corners,
                        fill=white,
                        inner sep=8pt,
                        minimum width=10cm,
                        minimum height=1.5cm
                    ] (legendBox) at (0, -11.0) {};

                    \node[safe, minimum width=2.5cm, text width=2cm, scale=0.8, label={}]
                        at ($(legendBox.west)+(2.0,0)$) {Safe move};

                    \node[heur, minimum width=2.5cm, text width=2cm, scale=0.8, label={}]
                        at ($(legendBox.west)+(5.5,0)$) {Heuristic};

                    \node[right, text=gray!80, font=\scriptsize\sffamily, align=left]
                        at ($(legendBox.west)+(7.0,0)$) {Labels \textbf{[S]} and \textbf{[H]}\\indicate strategy};
                \end{tikzpicture}%
            }
        \fi

        % Keep the caption free of inline math; it wraps poorly in some EPUB renderers.
        \caption{Transformation tree for the running example integral; badges \textbf{[S]}/\textbf{[H]} mark safe vs.\ heuristic moves; the dashed branch mirrors the sine substitution. Use it when diagnosing where a solve attempt branched and why backtracking was required.}
        \label{fig:lec3_transform_tree}
\end{figure}

Use \Cref{fig:lec3_transform_tree} as the derivation anchor for this section.

\paragraph{Example:} For the integral problem, the root node is the original integral. From there, we branch into applying different substitutions or algebraic manipulations, such as
\[
\begin{aligned}
\text{Apply substitution } u=\tan(x) &\Rightarrow \text{integration by parts} \\
&\Rightarrow \text{inverse trig identities} \Rightarrow \cdots
\end{aligned}
\]

\paragraph{Safe vs. Heuristic Transformations:}
\begin{itemize}
  \raggedright
  \item \textbf{Safe transformations} are guaranteed to preserve equivalence and progress towards a solution.
  \item \textbf{Heuristic transformations} may or may not lead to a solution; they are attempts that carry risk but can be beneficial.
\end{itemize}

\paragraph{Backtracking:} If a branch leads to no solution, the system must backtrack to a previous node and try alternative transformations. This requires the ability to:
\begin{itemize}
  \item \emph{Freeze} the current state before branching.
  \item \emph{Restore} previous states upon failure (e.g., by pushing serialized expression trees and associated metadata onto a stack for later reinstatement).
\end{itemize}
In practice this corresponds to pushing serialized expression trees (i.e., deep copies of the tree structure together with any transformation metadata) onto a stack so they can be reinstated after unsuccessful exploratory steps.

\subsection{Algorithmic Outline for Symbolic Problem Solving}
\label{sec:symbolic_algorithmic_outline_for_symbolic_problem_solving}

The general algorithm for solving symbolic problems such as integrals can be summarized as follows:

\begin{enumerate}
  \item \textbf{Define the goal:} For example, express the integral in terms of known functions from a table.
  \item \textbf{Enumerate transformations:} List all possible safe and heuristic transformations applicable to the current expression.
  \item \textbf{Apply safe transformations:} Attempt all safe transformations and check if the problem is solved.
  \item \textbf{If not solved, apply heuristic transformations:} Attempt heuristic transformations to explore alternative paths; common template hits include \(\int f'(x)/f(x)\,dx \to \log|f(x)|+C\) and \(\int r'(x)e^{r(x)}\,dx \to e^{r(x)}+C\).
  \item \textbf{Branch and backtrack:} For each transformation, branch the search tree. If a branch fails, backtrack and try other branches.
  \item \textbf{Use heuristics to guide search:} For example, use functional composition depth or cost metrics to prioritize branches.
  \item \textbf{Terminate cleanly:} Stop when a closed-form antiderivative is found, or when depth/time budgets are exceeded without success; optional numeric residual tests can accept approximate solutions.
\end{enumerate}

\paragraph{Note:} This approach resembles a \emph{greedy search} with backtracking, but it does not guarantee an optimal or even successful solution in all cases.

    \begin{tcolorbox}[summarybox,breakable,title={Transformation-tree search (pseudocode)}]
\footnotesize
\begin{verbatim}
function SolveIntegral(f0,
                       domain=(-1,1),
                       depth_limit=8,
                       time_limit=2s,
                       eps_abs=1e-6,
                       eps_rel=1e-6,
                       samples=24):
    stack <- [(f0, domain, empty_history, depth=0)]
    start <- clock()
    best  <- {status="fail",
              F=None,
              residual=None,
              history=[],
              domain=domain}
    while stack not empty:
        current, dom, history, depth <- pop(stack)
        if clock() - start > time_limit
           or depth > depth_limit:
            continue
        if passes_residual_test(current, f0, dom,
                                eps_abs, eps_rel, samples):
            res = residual(current, f0, dom, samples)
            return {status="closed_form",
                    F=current,
                    residual=res,
                    history=history,
                    domain=dom}
        safe, heuristic <- enumerate_transforms(current, dom)
        for T in safe:
            g, new_dom <- apply(T, current, dom)
            push(stack,
                 (g, new_dom, history + [T], depth+1))
        for H in heuristic (ordered by cost)
             when depth+1 <= depth_limit:
            g, new_dom <- apply(H, current, dom)
            push(stack,
                 (g, new_dom, history + [H], depth+1))
    return best
\end{verbatim}
\normalsize
The pseudocode mirrors the narrative: record the original integrand \(f_0\), track the current domain/branch, apply safe transforms eagerly, explore heuristic branches within time/depth budgets, and only accept a candidate antiderivative once a sampled residual check (absolute or relative) passes on points inside the declared domain; return a clear status/history/domain either way.

\paragraph{Residual test implementation.} At each candidate \(F\), sample \(K\) points inside the current domain but away from singularities and branch points, and form \(R=\max_i |F'(x_i)-f_0(x_i)|\). Accept if \(R \le \varepsilon_{\text{abs}}\) or \(R/(1+\max_i |f_0(x_i)|)\le \varepsilon_{\text{rel}}\). Automatic differentiation or a high-order finite difference (step \(h=O(\sqrt{\varepsilon_{\text{machine}}})\)) keeps the check numerically stable. As substitutions shrink the domain, shrink the sample set and log the updated interval alongside the history.
\end{tcolorbox}

\begin{tcolorbox}[summarybox,title={Termination policies and numeric fallbacks}]
\begin{itemize}
    \item \textbf{Budgeting:} Cap depth, number of heuristic branches, and runtime (e.g., depth limit \(D=8\), two-second wall clock). When limits are reached, report ``no elementary antiderivative within budget \(D\)'' rather than looping forever.
    \item \textbf{Residual checks:} Differentiate candidate antiderivatives symbolically and numerically. Sample points inside the declared domain (away from poles) and accept only if \( \max_i |F'(x_i) - f(x_i)| \le \varepsilon_{\text{abs}}\) or the relative tolerance passes; otherwise prune or refine before returning.
    \item \textbf{Numeric escape hatch:} Switch to adaptive quadrature (e.g., Gauss--Kronrod) once symbolic attempts fail; return both the numeric estimate and the failed transformation history so users can adjust heuristics, noting that the numeric value is not a closed form.
    \item \textbf{Domain reminders:} When substitutions shrink domains (e.g., \(x=\sin y\) enforces \(|x|\le1\)), log the restriction and branch choice so the numeric fallback samples within the valid range and the report is reproducible.
\end{itemize}
\end{tcolorbox}
\begin{table}[t]
\centering
% Avoid inline math in captions; it wraps poorly in some EPUB renderers.
\caption{Table: Transformation toolkit (safe vs.\ heuristic). Preconditions keep domains/branches explicit (e.g., restrictions like ``x in (-1,1)'' for square-root expressions); principal branches unless noted.}
\small
\begin{tabularx}{0.96\linewidth}{@{}>{\raggedright\arraybackslash}p{0.22\linewidth} >{\raggedright\arraybackslash}X >{\raggedright\arraybackslash}X @{}}
\toprule
 & \textbf{Safe} & \textbf{Heuristic} \\
\midrule
Constant factor & \(\int a\,g(x)\,dx = a\int g(x)\,dx\) (no domain change). & Completing the square before attempting trig substitutions; track any resulting branch cuts. \\
Linear substitution & \(u=ax+b,\;dx=du/a\) with \(a\neq0\), invertible on the stated interval. & Trigonometric substitutions \(x=\sin u,\;x=\tan u,\;t=\tan(x/2)\) with domains \(u\in(-\tfrac{\pi}{2},\tfrac{\pi}{2})\) or stated principal branches. \\
Polynomial division / partial fractions & Split improper rational functions into polynomial + proper fraction where denominators stay nonzero on the domain. & Rationalising substitutions such as \(x=1/u\) or \(x=u^2\) to expose hidden symmetry; avoid zeros/poles introduced by the map. \\
Log-derivative pattern & \(\int f'(x)/f(x)\,dx = \log|f(x)|+C\) when \(f(x)\neq 0\) on the domain. & Template lookups (Beta/Gamma forms with parameter sign conditions, exponential-times-polynomial motifs, etc.). \\
\bottomrule
\end{tabularx}
\end{table}

\paragraph{Worked example: Beta template vs.\ numeric fallback}
Consider the Beta integral
\[
I(a,b) = \int_0^1 x^{a-1}(1-x)^{b-1}\,\mathrm{d}x, \qquad a,b>0.
\]
Safe transformations (factor constants, recognize the Beta template) immediately identify the elementary value \(B(a,b)=\Gamma(a)\Gamma(b)/\Gamma(a+b)\). By contrast, the perturbed integral
\[
\int_0^1 x^{a-1}(1-x)^{b-1}\log(1+x)\,\mathrm{d}x
\]
fails the template check after all safe moves. The solver therefore (i) records the unmet template, (ii) pushes a heuristic branch such as differentiation under the integral sign, and (iii) if the branch exceeds time/depth budgets, falls back to adaptive quadrature with the reported residual \(|I_{\text{numeric}}-I_{\text{candidate}}|\). This concrete pattern---try Beta/Gamma reduction, else return a certified numeric answer---embodies the policy described in the termination box.

\paragraph{Failure path with certified numeric residual.} Setting \(a=\tfrac{3}{2}, b=2\) in the perturbed integral above illustrates the full fallback. Safe moves reduce the plain Beta integral to \(B(3/2,2)=4/15\), but the extra \(\log(1+x)\) term triggers every heuristic branch (integration by parts, differentiation under the integral sign, series expansion) without yielding a closed form before the default depth limit \(D=8\). The solver then hands the integrand to an adaptive Gauss--Kronrod routine, which returns \(I_{\text{numeric}} \approx 0.0915453885\) with an internal error certificate \(<3\times 10^{-7}\); this is a certified quadrature value rather than a closed form. The residual check
\[
|I_{\text{numeric}} - I_{\text{previous refine}}| \le 3\times 10^{-7}
\]
is attached to the report along with the failed transformation history, making it explicit that no elementary antiderivative was located within the allotted budget even though a numerically reliable answer exists.

\subsection{Discussion: What this example illustrates}
\label{sec:symbolic_discussion_what_this_example_illustrates}

Under the operational framing in \Cref{chap:intro}, the integrator exhibits several ingredients associated with intelligent problem solving: it maintains an explicit state (the current expression), chooses actions (transformations), manages contingencies (branching and backtracking), and verifies results with a crisp goal test (differentiate and check the residual). At the same time, it is limited: it does not learn new transformations from data, and its effectiveness depends on a human-designed library of moves and heuristics.

Not every heuristic is helpful. For instance, applying \(x = \tan y\) to \(\int (1 + x^2)^{3/2} \, dx\) looks attractive because \(1+\tan^2 y = \sec^2 y\), yet it transforms the problem into \(\int \sec^5 y \, dy\), which is more complicated than the original integral. In a transformation-tree implementation this branch simply backtracks and explores alternatives (e.g., \(x = \tanh u\) for \(|x|<1\) or \(x=\cosh u\) for \(|x|>1\)), underscoring why explicit search discipline and residual checks are essential.

This contrast helps position the data-driven chapters that follow, where the system's ``actions'' are parameter updates guided by loss functions and validation checks.

% Connection to statistical learning (keep as an in-flow bridge, not a TOC heading).
\paragraph{Connection to statistical learning.} Symbolic integration is a clean playground for thinking about representations, action sequences, and verification. In data-driven modeling, the objects change (datasets, models, and losses), but the system-level pattern is similar: choose a hypothesis class, optimize an objective under resource constraints, and validate that the result generalizes.

\begin{tcolorbox}[summarybox,title={Connection: transformation search vs.\ empirical risk minimization}]
\begin{itemize}
    \item \textbf{Goal test:} residual check \(\max_{x\in S}|F'-f|\le\varepsilon\) vs.\ performance on held-out data.
    \item \textbf{Inductive bias:} safe/heuristic precedence vs.\ model class and regularization that shape what is learnable.
    \item \textbf{Budget:} depth/time limits vs.\ compute/epoch budgets and early stopping.
\end{itemize}
\end{tcolorbox}

\medskip
\begin{tcolorbox}[summarybox,title={Key takeaways}]
\textbf{Minimum viable mastery}
\begin{itemize}
    \item Symbolic integration is a compact example of a goal-driven system: represent state, apply meaning-preserving actions, and verify outcomes.
    \item Safe moves encode guaranteed transformations; heuristic moves trade certainty for coverage and require backtracking discipline.
    \item Residual checks act as a crisp goal test: differentiate a candidate and measure whether it agrees with the original integrand on the declared domain.
\end{itemize}
\medskip
\textbf{Common pitfalls}
\begin{itemize}
    \item Losing the system-level point in calculus detail: always name the state, action, heuristic, and goal test.
    \item Ignoring domains/branches: substitutions can shrink domains, introduce branch cuts, and invalidate a ``correct'' algebraic rewrite.
    \item Treating heuristics as proofs: heuristic branches must be verified (or backtracked), not trusted.
\end{itemize}
\end{tcolorbox}

\begin{tcolorbox}[summarybox,title={Exercises and lab ideas}]
\begin{itemize}
    \item Implement a minimal example from this chapter and visualize intermediate quantities (plots or diagnostics) to match the pseudocode.
    \item Stress-test a key hyperparameter or design choice discussed here and report the effect on validation performance or stability.
    \item Re-derive one core equation or update rule by hand and check it numerically against your implementation.
\end{itemize}
\medskip
\noindent\textbf{If you are skipping ahead.} Keep the pattern vocabulary: safe vs.\ heuristic moves, explicit budgets, and residual/verification checks. The data-driven chapters reuse the same discipline (objective, constraints, and validation) even though the ``actions'' become parameter updates.
\end{tcolorbox}

\medskip
\paragraph{Where we head next.} For the data-driven thread (datasets, objectives, diagnostics, classification), continue to \Cref{chap:supervised,chap:logistic}. For nonlinear function classes and nonconvex training dynamics, continue through \Crefrange{chap:perceptron}{chap:mlp} and the chapters that follow.



% Chapter 3
\section{Supervised Learning Foundations}\label{chap:supervised}

\Cref{chap:symbolic} illustrated a non-statistical lens: solve problems by transformation search, with explicit goal tests. We now switch to the data-driven lens. Figure \Cref{fig:roadmap} marks this as the core supervised strand.

Building intelligent models is an imprecise science. If we know the relationship between the input and the output, there is no need to infer it: Celsius and Fahrenheit are linked by a simple formula, and many physical laws provide direct mappings from one quantity to another. In the problems that motivate machine learning, the mapping is unknown, messy, or only partially understood, so we settle for an approximation. That approximation might be statistical (learned from data), rule-based (encoded from experience), biologically inspired (neural computation), behavioral (fuzzy rules), or evolutionary (search over candidate solutions). In this chapter we focus on the statistical, data-driven strand: supervised learning.

In this sense, supervised learning is about prediction and inference: given evidence \(\mathbf{x}\), estimate an output \(y\) that you can act on or audit. Other modeling goals exist (summarizing structure, compressing representations, discovering clusters), but supervised learning is the cleanest place to learn the mechanics of fitting models, comparing alternatives, and checking whether your success is real or just memorization.

Supervised learning begins with three commitments: pick a functional form that can plausibly approximate the mapping, collect paired examples of inputs and outputs, and define a quantitative measure of ``how wrong'' a prediction is. Once those are in place, training becomes possible: we adjust the model parameters so the predictions align with the observed outputs on the examples we have.

The word \emph{fitting} is meant literally. In classical curve fitting---and in practical settings like sensor calibration---we choose parameters so a predicted curve (or surface) passes near measured points. Keep the camera thread from \Cref{chap:intro} in mind: a camera system is useful because it can predict something actionable from what it senses. A simple example is exposure calibration: we collect scenes with known reference targets, measure raw sensor readouts \(\mathbf{x}\), and learn parameters that map those readouts to a correction \(y\) so the system produces consistent brightness across conditions. The same pattern repeats at higher levels (object detection scores, tracking signals, alert decisions): the details change, but the core act is the same---use paired input/output examples to fit parameters that make predictions reliable.

This chapter builds the supervised-learning toolkit around that central act. We start by making the pieces explicit (data, models, and losses). Then we show what training is doing when it succeeds, and what it looks like when it fails (underfitting vs.\ overfitting). After that we formalize the standard objective (ERM) and the main ``anti-memorization'' tools (regularization and validation). Finally, we work through linear regression as the first fully transparent case study where you can see the entire pipeline end to end.

\begin{tcolorbox}[summarybox,title={Learning Outcomes}]
\begin{itemize}
    \item Formalize datasets, hypotheses, and empirical risk minimization (ERM) with consistent notation used in \Crefrange{chap:supervised}{chap:logistic}.
    \item Compare common regression/classification losses and regularizers, understanding when to prefer each.
    \item Diagnose under/overfitting with data splits, learning curves, and bias--variance reasoning; use these diagnostics to guide model selection and regularization.
\end{itemize}
\end{tcolorbox}

\begin{tcolorbox}[summarybox,title={Design motif}]
Data \(\rightarrow\) model \(\rightarrow\) objective \(\rightarrow\) audit. This workflow shows up repeatedly in later chapters, even when the models become deeper and the optimization less forgiving.
\end{tcolorbox}

\noindent Before we turn the ``fitting'' story into equations, let us fix the handful of symbols we will reuse for several pages. The goal is not to introduce new notation, but to keep the derivations readable while the ideas are still new.
\begin{itemize}
  \item Data \(X \in \mathbb{R}^{N\times d}\) with rows \(\mathbf{x}_i^\top\); targets \(y_i\in\mathbb{R}\) for regression and \(y_i \in \{0,1\}\) for binary classification. The affine map \(y_{\pm1}=2y-1\) switches to \(\{-1,+1\}\) when margin-based expressions are convenient.
  \item Parameters \(\theta\) (model-specific), weights \(\mathbf{w}\in\mathbb{R}^d\); predictions carry hats: \(\hat{y}_i = h_\theta(\mathbf{x}_i)\), \(\hat{\mathbf{y}}=X\mathbf{w}\).
  \item The loss \(\ell(\hat{y},y)\) is the teacher's grading rubric; the objective aggregates losses over data and adds regularization. Parameters are learned from data; hyperparameters (e.g., \(\lambda\) in regularization) are chosen by validation.
  \item Noise uses \(\varepsilon\); residuals use \(e=y-\hat{y}\). Vectors are bold lowercase, matrices bold uppercase; scalars are italic.
  \item Bias absorption (when used): augmented feature \(\tilde{\mathbf{x}}=[\mathbf{x};1]\) with corresponding augmented weights.
\end{itemize}

\subsection{Problem Setup and Notation}
\label{sec:supervised_problem_setup_and_notation}

We observe a dataset \(\mathcal D = \{(\mathbf x_i, y_i)\}_{i=1}^N\) drawn i.i.d. from an unknown distribution \(\mathcal P\) on the input--output space \(\mathcal X \times \mathcal Y\). A hypothesis (model) \(h_\theta : \mathcal X \to \mathcal Y\) with parameters \(\theta\) produces predictions \(\hat{y}_i = h_\theta(\mathbf x_i)\). A pointwise loss function \(\ell(\hat{y}, y)\) measures the penalty incurred by predicting \(\hat{y}\) when the true label is \(y\).

The \emph{population risk} and \emph{empirical risk} associated with \(h_\theta\) are
\begin{align}
R(h_\theta) &= \mathbb{E}_{(\mathbf x, y)\sim \mathcal P}\big[\ell\big(h_\theta(\mathbf x), y\big)\big], \\
\hat R_N(h_\theta) &= \frac{1}{N} \sum_{i=1}^N \ell\big(h_\theta(\mathbf x_i), y_i\big).
    \label{eq:auto:lecture_supervised:1}
\end{align}
Because \(\mathcal P\) is unknown, learning algorithms minimize empirical proxies of \(R(h_\theta)\). This is the formal version of the ``educated guess'' idea: we posit a model family \(h_\theta\), then use data to choose parameter values that make its predictions behave like the measured input--output pairs. In practice we do this on a \emph{training set} (the data used to fit parameters), and we reserve held-out data to check whether the fitted model is trustworthy; \Cref{sec:lec1_model_selection} makes these evaluation protocols precise.

\subsection{Fitting, Overfitting, and Underfitting}
\label{sec:supervised_fitting_overfitting_and_underfitting}

Fitting is the act of choosing parameters \(\theta\) so a model's predictions match observed data. Concretely, we pick a loss \(\ell\), evaluate it on examples \((\mathbf{x}_i,y_i)\), and use an optimization method to search for parameters that make the aggregate loss small. This is what practitioners usually mean by \emph{training}.

It helps to picture training as repeated adjustment under feedback. You make a prediction, measure the mistake with a loss, update parameters to reduce that mistake, and repeat. In sensor calibration, this feels familiar: if your measured output is consistently off, you change a gain or offset; if it is noisy, you adjust how aggressively you trust any one reading. Supervised learning packages that intuition into a general recipe that can be reused across problems.

The goal is not to ``fit the training set'' as an end in itself. A good fit is one that holds up on new data: the fitted model should behave sensibly on inputs it has not seen. When fitting fails, it tends to fail in one of two recognizable ways.

\paragraph{Underfitting.}
The model family is too rigid for the task, the features do not contain enough information, or the optimization did not do its job. This is the student who cannot solve the practice problems before the exam: the mismatch is obvious even on the training set. The remedy is to change the representation or the hypothesis class, improve the data, or fix the optimization.

\paragraph{Overfitting.}
The model is flexible enough to match the training set by memorizing its quirks. This is the student who memorizes the worked examples so well that a small twist on the exam causes failure. Overfitting can look like success until you test on held-out data.

\paragraph{What we aim for.}
We want a well-fitted model: low training error and comparable validation/test error. The tools below are designed to keep that distinction visible: objectives (losses and regularizers), validation protocols (splits and cross-validation), and diagnostics (learning curves and bias--variance reasoning).

\begin{figure}[!htbp]
    \centering
\begin{tikzpicture}[background rectangle/.style={fill=white}, show background rectangle]
        \begin{axis}[
            width=0.82\linewidth,
            height=0.34\linewidth,
            xlabel={Model complexity},
            ylabel={Error on data},
            xmin=0, xmax=10,
            ymin=0, ymax=1.15,
            xtick=\empty,
            ytick=\empty,
            axis lines=left,
            grid=both,
            minor grid style={gray!12},
            major grid style={gray!25},
            legend style={at={(0.02,0.98)},anchor=north west,draw=none,fill=white},
            legend cell align=left,
            axis background/.style={fill=white},
            clip=true
        ]
            \addplot[cbBlue, thick, domain=0:10, samples=200] {0.95*exp(-0.35*x)+0.05};
            \addlegendentry{Training error}
            \addplot[cbOrange, thick, dashed, domain=0:10, samples=200] {0.18 + 0.015*(x-4.5)^2};
            \addlegendentry{Validation error}

            \draw[gray!60, dashed] (axis cs:3.2,0) -- (axis cs:3.2,1.05);
            \draw[gray!60, dashed] (axis cs:6.2,0) -- (axis cs:6.2,1.05);

            \node[font=\scriptsize, align=center, color=gray!70!black, anchor=north] at (axis description cs:0.16,0.98) {underfit\\(high bias)};
            \node[font=\scriptsize, align=center, color=gray!70!black, anchor=north] at (axis description cs:0.50,0.98) {good fit};
            \node[font=\scriptsize, align=center, color=gray!70!black, anchor=north] at (axis description cs:0.84,0.98) {overfit\\(high variance)};

            \node[
                font=\scriptsize,
                color=cbOrange!80!black,
                fill=white,
                fill opacity=0.85,
                text opacity=1,
                inner sep=1.2pt,
                anchor=south
            ] at (axis cs:4.5,0.22) {best validation};
        \end{axis}
    \ensuretikzbackgroundlayers
    \end{tikzpicture}
    \caption{Underfitting and overfitting as a function of model complexity. Training error typically decreases with complexity, while validation error often has a U-shape. Regularization and model selection aim to operate near the minimum of the validation curve. Use it when deciding whether to add capacity, add data, or add regularization.}
    \label{fig:lec1_fit_regimes}
\end{figure}

\subsection{Empirical Risk Minimization and Regularization}
\label{sec:supervised_empirical_risk_minimization_and_regularization}

To make the informal idea of ``fitting'' mathematically precise, we choose an objective and minimize it. The supervised-learning baseline is \emph{empirical risk minimization}: minimize the average loss on the training data.

This is the first place where the chapter's opening promises become concrete. If we do not know the true input--output law, we still need a disciplined way to compare candidate models and to say whether one parameter choice is better than another. The loss plays the role of the teacher's rubric, and ERM is the simplest way to aggregate that rubric over many examples: instead of arguing about one example at a time, we ask for parameters that perform well on average across the dataset.

The \emph{empirical risk minimizer} (ERM) selects
\begin{equation}
\hat\theta_{\text{ERM}} = \arg\min_{\theta} \; \hat R_N(h_\theta).
\label{eq:auto_supervised_ecc27ebd80}
\end{equation}
To mitigate overfitting, we often add a regularizer \(\Omega(\theta)\) with strength \(\lambda \ge 0\):
\begin{equation}
\hat\theta_\lambda = \arg\min_{\theta} \; \hat R_N(h_\theta) + \lambda\,\Omega(\theta), \qquad \Omega(\theta) \in \{\|\theta\|_2^2,\; \|\theta\|_1,\ldots\}.
\label{eq:auto_supervised_11c23f9474}
\end{equation}

\noindent Regularization is not an arbitrary penalty. It is the mathematical version of a teaching move: if a student can memorize every worked example, you change the exercises so memorization is less effective and understanding is rewarded. Regularization plays the same role. It makes some parameter settings expensive, which pushes learning toward explanations that generalize better.

In supervised learning, this matters because a model can drive training loss down in ways that do not survive contact with new data. Regularization is one of the main tools we use to ``push back'' against memorization: we still fit the data, but we also express a preference for solutions that are stable, simple, or structured in ways that match the problem.

\paragraph{Ridge and lasso.}
Two penalties show up so often that they have become part of the basic vocabulary:
\begin{itemize}
    \item \textbf{Ridge (L2)} adds \(\|\theta\|_2^2\), which shrinks weights smoothly and stabilizes solutions when features are correlated.
    \item \textbf{Lasso (L1)} adds \(\|\theta\|_1\), which tends to set some weights to exactly zero, yielding sparse models and a form of feature selection.
\end{itemize}
The difference is easiest to remember geometrically: L2 has round level sets, while L1 has corners, and corners create exact zeros.

\begin{tcolorbox}[summarybox,title={Regularization: L1/L2 and scaling}]
\begin{itemize}
    \item \textbf{Why regularize?} Flexible models can fit training data by effectively memorizing idiosyncrasies (noise, quirks of the sample) rather than capturing stable structure. Regularization makes such memorization expensive and rewards explanations that survive on held-out data.
    \item \textbf{L2 (ridge)} shrinks weights smoothly, is rotationally invariant, and works well when features are dense and correlated.
    \item \textbf{L1 (lasso)} promotes sparsity, effectively performing feature selection when many coefficients should be zero.
    \item \textbf{Why the names?} ``Ridge'' refers to the ridge-like valleys that appear in least-squares objectives under multicollinearity; the L2 penalty lifts the valley floor and stabilizes the solution. ``LASSO'' is an acronym for \emph{Least Absolute Shrinkage and Selection Operator}.
    \item \textbf{Why L1 vs.\ L2 feels different:} the L2 penalty discourages large coefficients but rarely drives them exactly to zero, while the L1 penalty creates corners in the geometry that tend to set some coefficients to exactly zero.
    \item \textbf{Standardization} (zero mean, unit variance) is essential before applying L1/L2/elastic-net so the penalty treats all dimensions comparably.
    \item With an intercept term, centering \(y\) makes the algebra cleaner; ridge and lasso still apply directly once features are scaled.
\end{itemize}
\end{tcolorbox}

\begin{figure}[h]
    \centering
\begin{tikzpicture}[scale=0.95, background rectangle/.style={fill=white}, show background rectangle]
        \tikzstyle{ax}=[gray!60, -{Stealth[length=2mm]}]
        \tikzstyle{contour}=[gray!50, thick]
        \tikzstyle{ball}=[cbBlue!70!black, thick]

        % Left: L2 ball
        \begin{scope}[xshift=-4.4cm]
            \draw[ax] (-1.9,0) -- (2.1,0) node[anchor=west] {$\theta_1$};
            \draw[ax] (0,-1.9) -- (0,2.1) node[anchor=south] {$\theta_2$};
            \draw[contour] (0.7,0.2) circle (1.35);
            \draw[contour] (0.7,0.2) circle (0.95);
            \draw[contour] (0.7,0.2) circle (0.55);
            \draw[ball] (0,0) circle (1.25);
            \fill[cbOrange!80!black] (0.88,0.88) circle (1.6pt);
            \node[font=\scriptsize, anchor=west] at (1.05,0.95) {solution};
            \node[font=\scriptsize, align=center] at (0,-2.35) {L2 (ridge):\\round constraint};
        \end{scope}

        % Right: L1 ball
        \begin{scope}[xshift=4.4cm]
            \draw[ax] (-1.9,0) -- (2.1,0) node[anchor=west] {$\theta_1$};
            \draw[ax] (0,-1.9) -- (0,2.1) node[anchor=south] {$\theta_2$};
            \draw[contour] (0.7,0.2) circle (1.35);
            \draw[contour] (0.7,0.2) circle (0.95);
            \draw[contour] (0.7,0.2) circle (0.55);
            \draw[ball] (0,1.25) -- (1.25,0) -- (0,-1.25) -- (-1.25,0) -- cycle;
            \fill[cbOrange!80!black] (0,1.25) circle (1.6pt);
            \node[font=\scriptsize, anchor=west] at (0.15,1.33) {solution};
            \node[font=\scriptsize, align=center] at (0,-2.35) {L1 (lasso):\\corners encourage zeros};
        \end{scope}
    \ensuretikzbackgroundlayers
    \end{tikzpicture}
    \caption{Why L1 promotes sparsity. Minimizing loss subject to an L2 constraint tends to hit a smooth boundary; an L1 constraint has corners aligned with coordinate axes, so tangency often occurs at a point where some coordinates are exactly zero. Use it when choosing between L1 and L2 penalties for feature selection.}
    \label{fig:lec1_l1_l2_geometry}
\end{figure}

\begin{figure}[!htbp]
    \centering
\begin{tikzpicture}[background rectangle/.style={fill=white}, show background rectangle]
    \begin{axis}[
        width=0.95\linewidth,
        height=0.36\linewidth,
        xlabel={Regularization strength $\lambda$},
        ylabel={Coefficient value},
        xmin=0, xmax=5,
        ymin=-1.1, ymax=1.1,
        legend style={at={(0.985,0.03)},anchor=south east,draw=none,fill=white,fill opacity=0.85,text opacity=1},
            grid=both,
            minor grid style={gray!12},
            major grid style={gray!25},
            axis background/.style={fill=white}
        ]
            \addplot[cbBlue, thick, domain=0:5, samples=200] {max(0, 1 - 0.35*x)};
            \addlegendentry{$\beta_1$ (shrinks)}
            \addplot[cbOrange, thick, dashed, domain=0:5, samples=200] {max(0, 0.7 - 0.45*x)};
            \addlegendentry{$\beta_2$ (hits zero)}
            \addplot[cbGreen, thick, dash dot, domain=0:5, samples=200] {-max(0, 0.9 - 0.55*x)};
            \addlegendentry{$\beta_3$ (hits zero)}
        \end{axis}
    \ensuretikzbackgroundlayers
\end{tikzpicture}
    % Avoid inline math in captions; it wraps poorly in some EPUB renderers.
    \caption{A typical lasso path as the regularization strength increases. Coefficients shrink, and some become exactly zero, yielding sparse models. Use it when interpreting how penalty strength trades accuracy for sparsity.}
    \label{fig:lec1_lasso_path}
\end{figure}

\subsection{Elastic-net paths and cross-validation}
\label{sec:supervised_elastic_net_paths_and_cross_validation}
Pure L1 or L2 penalties rarely dominate modern workflows; the \emph{elastic net} mixes them to balance sparsity and stability:
\begin{equation}
    \hat{\theta}_{\alpha,\lambda} = \arg\min_{\theta} \hat{R}_N(h_\theta) + \lambda\Big(\alpha \|\theta\|_1 + \tfrac{1-\alpha}{2} \|\theta\|_2^2\Big), \qquad \alpha \in [0,1].
\label{eq:auto_supervised_dda9108db8}
\end{equation}
Setting \(\alpha=1\) recovers the lasso, \(\alpha=0\) yields ridge, and intermediate values trace a solution path that tends to group correlated features while still pruning irrelevant ones. In practice we standardize the features once, draw a logarithmic grid of \(\lambda\) values, and run \(K\)-fold cross-validation for each pair \((\alpha,\lambda)\). The ``one-standard-error'' rule selects the largest \(\lambda\) whose validation error is within one standard error of the minimum. It gives a stable operating point and avoids over-interpreting tiny validation differences.

\subsection{Common Loss Functions}
\label{sec:supervised_common_loss_functions}

Loss functions make the teacher signal quantitative: they decide what counts as a small mistake, what counts as a large one, and which kinds of errors matter most.
For binary classification with labels \(y \in \{-1, +1\}\) and margin \(z = y\,f(\mathbf x)\), two standard losses are
\begin{align}
\ell_{\text{hinge}}(y,z) &= \max\bigl(0, 1 - z\bigr), &
\ell_{\text{logistic}}(y,z) &= \log\bigl(1 + e^{-z}\bigr).
    \label{eq:auto:lecture_supervised:2}
\end{align}
Here \(y\in\{-1,+1\}\); when labels are instead coded as \(y\in\{0,1\}\) (common in probability-of-class formulations), the margin expression uses \(y_{\pm1}=2y-1\) to map between codings.
\Cref{fig:lec1_class_losses} visualizes these curves together with the squared hinge so you can match the algebra to the margin geometry.
For regression with residual \(e = y - \hat{y}\), we frequently use
\begin{align}
    \ell_{\text{sq}}(e) &= \tfrac{1}{2} e^2, &
    \ell_{\text{abs}}(e) &= |e|.
    \label{eq:auto:lecture_supervised:3}
\end{align}
The Huber loss interpolates between these: it is quadratic when \(|e| \le \delta\) and linear beyond that threshold (here the plot uses \(\delta=1\)), reducing sensitivity to outliers while remaining smooth around the origin.

\begin{table}[!htbp]
\centering
\caption{Common losses and typical use (reference for \Crefrange{chap:supervised}{chap:perceptron}). Use it when matching a loss to a modeling assumption and a downstream decision metric.}
\begin{tabular}{@{}>{\raggedright\arraybackslash}p{0.22\linewidth} >{\raggedright\arraybackslash}p{0.25\linewidth} >{\raggedright\arraybackslash}p{0.4\linewidth}@{}}
\toprule
\textbf{Loss} & \textbf{Convex?} & \textbf{Typical use} \\
\midrule
Squared error $\tfrac{1}{2}e^2$ & Yes & Regression when Gaussian noise is plausible; differentiable everywhere. \\
Absolute error $|e|$ & Yes & Robust regression with Laplacian noise assumptions; non-differentiable at 0. \\
Huber (quadratic $\rightarrow$ linear) & Yes & Regression when moderate outliers are present; smooth near zero. \\
Logistic (binary cross\hyp{}entropy) & Yes & Probabilistic classification; pairs naturally with sigmoid. \\
Hinge / squared hinge & Yes & Margin-based classifiers (SVMs, large-margin perceptrons). \\
\bottomrule
\end{tabular}
\end{table}

\begin{figure}[!htbp]
    \centering
\begin{tikzpicture}[background rectangle/.style={fill=white}, show background rectangle]
        \begin{axis}[
            width=0.75\linewidth,
            height=0.35\linewidth,
            xlabel={Margin $z$},
            ylabel={Loss},
            xmin=-3, xmax=3,
            ymin=0, ymax=2.5,
            legend style={at={(0.03,0.97)},anchor=north west},
            samples=300,
            grid=both,
            minor grid style={gray!15},
            major grid style={gray!30},
            axis background/.style={fill=white}
        ]
            \addplot[cbBlue, thick, domain=-3:3, samples=300]{ (x <= 1 ? max(0,1-x) : 0) };
            \addlegendentry{Hinge}
            \addplot[cbOrange, thick, domain=-3:3, samples=300]{ ln(1 + exp(-x)) };
            \addlegendentry{Logistic}
            \addplot[cbGreen, thick, dashed, domain=-3:3, samples=300]{ (x <= 1 ? max(0,1-x)^2 : 0) };
            \addlegendentry{Squared hinge}
        \end{axis}
    \ensuretikzbackgroundlayers
\end{tikzpicture}
    % Avoid inline math in captions; it wraps poorly in some EPUB renderers.
    \caption{Classification losses as functions of the signed margin z. Use it when comparing how different losses treat confident mistakes and near-boundary points.}
    \label{fig:lec1_class_losses}
\end{figure}

\begin{figure}[!htbp]
    \centering
\begin{tikzpicture}[background rectangle/.style={fill=white}, show background rectangle]
        \begin{axis}[
            width=0.75\linewidth,
            height=0.35\linewidth,
            xlabel={Error $e$},
            ylabel={Loss},
            xmin=-3, xmax=3,
            ymin=0, ymax=5,
            legend style={at={(0.02,0.98)},anchor=north west},
            samples=300,
            grid=both,
            minor grid style={gray!15},
            major grid style={gray!30},
            axis background/.style={fill=white}
        ]
            \addplot[cbBlue, thick, domain=-3:3, samples=300]{0.5*x^2};
            \addlegendentry{Squared}
            \addplot[cbOrange, thick, domain=-3:3, samples=300]{abs(x)};
            \addlegendentry{Absolute}
            \addplot[cbGreen, thick, dashed, domain=-3:3, samples=300]{ (abs(x) <= 1 ? 0.5*x^2 : abs(x) - 0.5) };
            \addlegendentry{Huber}
        \end{axis}
    \ensuretikzbackgroundlayers
    \end{tikzpicture}
    \caption{Regression losses versus prediction error. The Huber loss transitions from quadratic to linear to reduce sensitivity to outliers. Use it when choosing a loss that is robust to heavy-tailed noise.}
    \label{fig:lec1_reg_losses}
\end{figure}

\FloatBarrier

\subsection{Model Selection, Splits, and Learning Curves}\label{sec:lec1_model_selection}

Up to this point, we have defined what it means to fit: choose a model family, pick an objective (loss plus any regularizer), and tune parameters to reduce that objective on observed examples. The next question is how to choose among competing model families, hyperparameters, and training procedures without fooling ourselves. Model selection is the discipline of making those choices using validation data, while keeping one final dataset split untouched so that the reported performance remains honest.

In other words, this is where the chapter's ``audit'' step becomes operational: we decide what to trust by checking performance on data the model has not been allowed to fit.

Practical workflows allocate data into training, validation, and test portions. Training data are used to fit parameters; validation data guide choices such as hyperparameters and model families; and the test set provides an unbiased audit once those choices are fixed. The key habit is the separation of roles: training is where you allow the model to ``learn'' (and potentially overfit), validation is where you decide what kind of learning you trust, and the test set is the final audit.

\begin{tcolorbox}[summarybox,title={Risk \& audit}]
\begin{itemize}
    \item \textbf{Leakage:} avoid split mistakes (duplicates, near-duplicates, time leakage) that inflate validation accuracy.
    \item \textbf{Metric mismatch:} align the loss you optimize with the metric you report (and the decision you must make).
    \item \textbf{Overfitting signals:} track training vs.\ validation curves and use learning curves to diagnose data hunger vs.\ excess capacity.
    \item \textbf{Distribution shift:} audit performance by slice (population, device, lighting, region) rather than relying on one aggregate score.
    \item \textbf{Calibration:} check reliability when probabilities drive actions (thresholds, alerts, resource allocation).
    \item \textbf{Reporting discipline:} log data split policy, seeds, and selection criteria; \Cref{app:repro_standards} defines the book-wide template.
\end{itemize}
\end{tcolorbox}

\begin{tcolorbox}[summarybox,title={Proper scoring rules and calibration},breakable]
\small
\begin{itemize}
    \item \textbf{Log loss (cross\hyp{}entropy)} and the \textbf{Brier score} are \emph{proper} scoring rules: in expectation, they are minimized by predicting the true class probability.
    \item \textbf{Brier} is squared error in probability space; it penalizes confident mistakes less harshly than log loss and is often paired with reliability diagrams.
    \item \textbf{Log loss} heavily punishes overconfident errors (loss \(\to \infty\) as predicted probability \(\to 0\) on the true class), so it is a natural objective when probabilities will be thresholded downstream.
    \item \textbf{Practical tip:} train with log loss, but monitor both log loss and Brier score on validation data to catch calibration issues early.
\end{itemize}
\normalsize
\end{tcolorbox}

\begin{figure}[h]
    \centering
\begin{tikzpicture}[x=0.9\linewidth,y=1.0cm, font=\scriptsize, background rectangle/.style={fill=white}, show background rectangle]
        \def\barheight{0.6}
        \def\train{0.7}
        \def\val{0.15}
        \def\test{0.15}
        \path[rounded corners=2pt, draw=gray!60, line width=0.6pt] (0,0) rectangle (1,\barheight);
        \path[rounded corners=2pt, fill=cbBlue!25] (0,0) rectangle (\train,\barheight);
        \path[fill=cbOrange!30] (\train,0) rectangle (\train+\val,\barheight);
        \path[rounded corners=2pt, fill=cbGreen!25] (\train+\val,0) rectangle (1,\barheight);
        \draw[gray!60, line width=0.6pt, rounded corners=2pt] (0,0) rectangle (1,\barheight);
        \draw[gray!50, line width=0.5pt] (\train,0) -- (\train,\barheight);
        \draw[gray!50, line width=0.5pt] (\train+\val,0) -- (\train+\val,\barheight);
        \node[align=center] at (\train/2,0.5*\barheight) {Train\\70\%};
        \node[align=center] at (\train+\val/2,0.5*\barheight) {Validation\\15\%};
        \node[align=center] at (1-\test/2,0.5*\barheight) {Test\\15\%};
        \draw[cbGray, -{Stealth[length=2.2mm]}, line width=0.8pt] (0,-0.38) -- node[below, text=cbGray, yshift=-2pt]{Shuffle, then split once or use $K$-fold CV} (1,-0.38);
    \ensuretikzbackgroundlayers
    \end{tikzpicture}
    \caption{Dataset partitioning into training, validation, and test segments. Any resampling scheme should preserve disjoint evaluation data; when classes are imbalanced, shuffle within strata so each split reflects the overall class mix. Use it when designing splits that support trustworthy model selection and reporting.}
    \label{fig:lec1_splits}
\end{figure}

\begin{tcolorbox}[summarybox,title={A concrete toy task}]
As a reference point, keep in mind one small binary classification problem: a two\hyp{}moons toy with a standard train/validation/test split. It is deliberately simple, but it is rich enough to reveal the recurring failure modes (memorization, metric mismatch, split leakage) and the recurring remedies (regularization, validation, and diagnostics).
\end{tcolorbox}

To make the workflow concrete, \Cref{fig:lec1_pipeline} summarizes the standard ERM pipeline from dataset to model selection.

\begin{figure}[h]
    \centering
\begin{tikzpicture}[scale=0.90, transform shape, node distance=0.60cm, every node/.style={font=\scriptsize}, background rectangle/.style={fill=white}, show background rectangle]
    \tikzstyle{block}=[draw, rounded corners, fill=gray!10, minimum width=1.46cm, minimum height=0.85cm, align=center]
    \node[block] (data) {Dataset\\$\mathcal{D}$};
    \node[block, right=of data] (split) {Stratified\\train/val/test\\split};
    \node[block, right=of split] (train) {Train model\\(ERM + regularizer)};
        \node[block, right=of train] (val) {Validate / tune\\hyperparameters};
        \node[block, right=of val] (test) {Frozen model\\tested once\\{\scriptsize(best model only)}};
        \draw[->, thick] (data) -- (split);
        \draw[->, thick] (split) -- (train);
        \draw[->, thick] (train) -- (val);
        \draw[->, thick] (val) -- (test);
        \draw[->, thick, bend right=40] (train.south) to node[below, font=\scriptsize]{hyperparameter update} (val.south);
    \ensuretikzbackgroundlayers
    \end{tikzpicture}
    \caption{Mini ERM pipeline (split once, iterate train/validate, then test only the best model on the held-out set). Use it when enforcing a clean separation between tuning and final reporting.}
    \label{fig:lec1_pipeline}
\end{figure}

Learning curves plot training and validation error against the number of training examples, revealing underfitting or overfitting regimes.

\begin{figure}[h]
    \centering
\begin{tikzpicture}[background rectangle/.style={fill=white}, show background rectangle]
        \begin{axis}[
            width=0.7\linewidth,
            height=0.35\linewidth,
            xlabel={Training examples},
            ylabel={Loss / metric},
            xmin=0, xmax=100,
            ymin=0, ymax=0.5,
            legend style={at={(0.5,1.02)},anchor=south,legend columns=2},
            grid=both,
            minor grid style={gray!15},
            major grid style={gray!30},
            axis background/.style={fill=white}
        ]
            \addplot[cbBlue, thick, smooth, mark=*, mark repeat=2, mark options={fill=cbBlue}] table {
                n err
                5 0.45
                10 0.38
                20 0.30
                40 0.22
                60 0.18
                80 0.16
                100 0.15
            };
            \addlegendentry{Training error}
            \addplot[cbOrange, thick, dashed, smooth, mark=square*, mark repeat=2, mark options={fill=cbOrange}] table {
                n err
                5 0.46
                10 0.42
                20 0.36
                40 0.30
                60 0.27
                80 0.25
                100 0.24
            };
            \addlegendentry{Validation error}
            % Patience band
            \addplot[fill=cbOrange!20, opacity=0.3, draw=none] coordinates {(55,0) (75,0) (75,0.5) (55,0.5)} -- cycle;
            \node[anchor=south east, font=\scriptsize, color=cbOrange!80!black] at (axis cs:72,0.48) {patience window};
        \end{axis}
\ensuretikzbackgroundlayers
\end{tikzpicture}
    \caption{Learning curves reveal under/overfitting: the validation curve flattens while additional data continue to decrease training error only marginally. A shaded patience window marks when early stopping would halt if no validation improvement occurs. Use it when deciding whether you need more data, more capacity, or different regularization.}
    \label{fig:lec1_learning_curves}
\end{figure}

\begin{tcolorbox}[summarybox,title={Data-leakage checklist}]
\begin{itemize}
    \item Split data before any preprocessing or feature selection.
    \item Fit scalers/imputers/dimensionality-reduction transforms on the training fold only; reuse fitted parameters on validation/test (or within each CV fold via pipelines).
    \item Respect temporal order for time-series; avoid target/future-derived features.
    \item Wrap preprocessing + model in a pipeline for cross-validation so transformers refit inside each fold.
\end{itemize}
\end{tcolorbox}

\begin{tcolorbox}[summarybox,title={Bias--variance at a glance}]
\begin{itemize}
    \item \textbf{High bias (underfit):} train and validation errors both plateau high and together; add capacity/features or reduce regularization.
    \item \textbf{High variance (overfit):} train error low, validation error high/diverging; add data, strengthen regularization, or use early stopping.
    \item \textbf{Well fit:} train/validation track closely and decrease or level off at low error; further gains require better data or priors.
\end{itemize}
\end{tcolorbox}

\noindent Learning curves explain \emph{why} the train/validation split is useful: they show whether more data, more capacity, or more regularization is the lever that actually moves the validation error. Once you can read these curves, a natural next question is what happens as we scale up data and model size. The aside below summarizes two modern empirical patterns that are best treated as guidance, not as a recipe.

\begin{tcolorbox}[summarybox,title={Aside: scaling laws and double descent}]
The simplest story is the classical bias--variance picture: as model capacity grows, training error falls, and validation error often has a U-shape. In modern overparameterized models, that picture can be incomplete. You may see \emph{double descent}: after the classical U-shape, error can decrease again once model size exceeds the interpolation threshold \citep{Belkin2019}. You may also hear \emph{scaling laws}: in some regimes, validation loss decreases roughly as a power law of compute, data, and model size \citep{Kaplan2020,Hoffmann2022}.

Treat both as diagnostics rather than guarantees. Use them to decide whether to collect more data, shrink or expand a model, or regularize more aggressively, but still make final choices by comparing validation curves. Do not chase the interpolation peak as a goal.
\end{tcolorbox}

\begin{figure}[t]
    \centering
    \includegraphics[width=0.78\linewidth]{assets/lec1/lec1_reliability_double}
    \caption[Calibration and capacity diagnostics (reliability and double descent)]{Calibration and capacity diagnostics. Left: reliability diagram with binned predicted probabilities vs. empirical accuracy; Expected Calibration Error (ECE) measures deviation from the diagonal. Right: illustrative double\hyp{}descent risk vs.\ model size (log\hyp{}scale on the x\hyp{}axis); the dashed line sketches a scaling\hyp{}law trend for choosing capacity/regularization. Use it when deciding whether to prioritize calibration, more data, more capacity, or stronger regularization.}
    \label{fig:lec1-calibration-double-descent}
\end{figure}

Regularization trades model complexity for generalization; \Cref{fig:lec1_ridge} depicts the effect of ridge penalties on the weight norm.

\begin{figure}[h]
    \centering
\begin{tikzpicture}[background rectangle/.style={fill=white}, show background rectangle]
        \begin{axis}[
            width=0.7\linewidth,
            height=0.35\linewidth,
            xlabel={$\lambda$},
            ylabel={$\|\theta\|_2$},
            xmin=0, xmax=5,
            ymin=0, ymax=1.1,
            samples=200,
            axis background/.style={fill=white}
        ]
            \addplot[cbBlue, thick, domain=0:5] {1/(1+0.5*x)};
        \end{axis}
    \ensuretikzbackgroundlayers
    \end{tikzpicture}
    % Avoid inline math in captions; it wraps poorly in some EPUB renderers.
    \caption{Ridge regularization shrinks parameter norms as the penalty strength increases. Use it when tuning weight decay to control variance without forcing sparsity.}
    \label{fig:lec1_ridge}
\end{figure}

\subsection{Linear regression: a first full case study}\label{sec:linear_regression_closed}

Up to this point, the supervised-learning pipeline has been described in abstract terms: a dataset, a hypothesis class, an objective, and an audit. Linear regression is where those pieces become concrete enough that you can see every moving part at once.

\noindent A useful habit, before you commit to any model family, is to ask whether there is any signal to model in the first place. If the relationship were deterministic (like Celsius \(\leftrightarrow\) Fahrenheit), there would be nothing to learn. In supervised learning we assume the relationship is statistical: the same input can map to different outputs because of noise, missing variables, or genuine uncertainty. For simple problems, a scatter plot and a correlation coefficient can reveal whether a linear trend is even plausible. For high-dimensional data, analogous sanity checks (feature scaling, collinearity) help you decide whether linear regression is a sensible starting point or merely a baseline.

\noindent In this section, the coefficients \(\boldsymbol{\beta}\) are the knobs we turn. ``Learning'' means using training pairs \((X,\mathbf{y})\) to estimate \(\boldsymbol{\beta}\) so predictions match observed targets as well as they can under the chosen objective. Because least squares is a convex, closed-form problem, repeating the fit with the same data returns the same solution (up to numerical tolerance). That transparency is exactly why linear regression is worth treating carefully: it makes the ideas of losses, optimization, regularization, and validation feel concrete before we move on to models where the same pipeline is less visible.

\paragraph{Model.}
Given inputs \(\mathbf{x}_i\in\mathbb{R}^d\) and continuous targets \(y_i\in\mathbb{R}\), the linear model predicts
\begin{equation}
    \hat{\mathbf{y}} = X\boldsymbol{\beta}, \qquad X\in\mathbb{R}^{N\times d}.
    \label{eq:linear_model}
\end{equation}
Equivalently, \(\hat{y}_i = \mathbf{x}_i^\top \boldsymbol{\beta}\) for each data point. The vector \(\boldsymbol{\beta}\) is the set of adjustable parameters: \emph{fitting} the model means choosing \(\boldsymbol{\beta}\) so that predictions align with observed outputs. The residual \(\mathbf{e}=\mathbf{y}-\hat{\mathbf{y}}\) captures what the model fails to explain on the data at hand.

\paragraph{A noise model (why squared error shows up).}
A common way to formalize ``measurement scatter'' is to write
\begin{equation}
    y_i = \mathbf{x}_i^\top\boldsymbol{\beta} + \varepsilon_i,
    \qquad \varepsilon_i \sim \mathcal{N}(0,\sigma^2),
\label{eq:auto_supervised_2dec229a60}
\end{equation}
where \(\varepsilon_i\) is observation noise (sensor noise, unmodeled effects, annotation noise, etc.). Under this assumption,
\begin{equation}
    p(y_i\mid \mathbf{x}_i,\boldsymbol{\beta})=\mathcal{N}(\mathbf{x}_i^\top\boldsymbol{\beta},\sigma^2),
\label{eq:auto_supervised_4f4ef34b0d}
\end{equation}
and (assuming i.i.d.\ observations) the likelihood factorizes:
\begin{equation}
    p(\mathbf{y}\mid X,\boldsymbol{\beta})=\prod_{i=1}^N p(y_i\mid \mathbf{x}_i,\boldsymbol{\beta}).
\label{eq:auto_supervised_bb73b50b8f}
\end{equation}
Maximizing the (log) likelihood is equivalent to minimizing the negative log-likelihood, and for Gaussian noise that becomes (up to constants and a scale factor \(1/(2\sigma^2)\)) the familiar sum of squared errors:
\begin{equation}
    -\log p(\mathbf{y}\mid X,\boldsymbol{\beta})
    = \frac{1}{2\sigma^2}\sum_{i=1}^N (y_i-\mathbf{x}_i^\top\boldsymbol{\beta})^2 + \text{const}.
\label{eq:auto_supervised_18a9c6ace6}
\end{equation}
This is the simplest example of a recurring theme: if you propose an ``educated guess'' model for how data are generated, training often becomes ``minimize a loss''.

\paragraph{Objective.}
To fit \(\boldsymbol{\beta}\), we need a grading rubric. Squared error is the standard starting point because it is smooth and strongly penalizes large mistakes:
\begin{equation}
    L(\boldsymbol{\beta}) = \tfrac{1}{2}\|\mathbf{y}-X\boldsymbol{\beta}\|_2^2.
\label{eq:auto_supervised_57c930d476}
\end{equation}
The gradient is simple,
\begin{equation}
    \nabla_{\boldsymbol{\beta}}L(\boldsymbol{\beta}) = X^\top(X\boldsymbol{\beta}-\mathbf{y}),
\label{eq:auto_supervised_9bb8d7206b}
\end{equation}
so gradient descent is explicit: \(\boldsymbol{\beta}\leftarrow \boldsymbol{\beta}-\eta X^\top(X\boldsymbol{\beta}-\mathbf{y})\). This same loop reappears later when the model is no longer linear and the loss is no longer quadratic.

\paragraph{Closed form and geometry.}
Least squares is convex and satisfies the normal equations:
\begin{equation}
    X^\top X\,\hat{\boldsymbol{\beta}} = X^\top \mathbf{y}.
\label{eq:auto_supervised_bce6291d37}
\end{equation}
When \(X^\top X\) is invertible, the solution can be written explicitly as
\begin{equation}
    \hat{\boldsymbol{\beta}} = (X^\top X)^{-1}X^\top \mathbf{y}.
\label{eq:auto_supervised_44e6ba213f}
\end{equation}
Geometrically, the prediction \(\hat{\mathbf{y}}\) is the orthogonal projection of \(\mathbf{y}\) onto the column space of \(X\). In code, solve the linear system (QR/SVD) rather than forming \((X^\top X)^{-1}\) explicitly; collinearity can make \(X^\top X\) poorly conditioned even when the mathematics is correct.

\paragraph{Where overfitting enters.}
With raw features, a linear model may underfit; with aggressive feature expansions (polynomials, splines, kernels, learned features), the same least-squares machinery can overfit. This is where the earlier tools matter. Regularization (ridge, lasso, elastic net) makes memorization harder; validation selects hyperparameters; learning curves diagnose whether error is limited by bias, variance, or data.

\paragraph{Ridge and lasso in one line.}
Ridge adds \(\lambda\|\boldsymbol{\beta}\|_2^2\) to the objective, shrinking coefficients and stabilizing solutions when features are correlated; lasso uses \(\|\boldsymbol{\beta}\|_1\) and tends to drive some coefficients to exactly zero. The ridge shrinkage behavior is visualized in \Cref{fig:lec1_ridge}.

\medskip
\noindent The discipline of supervised learning is reusable across models: define a dataset and a hypothesis class, choose an objective (loss plus regularizer), optimize it, and then audit generalization with clean train/validation/test separation. In \Cref{chap:logistic}, we apply this toolkit to classification, where the loss becomes a Bernoulli negative log-likelihood (cross\hyp{}entropy) and the evaluation tools expand (confusion matrices, ROC/PR curves, and calibration).

\begin{tcolorbox}[summarybox,title={Key takeaways}]
\textbf{Minimum viable mastery}
\begin{itemize}
    \item Supervised learning chooses a hypothesis class and fits parameters by minimizing empirical risk, then audits generalization on held-out data.
    \item Overfitting is a training success but a deployment failure; regularization and validation protocols are the practical defenses.
    \item Learning curves and bias--variance reasoning are diagnostics: they help decide whether to add data, change capacity, or adjust regularization.
\end{itemize}
\medskip
\textbf{Common pitfalls}
\begin{itemize}
    \item Tuning on the test set (or peeking repeatedly) turns the test set into training data.
    \item Leakage through preprocessing: fit scalers/encoders/imputers on the training split only, then apply to val/test.
    \item Over-interpreting a single metric: use learning curves and slice audits, not only one headline number.
\end{itemize}
\end{tcolorbox}

\begin{tcolorbox}[summarybox,title={Exercises and lab ideas}]
\begin{itemize}
    \item Implement linear regression with ridge regularization using both (i) a closed-form solve (QR/SVD) and (ii) gradient descent; compare validation curves as \(\lambda\) varies.
    \item Create a controlled overfitting experiment: increase polynomial feature degree or add noisy features, then use learning curves (\Cref{fig:lec1_learning_curves}) to diagnose bias vs.\ variance and decide how much regularization is needed.
    \item Demonstrate a leakage failure mode by fitting preprocessing on the full dataset (incorrect) versus on the training split only (correct); report the difference in test error.
\end{itemize}
\medskip
\noindent\textbf{If you are skipping ahead.} Keep the ERM loop and split hygiene close: model class, loss/regularizer, optimizer, and a clean train/val/test protocol. Those ideas are assumed immediately in \Cref{chap:logistic} and reused throughout the neural chapters.
\end{tcolorbox}

\medskip
\paragraph{Where we head next.} \Cref{chap:logistic} extends this ERM setup from continuous targets to discrete labels: outputs become class probabilities, the loss becomes cross\hyp{}entropy, and evaluation adds confusion matrices, ROC/PR curves, and threshold selection.

\paragraph{References.} Full citations for works mentioned in this chapter appear in the book-wide bibliography.

\FloatBarrier

%Chapter 4
\section{Classification and Logistic Regression}\label{chap:logistic}
\begin{tcolorbox}[summarybox,title={Learning Outcomes}]
After this chapter, you should be able to:
\begin{itemize}
  \item Derive the logistic log\hyp{}likelihood and its gradient.
  \item Explain the NLL (cross\hyp{}entropy) connection and convexity.
  \item Extend to softmax regression for multiclass problems.
\end{itemize}
\end{tcolorbox}

\Cref{chap:symbolic} illustrated one view of intelligent behavior as transformation search with explicit goal tests, while \Cref{chap:supervised} introduced the data-driven view: represent a task with data, choose a hypothesis class, minimize empirical risk under regularization, and audit performance on held-out data. This chapter extends that toolkit from regression to classification. The roadmap in \Cref{fig:roadmap} places this chapter on the core supervised path.

\begin{tcolorbox}[summarybox,title={Design motif}]
Keep the workflow, swap the likelihood: logistic regression keeps the linear score but changes the probabilistic model (Bernoulli likelihood) and the loss (negative log\hyp{}likelihood, i.e., cross\hyp{}entropy).
\end{tcolorbox}

\subsection{From regression to classification}
\label{sec:logistic_from_regression_to_classification}
Linear regression models a continuous target \(y\) and, under a Gaussian noise model, yields a closed-form solution via the normal equations (\Cref{chap:supervised}, \cref{sec:linear_regression_closed}). Classification changes the output space: \(y\) is a discrete label, and the model predicts class probabilities rather than raw responses. The ERM pipeline remains the same, but the loss becomes the negative log-likelihood (binary cross\hyp{}entropy) and optimization is typically iterative.

Throughout this chapter we use \(y\in\{0,1\}\) by default, switching to \(y_{\pm1}=2y-1\) only when margin-based expressions are convenient. Predictions carry hats (e.g., \(\hat{y}\)), \(\varepsilon\) denotes noise, and \(e=y-\hat{y}\) denotes residuals. For a refresher on data splits, learning curves, and the bias--variance vocabulary, see \Cref{chap:supervised}; here we focus on the logistic-specific modeling and diagnostics.

\subsection{Classification problem statement}
\label{sec:logistic_classification_problem_statement}

In this chapter, we shift our attention to a fundamentally different type of problem: \emph{classification}. Unlike regression, where the output \(y\) is continuous, classification predicts a \emph{discrete label}. We will start with the binary case \(y\in\{0,1\}\), where the goal is to estimate a probability
\[
\pi(\mathbf{x}) = P(y=1\mid \mathbf{x}),
\]
and then produce a decision by thresholding \(\pi(\mathbf{x})\) (or by comparing class probabilities when there are more than two classes). For multiclass problems the label belongs to one of \(K\) classes,
\[
y \in \{c_1, c_2, \ldots, c_K\},
\]
and the goal is to estimate \(P(y=c_k\mid \mathbf{x})\) for each \(k\); we return to the softmax extension later in the chapter.

\subsection{Bayes Optimal Classifier}
\label{sec:logistic_bayes_optimal_classifier}

A fundamental result in statistical pattern recognition is that the \emph{Bayes classifier} is the optimal classifier in terms of minimizing the expected classification error. The Bayes classifier assigns \(\mathbf{x}\) to the class:
\[
\hat{y} = \arg\max_{c_k \in \{c_1,\ldots,c_K\}} P(y = c_k \mid \mathbf{x}).
\]

Using Bayes' theorem, the posterior probability can be expressed as
\begin{equation}
    P(y = c_k \mid \mathbf{x}) = \frac{P(\mathbf{x} \mid y = c_k) P(y = c_k)}{P(\mathbf{x})}.
    \label{eq:bayes_theorem}
\end{equation}
Here
\begin{itemize}
    \item \(P(\mathbf{x} \mid y = c_k)\) is the \emph{class-conditional likelihood},
    \item \(P(y = c_k)\) is the \emph{prior probability} of class \(c_k\),
    \item \(P(\mathbf{x}) = \sum_{j=1}^K P(\mathbf{x} \mid y = c_j) P(y = c_j)\) is the marginal likelihood of the input.
\end{itemize}

\Cref{eq:bayes_theorem} provides a principled way to compute the posterior probabilities, and thus the optimal classification rule.

\paragraph{Challenges in Practice}

Despite its theoretical appeal, the Bayes classifier is rarely used directly in practice because:
\begin{itemize}
    \item The class-conditional densities \(P(\mathbf{x} \mid y = c_k)\) are typically unknown.
    \item The prior probabilities \(P(y = c_k)\) may also be unknown or difficult to estimate accurately.
    \item Estimating these distributions nonparametrically or parametrically can be challenging, especially in high-dimensional spaces.
\end{itemize}

Consequently, practical classification methods often rely on approximations or alternative formulations.

\paragraph{Running example: a two-cluster dataset}
To keep the discussion concrete, we reuse a small toy dataset in \Cref{fig:lec1_dataset} consisting of two Gaussian clusters. Under a simple generative assumption (equal covariances and similar priors), \Cref{fig:lec1_bayes} visualizes the Bayes\hyp{}optimal decision boundary: it is linear in this LDA setting, while unequal covariances yield a quadratic boundary. This running example will anchor the geometric intuition (what a decision boundary looks like) before we turn to discriminative models that learn \(\pi(\mathbf{x})\) directly.

    \begin{figure}[!htbp]
        \centering
\begin{tikzpicture}[background rectangle/.style={fill=white}, show background rectangle]
            \begin{axis}[
            width=0.78\linewidth,
            height=0.5\linewidth,
            axis equal image,
            xmin=-3.2,xmax=3.2,
            ymin=-2.4,ymax=2.4,
            xlabel={$x_1$},
            ylabel={$x_2$},
            axis line style={gray!60},
            tick style={gray!60},
            legend style={at={(0.02,0.02)},anchor=south west,fill=white,draw=none,font=\scriptsize},
            legend cell align=left,
            grid=both,
            minor grid style={gray!10},
            major grid style={gray!25},
            axis background/.style={fill=white}
        ]
            % covariance ellipses
            \addplot[draw=cbBlue!80!black, fill=cbBlue!12, line width=0.8pt, domain=0:360, samples=240]
                ({-1.7 + 0.9*cos(x) - 0.1*sin(x)},
                 {0.2 + 0.3*cos(x) + 0.7*sin(x)});
            \addlegendentry{Class $\mathcal C_0$ density contour}
            \addplot[draw=cbOrange!85!black, fill=cbOrange!12, line width=0.8pt, domain=0:360, samples=240]
                ({1.9 + 0.8*cos(x) + 0.05*sin(x)},
                 {0.0 + 0.25*cos(x) + 0.75*sin(x)});
            \addlegendentry{Class $\mathcal C_1$ density contour}

            % samples for each class
            \addplot+[only marks, mark=*, mark size=1.7pt, mark options={draw=white, line width=0.35pt}, color=cbBlue] coordinates {
                (-2.2,0.6) (-2.0,0.1) (-1.8,-0.4) (-1.4,-0.2) (-1.5,0.7)
                (-1.1,0.4) (-1.3,-0.6) (-1.9,1.0) (-2.3,-0.3) (-1.6,0.9)
                (-1.8,0.0) (-1.4,0.3) (-1.9,0.5) (-1.2,-0.3) (-2.1,0.2)
            };
            \addlegendentry{Samples from $\mathcal C_0$}
            \addplot+[only marks, mark=square*, mark size=1.7pt, mark options={draw=white, line width=0.35pt}, color=cbOrange] coordinates {
                (1.2,0.5) (1.6,1.0) (2.0,0.7) (2.4,0.1) (2.2,-0.5)
                (1.5,-0.6) (1.8,0.0) (2.3,0.3) (1.7,-0.8) (2.5,-0.2)
                (1.9,1.2) (2.1,-0.9) (1.6,0.7) (2.0,-0.1) (2.4,0.6)
            };
            \addlegendentry{Samples from $\mathcal C_1$}
        \end{axis}
        \ensuretikzbackgroundlayers
    \end{tikzpicture}
    \caption[Schematic: Synthetic binary dataset.]{Schematic: Synthetic binary dataset built from two anisotropic Gaussian clusters; shaded ellipses hint at the underlying density while the scattered samples are reused throughout the running examples.}
    \label{fig:lec1_dataset}
\end{figure}

\begin{figure}[!htbp]
    \centering
\begin{tikzpicture}[background rectangle/.style={fill=white}, show background rectangle]
        \begin{axis}[
            width=0.9\linewidth,
            height=0.55\linewidth,
            axis equal image,
            xmin=-3,xmax=3,
            ymin=-2.2,ymax=2.2,
            xlabel={$x_1$},
            ylabel={$x_2$},
            legend style={at={(0.02,0.98)},anchor=north west,draw=none,fill=white},
            legend cell align=left,
            axis background/.style={fill=white}
        ]
            % helper paths for region shading
            \path[name path=top] (axis cs:-3,2.2) -- (axis cs:3,2.2);
            \path[name path=bottom] (axis cs:-3,-2.2) -- (axis cs:3,-2.2);
            % vertical decision boundary at x = 0.15 (approx. equal posterior line)
            \path[name path=leftedge] (axis cs:-3,-2.2) -- (axis cs:-3,2.2);
            \path[name path=rightedge] (axis cs:3,-2.2) -- (axis cs:3,2.2);
            \addplot[name path=decision, cbPink, ultra thick] coordinates {(0.15,-2.2) (0.15,2.2)};
            % shade regions
            \addplot[draw=none, fill=cbBlue!10] fill between[of=leftedge and decision];
            \addplot[draw=none, fill=cbOrange!10] fill between[of=decision and rightedge];

            % sample clouds
            \addplot+[only marks, mark=*, mark options={scale=0.8}, color=cbBlue] coordinates {
                (-2.2,0.6) (-2.0,0.1) (-1.8,-0.4) (-1.4,-0.2) (-1.5,0.7)
                (-1.1,0.4) (-1.3,-0.6) (-1.9,1.0) (-2.3,-0.3) (-1.6,0.9)
                (-1.8,0.0) (-1.4,0.3) (-1.9,0.5) (-1.2,-0.3) (-2.1,0.2)
            };
            \addlegendentry{$\mathcal C_0$ samples}
            \addplot+[only marks, mark=square*, mark options={scale=0.8}, color=cbOrange] coordinates {
                (1.2,0.5) (1.6,1.0) (2.0,0.7) (2.4,0.1) (2.2,-0.5)
                (1.5,-0.6) (1.8,0.0) (2.3,0.3) (1.7,-0.8) (2.5,-0.2)
                (1.9,1.2) (2.1,-0.9) (1.6,0.7) (2.0,-0.1) (2.4,0.6)
            };
            \addlegendentry{$\mathcal C_1$ samples}
            \addlegendentry{decision boundary}
        \end{axis}
        \ensuretikzbackgroundlayers
    \end{tikzpicture}
    % Avoid inline math in captions; it wraps poorly in some EPUB renderers.
    \caption{Schematic: Bayes-optimal boundary for two Gaussian classes with equal covariances and similar priors (LDA setting), which yields a linear separator. Unequal covariances produce a quadratic boundary. We place the boundary near the equal-posterior line (vertical, pink); left/right regions correspond to predicted classes R0 and R1.}
    \label{fig:lec1_bayes}
\end{figure}

\paragraph{Naive Bayes Approximation}
One classical workaround is the Naive Bayes classifier, which assumes that the components of \(\mathbf{x}\) are conditionally independent given the class label. Under this assumption,
\[
P(\mathbf{x} \mid y = c_k) = \prod_{j=1}^p P(x_j \mid y = c_k),
\]
making the computation and estimation of the likelihood tractable. It is important to remember that this factorization is justified only under the conditional independence assumption; when the features are strongly correlated, Naive Bayes can suffer because the assumption is violated.

\subsection{Logistic Regression: A Probabilistic Discriminative Model}
\label{sec:logistic_logistic_regression_a_probabilistic_discriminative_model}

One widely used approach to classification, especially for binary problems, is \emph{logistic regression}. Logistic regression models the posterior probability \(P(y=1 \mid \mathbf{x})\) directly as a function of \(\mathbf{x}\), without explicitly modeling the class-conditional densities.

\begin{tcolorbox}[summarybox,title={Logistic regression at a glance}]
\textbf{Objective:} Minimize binary cross\hyp{}entropy (negative log-likelihood) between true labels \(y\in\{0,1\}\) and predicted probabilities \(\hat{p}=\sigma(\boldsymbol{\beta}^\top \tilde{\mathbf{x}})\).\\
\textbf{Key hyperparameters:} Regularization type/strength (L2 or L1, penalty \(\lambda\); many libraries use \(C=1/\lambda\)), feature scaling, optimization settings (step size, iterations).\\
\textbf{Defaults:} Standardize features; use L2 regularization with moderate strength; start with a 0.5 decision threshold and adjust only if class costs are asymmetric.\\
\textbf{Common pitfalls:} Strong collinearity between features, severe class imbalance, and uncalibrated probability outputs if the model is over-regularized or trained on a biased sample.
\end{tcolorbox}

\paragraph{Binary Classification Setup}

Consider the binary classification problem where \(y \in \{0,1\}\). The goal is to model the probability that the output is class 1 given the input \(\mathbf{x}\): \(P(y=1 \mid \mathbf{x}) = \pi(\mathbf{x})\).

\paragraph{Linear Model for the Log-Odds}

Logistic regression assumes that the \emph{log-odds} (also called the \emph{logit}) of the positive class is a linear function of the input features. Introducing the augmented feature vector \(\tilde{\mathbf{x}} = [1, x_1, \ldots, x_p]^\top\) and parameter vector \(\boldsymbol{\beta} = [\beta_0, \beta_1, \ldots, \beta_p]^\top\), we write
\begin{equation}
    \log \frac{\pi(\mathbf{x})}{1 - \pi(\mathbf{x})} = \boldsymbol{\beta}^\top \tilde{\mathbf{x}}.
    \label{eq:logit_linear}
\end{equation}

This implies that the posterior probability \(\pi(\mathbf{x})\) can be written as the \emph{logistic sigmoid} function applied to the linear predictor:
\begin{equation}
    \pi(\mathbf{x}) = \frac{1}{1 + \exp\left(-\boldsymbol{\beta}^\top \tilde{\mathbf{x}}\right)}.
    \label{eq:logistic_probability}
\end{equation}

\begin{tcolorbox}[summarybox,title={Author's note: why ``logistic'' and why ``regression''?}]
The name \emph{logistic} comes from the logistic (sigmoid) link in \Cref{eq:logistic_probability}, which maps a real\hyp{}valued score to a probability in \([0,1]\). The word \emph{regression} reflects what we model linearly: the \emph{log\hyp{}odds} (logit) in \Cref{eq:logit_linear} is a linear function of the features. The model itself outputs a continuous probability; we turn that probability into a class label by thresholding (or comparing class probabilities in the multiclass extension).
\end{tcolorbox}

\subsubsection{Likelihood, loss, and gradient}\label{sec:lec2_logistic_likelihood}
For data \(\{(\mathbf{x}_i,y_i)\}_{i=1}^N\) with \(y_i\in\{0,1\}\), define \(p_i=\pi(\mathbf{x}_i)=\sigma(\boldsymbol{\beta}^\top\tilde{\mathbf{x}}_i)\).
Under a Bernoulli model, the likelihood factorizes as
\begin{equation}
    p(\mathbf{y}\mid X,\boldsymbol{\beta})=\prod_{i=1}^N p_i^{y_i}(1-p_i)^{1-y_i},
    \label{eq:lec2_bernoulli_likelihood}
\end{equation}
so the log\hyp{}likelihood is
\begin{equation}
    \log p(\mathbf{y}\mid X,\boldsymbol{\beta})
    =\sum_{i=1}^N \Big(y_i\log p_i + (1-y_i)\log(1-p_i)\Big).
    \label{eq:lec2_loglik}
\end{equation}
Maximizing \Cref{eq:lec2_loglik} is equivalent to minimizing the negative log\hyp{}likelihood (binary cross\hyp{}entropy). With the design matrix \(X=[\tilde{\mathbf{x}}_1^\top;\ldots;\tilde{\mathbf{x}}_N^\top]\) and vector \(\mathbf{p}=(p_1,\ldots,p_N)^\top\), the gradient of the negative log\hyp{}likelihood is
\begin{equation}
    \nabla_{\boldsymbol{\beta}}\Big(-\log p(\mathbf{y}\mid X,\boldsymbol{\beta})\Big)
    =X^\top(\mathbf{p}-\mathbf{y}).
    \label{eq:lec2_logistic_grad}
\end{equation}
The Hessian has the form \(X^\top W X\) where \(W=\mathrm{diag}(p_i(1-p_i))\succeq 0\), which makes the objective convex and explains why second-order methods work well for moderate feature dimensions.

\begin{figure}[!htbp]
    \centering
\begin{tikzpicture}[background rectangle/.style={fill=white}, show background rectangle]
        \begin{groupplot}[
            group style={group size=3 by 1, horizontal sep=1.15cm},
            width=0.28\linewidth,
            height=0.27\linewidth,
            grid=both,
            minor grid style={gray!10},
            major grid style={gray!20},
            tick label style={font=\scriptsize},
            label style={font=\scriptsize},
            title style={font=\scriptsize, align=center},
            legend style={font=\scriptsize, draw=none, fill=white, at={(0.98,0.98)}, anchor=north east},
            legend cell align=left,
            axis background/.style={fill=white},
        ]
        \nextgroupplot[
            title={Sigmoid},
            xlabel={logit $z$},
            ylabel={$p$},
            xmin=-5.5,xmax=5.5,
            ymin=0, ymax=1.05,
        ]
            \addplot[cbBlue, very thick, domain=-5.5:5.5, samples=200] {1/(1+exp(-x))};
        \nextgroupplot[
            title={BCE loss},
            xlabel={logit $z$},
            ylabel={$\mathcal L$},
            xmin=-5.5,xmax=5.5,
            ymin=0, ymax=6.2,
        ]
            \addplot[cbBlue, very thick, domain=-5.5:5.5, samples=220] {-ln(1/(1+exp(-x)))};
            \addlegendentry{$y=1$}
            \addplot[cbOrange, very thick, dashed, domain=-5.5:5.5, samples=220] {-ln(1-1/(1+exp(-x)))};
            \addlegendentry{$y=0$}
        \nextgroupplot[
            title={L2 shrinkage},
            xlabel={\(\lambda\)},
            ylabel={},
            xmin=0, xmax=5,
            ymin=0, ymax=1.05,
        ]
            \addplot[cbGreen, very thick, domain=0:5, samples=120] {1/(1+0.9*x)};
            \node[
                font=\scriptsize\bfseries,
                cbGreen!60!black,
                fill=white,
                fill opacity=0.85,
                text opacity=1,
                inner sep=1.2pt,
                anchor=south east,
                align=right
            ] at (rel axis cs:0.96,0.08) {shrinks\\weights};
    \end{groupplot}
    \ensuretikzbackgroundlayers
\end{tikzpicture}
            % Avoid inline math in captions; it wraps poorly in some EPUB renderers.
            \caption{Schematic: The sigmoid maps logits to probabilities (left). The binary cross\hyp{}entropy (negative log\hyp{}likelihood) penalizes confident wrong predictions sharply (middle). Regularization typically shrinks parameter norms as the penalty strength increases (right).}
    \label{fig:lec2_sigmoid_bce}
\end{figure}

\paragraph{Optimization geometry (why iterative solvers)}
Unlike linear regression, logistic regression does not have a closed-form solution for \(\boldsymbol{\beta}\), even though the objective is convex. In practice we therefore rely on iterative solvers (gradient methods, quasi\hyp{}Newton methods, or Newton/IRLS in moderate dimensions). \Cref{fig:lec1_gd} is a convex quadratic toy that reminds us what an optimization trajectory looks like when we minimize a smooth objective: step size and conditioning shape how quickly iterates contract toward the minimizer.

\begin{figure}[!htbp]
    \centering
\begin{tikzpicture}[background rectangle/.style={fill=white}, show background rectangle]
        \begin{axis}[
            width=0.9\linewidth,
            height=0.58\linewidth,
            axis equal image,
            xmin=-2.6,xmax=1.4,
            ymin=-2.2,ymax=1.2,
            xlabel={$w_1$},
            ylabel={$w_2$},
            grid=both,
            minor grid style={gray!15},
            major grid style={gray!30},
            legend style={at={(0.97,0.03)},anchor=south east,draw=none,fill=white},
            legend cell align=left,
            axis background/.style={fill=white}
        ]
            % iso-contours (hand-crafted ellipses)
            \addplot[gray!35, domain=0:360, samples=200]
                ({-0.05 + sqrt(1.2)*cos(x) - 0.15*sin(x)},
                 {-0.08 + 0.4*cos(x) + sqrt(0.8)*sin(x)});
            \addplot[gray!45, domain=0:360, samples=200]
                ({-0.05 + sqrt(1.6)*cos(x) - 0.15*sin(x)},
                 {-0.08 + 0.4*cos(x) + sqrt(1.0)*sin(x)});
            \addplot[gray!55, domain=0:360, samples=200]
                ({-0.05 + sqrt(2.2)*cos(x) - 0.15*sin(x)},
                 {-0.08 + 0.4*cos(x) + sqrt(1.3)*sin(x)});
            \addplot[gray!65, domain=0:360, samples=200]
                ({-0.05 + sqrt(2.9)*cos(x) - 0.15*sin(x)},
                 {-0.08 + 0.4*cos(x) + sqrt(1.7)*sin(x)});
            \addplot[gray!75, domain=0:360, samples=200]
                ({-0.05 + sqrt(3.6)*cos(x) - 0.15*sin(x)},
                 {-0.08 + 0.4*cos(x) + sqrt(2.1)*sin(x)});

            % gradient-descent iterates
            \addplot[cbPink, very thick, -{Latex[length=2.5mm]}] coordinates {
                (-2.2,-1.8)
                (-1.35,-0.95)
                (-0.65,-0.35)
                (-0.18,-0.08)
                (-0.02,-0.01)
            };
            \addlegendentry{GD trajectory}
            \addplot+[only marks, mark=*, mark options={fill=white}, color=cbPink] coordinates {
                (-2.2,-1.8)
                (-1.35,-0.95)
                (-0.65,-0.35)
                (-0.18,-0.08)
            };
            \node[cbPink!80!black, font=\scriptsize, anchor=south west] at (axis cs:-2.2,-1.8) {$t=0$};
            \node[cbPink!80!black, font=\scriptsize, anchor=south west] at (axis cs:-1.35,-0.95) {$t=5$};
            \node[cbPink!80!black, font=\scriptsize, anchor=south west] at (axis cs:-0.65,-0.35) {$t=10$};
            \node[cbPink!80!black, font=\scriptsize, anchor=north east] at (axis cs:-0.02,-0.01) {$t\rightarrow \infty$};
            \node[cbGreen!50!black, font=\scriptsize] at (axis cs:0.05,0.05) {minimum};
        \end{axis}
        \ensuretikzbackgroundlayers
    \end{tikzpicture}
        \caption{Schematic: Gradient-descent iterates contracting toward the minimizer of a convex quadratic cost. Ellipses are level sets; arrows show the ``steepest descent along contours'' direction.}
        \label{fig:lec1_gd}
    \end{figure}
\paragraph{Geometry of the logistic surface.} The decision rule is linear in feature space even though the posterior itself is smoothly varying. \Cref{fig:lec2-logistic-boundary} depicts this duality: the white hyperplane slices the space into two half-spaces while the probability ``ramp'' shows how margins translate into calibrated confidences.
\begin{figure}[h]
    \centering
    \ifdefined\HCode
        \includegraphics[width=0.72\linewidth]{assets/lec2_part2/lec2_logistic_boundary.png}
    \else
        \includegraphics[width=0.72\linewidth]{assets/lec2_part2/lec2_logistic_boundary.pdf}
    \fi
    % Avoid inline math in captions; it wraps poorly in some EPUB renderers.
    \caption{Schematic: Illustrative logistic-regression boundary. The dashed line marks the linear decision boundary at probability 0.5; labeled contours show how the posterior varies smoothly with margin, enabling calibrated decisions and adjustable thresholds.}
    \label{fig:lec2-logistic-boundary}
\end{figure}

\FloatBarrier

\subsection{Probabilistic Interpretation: MLE and MAP}
\label{sec:logistic_probabilistic_interpretation_mle_and_map}

The ERM view in \Cref{chap:supervised} treats learning as minimizing an average loss plus (optionally) a regularizer. The probabilistic view arrives at the same objective from a different direction:
\begin{itemize}
    \item \textbf{MLE} maximizes the data likelihood under a chosen observation model (for logistic regression: Bernoulli with \(p_i=\sigma(\boldsymbol{\beta}^\top\tilde{\mathbf{x}}_i)\)).
    \item \textbf{MAP} maximizes the posterior, which multiplies the likelihood by a prior \(p(\boldsymbol{\beta})\). In optimization form, MAP adds a penalty \(-\log p(\boldsymbol{\beta})\), which is exactly regularization.
\end{itemize}
Two common priors explain the two penalties that appear most often in practice: a zero-mean Gaussian prior yields an L2 (ridge) penalty, while a Laplace prior yields an L1 (lasso) penalty. The schematic below illustrates the MLE\(\rightarrow\)MAP idea on a simple mean-estimation problem: with little data, the prior matters; with enough data, MAP approaches MLE.

    \begin{figure}[!htbp]
        \centering
\begin{tikzpicture}[background rectangle/.style={fill=white}, show background rectangle]
            \begin{axis}[
                width=0.65\linewidth,
                height=0.35\linewidth,
            xlabel={Sample size $n$},
            ylabel={Estimate},
            xmin=0,xmax=50,
            ymin=0.2,ymax=1.0,
            legend style={at={(0.02,0.98)},anchor=north west},
            axis background/.style={fill=white}
        ]
            \addplot[cbBlue, thick] table {
                n est
                0 0.5
                5 0.56
                10 0.60
                20 0.63
                30 0.64
                40 0.65
                50 0.65
            };
            \addlegendentry{MAP (prior $\mu_0=0.5$)}
            \addplot[cbOrange, thick, dashed] table {
                n est
                0 0.5
                5 0.58
                10 0.62
                20 0.66
                30 0.69
                40 0.71
                50 0.72
            };
            \addlegendentry{MLE}
                \draw[gray, dotted] (axis cs:0,0.7) -- node[anchor=south east, font=\scriptsize]{true mean $0.7$} (axis cs:50,0.7);
        \end{axis}
        \ensuretikzbackgroundlayers
    \end{tikzpicture}
        % Avoid inline math in captions; it wraps poorly in some EPUB renderers.
        \caption{Schematic: MAP estimates interpolate between the prior mean and the data-driven MLE. As the sample size grows, the MAP curve approaches the true mean.}
        \label{fig:lec1_mle_map}
    \end{figure}

\FloatBarrier

\subsection{Confusion Matrices and Derived Metrics}
\label{sec:logistic_confusion_matrices_and_derived_metrics}

Once we have a probabilistic classifier, we need diagnostics that quantify performance on held-out data. For multi-class prediction, the confusion matrix \(C_{ij}\) records the number of examples with true class \(i\) predicted as \(j\). From \(C\) we compute accuracy, per-class precision/recall, and aggregate metrics. \emph{Macro-averaged} precision/recall first evaluate the metric per class and then average them uniformly, whereas \emph{micro-averaged} precision/recall pool all true/false positives across classes before computing the ratio (equivalent to weighting each example equally). Visual inspection (\Cref{fig:lec1_confusion}) helps diagnose systematic errors across classes.

On highly imbalanced problems accuracy and AUROC can be misleading; prefer class-balanced metrics (macro-F1) and AUPRC. \Cref{fig:lec1-roc-pr} collects ROC and PR curves on one page so you can choose operating points explicitly.

\begin{figure}[!htbp]
    \centering
    \ifdefined\HCode
        \includegraphics[width=0.78\linewidth]{assets/lec2_part2/lec2_roc_pr.png}
    \else
        \includegraphics[width=0.78\linewidth]{assets/lec2_part2/lec2_roc_pr.pdf}
    \fi
    % Avoid inline math in captions; it wraps poorly in some EPUB renderers.
    \caption{Schematic: ROC and PR curves with an explicit operating point. Left: ROC curve with iso-cost lines; right: PR curve with a class-prevalence baseline and iso-F1 contours. Together they visualize threshold trade-offs and calibration quality.}
    \label{fig:lec1-roc-pr}
\end{figure}
\begin{tcolorbox}[summarybox,title={Imbalance and thresholds}]
Use class or sample weights (e.g., inverse prevalence) inside the loss, and pick thresholds via ROC/PR curves or explicit cost ratios rather than defaulting to 0.5. With symmetric priors but asymmetric costs, predict class 1 when the logit exceeds \(\log(c_{10}/c_{01})\); for rare positives, report PR-AUC alongside AUROC.
\end{tcolorbox}

    \begin{figure}[!htbp]
        \centering
\begin{tikzpicture}[background rectangle/.style={fill=white}, show background rectangle]
                \begin{axis}[
                    width=0.6\linewidth,
                    height=0.4\linewidth,
                view={0}{90},
            xmin=-0.5,xmax=2.5,
            ymin=-0.5,ymax=2.5,
            xtick={0,1,2},
            ytick={0,1,2},
            xticklabels={Pred A,Pred B,Pred C},
            yticklabels={True A,True B,True C},
                xlabel=Predicted,
                ylabel=True,
                colorbar,
                colormap/viridis,
                nodes near coords,
                nodes near coords align={center},
                every node near coord/.append style={
                    font=\scriptsize,
                    fill=white,
                    fill opacity=0.75,
                    text opacity=1,
                    inner sep=1.2pt
                },
                axis background/.style={fill=white}
            ]
            \addplot[matrix plot*, mesh/cols=3, point meta=explicit] table [meta=z] {
                x y z
                0 0 42
                1 0 3
                2 0 1
                0 1 4
                1 1 37
                2 1 5
                0 2 0
                1 2 6
                2 2 32
                };
        \end{axis}
        \ensuretikzbackgroundlayers
    \end{tikzpicture}
        \caption{Schematic: Confusion matrix for a three-class classifier; diagonals dominate, indicating strong accuracy with modest confusion between classes B and C.}
        \label{fig:lec1_confusion}
    \end{figure}

\FloatBarrier
\begin{tcolorbox}[summarybox,title={Key takeaways}]
\begin{itemize}
    \item Logistic regression models class probability with a sigmoid link and maximizes a concave log\hyp{}likelihood (equivalently minimizes a convex negative log\hyp{}likelihood); there is no closed-form solution.
    \item ROC and PR curves provide threshold\hyp{}independent evaluation; AUC summarizes performance.
    \item Proper feature scaling and regularization improve convergence and generalization.
\end{itemize}
\end{tcolorbox}

\begin{tcolorbox}[summarybox,title={Probability calibration}]
Discrimination metrics (ROC/PR, AUC) say how well a classifier ranks examples but not how reliable its probabilities are. Calibration methods such as Platt scaling and temperature scaling adjust the logits so that predicted probabilities match empirical frequencies (e.g., 0.8 scores correspond to \(\approx 80\%\) positives), often measured via Expected Calibration Error (ECE) and inspected with reliability diagrams \citep{Platt1999,Guo2017}.
\end{tcolorbox}

\begin{table}[h]
\centering
\caption{Schematic: Handling class imbalance for logistic models (\Cref{chap:logistic} reference table).}
\begin{tabularx}{0.98\linewidth}{@{}>{\raggedright\arraybackslash}p{0.32\linewidth} >{\raggedright\arraybackslash}X@{}}
\toprule
\textbf{Tactic} & \textbf{When/why} \\
\midrule
Stratified splits (and K-fold) & Preserve class ratios in train/validation/test to avoid optimistic validation scores. \\
Class weighting / cost-sensitive loss & Multiply the cross\hyp{}entropy (or hinge loss) by per-class weights so minority errors matter more. Useful when collecting more data is difficult. \\
Resampling (over/undersampling, SMOTE) & Balance the dataset prior to training. Helps tree ensembles and linear models; pair with cross-validation to avoid overfitting. Use simple baselines (logistic/SVM) as a tie-break to detect overfitting. \\
Threshold tuning & Choose a decision threshold based on PR curves or cost ratios rather than default 0.5; report PR-AUC when positives are rare. \\
\bottomrule
\end{tabularx}
\end{table}

\begin{tcolorbox}[summarybox,title={Exercises and lab ideas}]
\begin{itemize}
    \item Implement a minimal example from this chapter and visualize intermediate quantities (plots or diagnostics) to match the pseudocode.
    \item Stress-test a key hyperparameter or design choice discussed here and report the effect on validation performance or stability.
    \item Re-derive one core equation or update rule by hand and check it numerically against your implementation.
\end{itemize}
\end{tcolorbox}

\medskip
\paragraph{Where we head next.} Logistic regression still yields a linear decision boundary. \Cref{chap:perceptron} introduces biologically inspired neuron models and perceptrons as trainable building blocks; stacking nonlinearities breaks the linearity ceiling and sets up multilayer networks and backpropagation.

\paragraph{References.} Full citations for works mentioned in this chapter appear in the book-wide bibliography.

\begin{tcolorbox}[summarybox,title={Part I takeaways}]
\begin{itemize}
  \item Intelligence as engineered self-correction: represent state, choose actions, verify outcomes.
  \item Two recurring toolkits: safe vs.\ heuristic moves (search) and ERM (model--loss--optimize--audit).
  \item Classification as a probabilistic decision problem: Bayes optimality and calibrated scores precede thresholds.
  \item Reading paths are a dependency graph: later chapters reuse diagnostics, notations, and audit habits introduced here.
\end{itemize}
\end{tcolorbox}
\part*{Part II: Neural networks, sequence modeling, and NLP}
\addcontentsline{toc}{part}{Part II: Neural networks, sequence modeling, and NLP}
% Chapter 5
\section{Introduction to Neural Networks}\label{chap:perceptron}

\Cref{chap:logistic} established linear and logistic models as strong baselines, but their decision boundaries stay linear in the original feature space. We now shift from a statistical lens to a biological abstraction: simple neurons whose composition yields nonlinear decision surfaces. This chapter introduces neuron models, perceptrons, activation functions, and the first learning rules (perceptron and Adaline); \Cref{fig:roadmap} marks this as the start of the neural strand.

\begin{tcolorbox}[summarybox, title={Learning Outcomes}]
After this chapter, you should be able to:
\begin{itemize}
  \item Describe the core ingredients of neural networks (architecture, activations, learning).
  \item Explain at a high level why multilayer perceptrons rely on smooth activations and gradient\hyp{}based training (\Cref{chap:mlp,chap:backprop}).
  \item Identify common pitfalls (saturation, poor initialization) and basic remedies.
\end{itemize}
\end{tcolorbox}

\begin{tcolorbox}[summarybox, title={Design motif}]
Biology as engineering abstraction: start with simple units, make the update rule explicit, and use geometry to build intuition before the algebra gets deep.
\end{tcolorbox}

\subsection{Biological Inspiration}
\label{sec:perceptron_biological_inspiration}

Neural networks borrow a simple but powerful idea from biology: complex behavior can emerge from many simple units interacting through many simple connections. The goal is not to model the brain in detail, but to steal an engineering abstraction we can write down and implement: signals flow through a network of units, and learning adjusts connection strengths so the overall system changes its behavior.

\paragraph{Neurons and Neural Activity}

A biological neuron can be viewed as a processing unit that receives multiple inputs, integrates them, and produces an output if certain conditions are met. In the simplified picture we use here, the key parts are:

\begin{itemize}
    \item \textbf{Dendrites:} Receive incoming signals from other neurons.
    \item \textbf{Cell body (soma):} Integrates incoming signals.
    \item \textbf{Axon:} Transmits the output signal to other neurons.
    \item \textbf{Synapses:} Junctions where signals are transmitted between neurons.
\end{itemize}

Incoming signals can excite or inhibit the neuron. When the combined influence crosses a threshold, the neuron ``fires'' and sends an impulse down the axon to connected neurons. Real neurons are richer than this sketch (timing, frequency, and graded effects matter), but the abstraction is enough to motivate an artificial unit that combines inputs, applies a nonlinearity, and emits an output.

\paragraph{Complexities and unknowns (and what we steal anyway).}
Real neural tissue is a biophysical system: spikes, timing, chemistry, adaptation, and many interacting feedback loops. We do not pretend to model that faithfully here. Instead, we borrow an engineering abstraction that has two virtues: (i) it is composable, and (ii) it gives us a learning rule we can write down, debug, and scale.

So what do we \emph{keep} from biology? The idea that many small units, each doing a simple computation, can be wired into a larger system whose behavior is richer than any single unit. And what do we \emph{let go} of? The detailed mechanisms (spike timing, ion channels, and biological realism) that matter for neuroscience but are not necessary to build useful learning machines.

A good analogy is circuit design: you can build powerful systems out of simple, idealized components without simulating electron physics at every wire. In the same spirit, we keep the compositional story (many units, many connections) and insist that the update rule is explicit and learnable. This is a recurring theme in the book: we look to nature for candidate mechanisms behind intelligent behavior, describe them in scientific terms, and then simplify them until they become concrete engineering problems we can test and improve.

\subsection{From Biological to Artificial Neural Networks}
\label{sec:perceptron_from_biological_to_artificial_neural_networks}

Artificial neural networks (ANNs) are computational models inspired by the structure and function of biological neural systems. The goal is to create systems that can process information, learn from data, and perform tasks that require intelligence.

\paragraph{Key Features of Artificial Neural Networks}

To design an ANN that captures essential aspects of biological neural processing, several features must be considered:

\begin{enumerate}
    \item \textbf{Architecture:} The arrangement and connectivity of neurons within the network. This includes the number of layers, the pattern of connections (e.g., feedforward, recurrent), and the flow of information.

    \item \textbf{Signal Propagation:} How input signals are transmitted through the network, transformed by neurons, and produce outputs.

    \item \textbf{Learning Mechanism:} The method by which the network adjusts its parameters (e.g., weights) based on data to improve performance. This involves capturing and retaining knowledge from experience.

    \item \textbf{Activation Dynamics:} The rules governing neuron activation, including how neurons decide to fire based on inputs, the degree of activation, and inhibition mechanisms.
\end{enumerate}

\paragraph{Historical Context}

The concept of artificial neural networks dates back to the early 1940s, with pioneering work that attempted to mathematically model neuron behavior. Over the past eight decades, ANNs have evolved significantly, leading to a variety of architectures and learning algorithms. This evolution reflects ongoing efforts to better approximate biological intelligence and to address practical challenges in computation and learning.

\paragraph{Where this fits in Part II.}\label{sec:perceptron_outline_of_neural_network_study}
This chapter starts the neural strand by introducing the perceptron neuron (defined below) as the basic
trainable unit: a weighted sum followed by an activation/threshold, together with the first learning rules
(mistake-driven and gradient-driven). The point is not to model biology literally; the point is to steal an
engineering abstraction we can write down, debug, and scale.

\begin{tcolorbox}[summarybox, title={Part II local roadmap: what stays the same, what changes}]
\textbf{What stays the same.} We still follow the ERM loop: choose a representation, define a loss, optimize, then verify with honest evaluation and diagnostics. As we proceed, you will find that this theme generalizes across many of the algorithms we discuss.

\medskip
\textbf{What changes (the design space).} Instead of choosing one linear model, we build systems by composing
many simple units. Used on its own, a perceptron-like unit still induces a linear decision boundary in the
original feature space (as in logistic regression). Its importance is that it is \emph{composable}: by changing
how units are connected, what nonlinearities they apply, and what signals drive learning, we can move from a
single linear boundary to richer nonlinear decision surfaces when the task demands it. Many of these design
choices are loosely inspired by how we think biological systems process information (as sketched in the
section that follows), but the target remains an engineering model, not a faithful brain simulation.

\medskip
\textbf{How the rest of Part II builds.}
\begin{itemize}
  \item \textbf{Units \(\rightarrow\) multilayer systems:} \Cref{chap:mlp} chains units into multilayer networks;
  \Cref{chap:backprop} shows how to compute all gradients efficiently (the training engine).
  \item \textbf{Architectures as inductive bias:} later chapters introduce structured connectivity (e.g., convolution,
  recurrence, attention) that changes what patterns are easy to represent and what information can flow.
  \item \textbf{Audit hooks:} keep learning curves, calibration, and slice checks alongside accuracy. A good loss is
  not automatically a good decision.
\end{itemize}
\end{tcolorbox}

% Chapter 5 (continued)

\subsection{Neural Network Architectures}
\label{sec:perceptron_neural_network_architectures}

Neural networks can be broadly categorized based on the flow of information through their structure. Understanding these architectures is crucial for designing and analyzing neural models that mimic biological neural systems.

\paragraph{Feedforward Neural Networks}

Feedforward neural networks (FNNs) are characterized by a unidirectional flow of information from input to output layers without any cycles or loops. The information propagates forward through successive layers of neurons, each layer transforming the input received from the previous layer.

Conceptually, this can be thought of as a cascade of neuron activations where each neuron receives input signals, processes them, and passes the output to the next layer. This architecture aligns with the idea that sensory information in biological systems is processed in a hierarchical manner.

Mathematically, if we denote the input vector as \(\mathbf{x}\), the output of layer \(l\) as \(\mathbf{a}^{(l)}\), and the weight matrix connecting layer \(l-1\) to layer \(l\) as \(\mathbf{W}^{(l)}\), the feedforward operation is given by:
\begin{align}
    \mathbf{z}^{(l)} &= \mathbf{a}^{(l-1)} \mathbf{W}^{(l)} + \mathbf{b}^{(l)} \\
    \mathbf{a}^{(l)} &= f(\mathbf{z}^{(l)}).
    \label{eq:feedforward}
\end{align}
where \(\mathbf{b}^{(l)}\) is the bias vector and \(f(\cdot)\) is the activation function applied element-wise.
\paragraph{Shapes and convention.} We use the row-major (deep-learning) convention. A single example is a row vector \(\mathbf{a}^{(l)}\in\mathbb{R}^{1\times n_l}\), a mini\hyp{}batch stacks examples by rows \(\mathbf{A}^{(l)}\in\mathbb{R}^{B\times n_l}\), and weights map features by right multiplication \(\mathbf{W}^{(l)}\in\mathbb{R}^{n_{l-1}\times n_l}\). Biases \(\mathbf{b}^{(l)}\in\mathbb{R}^{n_l}\) broadcast across the batch: \(\mathbf{Z}^{(l)}=\mathbf{A}^{(l-1)}\mathbf{W}^{(l)}+\mathbf{1}(\mathbf{b}^{(l)})^\top\). For a concrete check, if \(B=4\), \(n_{l-1}=3\), and \(n_l=5\), then \(\mathbf{A}^{(l-1)}\in\mathbb{R}^{4\times 3}\), \(\mathbf{W}^{(l)}\in\mathbb{R}^{3\times 5}\), and \(\mathbf{Z}^{(l)}\in\mathbb{R}^{4\times 5}\). We reserve \(\phi(\cdot)\) for kernel feature maps (\Cref{app:kernels}).

\paragraph{Recurrent Neural Networks}

In contrast, recurrent neural networks (RNNs) allow information to flow in cycles, enabling feedback connections. This means that the network's state at a given time depends not only on the current input but also on previous states, effectively creating a form of memory.

The recurrent architecture is more flexible and biologically plausible since neurons can influence each other bidirectionally and inputs/outputs can be introduced or sampled at various points in the network. This allows modeling of temporal sequences and dynamic behaviors. A simple recurrent update is
\[
\mathbf{h}_t = f(\mathbf{x}_t \mathbf{W}_{xh} + \mathbf{h}_{t-1} \mathbf{W}_{hh} + \mathbf{b}_h), \qquad
\mathbf{y}_t = \mathbf{h}_t \mathbf{W}_{hy} + \mathbf{b}_y,
\]
with the full treatment deferred to \Cref{chap:rnn}.

\subsection{Activation Functions}
\label{sec:perceptron_activation_functions}

Activation functions determine how the input to a neuron is transformed into an output signal, effectively controlling the neuron's excitation level. They play a critical role in enabling neural networks to model complex, nonlinear relationships.

\paragraph{Biological Motivation}

In biological neurons, excitation occurs when the combined chemical signals exceed a certain threshold, triggering an action potential (a "fire"). Similarly, artificial neurons use activation functions to decide whether to activate (fire) based on their input.

\paragraph{Common Activation Functions}
Activation functions map the aggregated input \(z\) to a neuron's output \(y=f(z)\); they inject nonlinearity and control gradient flow during learning. Different choices trade off biological plausibility, numerical stability, and ease of optimization.

\begin{itemize}
    \item \textbf{Step Function (Heaviside)}:
    \[
        f(x) = \begin{cases}
            1 & x > 0 \\
            0 & x \leq 0
        \end{cases}
    \]
    Models a binary firing behavior but is not differentiable, limiting its use in gradient-based learning.

    \item \textbf{Sign Function}:
    \[
        f(x) = \begin{cases}
            1 & x > 0 \\
            0 & x = 0 \\
            -1 & x < 0
        \end{cases}
    \]
        Allows for inhibitory (negative) outputs, mimicking excitatory and inhibitory neuron behavior. We adopt the convention \(f(0)=0\); some authors either leave \(\mathrm{sign}(0)\) undefined or set it to \(+1\), so it is helpful to state the choice explicitly.

    \item \textbf{Linear Function}:
    \[
        f(x) = x
    \]
    Useful in some contexts but cannot model nonlinearities alone.

    \item \textbf{Sigmoid Function}:
    \[
        f(x) = \frac{1}{1 + e^{-x}}
    \]
        Smoothly maps inputs to \((0,1)\), differentiable, and historically popular. Because sigmoid outputs saturate near \(0\) and \(1\), gradients can become small in deep stacks; later chapters discuss practical workarounds and alternatives.

    \item \textbf{Hyperbolic Tangent (tanh)}:
    \[
        f(x) = \tanh(x) = \frac{e^{x} - e^{-x}}{e^{x} + e^{-x}}
    \]
    Maps inputs to \((-1,1)\), zero-centered, often preferred over sigmoid.

    \item \textbf{ReLU (Rectified Linear Unit)}:
    \[
        f(x) = \max(0, x)
    \]
        Computationally efficient and helps mitigate vanishing gradient problems.

\end{itemize}

\begin{tcolorbox}[summarybox, title={Notation note: activations and thresholds}]
In this chapter we use \(f(\cdot)\) as a generic placeholder for an activation function; when we need the logistic sigmoid specifically we write \(\sigma(\cdot)\). Elsewhere in the book, \(\phi(\cdot)\) denotes a kernel feature map. For thresholded functions we adopt \(H(0)=1\) (Heaviside) and \(\mathrm{sgn}(0)=0\) by convention. These choices do not affect continuous models but keep examples consistent.
\end{tcolorbox}

\subsection{Learning Paradigms in Neural Networks}
\label{sec:perceptron_learning_paradigms_in_neural_networks}

When building a neural network, whether feedforward or recurrent, the fundamental process involves producing an output, comparing it with a target, and then adjusting the network parameters based on the error. This iterative process is the essence of \emph{learning}. We distinguish several learning paradigms depending on the availability and nature of the target information:

\paragraph{Supervised Learning}
In supervised learning, the network is provided with input-output pairs. The network produces an output for a given input, compares it to the known target output, computes an error, and updates its parameters to reduce this error. This requires labeled data and is the most common learning paradigm in practice.

\paragraph{Unsupervised Learning}
In unsupervised learning, there is no explicit target output. The network must discover patterns or structure in the input data by itself. This often involves competition among different patterns, where some patterns become dominant and reinforce themselves, while others are suppressed. The network evolves until it reaches an equilibrium state where the learned representations stabilize.
Beyond competitive learning, unsupervised methods encompass clustering, density estimation, dimensionality reduction, autoencoders, and modern self-supervised objectives---any setting where structure is inferred directly from the inputs.

\paragraph{Reinforcement Learning}
Reinforcement learning (RL) models learning from interaction with feedback. An agent with policy \(\pi(a\mid s)\) selects actions, collects rewards, and updates \(\pi\) to improve expected return. Full RL treatments appear later; here the point is that not all learning is supervised, and neural-network controllers are common in modern RL.

\subsection{Fundamentals of Artificial Neural Networks}
\label{sec:perceptron_fundamentals_of_artificial_neural_networks}

The foundational model of artificial neural networks dates back to \citet{McCullochPitts1943}, who proposed a simple neuron model capturing essential features of biological neurons.

\paragraph{McCulloch-Pitts Neuron Model}

Consider a single neuron with multiple binary inputs \( x_i \in \{0,1\} \), \( i=1, \ldots, n \). Each input is associated with a weight \( w_i \), which can be positive (excitatory) or negative (inhibitory). The neuron computes a weighted sum of its inputs:

\begin{align}
    S = \sum_{i=1}^n w_i x_i.
    \label{eq:weighted_sum}
\end{align}

The output \( y \) of the neuron is determined by comparing \( S \) to a threshold \(\theta\):

\begin{align}
    y =
    \begin{cases}
        1, & \text{if } S \geq \theta, \\
        0, & \text{otherwise}.
    \end{cases}
    \label{eq:threshold_output}
\end{align}

Key characteristics of this model include:

\begin{itemize}
    \item \textbf{Binary inputs:} Inputs are either active (1) or inactive (0).
    \item \textbf{Excitatory and inhibitory weights:} Weights \( w_i > 0 \) excite the neuron, while \( w_i < 0 \) inhibit it.
    \item \textbf{Thresholding:} The neuron fires (outputs 1) only if the weighted sum exceeds the threshold.
\end{itemize}

\paragraph{Interpretation}

This simple neuron can be viewed as a linear classifier that partitions the input space into two regions separated by the hyperplane defined by the equation

\begin{align}
    \sum_{i=1}^n w_i x_i = \theta.
    \label{eq:auto:lecture_3_part_i:1}
\end{align}

The learning task reduces to finding appropriate weights \( w_i \) and threshold \(\theta\) that correctly classify inputs.

\paragraph{Excitation and Inhibition}

The neuron can be excited or inhibited depending on the sign and magnitude of the weights. For example:

\begin{itemize}
    \item If all \( w_i > 0 \), the neuron is purely excitatory.
    \item If some \( w_i < 0 \), those inputs inhibit the neuron.
    \item The balance of excitation and inhibition determines the neuron's response.
\end{itemize}
In biological circuits inhibition is carried by specialized interneurons, whereas here a negative weight is an abstract shortcut; the sign simply indicates whether an input pushes the weighted sum above or below the threshold. Artificial neurons are function approximators; similarity to biology is inspirational, not mechanistic.

\paragraph{Learning Objective}

In this model, learning can be interpreted as adjusting the weights \( w_i \) and threshold \(\theta\) to achieve desired input-output mappings. The challenge is to find these parameters such that the neuron outputs 1 for inputs belonging to a certain class and 0 otherwise.

\subsection{Mathematical Formulation of the Neuron Output}
\label{sec:perceptron_mathematical_formulation_of_the_neuron_output}

To summarize, the neuron output is given by

\begin{align}
    y = f\left( \sum_{i=1}^n w_i x_i - \theta \right),
    \label{eq:neuron_output}
\end{align}

where \( f(\cdot) \) is the activation function, which in the McCulloch-Pitts model is a Heaviside step function:

\begin{align}
    f(z) =
    \begin{cases}
        1, & z \geq 0, \\
        0, & z < 0.
    \end{cases}
    \label{eq:auto:lecture_3_part_i:2}
\end{align}
We explicitly set \(f(0)=1\); other texts sometimes use \(f(0)=\tfrac{1}{2}\), so documenting the convention avoids confusion when comparing derivations. It is also common to absorb the threshold into an augmented weight vector by defining \(x_0 = 1\) and \(w_0 = -\theta\), yielding a pure inner product \(\mathbf{w}^\top \mathbf{x}\) that we will reuse in the perceptron section.

This model laid the groundwork for later developments in neural networks, including the introduction of differentiable nonlinearities that enable gradient-based learning.

\subsection{McCulloch-Pitts neuron: examples and limits}
\label{sec:perceptron_mcculloch_pitts_neuron_examples_and_limits}

Recall the MP neuron definition in \eqref{eq:weighted_sum}--\eqref{eq:threshold_output} (equivalently \eqref{eq:neuron_output}); here we focus on logic-gate examples and the limitations that motivate learnable variants.

\paragraph{Example: AND and OR gates}

- For an AND gate with inputs \(x_1, x_2\), set weights \(w_1 = w_2 = 1\) and threshold \(\theta = 2\). The output is 1 only if both inputs are 1, matching the AND truth table.

- For an OR gate, keep weights \(w_1 = w_2 = 1\) but set \(\theta = 1\). The output is 1 if at least one input is 1, matching the OR truth table.

This demonstrates how the MP neuron can implement simple logical functions by appropriate choice of weights and threshold.

\paragraph{Limitations of the MP model}

Despite its conceptual simplicity, the MP neuron has significant limitations:

\begin{itemize}
    \item \textbf{No learning mechanism:} The weights and threshold must be manually assigned or guessed. There is no algorithmic way to adjust parameters based on data.
    \item \textbf{Limited computational power:} The MP neuron can only represent linearly separable functions. Complex patterns requiring nonlinear decision boundaries cannot be modeled.
    \item \textbf{Binary inputs and outputs:} The model is restricted to binary signals, limiting its applicability to real-valued data.
\end{itemize}

These limitations motivated the development of more sophisticated neuron models and learning algorithms.

\subsection{From MP Neuron to Perceptron and Beyond}
\label{sec:perceptron}

The MP neuron laid the groundwork for subsequent models that introduced learning capabilities and continuous-valued inputs and outputs.

\paragraph{Perceptron model}

The perceptron, introduced by Rosenblatt in 1958, extends the MP neuron by incorporating a learning algorithm to adjust weights based on training data. The perceptron output is
\begin{align}
    y = \begin{cases}
    1 & \text{if } \mathbf{w}^\top \mathbf{x} + b \geq 0, \\
    0 & \text{otherwise},
    \end{cases}
    \label{eq:perceptron}
\end{align}
where \(\mathbf{x}\) is the input vector, \(\mathbf{w}\) the weight vector, and \(b\) the bias (threshold).
\begin{tcolorbox}[summarybox, title={Targets and encodings}]
We switch between labels in \texttt{\{0,1\}} (probability view) and labels in \texttt{\{-1,+1\}} (margin view).
Convert with \texttt{y\_pm = 2*y01 - 1} and \texttt{y01 = (y\_pm + 1)/2}.
Perceptron updates below use the \texttt{\{-1,+1\}} encoding.
\end{tcolorbox}

The perceptron learning rule iteratively updates weights to reduce classification errors, enabling the model to learn from data rather than relying on manual parameter selection. The signed-margin derivation below yields the update used in practice; the induced separating hyperplane and signed distance are illustrated in \Cref{fig:lec3-perceptron-geometry}.

\paragraph{Perceptron update from the signed margin.} Let \(d_i = y_i(\mathbf{w}^\top \mathbf{x}_i + b)\) be the signed margin. If \(d_i \ge 0\) the example is correctly classified; if \(d_i < 0\) the example is misclassified. A common perceptron criterion is
\[
J(\mathbf{w}, b) = - \sum_{i \in \mathcal{M}} d_i = - \sum_{i \in \mathcal{M}} y_i (\mathbf{w}^\top \mathbf{x}_i + b),
\]
where \(\mathcal{M}\) is the set of misclassified examples. Taking a gradient step on \(J\) yields
\[
\mathbf{w} \leftarrow \mathbf{w} + \eta y_i \mathbf{x}_i,\qquad
b \leftarrow b + \eta y_i,
\]
which is exactly the perceptron update. In augmented form, set \(x_0=1\) and \(w_0=b\), and the update becomes \(\mathbf{w} \leftarrow \mathbf{w} + \eta y_i \mathbf{x}_i\). Geometrically, each mistake nudges the hyperplane so the signed distance \(d_i\) increases.

\begin{tcolorbox}[summarybox, title={Worked example: learning an OR gate by mistake-driven updates}]
We use labels \(y\in\{-1,+1\}\) (so OR has targets \((-1,+1,+1,+1)\) for \((0,0),(0,1),(1,0),(1,1)\)),
learning rate \(\eta=1\), and initialize \(\mathbf{w}=(0,0)\), \(b=0\). Cycle through the four inputs in that
fixed order; whenever \(y(\mathbf{w}^\top\mathbf{x}+b)\le 0\), apply \(\mathbf{w}\leftarrow \mathbf{w}+y\mathbf{x}\),
\(b\leftarrow b+y\).

\medskip
\noindent The sequence of updates (after each mistake) is:

% QC-BEGIN: perceptron_or_trace
% step epoch idx x1 x2 y w1 w2 b
% 1 1 1 0 0 -1 0 0 -1
% 2 1 2 0 1  1 0 1  0
% 3 1 3 1 0  1 1 1  1
% 4 2 1 0 0 -1 1 1  0
% 5 3 1 0 0 -1 1 1 -1
% 6 3 2 0 1  1 1 2  0
% 7 4 1 0 0 -1 1 2 -1
% 8 4 3 1 0  1 2 2  0
% 9 5 1 0 0 -1 2 2 -1
% QC-END: perceptron_or_trace
\[
\begin{array}{r|c|c|c}
\text{step} & (x_1,x_2),\,y & (w_1,w_2) & b \\
\hline
1 & (0,0),-1 & (0,0) & -1\\
2 & (0,1),+1 & (0,1) & 0\\
3 & (1,0),+1 & (1,1) & 1\\
4 & (0,0),-1 & (1,1) & 0\\
5 & (0,0),-1 & (1,1) & -1\\
6 & (0,1),+1 & (1,2) & 0\\
7 & (0,0),-1 & (1,2) & -1\\
8 & (1,0),+1 & (2,2) & 0\\
9 & (0,0),-1 & (2,2) & -1\\
\end{array}
\]
After these 9 updates, predicting \(+1\) when \(\mathbf{w}^\top\mathbf{x}+b\ge 0\) (and \(-1\) otherwise) with
\(\mathbf{w}=(2,2)\), \(b=-1\) labels all four OR inputs correctly.
\end{tcolorbox}

\paragraph{Perceptron convergence theorem.} If a training set is linearly separable with margin \(\gamma > 0\), the perceptron learning algorithm is guaranteed to find a separating hyperplane after at most \((R/\gamma)^2\) updates, where \(R\) bounds the input norms. Rescaling features changes \(R\) and \(\gamma\), so standardizing inputs tightens the bound. When the data are not separable the algorithm can cycle forever; \Cref{sec:mlp-limitations} (and \Cref{chap:mlp}) therefore emphasize feature scaling, bias terms, and the move to differentiable multilayer models to handle nonlinear problems.

\begin{tcolorbox}[summarybox, title={Perceptron convergence theorem (proof sketch)}]
Assume there exists a unit vector \(\mathbf{w}^\star\) such that \(y_i\,\mathbf{w}^\star\!\cdot\!\mathbf{x}_i \ge \gamma\) for all \(i\) and that \(\|\mathbf{x}_i\|\le R\).
Let w(t) denote the perceptron weights after t mistakes. Each mistake updates:
\[
\mathbf{w}^{(t+1)} = \mathbf{w}^{(t)} + y_i \mathbf{x}_i.
\]
\begin{enumerate}
    \item \textbf{Progress along the separator.} The inner product with w* grows by at least gamma each mistake:
    \[
    \begin{aligned}
    \mathbf{w}^{(t+1)} \cdot \mathbf{w}^\star
    &= \mathbf{w}^{(t)} \cdot \mathbf{w}^\star + y_i\,\mathbf{x}_i \cdot \mathbf{w}^\star \\
    &\ge \mathbf{w}^{(t)} \cdot \mathbf{w}^\star + \gamma.
    \end{aligned}
    \]
    Thus after T mistakes, the dot product with w* is at least T*gamma.
    \item \textbf{Bounding the norm.} The squared norm grows slowly:
    \[
    \begin{aligned}
    \|\mathbf{w}^{(t+1)}\|^2
    &= \|\mathbf{w}^{(t)}\|^2 + \| \mathbf{x}_i\|^2 + 2 y_i \mathbf{x}_i \cdot \mathbf{w}^{(t)} \\
    &\le \|\mathbf{w}^{(t)}\|^2 + R^2,
    \end{aligned}
    \]
    because the mistake condition implies \(y_i \mathbf{x}_i \cdot \mathbf{w}^{(t)} \le 0\). Inductively, \(\|\mathbf{w}^{(T)}\|^2 \le TR^2\).
    \item \textbf{Combine via Cauchy--Schwarz.}
    \[
    \begin{aligned}
    T\gamma
    &\le \mathbf{w}^{(T)} \cdot \mathbf{w}^\star \\
    &\le \|\mathbf{w}^{(T)}\| \|\mathbf{w}^\star\|
    \le \sqrt{T}\, R,
    \end{aligned}
    \]
    which implies \(T \le (R/\gamma)^2\).
\end{enumerate}
Therefore the perceptron halts after finitely many mistakes on separable data. If the data are not separable, some \(\gamma > 0\) cannot be found, and the above argument no longer applies; hence the need for multilayer networks.
\end{tcolorbox}

\begin{tcolorbox}[summarybox, title={Common perceptron pitfalls}]
\begin{itemize}
    \item \textbf{Feature scaling:} Large-magnitude features dominate updates; standardize inputs first.
    \item \textbf{Random seed sensitivity:} Different initial weights can lead to drastically different separating hyperplanes.
    \item \textbf{Non-separable data:} Without slack variables or kernels the perceptron will not converge; diagnose this before training indefinitely. XOR is the canonical counterexample.
\end{itemize}
\end{tcolorbox}

\begin{tcolorbox}[summarybox, title={Geometry intuition: beyond XOR (two triangles)}]
XOR is the canonical non-separable toy, but the limitation is geometric, not specific to a truth table. A second picture to keep in mind is \emph{two interleaved triangles}: one class occupies the vertices of a large triangle, while the other occupies the vertices of a smaller triangle rotated inside it. No single line separates the two sets; any separating boundary must ``bend'' or be assembled from multiple linear pieces.
\medskip

\noindent\textbf{Why this matters.} A perceptron draws exactly one hyperplane, so it cannot represent such boundaries. \Cref{chap:mlp} restores expressivity by composing units so the overall decision boundary can be piecewise-linear (or curved) while still being trainable by gradients.
\end{tcolorbox}

\begin{figure}[t]
    \centering
    \begin{tikzpicture}
        \begin{axis}[
            axis lines=middle,
            axis line style={gray!70},
            xmin=-3.2, xmax=3.2,
            ymin=-3.0, ymax=3.0,
            width=0.64\linewidth,
            height=0.5\linewidth,
            xtick={-2,0,2},
            ytick={-2,0,2},
            clip=false,
            ticklabel style={font=\scriptsize, gray!70},
            grid=both,
            minor grid style={gray!10},
            major grid style={gray!20},
            legend style={font=\scriptsize, draw=none, fill=none, at={(0.02,0.98)}, anchor=north west}
        ]
            \addplot[thick, gray!80] coordinates {(-3.2,2.1) (3.2,-2.1)}
                node[pos=0.82, rotate=-45, anchor=south west, font=\scriptsize, text=gray!70]
                {$\mathbf{w}^\top\mathbf{x}+b=0$};
            \addplot[dashed, gray!60] coordinates {(-3.2,2.9) (3.2,-1.3)};
            \addplot[dashed, gray!60] coordinates {(-3.2,1.3) (3.2,-2.9)};
            \node[anchor=south east, font=\scriptsize, gray!70] at (axis cs:3.1,-2.15) {$C{=}1$};

            \addplot[only marks, mark=*, mark size=1.8pt, mark options={draw=white, line width=0.4pt}, color=cbBlue]
                coordinates {(-1.2,1.5) (-0.8,2.1) (-2,1.3)};
            \addlegendentry{Class $+1$}
            \addplot[only marks, mark=triangle*, mark size=2.1pt, mark options={draw=white, line width=0.4pt}, color=cbOrange]
                coordinates {(1.4,-1.3) (2.2,-0.6) (0.9,-2.1)};
            \addlegendentry{Class $-1$}

            \addplot[thick, cbPink, -{Stealth[length=2.2mm]}] coordinates {(0.6,-0.3) (0.15,0.2)};
            \node[cbBlue!70!black, font=\scriptsize] at (0.85,0.3) {$d(\mathbf{x},\mathcal{H})$};
        \end{axis}
    \end{tikzpicture}
    % Avoid inline math in captions; it wraps poorly in some EPUB renderers.
    \caption{Perceptron geometry. Points on opposite sides of the separating hyperplane receive different labels, and signed distance to the boundary controls both prediction and update magnitude. Compare with \Cref{fig:lec2-logistic-boundary} in \Cref{chap:logistic}: both are linear separators, but logistic smooths the boundary into calibrated probabilities.}
    \label{fig:lec3-perceptron-geometry}
\end{figure}


\paragraph{Adaline model}

The Adaptive Linear Neuron (Adaline), developed in the 1960s, further improves on the perceptron by using a linear activation function and minimizing a continuous error function (mean squared error). This allows the use of gradient descent for training, leading to more stable convergence.

\paragraph{Adaline weight update (derivation)}
Adaline uses a linear output \(y = \mathbf{w}^\top \mathbf{x} + b\) and the squared error
\[
E = \tfrac{1}{2}(t-y)^2.
\]
The gradient is \(\partial E / \partial \mathbf{w} = -(t-y)\mathbf{x}\) and \(\partial E/\partial b = -(t-y)\), so the update is
\[
\mathbf{w} \leftarrow \mathbf{w} + \eta (t-y)\mathbf{x}, \qquad
b \leftarrow b + \eta (t-y).
\]
Unlike the perceptron, Adaline updates on every example and scales the step by the residual \(t-y\); this is the first explicit appearance of gradient-based weight optimization in the neural narrative.

The perceptron and Adaline models are limited to linearly separable problems. To overcome this, multilayer perceptrons (MLPs) with hidden layers were introduced; \Cref{chap:mlp} and \Cref{chap:backprop} develop the mechanics in full.

\begin{tcolorbox}[summarybox, title={Perceptron vs.\ logistic regression}]
Linear score \(s(\mathbf{x})=\mathbf{w}^\top\mathbf{x}+b\). The perceptron predicts \(\mathbb{1}[s\ge 0]\) and updates \(\mathbf{w}\leftarrow \mathbf{w}+\eta y_i\mathbf{x}_i\) (and \(b\leftarrow b+\eta y_i\)) only on mistakes, with \(y_i\in\{-1,+1\}\). Logistic regression predicts \(\sigma(s)\), minimizes cross\hyp{}entropy \(-\sum_i y_i\log\sigma(s_i)+(1-y_i)\log(1-\sigma(s_i))\), and steps by \(\sum_i(\sigma(s_i)-y_i)\mathbf{x}_i\). Prefer logistic when calibrated probabilities and smooth optimization are needed (\Cref{chap:logistic}).
\end{tcolorbox}

\begin{tcolorbox}[summarybox, title={Author's note: what a single perceptron cannot express}]
A single perceptron makes one global, all-or-none decision: one hyperplane, one threshold, one set of weights shared across every example. That simplicity is the point of the model, and it is also the source of the limitation.

Many real problems need \emph{communities} of units that can specialize. Different hidden units respond to different regions, features, or patterns, and the model combines those responses into a richer decision surface. Multi-layer networks do not only add parameters; they add internal structure that lets different parts of the model ``care about'' different parts of the data.
\end{tcolorbox}

\begin{tcolorbox}[summarybox, title={Key takeaways}]
\textbf{Minimum viable mastery}
\begin{itemize}
    \item The perceptron and Adaline turn threshold units into trainable classifiers by updating weights from data.
    \item Geometry (hyperplanes and signed distance) explains predictions and update magnitude.
    \item Logistic regression keeps the same linear score but learns calibrated probabilities via a smooth loss (\Cref{chap:logistic}).
    \item Nonlinear tasks (e.g., XOR) require multilayer networks and backpropagation (\Crefrange{chap:mlp}{chap:backprop}).
\end{itemize}
\medskip
\textbf{Common pitfalls}
\begin{itemize}
    \item Expecting a single hyperplane to solve nonconvex structure: without hidden units you cannot express XOR-like logic.
    \item Mixing label codings (\(\{-1,+1\}\) vs.\ \(\{0,1\}\)) without adjusting the loss/update formulas.
    \item Treating linear scores as probabilities: calibrated probabilities require a probabilistic model/loss (e.g., logistic).
\end{itemize}
\end{tcolorbox}

\begin{tcolorbox}[summarybox, title={Exercises and lab ideas}]
\begin{itemize}
    \item Implement a minimal example from this chapter and visualize intermediate quantities (plots or diagnostics) to match the pseudocode.
    \item Stress-test a key hyperparameter or design choice discussed here and report the effect on validation performance or stability.
    \item Re-derive one core equation or update rule by hand and check it numerically against your implementation.
\end{itemize}
\medskip
\noindent\textbf{If you are skipping ahead.} Remember the two bottlenecks that force multilayer networks: expressivity (nonlinear boundaries) and trainability (smooth gradients). \Cref{chap:mlp} and \Cref{chap:backprop} assume these motivations.
\end{tcolorbox}

\medskip
\paragraph{Where we head next.} Perceptrons are intentionally simple: hard thresholds and uniform updates. Their strengths (linear separation) and limits (for example XOR) motivate multilayer models. \Cref{chap:mlp} continues this thread by chaining units, defining a loss, and asking the key training question: \emph{how should weights change to improve performance?} That question leads directly to the chain rule and smooth activations. \Cref{chap:backprop} then scales the same cache-and-chain-rule accounting to arbitrary depth and efficient implementation.

\nocite{McCullochPitts1943, Rosenblatt1958, WidrowHoff1960, Rumelhart1986}

% Chapter 6
\section{Multi-Layer Perceptrons: Challenges and Foundations}\label{chap:mlp}

\Cref{chap:perceptron} introduced the perceptron and Adaline: single units that learn by updating weights from data, with Adaline giving our first explicit glimpse of gradient descent on a smooth performance function.

In this chapter we keep the story linear and concrete. We build the smallest possible network (two neurons in series), define a performance function, and ask the core question: \emph{how should the weights change to improve performance?} Answering that question forces us to use derivatives (the chain rule) and leads naturally to gradient descent. Along the way we encounter a practical obstacle: hard thresholds are not differentiable, so they do not support gradient-based learning. We replace them with smooth activations and immediately gain a clean update story. This is the conceptual bridge to full backpropagation in \Cref{chap:backprop}. \Cref{fig:roadmap} shows this as the hinge between single-unit models and multilayer training.

\begin{tcolorbox}[summarybox, title={Learning Outcomes}]
After this chapter, you should be able to:
\begin{itemize}
    \item Build the smallest multi\hyp{}layer network and write its forward equations.
    \item Define a simple performance (loss) function for the network output.
    \item Explain why gradient descent is the right tool for weight updates.
    \item Show why hard thresholds block gradients and motivate smooth activations.
    \item Derive the weight updates for the two\hyp{}neuron network using the chain rule.
\end{itemize}
\end{tcolorbox}

\begin{tcolorbox}[summarybox, title={Design motif}]
Build the smallest trainable network you can, keep every intermediate quantity visible, and let the chain rule explain how learning signals flow.
\end{tcolorbox}

\subsection{From a single unit to the smallest network}
\label{sec:mlp-limitations}

\paragraph{A short map: building a trainable network.}
The chapter follows one tight loop:
\begin{itemize}
    \item \textbf{Build:} write the forward computation (a two-neuron chain).
    \item \textbf{Judge:} define a performance function $P$ (we use squared error).
    \item \textbf{Move:} use derivatives to update parameters via gradient descent.
    \item \textbf{Fix:} choose a differentiable activation so those derivatives exist and carry signal.
\end{itemize}
Once you can execute this loop for two neurons, scaling to many neurons is mostly bookkeeping (\Cref{chap:backprop}).
\paragraph{How this chapter fits the workflow.}
The same objective\(\rightarrow\)audit workflow from the supervised toolkit still applies; what changes is the \emph{representation}.
\begin{itemize}
    \item From \Cref{chap:supervised}: diagnostics (learning curves, bias--variance) tell you \emph{what} is going wrong.
    \item From \Cref{chap:logistic}: a linear probabilistic baseline tells you \emph{how far} you can go without nonlinear features.
    \item Here: we build the smallest nonlinear network and derive its gradient updates, setting up the general backprop machinery in \Cref{chap:backprop}.
\end{itemize}

\paragraph{Function estimation as the unifying view.}
Learning is function estimation: approximate some unknown mapping $f: X \to Y$ from examples. A neural network is a structured way to represent a nonlinear $f$ by composing simple units. In this chapter we keep the bookkeeping minimal and use one tiny network plus a simple squared error objective so we can focus on the mechanics of learning.

\paragraph{From one unit to a chain of units.}
A perceptron computes a weighted sum and then applies an activation (often a threshold in the classical presentation):
\begin{equation}
 y = f(p), \qquad p = \mathbf{w}^\top\mathbf{x} + b,
 \label{eq:perceptron_forward}
\end{equation}
where $\mathbf{x}\in\mathbb{R}^n$, $\mathbf{w}\in\mathbb{R}^n$, and $b$ is a bias (threshold). Because the boundary $\mathbf{w}^\top\mathbf{x} + b = 0$ is a hyperplane, a single unit can only represent linear separations. This explains the classic XOR failure and motivates building a network of units.

The smallest network that is more than a single unit is a \emph{two\hyp{}neuron chain}: one neuron feeds another. Write
\begin{align}
 p_1 &= \mathbf{w}_1^\top \mathbf{x} + b_1, & y_1 &= f(p_1), \\
 p_2 &= w_2 y_1 + b_2, & y_2 &= f(p_2). \label{eq:two_neuron_forward}
\end{align}
Even this tiny network introduces the central idea of neural networks: intermediate computations (here $y_1$) are reused and influence the final output $y_2$. A single neuron is a linear classifier; chaining neurons gives a nonlinear representation. With multiple hidden units, that added structure is enough to solve XOR. We will reuse \Cref{fig:mlp_minimal_chain} as the bookkeeping diagram when we apply the chain rule.

\begin{figure}[t]
    \centering
    \begin{tikzpicture}[
        node distance=2.4cm,
        >=Stealth,
        box/.style={draw, rounded corners, minimum width=1.6cm, minimum height=0.9cm, align=center, fill=cbBlue!6},
        neuron/.style={draw, circle, minimum size=1.0cm, align=center, fill=cbGreen!8},
        outbox/.style={draw, rounded corners, minimum width=1.2cm, minimum height=0.9cm, align=center, fill=cbOrange!8},
        lab/.style={font=\footnotesize, inner sep=1pt, align=center},
    ]
        \node[box] (x) {$\mathbf{x}$};
        \node[neuron, right=2.6cm of x] (n1) {$f$};
        \node[neuron, right=2.4cm of n1] (n2) {$f$};
        \node[outbox, right=3.0cm of n2] (y) {$y_2$};

        \draw[->] (x) -- node[lab, above, pos=0.55, yshift=6pt] {$p_1=\mathbf{w}_1^\top\mathbf{x}+b_1$} (n1);
        \draw[->] (n1) -- node[lab, below, pos=0.5, yshift=-2pt] {$y_1=f(p_1)$} (n2);
        \draw[->] (n2) -- node[lab, above, pos=0.5, xshift=-6pt, yshift=8pt] {$p_2=w_2 y_1+b_2$} (y);
        \node[lab, below=4pt of y] {$y_2=f(p_2)$};
    \end{tikzpicture}
    % Avoid inline math in captions; it wraps poorly in some EPUB renderers.
    \caption{The minimal neural network used in this chapter is a two-neuron chain. The first unit produces an intermediate signal, and the second unit maps that signal to the final output. Use it when tracking variables and gradients in a toy network before scaling to deeper models.}
    \label{fig:mlp_minimal_chain}
\end{figure}

\paragraph{Author's note: why a network changes the story.}
With one perceptron, every training example pushes on the same single separator. A network introduces \emph{intermediate representations}: different hidden units can respond to different patterns, and the output unit can combine those responses. That added structure is what lets neural networks model nonlinearity while still using simple building blocks.

\paragraph{A checklist of what we must settle (and why).}
To turn ``a diagram of neurons'' into something trainable, we need:
\begin{itemize}
    \item \textbf{A parameterization:} weights and biases that control the mapping from inputs to outputs.
    \item \textbf{A performance function:} a scalar $P$ that is lower when the output is better.
    \item \textbf{An update rule:} a systematic way to change parameters to reduce $P$.
    \item \textbf{Differentiability:} if we want to use gradient descent, every link from parameters to $P$ must be differentiable so the chain rule can propagate credit (and blame).
\end{itemize}
The chapter keeps everything small so you can see all four ingredients in one place.

\paragraph{Bias as a learned threshold.}
A hard threshold $\theta$ can be absorbed into a bias term by writing $p = \mathbf{w}^\top\mathbf{x} - \theta$ and setting $b=-\theta$. In practice we append $x_0=1$ and treat $b$ as another weight $w_0$; the algebra is identical. The bias handles \emph{where} the unit switches, while the weights handle \emph{which direction} it prefers.

\subsection{Performance: what are we trying to improve?}
\label{sec:mlp_performance_what_are_we_trying_to_improve}

Once we have a forward computation, we need a performance function that tells us whether the output is good. For one training example with target $t$, a simple choice is the squared error
\begin{equation}
 P = \frac{1}{2}(y_2 - t)^2.
 \label{eq:performance}
\end{equation}
The factor $\tfrac{1}{2}$ makes derivatives cleaner. If you prefer to \emph{maximize} a score rather than minimize an error, you could take $-\tfrac{1}{2}(y_2-t)^2$ instead. The math below is identical up to a sign. We will minimize $P$.

\paragraph{Why a square?}
The signed error $e = y_2 - t$ can be positive or negative. Squaring removes the sign and penalizes large deviations more heavily, while keeping $P$ smooth so a small change in a weight produces a small change in performance. That smoothness is exactly what makes derivative-based updates meaningful.

\begin{tcolorbox}[summarybox, title={Author's note: one objective is enough for the first derivation}]
We only need a performance function that (1) is easy to differentiate and (2) rewards outputs that move toward the target. The squared error does both, so it is a good stand-in while we learn the mechanics. Once the chain rule story is clear, swapping in other objectives is mostly a matter of changing a few local derivatives at the output layer.
\end{tcolorbox}

\paragraph{A geometric intuition.}
For a fixed input, the performance becomes a surface over the weights. If the surface looks like a ``bowl,'' then the bottom is the optimum. The goal is to move weights along this surface in the direction that improves performance. \Cref{fig:mlp_gd_surface} visualizes this geometry and contrasts vector updates with coordinate-wise zig-zag steps.

\subsection{Gradient descent: how do weights move?}
\label{sec:mlp_gradient_descent_how_do_weights_move}

We now ask: how should $\mathbf{w}_1, w_2, b_1, b_2$ change to reduce $P$? The standard answer is gradient descent:
\begin{equation}
\theta \leftarrow \theta - \eta \nabla_{\theta} P,
\label{eq:gd_update}
\end{equation}
where $\theta$ stands for any parameter and $\eta>0$ is the step size. Geometrically, you can picture the performance surface as a landscape: the gradient points uphill, so we step in the opposite direction to descend toward a minimum. The step size controls how far we move; too large can overshoot, too small can crawl.

For a weight vector, the update is a vector step:
\begin{equation}
\Delta \mathbf{w} = -\eta \nabla_{\mathbf{w}} P.
\label{eq:vectorized_update}
\end{equation}
This is the ``move the weights in the right direction'' story made precise: we do not guess the direction; we compute it from the derivative of performance. Importantly, we update \emph{all} weights at once (a vector step), not one coordinate at a time.

\paragraph{Step size is a design choice.}
The gradient gives a direction; the step size $\eta$ sets the distance. In practice you pick $\eta$ small enough to avoid oscillation and large enough to make progress. \Cref{chap:backprop} returns to this choice (learning-rate schedules, momentum, and other practical stabilizers) once the core derivative story is solid.

\begin{figure}[t]
    \centering
    \begin{tikzpicture}[background rectangle/.style={fill=white}, show background rectangle]
        \begin{axis}[
            width=0.72\linewidth,
            height=0.46\linewidth,
            xmin=-2.5, xmax=2.5,
            ymin=-2.0, ymax=2.2,
            axis lines=middle,
            xlabel={$w_1$},
            ylabel={$w_2$},
            grid=both,
            minor grid style={gray!10},
            major grid style={gray!20},
            tick label style={font=\scriptsize},
            label style={font=\scriptsize},
            axis background/.style={fill=white},
            clip=false,
        ]
            % Elliptical contours to suggest a convex bowl.
            \addplot[domain=0:360, samples=200, gray!55] ({1.6*cos(x)},{0.7*sin(x)});
            \addplot[domain=0:360, samples=200, gray!55] ({2.1*cos(x)},{0.95*sin(x)});
            \addplot[domain=0:360, samples=200, gray!55] ({2.6*cos(x)},{1.2*sin(x)});

            % Coordinate-wise updates (zig-zag).
            \addplot[cbOrange, thick, -{Stealth[length=2mm]}] coordinates {(-2.0,1.3) (-0.6,1.3)};
            \addplot[cbOrange, thick, -{Stealth[length=2mm]}] coordinates {(-0.6,1.3) (-0.6,0.2)};
            \addplot[cbOrange, thick, -{Stealth[length=2mm]}] coordinates {(-0.6,0.2) (0.0,0.2)};

            % Gradient step (single vector move).
            \addplot[cbBlue, very thick, -{Stealth[length=2.2mm]}] coordinates {(-2.0,1.3) (-0.8,0.35)};

            \node[font=\scriptsize, cbBlue!70!black] at (axis cs:-1.25,0.55) {$-\nabla P$};
            \node[font=\scriptsize, cbOrange!80!black, align=center] at (axis cs:-0.1,1.45) {one weight\\at a time};
        \end{axis}
    \end{tikzpicture}
    % Avoid inline math in captions; it wraps poorly in some EPUB renderers.
    \caption{Think of performance as a surface over the weights. Gradient descent moves in one vector step (blue), whereas coordinate-wise updates can zig-zag (orange). Use it when building intuition for why gradient directions reduce loss more efficiently than axis-aligned steps.}
    \label{fig:mlp_gd_surface}
\end{figure}

\subsection{Why hard thresholds block learning}
\label{sec:mlp_why_hard_thresholds_block_learning}

At this point the story is simple: define $P$, compute $\nabla P$, and update weights. The catch is that computing $\nabla P$ requires derivatives through the activation. When we apply the chain rule to the forward-pass equations in \eqref{eq:two_neuron_forward}, factors like $f'(p_1)$ and $f'(p_2)$ appear immediately. \Cref{fig:mlp_step_vs_sigmoid} makes this derivative contrast concrete.

If $f$ is a hard threshold (a step function), it is discontinuous and non\hyp{}differentiable at the threshold. That breaks the gradient story: $f'(p)$ either does not exist or is zero almost everywhere, so derivatives cannot guide learning. This is the core reason we replace thresholds with \emph{smooth, differentiable activations}.

\paragraph{Absorbing the threshold.}
The threshold itself can be folded into a bias term, but the discontinuity remains. We remove the discontinuity by choosing a smooth $f$.

\begin{figure}[t]
    \centering
    \begin{tikzpicture}[background rectangle/.style={fill=white}, show background rectangle]
        \begin{groupplot}[
            group style={group size=2 by 1, horizontal sep=1.1cm},
            width=0.44\linewidth,
            height=0.32\linewidth,
            axis lines=middle,
            xmin=-4, xmax=4,
            samples=200,
            tick label style={font=\scriptsize},
            label style={font=\scriptsize},
            title style={font=\scriptsize},
            axis background/.style={fill=white},
        ]
        \nextgroupplot[
            title={Hard threshold (step)},
            ymin=-0.15, ymax=1.15,
            ytick={0,1},
            xlabel={$p$},
        ]
            \addplot[cbBlue, thick, domain=-4:0] {0};
            \addplot[cbBlue, thick, domain=0:4] {1};
            % Derivative is zero almost everywhere (and undefined at 0).
            \addplot[cbBlue, dashed, domain=-4:4] {0};
        \nextgroupplot[
            title={Smooth activation (sigmoid)},
            ymin=-0.15, ymax=1.15,
            ytick={0,0.5,1},
            xlabel={$p$},
        ]
            \addplot[cbOrange, thick, domain=-4:4] {1/(1+exp(-x))};
            \addplot[cbOrange, dashed, domain=-4:4] {1/(1+exp(-x))*(1-1/(1+exp(-x)))};
        \end{groupplot}
    \end{tikzpicture}
    % Avoid inline math in captions; it wraps poorly in some EPUB renderers.
    \caption{Hard thresholds block gradient-based learning because the derivative is zero almost everywhere. A smooth activation like the sigmoid provides informative derivatives across a wide range of inputs. Use it when motivating why smooth nonlinearities enable gradient-based training.}
    \label{fig:mlp_step_vs_sigmoid}
\end{figure}


\subsection{Differentiable activations and the sigmoid trick}
\label{sec:mlp_differentiable_activations_and_the_sigmoid_trick}

A classic choice is the logistic (sigmoid) function
\begin{equation}
\sigma(p) = \frac{1}{1+e^{-p}}.
\label{eq:sigmoid}
\end{equation}
It maps real inputs to $(0,1)$ and is differentiable everywhere. The key identity is
\begin{equation}
\sigma'(p) = \sigma(p)\,[1-\sigma(p)].
\label{eq:sigmoid_derivative}
\end{equation}
This is a useful trick: the derivative is a function of the \emph{output} itself. If $y=\sigma(p)$ is already computed in the forward pass, then $\sigma'(p)=y(1-y)$ is immediately available in the backward pass. No extra exponentials are needed.

\begin{tcolorbox}[summarybox, title={Author's note: the derivative is already in the forward pass}]
In practice you rarely want to recompute expensive expressions during learning. For the sigmoid, once you have computed the output \texttt{y = sigmoid(p)}, you also have its slope for free: \texttt{sigmoid\_prime = y*(1-y)}. This is a small example of a bigger pattern: backpropagation works because we cache intermediate results on the forward pass and reuse them on the backward pass.
\end{tcolorbox}

\paragraph{Derivation sketch.}
Let $\beta = \sigma(\alpha) = (1+e^{-\alpha})^{-1}$. Differentiate:
\[
\frac{d\beta}{d\alpha} = \frac{e^{-\alpha}}{(1+e^{-\alpha})^2}
= \left(\frac{1}{1+e^{-\alpha}}\right)\left(1-\frac{1}{1+e^{-\alpha}}\right)
 = \beta(1-\beta).
\]
This is the exact algebraic shortcut used in neural networks.

\subsection{Deriving weight updates for the two\hyp{}neuron network}
\label{sec:mlp_deriving_weight_updates_for_the_two_neuron_network}

The diagram in \Cref{fig:mlp_minimal_chain} is also a derivative map: to update a weight, follow how a small change in that weight would flow forward to the output and then back to the performance. The chain rule turns that story into algebra.

We now compute the gradients from the forward-pass equations in \eqref{eq:two_neuron_forward} using the chain rule. First note the easy derivatives:
\begin{itemize}
    \item $\displaystyle \frac{\partial P}{\partial y_2} = y_2 - t$.
    \item $\displaystyle \frac{\partial y_i}{\partial p_i} = f'(p_i)$.
    \item $\displaystyle \frac{\partial p_2}{\partial w_2} = y_1$ and $\displaystyle \frac{\partial p_2}{\partial y_1} = w_2$.
    \item $\displaystyle \frac{\partial p_1}{\partial \mathbf{w}_1} = \mathbf{x}$.
\end{itemize}

\paragraph{Second layer.}
\begin{align}
\frac{\partial P}{\partial w_2}
&= \frac{\partial P}{\partial y_2}\frac{\partial y_2}{\partial p_2}\frac{\partial p_2}{\partial w_2}
= (y_2 - t)\, f'(p_2)\, y_1,
\label{eq:grad_w2}
\end{align}
and similarly
\begin{equation}
\frac{\partial P}{\partial b_2} = (y_2 - t) f'(p_2).
\label{eq:grad_b2}
\end{equation}

\paragraph{First layer.}
The first layer feels the effect of the second layer through the chain rule:
\begin{align}
\frac{\partial P}{\partial \mathbf{w}_1}
&= \frac{\partial P}{\partial y_2}\frac{\partial y_2}{\partial p_2}\frac{\partial p_2}{\partial y_1}\frac{\partial y_1}{\partial p_1}\frac{\partial p_1}{\partial \mathbf{w}_1} \\
&= (y_2 - t)\, f'(p_2)\, w_2\, f'(p_1)\, \mathbf{x},
\label{eq:grad_w1}
\end{align}
with bias derivative
\begin{equation}
\frac{\partial P}{\partial b_1} = (y_2 - t) f'(p_2) w_2 f'(p_1).
\label{eq:grad_b1}
\end{equation}

\paragraph{Error terms (backprop view).}
Define
\begin{equation}
\delta_2:= \frac{\partial P}{\partial p_2} = (y_2 - t) f'(p_2).
\label{eq:delta2}
\end{equation}
Then $\partial P/\partial w_2 = \delta_2 y_1$ and $\partial P/\partial b_2=\delta_2$. The first layer receives a backpropagated error
\begin{equation}
\delta_1:= \frac{\partial P}{\partial p_1} = \delta_2 w_2 f'(p_1),
\label{eq:delta1}
\end{equation}
so $\partial P/\partial \mathbf{w}_1 = \delta_1 \mathbf{x}$ and $\partial P/\partial b_1 = \delta_1$.

This is the central lesson: once we compute a local error term, it can be reused across many gradients. That reuse is exactly what makes backpropagation efficient and is why deeper networks remain tractable.

\begin{tcolorbox}[summarybox, title={Worked example: one numerical gradient step (sanity check)}]
Take a single input $\mathbf{x} = [1,-1]^\top$, target $t=1$, sigmoid activation $f=\sigma$, and parameters
$\mathbf{w}_1=[0.8,\,0.2]^\top$, $b_1=0$, $w_2=1$, $b_2=0$.
\medskip

\noindent\textbf{Forward:} $p_1=0.6$, $y_1=\sigma(p_1)\approx 0.646$; $p_2=y_1$, $y_2=\sigma(p_2)\approx 0.656$; $P=\tfrac12(y_2-t)^2\approx 0.059$.
\medskip

\noindent\textbf{Backward:} $\sigma'(p)=y(1-y)$, so $\delta_2=(y_2-t)\sigma'(p_2)\approx -0.078$ and $\delta_1=\delta_2 w_2 \sigma'(p_1)\approx -0.018$.
Thus $\nabla_{\mathbf{w}_1}P=\delta_1 \mathbf{x} \approx [-0.018,\,+0.018]^\top$ and $\nabla_{w_2}P=\delta_2 y_1\approx -0.050$.
\medskip

\noindent\textbf{Update:} with $\eta=0.5$, gradient descent increases $w_2$ slightly (since the gradient is negative) and nudges $\mathbf{w}_1$ in a direction that increases $y_2$ toward the target.
\end{tcolorbox}

\subsection{From two neurons to multi\hyp{}
\label{sec:mlp_from_two_neurons_to_multi}layer networks}
\label{sec:mlp_from_two_neurons_to_multi_sec_mlp_from_two_neurons_to_multi_layer_networks}

Nothing essential changes for deeper networks; we simply apply the same chain rule repeatedly. For a layer $l$ with pre\hyp{}activations $\mathbf{p}^{(l)}$ and weights $\mathbf{W}^{(l+1)}$, the error signal satisfies
\begin{equation}
\boldsymbol{\delta}^{(l)} = \bigl(\boldsymbol{\delta}^{(l+1)} (\mathbf{W}^{(l+1)})^\top\bigr) \circ f'(\mathbf{p}^{(l)}),
\label{eq:delta_recursion}
\end{equation}
where $\circ$ denotes element\hyp{}wise multiplication. This recursion is the heart of backpropagation, which we derive and operationalize in \Cref{chap:backprop}.

\begin{tcolorbox}[summarybox, title={Author's note: what backprop adds}]
Conceptually, nothing new happens when you go from two neurons to many layers: it is still the chain rule and the same local derivatives. What changes is the \emph{organization}: we run a forward pass that caches intermediate values, then a backward pass that reuses those caches to compute all gradients efficiently (and stably) for an entire batch. \Cref{chap:backprop} turns the recursion into an implementable algorithm and shows the standard bookkeeping.
\end{tcolorbox}

\subsection{Summary}
\label{sec:mlp_summary}
\begin{itemize}
    \item A two\hyp{}neuron chain is the smallest network that goes beyond a single perceptron.
    \item Learning starts by defining a performance function and asking how weights change it.
    \item Gradient descent uses derivatives to choose the correct update direction.
    \item Hard thresholds obstruct gradients; smooth activations fix the problem.
    \item The sigmoid derivative $\sigma'(p)=\sigma(p)(1-\sigma(p))$ is a convenient identity because it reuses the output.
    \item The two\hyp{}neuron derivation already contains the backpropagation pattern used in deep networks.
\end{itemize}

\begin{tcolorbox}[summarybox, title={Derivation closure: implement, cache, fail-fast}]
\begin{itemize}
    \item \textbf{Implement:} treat the chapter equations as a forward function plus a scalar loss; write them once in vector form before batching.
    \item \textbf{Cache:} keep \((\mathbf{x}, p_1, y_1, p_2, y_2)\) from the forward pass so each gradient term is a local reuse, not a re-derivation.
    \item \textbf{Fail-fast checks:} run finite-difference gradient checks on a tiny example, then track train/validation divergence to catch update-sign or step-size mistakes early.
    \item \textbf{Handoff:} the same cache-then-backward discipline scales directly in \Cref{chap:backprop}; only bookkeeping grows.
\end{itemize}
\end{tcolorbox}

\begin{tcolorbox}[summarybox, title={Key takeaways}]
\textbf{Minimum viable mastery}
\begin{itemize}
    \item Training is a loop: define a forward computation, define a scalar performance, then use derivatives to update parameters.
    \item Hard thresholds break the gradient story; smooth activations (e.g., sigmoid) restore informative derivatives.
    \item The two-neuron derivation already contains the reusable ``local error'' pattern that scales to deep networks.
\end{itemize}
\medskip
\textbf{Common pitfalls}
\begin{itemize}
    \item Trying to differentiate through discontinuities (step functions) and then ``patching'' gradients by hand.
    \item Losing the chain rule in bookkeeping: cache intermediate values and reuse them consistently.
    \item Confusing notation: distinguish pre-activation vs.\ activation, and keep shapes explicit when batching.
\end{itemize}
\end{tcolorbox}

\begin{tcolorbox}[summarybox, title={Exercises and lab ideas}]
\noindent\textbf{Setup.} These reinforce the two-neuron derivation and prepare you for the multi-layer bookkeeping in \Cref{chap:backprop}.
\begin{itemize}
    \item \textbf{Numerical gradient check:} Implement finite differences for the two\hyp{}neuron chain and compare to your analytic gradients; report relative error.
    \item \textbf{Step vs.\ sigmoid:} Replace the smooth activation with a hard threshold and observe what breaks when you try to compute updates via derivatives.
    \item \textbf{XOR with two hidden units:} Train a tiny MLP on XOR and plot its decision regions; note sensitivity to initialization and step size.
\end{itemize}
\medskip
\noindent\textbf{If you are skipping ahead.} Be able to read a forward graph and a backward recursion: local derivatives, cached activations, and a scalar loss driving parameter updates. That is exactly what \Cref{chap:backprop} scales up.
\end{tcolorbox}

\medskip
\paragraph{Where we head next.} \Cref{chap:backprop} lifts this two-neuron derivation to deep networks and shows how to implement the same backward logic efficiently in batch training.

% Chapter 7
\section{Backpropagation Learning in Multi-Layer Perceptrons}\label{chap:backprop}
\graphicspath{{assets/lec4/}}

Building on the two\hyp{}neuron derivation in \Cref{chap:mlp}, we derive backpropagation for an $L$-layer network as a systematic application of the chain rule. The idea is unchanged: compute local error terms (the \(\delta\)'s), then reuse them to obtain all weight gradients efficiently. \Cref{fig:roadmap} marks this chapter as the training engine for deep models.

\begin{tcolorbox}[summarybox, title={Learning Outcomes}]
\begin{itemize}
    \item Derive the layerwise backpropagation recursions for arbitrary-depth MLPs.
    \item Connect theoretical gradients to implementation details (vectorization, caching, numerical stability).
    \item Translate training diagnostics (learning curves, early stopping) into concrete optimization policies.
\end{itemize}
\end{tcolorbox}

\begin{tcolorbox}[summarybox, title={Design motif}]
Compute local error signals once, then reuse them to update every parameter efficiently. The organization (cache forward values, then sweep backward) is the algorithm.
\end{tcolorbox}

\subsection{Context and Motivation}
\label{sec:backprop_context_and_motivation}

Multi\hyp{}layer networks raise a specific question: \emph{How do we update the weights across multiple layers when the only explicit error signal is at the output?} In a single\hyp{}layer perceptron, the output error touches the weights directly; in a deep network, a change in one layer propagates through subsequent layers and alters the output in a nonlinear, intertwined way.

Shallow networks (one hidden layer) already move beyond linear separability, but more complex tasks demand deeper hierarchies of features. The multi\hyp{}layer perceptron stacks these layers to learn richer decision boundaries, and backpropagation is the mechanism that makes that depth trainable.

\paragraph{Implementation lens.}
A practical MLP rarely updates one coordinate at a time; gradients are treated as full vectors so every weight moves coherently. Backpropagation is what turns a multilayer diagram into a trainable model: it reuses local error signals so you can compute \emph{all} gradients efficiently, making it practical to learn hidden representations (not just tune a final linear layer) while keeping the same validation\(\rightarrow\)audit discipline (learning curves, early stopping, and slice checks).

\subsection{Problem Setup}
\label{sec:backprop_problem_setup}

Consider a multi-layer perceptron with layers indexed by $l = 0, 1, \ldots, L$, where $l=0$ is the input layer and $L$ is the output layer. Each layer $l$ contains neurons indexed by $i$, and the output of neuron $i$ in layer $l$ is denoted by $a_i^{(l)}$. The input to this neuron before activation is denoted by $z_i^{(l)}$. The weights connecting neuron $i$ in layer $l-1$ to neuron $j$ in layer $l$ are denoted by $w_{ij}^{(l)}$.

The forward pass through the network is given by:
\begin{align}
    z_j^{(l)} &= \sum_i a_i^{(l-1)} w_{ij}^{(l)} + b_j^{(l)}, \label{eq:forward_z} \\
    a_j^{(l)} &= f\big(z_j^{(l)}\big), \label{eq:forward_a}
\end{align}
where $b_j^{(l)}$ is the bias term for neuron $j$ in layer $l$, and $f(\cdot)$ is the activation function, typically nonlinear (e.g., sigmoid, ReLU). Equation~\eqref{eq:forward_z} makes it explicit that we sum over every incoming neuron $i$ in layer $l-1$ to form the affine pre-activation $z_j^{(l)}$.

\subsection{Loss and Objective}
\label{sec:backprop_loss_and_objective}

To keep the story linear (and aligned with \Cref{chap:mlp}), we will use a simple squared\hyp{}error objective. Let the network output be \(\mathbf{a}^{(L)}\) and let \(\mathbf{t}\) be the target (one\hyp{}hot targets for classification are fine; we do not need a separate regression/classification split yet). A standard loss is
\begin{equation}
    \mathcal{L} = \frac{1}{2} \sum_k \left( t_k - a_k^{(L)} \right)^2. \label{eq:error_function}
\end{equation}
The goal of learning is to adjust the weights \(\{w_{ij}^{(l)}\}\) to minimize \(\mathcal{L}\). Later in this chapter, we briefly note how common alternatives (notably cross\hyp{}entropy with sigmoid/softmax outputs) simplify the output\hyp{}layer error term; the backprop recursion itself does not change.

\subsection{Challenges in Weight Updates}
\label{sec:backprop_challenges_in_weight_updates}

In a shallow network, weight updates can be computed directly from the output error. However, in a deep network, the output error depends on all weights in a complex way. A change in a weight in an earlier layer affects the activations of subsequent layers, ultimately influencing the output.

For example, consider a weight $w_{ij}^{(l)}$ connecting neuron $i$ in layer $l-1$ to neuron $j$ in layer $l$. Changing this weight affects $z_j^{(l)}$, which affects $a_j^{(l)}$, which in turn affects all neurons in layers $l+1, l+2, \ldots, L$. Therefore, the total effect of changing $w_{ij}^{(l)}$ on the loss is a composition of many intermediate effects.

\subsection{Notation for Layers and Neurons}
\label{sec:backprop_notation_for_layers_and_neurons}

To formalize this, we introduce the following notation:
\begin{itemize}
    \item $l$: layer index, with $l=0$ representing the input layer, and $l=L$ the output layer.
    \item $i$: neuron index in layer $l-1$.
    \item $j$: neuron index in layer $l$.
    \item $k$: neuron index in layer $L$ (output layer).
    \item $a_i^{(l)}$: activation of neuron $i$ in layer $l$.
    \item $z_j^{(l)}$: weighted input to neuron $j$ in layer $l$.
    \item $w_{ij}^{(l)}$: weight from neuron $i$ in layer $l-1$ to neuron $j$ in layer $l$.
    \item $b_j^{(l)}$: bias of neuron $j$ in layer $l$.
    \item $f(\cdot)$: activation function.
\end{itemize}
These definitions carry directly into the forward-pass recap below, where we chain the affine map and nonlinearity across layers.
\paragraph{Notation handoff.}
Across this chapter, \(a\) denotes activations and \(z\) denotes pre-activations; this pairing is reused in later deep-learning chapters. If you jump into chapters out of order, keep \Cref{app:notation_collisions} nearby for symbol overloads.

\subsection{Forward Pass Recap}
\label{sec:backprop_forward_pass_recap}

The forward pass computes activations layer by layer:
\begin{align}
    z_j^{(l)} &= \sum_i a_i^{(l-1)} w_{ij}^{(l)} + b_j^{(l)}, \\
    a_j^{(l)} &= f\big(z_j^{(l)}\big).
    \label{eq:auto:lecture_4_part_i:1}
\end{align}

The output layer activations $a_k^{(L)}$ are compared to the

\begin{tcolorbox}[summarybox, title={Mini example: two-layer backprop in practice}]
\footnotesize
\begin{verbatim}
# Shapes: X in R^{B x d}, W1 in R^{d x h}, W2 in R^{h x c}
def step(X, Y, params, eta, wd=1e-4, p_drop=0.1):
    W1, b1, W2, b2 = params
    B = X.shape[0]
    # Forward pass
    Z1 = X @ W1 + b1
    H1 = relu(Z1)
    mask1 = (np.random.rand(*H1.shape) > p_drop).astype(
        H1.dtype
    )
    # inverted dropout
    H1 = H1 * mask1 / (1 - p_drop)
    Z2 = H1 @ W2 + b2
    Yhat = softmax(Z2)
    # Backward pass
    delta2 = (Yhat - Y) / B            # CE output error
    grad_W2 = H1.T @ delta2
    grad_W2 += wd * W2              # L2 decay
    grad_b2 = delta2.sum(axis=0)
    delta1 = (delta2 @ W2.T) * relu_deriv(Z1)
    # dropout backprop
    delta1 = delta1 * mask1 / (1 - p_drop)
    grad_W1 = X.T @ delta1
    grad_W1 += wd * W1
    grad_b1 = delta1.sum(axis=0)
    # SGD step
    return (W1 - eta * grad_W1, b1 - eta * grad_b1,
            W2 - eta * grad_W2, b2 - eta * grad_b2)
\end{verbatim}
\normalsize
The elementwise product \texttt{*} mirrors the Hadamard notation from \Crefrange{eq:forward_z}{eq:forward_a}. This miniature example bridges the algebra to vectorized code before we scale to \(L\)-layer MLPs later in the chapter.
\end{tcolorbox}

\begin{tcolorbox}[summarybox, title={Shape ledger for an $L$-layer MLP (batch size $B$)}]
\begin{itemize}
    \item \(A^{(l-1)}\in\mathbb{R}^{B\times n_{l-1}},\; Z^{(l)},\delta^{(l)}\in\mathbb{R}^{B\times n_l}\)
    \item \(W^{(l)}\in\mathbb{R}^{n_{l-1}\times n_l},\; b^{(l)}\in\mathbb{R}^{n_l}\)
    \item \(\partial L/\partial W^{(l)} = (A^{(l-1)})^\top \delta^{(l)} / B \in \mathbb{R}^{n_{l-1}\times n_l}\)
    \item \(\partial L/\partial b^{(l)} = \mathrm{batch\_mean}(\delta^{(l)}) \in \mathbb{R}^{n_l}\)
\end{itemize}
Layers share this structure; convolutional/sequence models reuse the same calculus with different bookkeeping.
\end{tcolorbox}

\subsection{Backpropagation: Recursive Computation of Error Terms}
\label{sec:backprop_backpropagation_recursive_computation_of_error_terms}

Recall that our goal is to compute the gradient of the loss with respect to the weights in the network, specifically for weights connecting layer \( l \) to layer \( l+1 \). We denote the weight connecting neuron \( i \) in layer \( l \) to neuron \( j \) in layer \( l+1 \) as \( w_{ij}^{(l)} \).

We will continue with the squared\hyp{}error loss from \Cref{chap:mlp}:
\begin{equation}
\mathcal{L} = \frac{1}{2} \sum_{k} (t_k - a_k^{(L)})^2.
\label{eq:auto_backprop_4c440b0502}
\end{equation}
where \( t_k \) is the target output and \( a_k^{(L)} \) is the activation of output neuron \( k \). Other losses change only a few local derivatives (most notably at the output layer), but the backprop recursion and bookkeeping are the same.

To update the weights using gradient descent, we need to compute
\[
\frac{\partial \mathcal{L}}{\partial w_{ij}^{(l)}}.
\]

\paragraph{Chain rule decomposition}

By the chain rule, we have
\begin{equation}
\frac{\partial \mathcal{L}}{\partial w_{ij}^{(l)}} = \frac{\partial \mathcal{L}}{\partial z_j^{(l+1)}} \cdot \frac{\partial z_j^{(l+1)}}{\partial w_{ij}^{(l)}}.
\label{eq:chain_rule_weight}
\end{equation}
where \( z_j^{(l+1)} \) is the weighted input to neuron \( j \) in layer \( l+1 \):
\[
z_j^{(l+1)} = \sum_i a_i^{(l)} w_{ij}^{(l)} + b_j^{(l+1)}.
\]
Here \( a_i^{(l)} \) is the activation of neuron \( i \) in layer \( l \), and \( b_j^{(l+1)} \) the bias term.

Since \( z_j^{(l+1)} \) is linear in \( w_{ij}^{(l)} \), we have
\[
\frac{\partial z_j^{(l+1)}}{\partial w_{ij}^{(l)}} = a_i^{(l)}.
\]

Thus,
\begin{equation}
\frac{\partial \mathcal{L}}{\partial w_{ij}^{(l)}} = \delta_j^{(l+1)} a_i^{(l)},
\label{eq:weight_gradient}
\end{equation}
where we define the \emph{error term}
\[
\delta_j^{(l+1)}:= \frac{\partial \mathcal{L}}{\partial z_j^{(l+1)}}.
\]
Collecting the \(\delta_j^{(l+1)}\) for all neurons in layer \(l+1\) forms a vector \(\boldsymbol{\delta}^{(l+1)}\) with the same dimension as \(z^{(l+1)}\), ensuring the gradient \(\frac{\partial \mathcal{L}}{\partial W^{(l)}}\) has the same shape as the weight matrix.

\paragraph{Interpretation of \(\delta_j^{(l+1)}\)}

The term \(\delta_j^{(l+1)}\) measures how sensitive the error is to changes in the net input \( a_j^{(l+1)} \). Our task reduces to computing these \(\delta\) terms for all neurons in the network.

\subsubsection{Output layer error terms}
\label{sec:backprop_output_layer_error_terms_sub}

For the output layer \( L \), the activation of neuron \( k \) is
\[
a_k^{(L)} = f\big(z_k^{(L)}\big),
\]
where \(f(\cdot)\) is the activation function.

The error term for output neuron \( k \) is
\begin{align}
\delta_k^{(L)} &= \frac{\partial \mathcal{L}}{\partial z_k^{(L)}} \\
&= \frac{\partial \mathcal{L}}{\partial a_k^{(L)}} \frac{\partial a_k^{(L)}}{\partial z_k^{(L)}} \\
&= \big(a_k^{(L)} - t_k\big) \, \phi'\big(z_k^{(L)}\big),
\label{eq:delta_output}
\end{align}
where \(\phi'\) denotes the derivative of the activation function evaluated element-wise.
For cross\hyp{}entropy with sigmoid/softmax output, \(\phi'\) cancels and \(\delta_k^{(L)} = a_k^{(L)}-t_k\); for MSE retain the factor above.

\subsubsection{Hidden layer error terms}
\label{sec:backprop_hidden_layer_error_terms_sub}

For a hidden neuron \( j \) in layer \( l \), the error term \(\delta_j^{(l)}\) depends on the error terms of the neurons in the next layer \( l+1 \) to which it connects. Using the chain rule,
\begin{align}
\delta_j^{(l)} &= \frac{\partial \mathcal{L}}{\partial z_j^{(l)}} \\
&= \sum_{k} \frac{\partial \mathcal{L}}{\partial z_k^{(l+1)}} \frac{\partial z_k^{(l+1)}}{\partial z_j^{(l)}} \\
&= \sum_{k} \delta_k^{(l+1)} \frac{\partial z_k^{(l+1)}}{\partial z_j^{(l)}}.
\label{eq:delta_hidden_chain}
\end{align}

Since
\[
z_k^{(l+1)} = \sum_m a_m^{(l)}\, w_{mk}^{(l)} + b_k^{(l+1)},
\]
and \( a_j^{(l)} = \phi\big(z_j^{(l)}\big) \), we have
\[
\frac{\partial z_k^{(l+1)}}{\partial z_j^{(l)}} = w_{jk}^{(l)} \, \phi'\big(z_j^{(l)}\big).
\]

Substituting into \eqref{eq:delta_hidden_chain} yields
\begin{equation}
\delta_j^{(l)} = \phi'\big(z_j^{(l)}\big) \sum_{k} w_{jk}^{(l)} \delta_k^{(l+1)}.
\label{eq:delta_hidden}
\end{equation}

For sigmoid activations $\phi$, the derivative simplifies to $\phi'(z_j^{(l)}) = a_j^{(l)} (1 - a_j^{(l)})$; other activations require substituting their respective derivatives in \eqref{eq:delta_hidden}.

\paragraph{Summary: Backpropagation recursion}
Backpropagation is reverse-mode automatic differentiation on the network graph. A forward pass caches intermediates; a reverse pass reuses caches to get all gradients in \(O(P)\) time (versus \(O(P^2)\) for finite differences) for \(P\) parameters. Frameworks (PyTorch/JAX/TF) automate this; the algebra below is its manual derivation.

\subsection{Backpropagation Algorithm: Detailed Derivation}
\label{sec:backprop_backpropagation_algorithm_detailed_derivation}

Recall that the goal of backpropagation is to compute the gradient of the loss with respect to each weight in the network, enabling gradient descent updates. Consider a single neuron \( k \) in the output layer with output \( o_k \) and activation \( a_k \). The target output is \( t_k \).

\paragraph{Error function and its derivatives}

We use the squared-error loss for a single output neuron:
\begin{equation}
    \mathcal{L}_{\text{SE}} = 0.5 \, (t_k - o_k)^2.
\label{eq:auto_backprop_6569a13e6f}
\end{equation}

Our objective is to compute \(\frac{\partial \mathcal{L}_{\text{SE}}}{\partial w_{jk}}\), where \(w_{jk}\) is the weight from neuron \(j \) in the previous layer to neuron \(k\).

By the chain rule,
\begin{equation}
    \frac{\partial \mathcal{L}_{\text{SE}}}{\partial w_{jk}} = \frac{\partial \mathcal{L}_{\text{SE}}}{\partial o_k} \cdot \frac{\partial o_k}{\partial a_k} \cdot \frac{\partial a_k}{\partial w_{jk}}.
    \label{eq:chain_rule}
\end{equation}

\paragraph{Step 1: Derivative of error with respect to output}
From the squared-error definition above,
\begin{equation}
    \frac{\partial \mathcal{L}_{\text{SE}}}{\partial o_k} = o_k - t_k.
    \label{eq:dE_do}
\end{equation}

\paragraph{Step 2: Derivative of output with respect to activation}

Assuming the activation function \( f \) is the sigmoid,
\[
    o_k = f(a_k) = \frac{1}{1 + e^{-a_k}},
\]
its derivative is
\begin{equation}
    \frac{\partial o_k}{\partial a_k} = f'(a_k) = o_k (1 - o_k).
    \label{eq:do_da}
\end{equation}

\paragraph{Step 3: Derivative of activation with respect to weight}

The activation \( a_k \) is the weighted sum of inputs:
\[
    a_k = \sum_j w_{jk} x_j.
\]
Here \( x_j \) is the output from neuron \( j \) in the previous layer. Thus,
\begin{equation}
    \frac{\partial a_k}{\partial w_{jk}} = x_j.
    \label{eq:da_dw}
\end{equation}

\paragraph{Putting it all together}

Substituting \eqref{eq:dE_do}, \eqref{eq:do_da}, and \eqref{eq:da_dw} into \eqref{eq:chain_rule}:
\begin{equation}
    \frac{\partial \mathcal{L}_{\text{SE}}}{\partial w_{jk}} = (o_k - t_k) \, o_k (1 - o_k) \, x_j.
    \label{eq:dE_dw}
\end{equation}

Define the \emph{error signal} for neuron \( k \) as
\begin{equation}
    \delta_k = (o_k - t_k) \, o_k (1 - o_k).
    \label{eq:error_signal_output}
\end{equation}

Then,
\begin{equation}
    \frac{\partial \mathcal{L}_{\text{SE}}}{\partial w_{jk}} = \delta_k x_j.
    \label{eq:dE_dw_delta}
\end{equation}

The gradient descent update therefore becomes
\begin{equation}
    \Delta w_{jk} = -\eta \, \delta_k x_j,
    \label{eq:weight_update_output}
\end{equation}
where \(\eta\) is the learning rate.

\subsection{Backpropagation for Hidden Layers}
\label{sec:backprop_backpropagation_for_hidden_layers}

For neurons in hidden layers, the error signal \(\delta_j\) is computed by propagating the error backward from the next layer. Consider a hidden neuron \( j \) with pre-activation \( z_j \) and activation \( o_j = f(z_j) \). Its error signal is
\begin{equation}
    \delta_j = o_j (1 - o_j) \sum_k w_{jk} \delta_k,
    \label{eq:error_signal_hidden}
\end{equation}
where the sum is over all neurons \( k \) in the next layer to which neuron \( j \) connects.

The weight update for weights \( w_{ij} \) feeding into neuron \( j \) is then
\begin{equation}
    \Delta w_{ij} = -\eta \, \delta_j x_i,
    \label{eq:weight_update_hidden}
\end{equation}
where \( x_i \) is the output from the previous layer neuron \( i \).

\subsection{Batch and Stochastic Gradient Descent}
\label{sec:backprop_batch_and_stochastic_gradient_descent}

Given a training set of \( N \) examples \(\{(x^{(n)}, t^{(n)})\}_{n=1}^N\), the weight updates can be computed in different ways:

\begin{itemize}
    \item \textbf{Batch gradient descent:} Compute the gradient over the entire dataset and update weights once per epoch:
    \[
        \Delta w = -\frac{\eta}{N} \sum_{n=1}^N \delta^{(n)} x^{(n)}.
    \]

    \item \textbf{Stochastic gradient descent (SGD):} Update weights after each training example using the instantaneous gradient \(-\eta \, \delta^{(n)} x^{(n)}\). Although the updates are noisy, SGD often converges faster in practice and can escape shallow local minima.
\end{itemize}

\begin{tcolorbox}[summarybox, title={Optimizer and stability notes}]
SGD remains the backbone; momentum and Adam/AdamW from \Cref{chap:cnn} accelerate convergence. Add L2 weight decay to the gradient or decouple it (AdamW) to avoid biasing adaptive steps. For deep or ill-conditioned nets, gradient clipping can prevent explosions; for classification, pair these with cross\hyp{}entropy and the log-sum-exp stability tricks introduced alongside CNNs in \Cref{chap:cnn}. Reverse-mode AD underlies all of these updates.
\end{tcolorbox}


\begin{figure}[t]
    \centering
    \resizebox{\linewidth}{!}{%
    \begin{tikzpicture}[
        >={Stealth[round, length=2.5mm, width=1.75mm]},
        node distance=1.2cm and 1.5cm,
        font=\small\sffamily,
        base/.style={draw, line width=0.9pt, rounded corners=4pt, align=center},
        var/.style={base, circle, fill=white, draw=gray!40, minimum size=12mm, inner sep=1pt},
        block/.style={base, rectangle, fill=cbBlue!18, draw=cbBlue!75!black, minimum height=12mm, minimum width=18mm},
        act/.style={base, rectangle, fill=cbBlue!8, draw=cbBlue!75!black, minimum height=12mm, minimum width=18mm},
        param/.style={base, rectangle, fill=cbGreen!18, draw=cbGreen!75!black, minimum height=10mm, minimum width=26mm},
        loss/.style={base, rectangle, fill=cbPink!12, draw=cbPink!80!black, minimum height=12mm, minimum width=18mm},
        fwd/.style={->, line width=1.3pt, draw=gray!55},
        bwd/.style={->, line width=1.1pt, draw=cbOrange!85!black, dashed, rounded corners=10pt},
        grad/.style={->, line width=1.1pt, draw=cbGreen!75!black, dashed}
    ]
        \node[var] (x) {$\mathbf{x}$};

        \node[block, right=of x] (Z1) {$Z^{(1)}$};
        \node[param, above=1.5cm of Z1] (W1) {$W^{(1)},\, b^{(1)}$};
        \node[act, right=0.8cm of Z1] (A1) {$A^{(1)}$\\[-2pt]{\scriptsize $f(\cdot)$}};

        \node[right=0.8cm of A1, font=\large\bfseries, text=gray!40] (dots) {$\cdots$};

        \node[block, right=0.8cm of dots] (ZL) {$Z^{(L)}$};
        \node[param, above=1.5cm of ZL] (WL) {$W^{(L)},\, b^{(L)}$};
        \node[act, right=0.8cm of ZL] (AL) {$A^{(L)}$\\[-2pt]{\scriptsize $\sigma(\cdot)$}};

        \node[loss, right=1.2cm of AL] (L) {$\mathcal{L}$};
        \node[var, above=1.5cm of L] (t) {$t$};

        % Cache box (manual corners; no fit/backgrounds/shadows libraries needed)
        % Raise the cache-box top so the legend sits above the parameter nodes without overlap.
        \coordinate (boxTop) at ($(W1.north)+(0,2.2cm)$);
        \coordinate (cacheNW) at ($(Z1.west |- boxTop)+(-0.6,0)$);
        \coordinate (cacheSE) at ($(AL.south east)+(0.6,-0.55)$);
        % Use fill opacity so the cache box doesn't obscure the nodes (drawn after nodes).
        \draw[draw=gray!20, dashed, rounded corners=12pt, fill=gray!5, fill opacity=0.35, draw opacity=1]
            (cacheNW) rectangle (cacheSE);
        \node[anchor=north east, font=\scriptsize\itshape, text=gray!60, inner sep=4pt]
            at (cacheSE) {Forward cache};

        % Legend (top-left inside cache box)
        \node[anchor=north west, font=\scriptsize\sffamily, fill=white, draw=gray!20, rounded corners=3pt, inner sep=5pt]
            at ($(cacheNW)+(0.25,-0.25)$) {%
            \begin{tabular}{@{}l@{\,\,}l@{}}
            \textcolor{gray!55}{\rule{6mm}{1.5pt}} & Forward \\
            \textcolor{cbOrange!85!black}{\rule[2pt]{6mm}{1.5pt}} & Backprop ($\delta$) \\
            \textcolor{cbGreen!75!black}{\rule[2pt]{6mm}{1.5pt}} & Grads ($\nabla$) \\
            \end{tabular}
        };

        % Forward pass
        \draw[fwd] (x) -- (Z1);
        \draw[fwd] (W1) -- (Z1);
        \draw[fwd] (Z1) -- (A1);
        \draw[fwd] (A1) -- (dots);
        \draw[fwd] (dots) -- (ZL);
        \draw[fwd] (WL) -- (ZL);
        \draw[fwd] (ZL) -- (AL);
        \draw[fwd] (AL) -- (L);
        \draw[fwd] (t) -- (L);

        % Backward pass
        \draw[bwd] (L.south) |- ++(0,-1.3) -|
            node[pos=0.25, below, font=\scriptsize] {$\frac{\partial \mathcal{L}}{\partial A^{(L)}}$} (AL.south);
        \draw[bwd] (AL.south) |- ++(0,-1.3) -|
            node[pos=0.25, below, font=\scriptsize] {$\delta^{(L)}$} (ZL.south);
        \draw[bwd] (ZL.south) |- ++(0,-1.3) -|
            node[pos=0.5, below, font=\scriptsize, fill=white, inner sep=3pt] {chain rule via $W^\top, f'$} (Z1.south);
        \node[font=\scriptsize, text=cbOrange!85!black, below=0.1cm of Z1.south, xshift=0.4cm, yshift=-0.5cm] {$\delta^{(1)}$};

        % Gradient extraction (curved to avoid overlaps)
        \draw[grad] (ZL.north) to[bend right=45]
            node[midway, right, font=\tiny, text=cbGreen!75!black, align=left, xshift=2pt]
            {$\nabla_{W^{(L)}}$\\$\nabla_{b^{(L)}}$} (WL.south);
        \draw[grad] (Z1.north) to[bend right=45]
            node[midway, right, font=\tiny, text=cbGreen!75!black, align=left, xshift=2pt]
            {$\nabla_{W^{(1)}}$\\$\nabla_{b^{(1)}}$} (W1.south);
    \end{tikzpicture}%
    }
    % Avoid dense inline math in captions; it wraps poorly in EPUB renderers.
    \caption[Computational graph for backpropagation (reverse-mode AD)]{Computational graph for a feedforward network. Backpropagation is reverse\hyp{}mode AD: the forward sweep caches intermediate values, and the reverse sweep propagates deltas while accumulating weight/bias gradients from those cached values. Use it when debugging backprop implementations and tracing each gradient term.}
    \label{fig:backprop-computational-graph}
\end{figure}


\begin{tcolorbox}[summarybox, title={Debugging and gradient-check checklist}]
\begin{itemize}
    \item \textbf{Overfit a tiny batch:} Ensure the loss can be driven near zero on a handful of samples.
    \item \textbf{Gradient norms:} Track \(\|\nabla W^{(l)}\|\) per layer; look for dead layers (all zeros) or explosions.
    \item \textbf{Finite-difference check:} Compare analytic gradients to numerical finite differences on a tiny network with fixed seeds; relative error should be \(<10^{-6}\).
    \item \textbf{Shape assertions:} Verify that \(Z^{(l)}, A^{(l)},\delta^{(l)}\) have the expected batch shapes and that bias broadcasts correctly.
    \item \textbf{Layerwise sanity:} For a one-layer linear model, backprop gradients should match the closed-form linear-regression gradients.
\end{itemize}
\end{tcolorbox}

\subsection{Backpropagation Algorithm: Brief Numerical Check}
\label{sec:backprop_backpropagation_algorithm_brief_numerical_check}

For a quick sanity check, take a tiny 2--2--1 network with sigmoid output and cross\hyp{}entropy loss. Using
\begin{align*}
W^{(1)}&=\begin{bmatrix}0.5&-0.3\\0.8&0.2\end{bmatrix},\quad b^{(1)}=[0.1,-0.2],\\
W^{(2)}&=\begin{bmatrix}0.7\\-0.4\end{bmatrix},\quad b^{(2)}=0.05,\\
\mathbf{x}&=[0.6,-1.2],\quad t=1,
\end{align*}
the forward pass yields
\[
z^{(1)}=[-0.56,-0.62],\quad
a^{(1)}=[0.3635,0.3498],\quad
z^{(2)}=0.1646,\quad
a^{(2)}=0.5411,
\]
with loss \(\mathcal{L}\approx 0.6142\). The cross\hyp{}entropy output error is \(\delta^{(2)}=a^{(2)}-t=-0.4590\). Backpropagating gives
\[
\delta^{(1)}=[-0.0743,0.0418],\quad
\nabla_{W^{(2)}}=[-0.1669,-0.1605]^\top,\quad
\nabla_{b^{(2)}}=-0.4590,
\]
and
\[
\nabla_{W^{(1)}}=
\begin{bmatrix}
-0.0446&0.0251\\
0.0892&-0.0501
\end{bmatrix},\quad
\nabla_{b^{(1)}}=[-0.0743,0.0418].
\]
Finite-difference checks on the same network match to numerical precision, validating the implementation.

\paragraph{Aside: squared-error loss (alternative)}

The remainder of this subsection sketches the classic squared-error backprop derivation as a separate reminder; it is \emph{not} a continuation of the cross\hyp{}entropy numerical check above.

The error at the output neuron is:
\begin{equation}
    e = y - t,
\label{eq:auto_backprop_b55302fe32}
\end{equation}
and the squared error is:
\begin{equation}
    \mathcal{L}_{\text{SE}} = \frac{1}{2} e^2.
\label{eq:auto_backprop_7c0067d4dd}
\end{equation}

\paragraph{Backward Propagation of Error}

Define the error term \(\delta_j\) for each neuron \( j \) as:
\begin{equation}
    \delta_j = e_j \sigma'(net_j),
\label{eq:auto_backprop_88f14b071c}
\end{equation}
where \( e_j \) is the error at neuron \( j \), and
\[
\sigma'(z) = \sigma(z)(1 - \sigma(z)).
\]
is the derivative of the sigmoid function.

For the output neuron:
\[
\delta_{\text{out}} = (y_{\text{out}} - t) \, y_{\text{out}} (1 - y_{\text{out}}).
\]

For hidden neurons, the error term is computed by backpropagating the weighted sum of the downstream error terms:
\begin{equation}
    \delta_j = y_j (1 - y_j) \sum_{k} w_{kj} \, \delta_k,
\label{eq:auto_backprop_1f3342dea6}
\end{equation}
where the sum is over neurons \( k \) in the next layer and \( w_{kj} \) denotes the weight from neuron \( j \) to neuron \( k \).

\paragraph{Weight Update Rule}

Weights are updated using gradient descent with momentum:
\begin{equation}
    \Delta w_{ij}(n) = -\eta \, \delta_j x_i + \gamma \Delta w_{ij}(n-1),
    \label{eq:weight_update}
\end{equation}
where
\begin{itemize}
    \item \(\eta\) is the learning rate,
    \item \(\gamma\) is the momentum coefficient (typically \(0 \leq \gamma < 1\)),
    \item \(\Delta w_{ij}(n-1)\) is the previous weight change,
    \item \(n\) indexes the update step (e.g., the current training example in stochastic gradient descent).
\end{itemize}
The leading negative sign ensures that the update follows the negative gradient direction because each \(\delta_j\) equals \(\partial \mathcal{L}_{\text{SE}} / \partial z_j\).

The new weight is then:
\[
w_{ij}(n) = w_{ij}(n-1) + \Delta w_{ij}(n).
\]

\paragraph{Interpretation of Learning Rate and Momentum}

\begin{itemize}
    \item The \textbf{learning rate} \(\eta\) controls the step size in the weight update. A small \(\eta\) leads to slow convergence, while a large \(\eta\) can cause oscillations or divergence.
    \item The \textbf{momentum} term \(\gamma\) helps smooth the updates by incorporating a fraction of the previous weight change, reducing oscillations and potentially accelerating convergence.
\end{itemize}

\paragraph{Step-by-Step Example}

\begin{enumerate}
    \item \textbf{Initialization:} Draw weights \( w_{ij}\sim \mathcal{N}(0,\sigma^2)\) with \(\sigma\) set by He/Xavier rules; set biases to zero and \(\Delta w_{ij}(0) = 0\).
    \item \textbf{Feedforward:} Compute \( net_j \) and \( y_j \) for all neurons.
    \item \textbf{Compute output error:} Calculate \( \delta_{out} \).
    \item \textbf{Backpropagate error:} Compute \(\delta_j\) for hidden neurons.
    \item \textbf{Update weights:} Use equation \eqref{eq:weight_update} to update all weights.
    \item \textbf{Repeat:} Iterate over all training patterns until error \( E \) is below threshold or maximum epochs reached.
\end{enumerate}

\begin{tcolorbox}[summarybox, title={Mini\hyp{}batch backprop with explicit regularization}]
\textbf{Inputs:} mini\hyp{}batch \(\{\mathbf{x}_b,\mathbf{t}_b\}_{b=1}^B\), learning rate \(\eta\), L2 coefficient \(\lambda\), dropout keep probability \(q=1-p\).
\begin{enumerate}[leftmargin=*]
    \item \textbf{Forward pass:} propagate activations layer by layer; for each hidden layer draw a dropout mask \(\mathbf{m}\sim \operatorname{Bernoulli}(q)\), apply \(\tilde{\mathbf{a}}=\mathbf{m}\odot \mathbf{a}/q\), and cache \(\mathbf{m}\) for the backward step.
    \item \textbf{Backward pass:} compute \(\nabla_{W^{(\ell)}} \mathcal{L}\) using the cached activations/masks so that dropped units contribute zero gradient.
    \item \textbf{Update block (per layer):}
    \[
        \mathbf{g}_\ell = \frac{1}{B}\nabla_{W^{(\ell)}} \mathcal{L} + \lambda W^{(\ell)}, \qquad
        W^{(\ell)} \leftarrow W^{(\ell)} - \eta\, \mathbf{g}_\ell.
    \]
    Biases skip the weight-decay term. With Adam/SGD+momentum, \(\mathbf{g}_\ell\) replaces the raw gradient inside the optimizer step so L2 regularization and dropout are always enforced explicitly.
\end{enumerate}
\end{tcolorbox}

\paragraph{Remarks}

\begin{itemize}
    \item Monitor the training error over epochs; a plateau may indicate the need to adjust learning rate or introduce regularization.
    \item Shuffle training patterns between epochs when using SGD to avoid cyclic behaviors.
    \item Always track validation error to detect overfitting and decide when to stop training.
\end{itemize}

\subsection{Training Procedure and Epochs in Multi-Layer Perceptrons}
\label{sec:backprop_training_procedure_and_epochs_in_multi_layer_perceptrons}

Recall that during training of a multi-layer perceptron (MLP), we iteratively update the weights based on each training pattern. The process for one epoch can be summarized as follows:

\begin{enumerate}
    \item Present the first input pattern to the network.
    \item Perform a forward pass to compute the output.
    \item Calculate the error between the actual output and the desired output.
    \item Use backpropagation to compute the gradients and update the weights accordingly.
    \item Repeat steps 1--4 for all training patterns.
\end{enumerate}

After completing one epoch (i.e., one full pass through all training patterns), we evaluate the overall error. If the error is greater than a predefined tolerance, we continue training for additional epochs until the error converges below the threshold or a maximum number of epochs is reached.

\paragraph{Remarks:}
\begin{itemize}
    \item The weight updates after each pattern are typically small adjustments aimed at reducing the error.
    \item The initial weights strongly influence the convergence behavior and final solution.
    \item This iterative process is computationally intensive but essential for learning complex mappings.
\end{itemize}

\subsection{Role and Design of Hidden Layers}
\label{sec:backprop_role_and_design_of_hidden_layers}

In an MLP, the architecture consists of an input layer, one or more hidden layers, and an output layer. The hidden layers are crucial because they enable the network to learn nonlinear mappings.

\paragraph{Key Questions Regarding Hidden Layers:}
\begin{itemize}
    \item \textbf{How many hidden layers should be used?} There is no fixed rule; it depends on the complexity of the problem.
    \item \textbf{How many neurons per hidden layer?} This choice affects the network's capacity and generalization ability.
    \item \textbf{What activation functions to use in each layer?} Different layers can use different activation functions, such as sigmoid, ReLU, or tanh.
\end{itemize}

\paragraph{Design Considerations:}
\begin{itemize}
    \item \textbf{Number of neurons:} More neurons increase the capacity to learn complex functions but also increase the risk of overfitting and computational cost.
    \item \textbf{Number of layers:} Deeper networks can represent more complex functions but are harder to train.
    \item \textbf{Activation functions:} Choice affects gradient flow and convergence.
\end{itemize}

Ultimately, these design choices are made by the practitioner based on experimentation, domain knowledge, and validation performance.

\paragraph{Trade-offs:}
\begin{itemize}
    \item \textbf{Too many neurons/layers:} Requires more training data to avoid overfitting; increases computational burden.
    \item \textbf{Too few neurons/layers:} Limits the network's ability to approximate complex functions.
\end{itemize}

\subsection{Case Study: Learning the Function \texorpdfstring{\( y = x \sin x \)}{y = x sin x}}
\label{sec:backprop_case_study_learning_the_function_y_x_x_y_x_sin_x}

Consider the problem of training an MLP to approximate the function
\[
y = x \sin x.
\]

\paragraph{Setup:}
\begin{itemize}
    \item Generate a dataset of input-output pairs \(\{(x_i, y_i)\}\) where \(y_i = x_i \sin x_i\).
    \item Use this dataset to train an MLP regressor.
    \item Evaluate the network's ability to generalize by testing on inputs not seen during training.
\end{itemize}

\paragraph{Questions to Explore:}
\begin{itemize}
    \item How many hidden layers and neurons per layer are needed to approximate this nonlinear function well?
    \item What activation functions yield better performance?
    \item How does the size of the training set affect generalization?
\end{itemize}

\paragraph{Remarks:}
\begin{itemize}
    \item This is a regression problem, not a classification problem.
    \item The function is nonlinear and periodic, which challenges the network's approximation capabilities.
    \item Experimentation with different architectures and hyperparameters is essential.
\end{itemize}

As a representative example (illustrative; depends on initialization and the training recipe), a two-hidden-layer MLP with widths $[64, 32]$, ReLU activations, Adam optimization, and early stopping on a validation split can achieve mean absolute error on the order of $10^{-3}$ on held-out samples when trained on 2{,}000 uniformly spaced points in $[-3\pi, 3\pi]$.

\subsection{Applications of Multi-Layer Perceptrons}
\label{sec:backprop_applications_of_multi_layer_perceptrons}

Multi-layer perceptrons have found widespread applications across various domains due to their ability to approximate complex nonlinear functions. Some notable applications include:

\begin{itemize}
    \item \textbf{Signal processing:} Noise reduction, filtering, and feature extraction.
    \item \textbf{Weather forecasting:} Modeling complex atmospheric patterns.
    \item \textbf{Data compression:} Dimensionality reduction and encoding.
    \item \textbf{Pattern recognition:} Handwriting recognition, face detection.
    \item \textbf{Financial market prediction:} Time series forecasting and anomaly detection.
    \item \textbf{Image recognition:} Object detection and classification.
    \item \textbf{Voice recognition:} Speech-to-text and speaker identification.
\end{itemize}

\paragraph{Summary:} MLPs are versatile and powerful tools that serve as foundational building blocks in many machine learning systems.

\subsection{Limitations of Multi-Layer Perceptrons}
\label{sec:backprop_limitations_of_multi_layer_perceptrons}

Despite their versatility, MLPs have several limitations that practitioners must be aware of:

\begin{itemize}
\item \textbf{Convergence to local minima:} Due to the non-convex nature of the loss surface, training may converge to different local minima depending on the initial weights.
    \item \textbf{Sensitivity to initialization:} Different random initializations can lead to significantly different outcomes.
    \item \textbf{Hyperparameter tuning:} Learning rates, momentum, and regularization require careful tuning for stable convergence.
\end{itemize}

\subsection{Conclusion of Multi-Layer Perceptron Derivations}
\label{sec:backprop_conclusion_of_multi_layer_perceptron_derivations}

In this final segment of the chapter, we complete the derivations and discussions related to the multi-layer perceptron (MLP) and its learning algorithm, backpropagation.

Recall that the MLP consists of multiple layers of neurons, each performing an affine transformation followed by a nonlinear activation. The key to training the MLP is to minimize a loss \(\mathcal{L}\) defined over outputs and targets.

\paragraph{Backpropagation Algorithm Recap}

The backpropagation algorithm efficiently computes the gradient of the loss function with respect to all network parameters by applying the chain rule of calculus through the network layers. For a network with \( L \) layers, denote by:
\[
Z^{(l)} = A^{(l-1)}W^{(l)} + \mathbf{1}(b^{(l)})^\top,\quad A^{(l)} = \phi^{(l)}(Z^{(l)}),
\]
where \(W^{(l)}\) and \(b^{(l)}\) are the weights and biases of layer \( l \), \(A^{(l-1)}\) is the previous layer activation (rows are samples), and \(\phi^{(l)}\) is the activation function.

The error term at layer \( l \) is defined as:
\[
\delta^{(l)} = \frac{\partial \mathcal{L}}{\partial Z^{(l)}}.
\]

Using the chain rule, the error terms propagate backward as:
\begin{align}
\delta^{(L)} &= \nabla_{A^{(L)}} \mathcal{L} \odot \phi^{(L)\prime}(Z^{(L)}), \\
\delta^{(l-1)} &= \left(\delta^{(l)}(W^{(l)})^\top\right) \odot \phi^{(l-1)\prime}(Z^{(l-1)}), \quad l = L, \ldots, 2,
    \label{eq:auto:lecture_4_part_i:2}
\end{align}
where \(\odot\) denotes element-wise multiplication and \(\phi^{(l)\prime}\) is the derivative of the activation function at layer \( l \).

The gradients of the loss with respect to the parameters are then:
\begin{align}
\frac{\partial \mathcal{L}}{\partial W^{(l)}} &= (A^{(l-1)})^\top \delta^{(l)}, \label{eq:grad_W}\\
\frac{\partial \mathcal{L}}{\partial b^{(l)}} &= \mathbf{1}^\top \delta^{(l)}. \label{eq:grad_b}
\end{align}

These gradients are used in gradient-based optimization methods (e.g., stochastic gradient descent) to update the parameters and minimize the loss. \Cref{fig:lec4_backprop_flow} complements the algebra by showing how cached activations (blue) line up with the backward error signals (orange) in a simple two-layer network.

\begin{figure}[h]
    \centering
    \ifdefined\HCode
    % EPUB/HTML: use a simplified version of the original "wide" diagram, but with
    % fewer colors and without tiny per-edge weight labels (which rasterize poorly).
    \begin{tikzpicture}[
        >=Stealth,
        font=\small\sffamily,
        node/.style={circle, draw=black!50, line width=0.9pt, minimum size=12mm, fill=white},
        fwd/.style={->, line width=1.4pt, draw=cbBlue!80!black},
        bwd/.style={->, line width=1.4pt, draw=black!55, dashed},
        lbl/.style={font=\footnotesize\sffamily, text=black!70},
        background rectangle/.style={fill=white}, show background rectangle
    ]

        % Forward nodes (inputs -> hidden -> output)
        % Shift right to leave a clean legend margin on the left.
        \node[node] (x1) at (0.8,0.9) {$x_1$};
        \node[node] (x2) at (0.8,-0.9) {$x_2$};
        \node[node] (h1) at (2.6,0.9) {$a_1^{(1)}$};
        \node[node] (h2) at (2.6,-0.9) {$a_2^{(1)}$};
        \node[node] (y)  at (5.2,0) {$a^{(2)}$};

        % Forward connections (no per-edge weights in EPUB)
        \draw[fwd] (x1) -- (h1);
        \draw[fwd] (x1) -- (h2);
        \draw[fwd] (x2) -- (h1);
        \draw[fwd] (x2) -- (h2);
        \draw[fwd] (h1) -- (y);
        \draw[fwd] (h2) -- (y);

        % Backward/error signals (labels placed away from nodes to avoid overlap)
        \node[lbl, anchor=west] (d2) at ($(y)+(0.9,1.05)$) {$\boldsymbol{\delta}^{(2)}$};
        \node[lbl, anchor=west] (d1) at ($(y)+(0.9,-1.3)$) {$\boldsymbol{\delta}^{(1)}_1$};
        \node[lbl, anchor=west] (d0) at ($(y)+(0.9,-2.1)$) {$\boldsymbol{\delta}^{(1)}_2$};

        \draw[bwd] ($(y)+(0,0.55)$) -- (d2);
        \draw[bwd] ($(y)+(-0.1,-0.2)$).. controls (5.0,-0.85) and (4.6,-1.1).. (d1);
        \draw[bwd] ($(y)+(-0.1,-0.2)$).. controls (5.0,-1.25) and (4.6,-1.65).. (d0);

        % No in-figure legend for EPUB: small labels tend to overlap after rasterization.
        % The caption explains the color/linestyle mapping instead.
    \end{tikzpicture}
    \else
    \begin{tikzpicture}[
        >=Stealth,
        font=\small\sffamily,
        node distance=1.55cm and 2.0cm,
        var/.style={circle, draw=gray!60, line width=0.7pt, minimum size=10mm, fill=cbBlue!12, inner sep=0pt},
        hnode/.style={circle, draw=gray!60, line width=0.7pt, minimum size=10mm, fill=cbGreen!12, inner sep=0pt},
        outnode/.style={circle, draw=gray!60, line width=0.7pt, minimum size=10.5mm, fill=cbPink!15, inner sep=0pt},
        fwd/.style={->, line width=1.0pt, draw=cbBlue!80!black},
        bwd/.style={->, line width=1.0pt, draw=cbOrange!85!black, dashed},
        wlab/.style={font=\scriptsize, text=cbBlue!80!black},
        dlab/.style={font=\scriptsize, text=cbOrange!85!black}
    ]
        \node[var] (x1) {\(x_1\)};
        \node[var, below=0.9cm of x1] (x2) {\(x_2\)};
        \node[hnode, right=2.1cm of x1] (h1) {\(a_1^{(1)}\)};
        \node[hnode, below=0.9cm of h1] (h2) {\(a_2^{(1)}\)};
        \node[outnode, right=2.2cm of h1] (y) {\(a^{(2)}\)};

        \draw[fwd] (x1) -- node[above, sloped, wlab]{\(w_{11}^{(1)}\)} (h1);
        \draw[fwd] (x1) -- node[above, sloped, wlab]{\(w_{21}^{(1)}\)} (h2);
        \draw[fwd] (x2) -- node[below, sloped, wlab]{\(w_{12}^{(1)}\)} (h1);
        \draw[fwd] (x2) -- node[below, sloped, wlab]{\(w_{22}^{(1)}\)} (h2);
        \draw[fwd] (h1) -- node[above, wlab]{\(w_{1}^{(2)}\)} (y);
        \draw[fwd] (h2) -- node[below, wlab]{\(w_{2}^{(2)}\)} (y);

        \draw[bwd] (y) -- ++(0,1.2) node[right, xshift=0.1cm, dlab]{\(\delta^{(2)}\)};
        \draw[bwd] (h1) -- ++(0,-1.2) node[right, xshift=0.1cm, dlab]{\(\delta_1^{(1)}\)};
        \draw[bwd] (h2) -- ++(0,-1.2) node[right, xshift=0.1cm, dlab]{\(\delta_2^{(1)}\)};

        % Keep legend labels *inside* the diagram bounds; otherwise rasterizers can clip them.
        \path (current bounding box.north west) ++(2mm,-2mm)
            node[anchor=north west, font=\scriptsize, text=cbBlue!80!black] {forward};
        \path (current bounding box.north west) ++(2mm,-7mm)
            node[anchor=north west, font=\scriptsize, text=cbOrange!85!black] {backward};
    \end{tikzpicture}
    \fi
    % Avoid inline math in captions; it wraps poorly in some EPUB renderers.
    \caption{Forward (blue) and backward (orange) flows for a two-layer MLP. Cached activations and layerwise deltas travel along these arrows; backward signals use next-layer weights and activation derivatives. Use it when implementing backprop to confirm what to cache and where gradients flow.}
    \label{fig:lec4_backprop_flow}
\end{figure}


\paragraph{Example Execution}

An example was provided illustrating the forward pass computation of activations and the backward pass calculation of gradients for a simple MLP with one hidden layer. This example concretely demonstrated how the chain rule is applied layer-by-layer and how the error signals are propagated backward.

\paragraph{Remarks on Convergence and Practical Considerations}

While the backpropagation algorithm provides the exact gradients for the MLP, practical training involves additional considerations such as:
\begin{itemize}
    \item Initialization of weights to avoid vanishing or exploding gradients.
    \item Choice of activation functions (e.g., ReLU, sigmoid, tanh) affecting gradient flow.
    \item Regularization techniques (dropout, weight decay) to prevent overfitting.
    \item Optimization algorithms (momentum, Adam) to accelerate convergence.
\end{itemize}

These topics will be explored in subsequent chapters; for now, we compare the canonical activation choices in one place.

\paragraph{Comparing canonical nonlinearities}

With the full MLP and backpropagation machinery in place, it is useful to compare the most common nonlinearities side-by-side. \Cref{fig:lec4-activations} overlays the step, sigmoid, tanh, and ReLU curves so the saturation regions and derivative behavior are visually apparent before we move on to deeper architectures.

\begin{figure}[t]
    \centering
\begin{tikzpicture}[background rectangle/.style={fill=white}, show background rectangle]
        \begin{groupplot}[
            group style={group size=2 by 2, horizontal sep=1.2cm, vertical sep=1.0cm},
            width=0.42\linewidth,
            height=0.30\linewidth,
            axis lines=middle,
            xmin=-3, xmax=3,
            samples=200,
            tick label style={font=\scriptsize},
            label style={font=\scriptsize},
            title style={font=\scriptsize, align=center},
            axis background/.style={fill=white},
        ]
        \nextgroupplot[
            title={Step},
            ymin=-0.1, ymax=1.1,
            ytick={0,1},
        ]
            \addplot[cbBlue, thick, domain=-3:0] {0};
            \addplot[cbBlue, thick, domain=0:3] {1};
            \addplot[cbBlue, dashed, domain=-3:3] {0};
        \nextgroupplot[
            title={Sigmoid},
            xmin=-4, xmax=4,
            ymin=-0.1, ymax=1.1,
            ytick={0,0.5,1},
        ]
            \addplot[cbOrange, thick, domain=-4:4] {1/(1+exp(-x))};
            \addplot[cbOrange, dashed, domain=-4:4] {1/(1+exp(-x))*(1-1/(1+exp(-x)))};
        \nextgroupplot[
            title={tanh},
            ymin=-1.1, ymax=1.1,
            ytick={-1,0,1},
        ]
            \addplot[cbGreen, thick, domain=-3:3] {tanh(x)};
            \addplot[cbGreen, dashed, domain=-3:3] {1 - tanh(x)^2};
        \nextgroupplot[
            title={ReLU},
            ymin=-0.2, ymax=2.2,
            ytick={0,1,2},
        ]
            \addplot[cbPink, thick, domain=-3:0] {0};
            \addplot[cbPink, thick, domain=0:3] {x};
            \addplot[cbPink, dashed, domain=-3:0] {0};
            \addplot[cbPink, dashed, domain=0:3] {1};
        \end{groupplot}
    \ensuretikzbackgroundlayers
\end{tikzpicture}
    % Avoid inline math in captions; it wraps poorly in some EPUB renderers.
    \caption{Canonical activation functions on a common axis. Solid curves show the activation; dashed curves show its derivative. Use it when choosing an activation and checking whether its derivative will saturate or die out.}
    \label{fig:lec4-activations}
\end{figure}


For reference, \(\sigma'(z)=\sigma(z)\bigl(1-\sigma(z)\bigr)\), \(\tanh'(z)=1-\tanh^2(z)\), and the ReLU derivative is \(0\) for negative inputs and \(1\) for positive inputs (take \(0\) at the origin).

\paragraph{Trade-offs}
While some activation functions are inspired by biological neurons, others are chosen for mathematical convenience and training efficiency. Sigmoid and tanh saturate at large magnitude inputs, which slows gradients in deep networks. ReLU avoids saturation on the positive side but can produce ``dying ReLUs'' when biases push units negative and the gradients become zero; if many units stall, use He initialization, reduce the learning rate, or swap to a leaky ReLU with a small negative slope (e.g., 0.01).
This closes the core backpropagation story for MLPs. Next we summarize practical stability considerations and the key takeaways that guide real training.

\clearpage

    \begin{tcolorbox}[summarybox, title={Key takeaways}]
    \textbf{Minimum viable mastery}
    \begin{itemize}
        \item MLP training relies on stable optimization: proper initialization, learning\hyp{}rate schedules, and normalization help.
        \item Regularization (weight decay, dropout) reduces overfitting; validation curves guide early stopping.
        \item Despite power, MLPs remain sensitive to hyperparameters, so debugging and audits matter.
    \end{itemize}
    \medskip
    \textbf{Common pitfalls}
    \begin{itemize}
        \item Silent shape mistakes: mismatched dimensions can yield plausible but wrong gradients.
        \item Learning-rate pathologies: too large diverges, too small stalls; diagnose with train/val curves.
        \item Overfitting by optimization: improving training loss without validation gains is a signal to stop or regularize.
    \end{itemize}
    \end{tcolorbox}

    \begin{tcolorbox}[summarybox, title={Practical early stopping and checkpointing}]
    \begin{itemize}
        \item Maintain a validation split distinct from the training mini\hyp{}batches. After each epoch, record the validation loss \(L_{\text{val}}^{(e)}\).
        \item Stop training when \(L_{\text{val}}\) has not improved for \(k\) consecutive epochs (typical patience \(k \in [5,10]\)). Optionally require a minimum relative improvement (e.g., \(0.1\%\)) to smooth noise.
    \item Always checkpoint the parameters that achieved the best validation score and restore them before testing; averaging the last \(m\) checkpoints (``Polyak averaging'') can further stabilize performance.
\end{itemize}
\end{tcolorbox}

\begin{tcolorbox}[summarybox, title={Derivation closure: implement, cache, fail-fast}]
\begin{itemize}
    \item \textbf{Implement:} encode one canonical forward signature \((A^{(l-1)}, W^{(l)}, b^{(l)}) \mapsto (Z^{(l)}, A^{(l)})\), then reuse it for every layer.
    \item \textbf{Cache:} store \((A^{(l-1)}, Z^{(l)})\) per layer so backward passes only apply local Jacobians and matrix multiplications.
    \item \textbf{Fail-fast checks:} do a numerical gradient check on one mini-batch, then verify gradient norms and validation curves before long runs.
    \item \textbf{Reproducibility:} log seed, optimizer, schedule, and stopping rule with each run; the reporting template in \Cref{app:repro_standards} keeps comparisons fair.
\end{itemize}
\end{tcolorbox}

\begin{tcolorbox}[summarybox, title={Exercises and lab ideas}]
\begin{itemize}
    \item Implement a minimal example from this chapter and visualize intermediate quantities (plots or diagnostics) to match the pseudocode.
    \item Stress-test a key hyperparameter or design choice discussed here and report the effect on validation performance or stability.
        \item Re-derive one core equation or update rule by hand and check it numerically against your implementation.
    \end{itemize}
    \end{tcolorbox}
    \medskip
    \noindent\textbf{If you are skipping ahead.} Keep one practical habit: do a gradient check on a tiny case before scaling up. That discipline mirrors residual checks in \Cref{chap:symbolic} and prevents most wasted training runs in later deep chapters.

    \medskip
    \paragraph{Where we head next.} \Cref{chap:rbf} introduces radial basis function networks, an alternative nonlinear route where hidden features are mostly fixed and only the output layer is solved linearly. This provides a clean contrast to end-to-end backpropagation.

    \nocite{Rumelhart1986, Haykin2009}

% Chapter 8
\section{Radial Basis Function Networks (RBFNs)}\label{chap:rbf}

Building on the multilayer perceptron (MLP) architecture (\Cref{chap:mlp}) and its training machinery (\Cref{chap:backprop}), this chapter introduces radial basis function networks (RBFNs): three\hyp{}layer models with fixed nonlinear bases and a linear readout. \Cref{fig:roadmap} places this as the kernel/prototype branch alongside the MLP path.

\begin{tcolorbox}[summarybox, title={Learning Outcomes}]
\begin{itemize}
    \item Explain the architecture and training stages of RBF networks (center selection, width tuning, linear solve).
    \item Relate RBF solutions to linear estimators (normal equations, pseudoinverse, Wiener filtering) and know when ridge regularization is needed.
    \item Compare RBFNs to kernelized methods and other nonlinear classifiers to choose appropriate models in practice.
\end{itemize}
\end{tcolorbox}

\begin{tcolorbox}[summarybox, title={Design motif}]
Make the nonlinearity explicit: use a fixed (or lightly tuned) basis expansion in the hidden layer, then learn the output weights with linear-algebra tools.
\end{tcolorbox}

\subsection{Overview and Motivation}
\label{sec:rbf_overview_and_motivation}

\paragraph{How to read this chapter.}
We start with the three-layer picture and the basis-function intuition, then write the model in matrix form and show how training becomes a regularized least-squares solve. Near the end we connect the finite-basis view to the kernel viewpoint. The Wiener-filter box is optional context if you want the signal-processing parallel.

Unlike MLPs, which learn weights in every layer, an RBFN separates the job into two parts. The hidden layer provides a fixed nonlinear feature map (set by centers and widths). The output layer then learns a linear combination of those features. That split is the main story in this chapter: choose the basis, then solve for the readout.

The key idea is that ``nonlinear'' can come from the representation rather than the readout. Radial basis functions act like localized, kernel-style features; once you lift inputs into that feature space, the output layer is still linear, and the weights are responsible for finding the separating border. This is closely related to the kernel trick used in SVMs; \Cref{app:kernels} collects the classical kernel viewpoint.

\Cref{chap:supervised} frames this as a bias--variance tuning problem (choose capacity, regularization, and diagnostics via learning curves). Kernel methods such as kernel ridge regression and SVMs interpret the same trade-off through an RBF kernel matrix; here we keep the bases explicit, then connect to the dual/kernel view later in the chapter.

\paragraph{Running example: XOR as a representation problem.}
XOR is the smallest reminder that a single line in the input space is not always enough. The point of RBF networks is not that the output layer becomes complicated; it stays linear. The complication is pushed into the feature map: the hidden units apply localized, kernel-like transforms, and the output weights are responsible for finding the separating border in that transformed space. We will return to XOR twice: first to see the feature map, then to see how the linear solve chooses the readout weights.

\paragraph{Centers from clustering (a practical default).}
The hidden layer of an RBF network is easiest to understand when its centers are viewed as K-means prototypes: pick a coverage of the input space that reflects the data distribution, assign widths accordingly, and let the output layer learn the linear weights on top of those prototypes. Unsupervised clustering up front makes the later supervised solve far more stable.

\subsection{Architecture of RBFNs}
\label{sec:rbf_architecture_of_rbfns}

The RBFN consists of three layers:

\begin{tcolorbox}[summarybox, title={Notation and shapes}]
We denote each basis response by \(\varphi_i(\mathbf{x})\); stacking them yields \(\mathbf{G}(\mathbf{x})\in\mathbb{R}^M\) with entries \(G_i(\mathbf{x})=\varphi_i(\mathbf{x})\). For a dataset of \(N\) samples, the corresponding design matrix \(\boldsymbol{\Phi}\in\mathbb{R}^{N\times M}\) stacks one transformed sample per row, with entries \(\boldsymbol{\Phi}_{ji}=\varphi_i(\mathbf{x}_j)=G_i(\mathbf{x}_j)\). This matches the design-matrix convention used in \Cref{chap:supervised}.
\end{tcolorbox}


\begin{figure}[t]
    \centering
    \begin{tikzpicture}[>=stealth, node distance=1.6cm]
        \tikzset{
            inputnode/.style={circle, draw, fill=gray!10, minimum size=0.8cm},
            rbf/.style={circle, draw, fill=cbBlue!15, minimum size=1cm},
            outputnode/.style={circle, draw, fill=cbOrange!20, minimum size=0.9cm}
        }
        % inputs
        \node[inputnode] (x1) {$x_1$};
        \node[inputnode, below=0.8cm of x1] (x2) {$x_2$};
        \node[below=0.6cm of x2] (dots) {$\vdots$};
        \node[inputnode, below=0.6cm of dots] (xd) {$x_d$};
        % rbf layer
        \node[rbf, right=2.0cm of x1] (h1) {$\varphi_1$};
        \node[rbf, right=2.0cm of x2] (h2) {$\varphi_2$};
        \node[right=2.0cm of dots] (dots2) {$\vdots$};
        \node[rbf, right=2.0cm of xd] (hm) {$\varphi_M$};
        % output
        \node[outputnode, right=2.5cm of h2] (y) {$\hat{y}$};
        % connections input to rbf
        \draw[->, gray!70] (x1) -- (h1);
        \draw[->, gray!70] (x1) -- (h2);
        \draw[->, gray!70] (x2) -- (h1);
        \draw[->, gray!70] (x2) -- (h2);
        \draw[->, gray!70] (xd) -- (hm);
        % rbf to output
        \draw[->, thick, cbOrange] (h1) -- node[above, sloped, font=\scriptsize] {$w_1$} (y);
        \draw[->, thick, cbOrange] (h2) -- node[above, sloped, font=\scriptsize] {$w_2$} (y);
        \draw[->, thick, cbOrange] (hm) -- node[below, sloped, font=\scriptsize] {$w_M$} (y);
        % bias
        \node[below=1cm of y, font=\scriptsize] (bias) {bias $b$};
        \draw[->, cbOrange, dashed] (bias) -- (y);
        % notes
        \node[align=left, font=\scriptsize, above=0.2cm of h1] {Centers/widths\\\((\boldsymbol{\mu}_i,\sigma_i)\) set by\\k-means or heuristics};
        \node[align=left, font=\scriptsize, right=0.3cm of y] {Linear weights \(w_i\)\\learned from data};
    \end{tikzpicture}
    % Avoid inline math in captions; it wraps poorly in some EPUB renderers.
\caption{RBFN architecture. Inputs feed fixed radial units parameterized by centers and widths; a linear readout with weights and bias is trained by a regression or classification loss. Only the output weights are typically learned, while centers and widths come from clustering or spacing heuristics.}
    \label{fig:rbf_architecture_weights}
\end{figure}
\FloatBarrier

\Cref{fig:rbf_architecture_weights} highlights the split between fixed radial features and a trained linear readout.

\paragraph{A picture to keep in mind}
Once you have the architecture in mind, it helps to visualize what the hidden layer \emph{does}. In one dimension, you can literally draw the bases as overlapping Gaussian bumps; the model output is a weighted sum of those bumps. \Cref{fig:rbf_gaussian_bumps} is the mental model we will reuse as we introduce centers, widths, and the final linear solve.

\begin{figure}[t]
    \centering
    \begin{tikzpicture}
        \begin{axis}[
            width=0.68\linewidth,
            height=0.34\linewidth,
            xlabel={$x$},
            ylabel={Activation},
            xmin=-4, xmax=4,
            ymin=-0.2, ymax=1.4,
            legend style={at={(0.02,0.98)}, anchor=north west}
        ]
            \addplot[cbBlue, dashed, mark=*, mark repeat=25, mark options={fill=cbBlue}, domain=-4:4, samples=200]{exp(-((x+2)^2))};
            \addlegendentry{$\varphi_1$}
            \addplot[cbOrange, dashed, mark=square*, mark repeat=25, mark options={fill=cbOrange}, domain=-4:4, samples=200]{0.8*exp(-((x-0.5)^2)/0.5)};
            \addlegendentry{$\varphi_2$}
            \addplot[cbGreen, dashed, mark=triangle*, mark repeat=25, mark options={fill=cbGreen}, domain=-4:4, samples=200]{0.9*exp(-((x-2.2)^2)/0.7)};
            \addlegendentry{$\varphi_3$}
            \addplot[cbPink, thick, mark=diamond*, mark repeat=20, mark options={fill=cbPink}, domain=-4:4, samples=200]{exp(-((x+2)^2)) + 0.8*exp(-((x-0.5)^2)/0.5) + 0.9*exp(-((x-2.2)^2)/0.7)};
            \addlegendentry{$\sum_j w_j \varphi_j(x)$}
        \end{axis}
    \end{tikzpicture}
    \caption{Localized Gaussian basis functions (dashed) and their weighted sum (solid). Overlapping bumps allow RBF networks to interpolate complex signals smoothly.}
    \label{fig:rbf_gaussian_bumps}
\end{figure}
\FloatBarrier


\Cref{fig:rbf_centres} compares center-placement strategies used during initialization.

\begin{figure}[t]
    \centering
    \begin{tikzpicture}[
        font=\small\sffamily,
        panel/.style={draw=gray!35, fill=gray!8, rounded corners=2pt},
        kdot/.style={circle, fill=cbBlue!80!black, inner sep=1.3pt},
        rdot/.style={circle, fill=cbOrange!85!black, inner sep=1.3pt},
        khalo/.style={draw=cbBlue!60, fill=cbBlue!15, opacity=0.45},
        rhalo/.style={draw=cbOrange!60, fill=cbOrange!15, opacity=0.45}
    ]
        % --- Top panel: K-means ---
        \node[panel, minimum width=7.2cm, minimum height=3.0cm] (p1) at (0,0) {};
        \node[anchor=north west, font=\scriptsize] at ([xshift=5pt, yshift=-5pt]p1.north west) {(a) K-means centers};
        \begin{scope}[shift={(0,0)}]
            \foreach \x/\y in {-2/0,0/0,2/0}{
                \draw[khalo] (\x,\y) circle (0.8);
                \node[kdot] at (\x,\y) {};
            }
        \end{scope}

        % --- Bottom panel: random ---
        \node[panel, minimum width=7.2cm, minimum height=3.0cm] (p2) at (0,-4.0) {};
        \node[anchor=north west, font=\scriptsize] at ([xshift=5pt, yshift=-5pt]p2.north west) {(b) Random centers};
        \begin{scope}[shift={(0,-4.0)}]
            \foreach \x/\y in {-2.1/0.6,-0.6/-0.4,1.5/0.2,2.4/-0.7}{
                \draw[rhalo] (\x,\y) circle (0.8);
                \node[rdot] at (\x,\y) {};
            }
        \end{scope}
    \end{tikzpicture}
    % Avoid inline math in captions; it wraps poorly in some EPUB renderers.
\caption{Center placement and overlap. Top: K-means prototypes roughly tile the data manifold, giving even overlap; bottom: random centers can leave gaps or excessive overlap, influencing the width (sigma) choice and conditioning.}
    \label{fig:rbf_centres}
\end{figure}
\FloatBarrier


Later in the chapter we contrast this finite-basis (``primal'') view with the kernel (``dual'') view and show how Nystr\"om-style approximations fit into the same story.

\Cref{fig:rbf_architecture_weights} plus the overview above summarize the three-layer flow. Next we write the same story algebraically, so the training step becomes a linear solve with shapes you can check.
The key distinction is that the input-to-hidden layer connections do not have trainable weights; instead, the hidden layer units themselves perform nonlinear transformations of the input.

\subsubsection{Mathematical Formulation}
\label{sec:rbf_mathematical_formulation_sub}

Let the input vector be \(\mathbf{x} \in \mathbb{R}^n\). The hidden layer computes the vector
\[
\mathbf{G}(\mathbf{x}) = \begin{bmatrix}
G_1(\mathbf{x}) \\
G_2(\mathbf{x}) \\
\vdots \\
G_M(\mathbf{x})
\end{bmatrix} \in \mathbb{R}^M.
\]
where each \(G_i(\mathbf{x})\) is a radial basis function centered at some point \(\mathbf{c}_i \in \mathbb{R}^n\); stacking all \(M\) responses into \(\mathbf{G}(\mathbf{x})\) makes it clear that \(M\) controls the dimensionality of the transformed feature space.

The output layer then computes
\begin{align}
\mathbf{y}(\mathbf{x}) &= \mathbf{W}^\top \mathbf{G}(\mathbf{x}) + \mathbf{b}, \label{eq:rbfn_output}
\end{align}
where \(\mathbf{W} \in \mathbb{R}^{M \times K}\) is the weight matrix connecting the hidden layer to the output layer, and \(\mathbf{b} \in \mathbb{R}^K\) is a bias vector.

\paragraph{Interpretation:} The hidden layer maps the input \(\mathbf{x}\) into a new feature space via nonlinear functions \(G_i\), and the output layer performs a linear combination of these features to produce the final output.

\subsection{Radial Basis Functions}
\label{sec:rbf_radial_basis_functions}

The functions \(G_i(\mathbf{x})\) are typically chosen to be radially symmetric functions centered at \(\mathbf{c}_i\), such as Gaussian functions:
\begin{align}
G_i(\mathbf{x}) &= \varphi\left(\|\mathbf{x} - \mathbf{c}_i\|\right) = \exp\left(-\frac{\|\mathbf{x} - \mathbf{c}_i\|^2}{2\sigma_i^2}\right), \label{eq:gaussian_rbf}
\end{align}
where \(\sigma_i\) is the width (spread) parameter controlling the receptive field of the \(i\)-th basis function.

Other choices of radial basis functions are possible, but the Gaussian is the most common due to its smoothness and locality properties.

\paragraph{Normalized RBFs.} Some texts normalize the hidden responses as \(\tilde{G}_i(\mathbf{x}) = G_i(\mathbf{x})/\sum_j G_j(\mathbf{x})\) to smooth predictions when center density is uneven; the linear readout then uses \(\tilde{\mathbf{G}}(\mathbf{x})\) in place of \(\mathbf{G}(\mathbf{x})\).

\subsection{Key Properties and Advantages}
\label{sec:rbf_key_properties_and_advantages}

\begin{itemize}
    \item \textbf{Nonlinear transformation without weights:} The input-to-hidden layer mapping is fixed by the choice of centers \(\{\mathbf{c}_i\}\) and widths \(\{\sigma_i\}\), not by trainable weights.
    \item \textbf{Linear output layer:} Training reduces to finding the optimal weights \(\mathbf{W}\) in a linear model, which can be done efficiently using linear regression techniques.
\item \textbf{Universal approximation:} With sufficiently many radial basis functions placed densely over a compact domain (and with nondegenerate widths), RBFNs can approximate any continuous function to arbitrary accuracy \citep{ParkSandberg1991,Micchelli1986}.
\item \textbf{Interpretability:} Each hidden unit corresponds to a localized region in input space, making it easier to understand which prototypes influence a given prediction.
\end{itemize}
\paragraph{Curse of dimensionality.} In high dimensions Euclidean distances concentrate, so widths and center counts must scale with dimension; kernel ridge regression or learned features (e.g., CNNs) often dominate for images/audio.

% Chapter 8 (continued)

\subsection{Transforming Nonlinearly Separable Data into Linearly Separable Space}
\label{sec:rbf_transforming_nonlinearly_separable_data_into_linearly_separable_space}

Some datasets are not linearly separable in the original input space. A nonlinear feature map can move the same points into a space where a single linear boundary is enough.

Consider a nonlinear transformation function \( g(\cdot) \) applied to the input vector \( \mathbf{x} \in \mathbb{R}^n \), producing a transformed vector \( \mathbf{g}(\mathbf{x}) \in \mathbb{R}^m \). The goal is to find a weight vector \( \mathbf{w} \in \mathbb{R}^m \) such that the linear combination \( \mathbf{w}^\top \mathbf{g}(\mathbf{x}) \) separates the classes.

\paragraph{Example setup (XOR).}
\begin{itemize}
    \item Inputs: \(\mathbf{x}\in\{0,1\}^2\) with the four corners \((0,0),(0,1),(1,0),(1,1)\).
    \item Two radial units centered at \(\mathbf{v}_1=(0,0)^\top\) and \(\mathbf{v}_2=(1,1)^\top\).
    \item Feature map: \(\mathbf{g}(\mathbf{x})=[g_1(\mathbf{x}),g_2(\mathbf{x})]^\top\) with Gaussian \(g_i\).
    \item Linear readout: \(y = \mathbf{w}^\top \mathbf{g}(\mathbf{x})\).
\end{itemize}

\paragraph{Assumptions:}
- For simplicity, set \(\sigma^2 = 1\) (so \(2\sigma^2 = 2\)) in the Gaussian kernel activation function.
- Assume \(\mathbf{v}_1 = (0,0)^\top\) and \(\mathbf{v}_2 = (1,1)^\top\).
- The activation function is Gaussian radial basis function (RBF):
\[
g_i(\mathbf{x}) = \exp\left(-\frac{\|\mathbf{x} - \mathbf{v}_i\|^2}{2\sigma^2}\right).
\]

\paragraph{Transformation Results:}
Applying the transformation to the inputs yields new points in the \(g_1\)-\(g_2\) space. For example, the input \(\mathbf{x}=(0,0)\) maps to \((g_1,g_2)=(1,e^{-1})\), and \(\mathbf{x}=(1,1)\) maps to \((e^{-1},1)\). The two off-diagonal corners \((0,1)\) and \((1,0)\) map to the same feature point \((e^{-1/2},e^{-1/2})\). In this feature plane the XOR labels become linearly separable; for instance, the separator \(g_1+g_2=1.3\) places the diagonal corners on one side and the off-diagonal corners on the other (\Cref{fig:rbf_xor_feature_map}).

\begin{figure}[t]
    \centering
\begin{tikzpicture}[background rectangle/.style={fill=white}, show background rectangle]
        \begin{groupplot}[
            group style={group size=2 by 1, horizontal sep=1.3cm},
            width=0.44\linewidth,
            height=0.34\linewidth,
            axis lines=middle,
            tick label style={font=\scriptsize},
            label style={font=\scriptsize},
            title style={font=\scriptsize, align=center},
            axis background/.style={fill=white},
            clip=false,
        ]
        \nextgroupplot[
            title={Input space \((x_1,x_2)\)},
            xlabel={$x_1$},
            ylabel={$x_2$},
            xmin=-0.1, xmax=1.1,
            ymin=-0.1, ymax=1.1,
            xtick={0,1},
            ytick={0,1},
        ]
            % Class 0: (0,0) and (1,1)
            \addplot[
                only marks,
                mark=o,
                mark options={fill=white, draw=black},
                mark size=2.6pt,
            ] coordinates {(0,0) (1,1)};
            % Class 1: (0,1) and (1,0)
            \addplot[
                only marks,
                mark=square*,
                mark options={fill=black, draw=black},
                mark size=2.6pt,
            ] coordinates {(0,1) (1,0)};
            \node[font=\scriptsize, text=black!60, anchor=south] at (axis cs:0.5,0.05) {no single line separates XOR};

        \nextgroupplot[
            title={Feature space \((g_1,g_2)\)},
            xlabel={$g_1$},
            ylabel={$g_2$},
            xmin=0.3, xmax=1.05,
            ymin=0.3, ymax=1.05,
            xtick={0.368,0.607,1.0},
            ytick={0.368,0.607,1.0},
        ]
            % Class 0 in feature space.
            \addplot[
                only marks,
                mark=o,
                mark options={fill=white, draw=black},
                mark size=2.6pt,
            ] coordinates {(1.000000,0.367879) (0.367879,1.000000)};
            % Class 1 corners coincide at the same feature point for this two-center example.
            \addplot[
                only marks,
                mark=square*,
                mark options={fill=black, draw=black},
                mark size=2.6pt,
            ] coordinates {(0.606531,0.606531)};
            \node[font=\scriptsize, text=black!60, anchor=south] at (axis cs:0.606531,0.62) {(0,1) and (1,0) coincide};
            % One valid separating border in feature space: g1 + g2 = 1.3.
            \addplot[black!70, dashed, domain=0.3:1.05, samples=2] {1.3 - x};
            \node[font=\scriptsize, text=black!60, anchor=north] at (axis cs:0.78,0.52) {$g_1+g_2=1.3$};
        \end{groupplot}
    \end{tikzpicture}
    \caption{XOR before and after an RBF feature map. Left: in \((x_1,x_2)\), no single line separates the labels. Right: in \((g_1,g_2)\), the transformed points are linearly separable; one valid separating border is \(g_1+g_2=1.3\) (equivalently \(w=[1,1]\), \(b=-1.3\)).}
    \label{fig:rbf_xor_feature_map}
\end{figure}
\FloatBarrier

\subsection{Finding the Optimal Weight Vector \texorpdfstring{\(\mathbf{w}\)}{w}}
\label{sec:rbf_finding_the_optimal_weight_vector_w_w}

Given the transformed data \(\mathbf{g}(\mathbf{x})\) and desired outputs \(\mathbf{d}\), we want to find \(\mathbf{w}\) that minimizes the squared error between the predicted output and the target. Using the design matrix \(\boldsymbol{\Phi}\) defined in the notation box, the model predicts \(\hat{\mathbf{d}}=\boldsymbol{\Phi}\mathbf{w}\), and the least-squares objective is
\begin{equation}
J(\mathbf{w}) = \|\mathbf{d} - \boldsymbol{\Phi} \mathbf{w}\|^2.
\label{eq:cost_function}
\end{equation}

\paragraph{Normal Equations for the Weights:}
Differentiating \eqref{eq:cost_function} with respect to \(\mathbf{w}\) and setting the gradient to zero yields
\begin{equation}
\boldsymbol{\Phi}^\top \boldsymbol{\Phi} \,\mathbf{w} = \boldsymbol{\Phi}^\top \mathbf{d}.
\label{eq:normal_eq_weights}
\end{equation}
When \(\boldsymbol{\Phi}^\top\boldsymbol{\Phi}\) is well conditioned, the closed-form solution is \(\mathbf{w}^\star=(\boldsymbol{\Phi}^\top\boldsymbol{\Phi})^{-1}\boldsymbol{\Phi}^\top\mathbf{d}\); in practice we almost always add ridge regularization as described in the training section below.

\paragraph{Conditioning and capacity.} When \(M\) is large and Gaussians overlap heavily, \(\boldsymbol{\Phi}^\top\boldsymbol{\Phi}\) can become ill-conditioned. Ridge regularization (adding \(\lambda I\)) stabilizes the solve and controls variance, mirroring the bias--variance trade-off from \Cref{chap:supervised}. Choosing \(M\), \(\sigma\), and \(\lambda\) together is essential for good generalization; \Cref{chap:supervised}'s learning-curve diagnostics apply directly, and kernel methods (e.g., kernel ridge regression or SVMs) interpret the same trade-off via RBF kernels.

\subsection{The Role of the Transformation Function \texorpdfstring{\(g(\cdot)\)}{g(.)}}
\label{sec:rbf_the_role_of_the_transformation_function_g_g}

The nonlinear map \(g(\cdot)\) and its role in constructing \(\boldsymbol{\Phi}\) were defined in the transformation example above; here we focus on its parameters and how to choose it.

Two parameters characterize \(g(\cdot)\):
\begin{itemize}
    \item \(\mathbf{v}_i\): the centroid or center of the \(i\)-th basis function.
    \item \(\sigma_i\): the width or spread parameter controlling the receptive field of the basis function.
\end{itemize}

\paragraph{Choosing \(g(\cdot)\):} The choice of \(g(\cdot)\) is crucial. It defines how the input space is mapped into the feature space where linear separation is possible. A common rule-of-thumb for Gaussian widths is to set \(\sigma\) so that neighboring centers at average spacing \(\bar{r}\) overlap with height \(\exp(-\bar{r}^2/(2\sigma^2))\approx 0.5\)--0.7; too small \(\sigma\) fragments the boundary, too large washes out locality.

\subsection{Examples of Kernel Functions}
\label{sec:rbf_examples_of_kernel_functions}

\paragraph{1. Inverse Distance Function:}
\[
g(r) = \frac{1}{r + \epsilon}, \quad \epsilon > 0,
\]
where \(r = \|\mathbf{x} - \mathbf{v}\|\). This function decreases as the distance increases but can become unbounded near zero, potentially causing numerical instability.

\paragraph{2. Gaussian Radial Basis Function:}
\[
g(r) = \exp\left(-\frac{r^2}{2\sigma^2}\right).
\]
This function is smooth, bounded, and has a clear interpretation as a localized receptive field centered at \(\mathbf{v}\) with width \(\sigma\). It is the most commonly used kernel in RBF networks.

\begin{tcolorbox}[summarybox, title={Why ``radial''? Why a Gaussian?}]
An RBF unit is called \emph{radial} because its response depends primarily on distance from a center: points at the same radius (in the chosen metric) produce the same activation. The Gaussian basis is popular because it is smooth, has a clear center, and its width parameter \(\sigma\) directly controls locality: large \(\sigma\) makes each unit ``see'' broadly (risking underfit), while small \(\sigma\) makes units highly local (risking overfit and poor conditioning). The practical art is to pick centers that cover the data and then tune \(\sigma\) (and ridge \(\lambda\)) by validation, as in \Cref{fig:rbf_sigma_sweep}.
\end{tcolorbox}

\subsection{Interpretation of the Width Parameter \texorpdfstring{\(\sigma\)}{sigma}}
\label{sec:rbf_interpretation_of_the_width_parameter_sigma}

The parameter \(\sigma\) controls the spread of the basis function. Conceptually, increasing \(\sigma\) broadens the Gaussian bell, while decreasing \(\sigma\) produces a narrow spike around the centroid.

\begin{itemize}
    \item \(\sigma = 1\): The function is broad, covering a large region of the input space.
    \item \(\sigma = 0.3\): The function is narrow and sharply peaked around the centroid.
\end{itemize}

Choosing \(\sigma\) appropriately is critical for the network's performance:
\begin{itemize}
    \item If \(\sigma\) is too large, the basis functions overlap excessively, leading to smooth but potentially underfitting models.
    \item If \(\sigma\) is too small, the basis functions become too localized, which may cause overfitting and poor generalization.
\end{itemize}

\subsection{Effect of \texorpdfstring{\(\sigma\)}{sigma} on Classification Boundaries}
\label{sec:rbf_effect_of_sigma_on_classification_boundaries}

Consider a one-dimensional dataset with two classes (e.g., red and blue points). Projecting a sample \(x\) through the Gaussian basis functions produces feature activations
\[
\varphi_i(x) = \exp\!\left(-\frac{(x - v_i)^2}{2\sigma^2}\right),
\]
which serve as localized similarity measures to each centroid \(v_i\). When \(\sigma\) is large, many points activate the same basis functions with comparable strength, leading to smooth decision boundaries after the linear output layer. When \(\sigma\) is small, only points very close to a centroid elicit large activations, yielding sharply varying boundaries that can overfit noise. Visualizing \(\varphi_i(x)\) for several centroids illustrates how tuning \(\sigma\) controls the flexibility of the classifier.

\begin{figure}[t]
    \centering
    \begin{tikzpicture}[scale=0.9]
        \node at (0,3.2) {\small$\sigma$ sweep (schematic)};
        % three panels
        \begin{scope}[shift={(-4,0)}]
            \draw[gray!40] (-1.8,-1.3) rectangle (1.8,1.3);
            \node at (0,1.6) {\scriptsize underfit (large $\sigma$)};
            \draw[thick, blue!40] (-1.8,0) -- (1.8,0);
            \draw[thick, blue!40] (0,-1.3) -- (0,1.3);
        \end{scope}
        \begin{scope}
            \draw[gray!40] (-1.8,-1.3) rectangle (1.8,1.3);
            \node at (0,1.6) {\scriptsize just right};
            \draw[thick, blue!60, rounded corners] (-1.6,-0.4).. controls (-0.6,-0.6) and (0.2,0.6).. (1.6,0.4);
        \end{scope}
        \begin{scope}[shift={(4,0)}]
            \draw[gray!40] (-1.8,-1.3) rectangle (1.8,1.3);
            \node at (0,1.6) {\scriptsize overfit (small $\sigma$)};
            \draw[thick, blue!60, rounded corners] (-1.7,-0.8) -- (-0.6,-0.2) -- (0.1,-1.0) -- (1.7,-0.1);
        \end{scope}
    \end{tikzpicture}
    % Avoid inline math in captions; it wraps poorly in some EPUB renderers.
\caption{Schematic illustration of how the width parameter \(\sigma\) influences decision boundaries: too-large \(\sigma\) underfits (bases overlap heavily and wash out locality), intermediate \(\sigma\) captures the boundary, and too-small \(\sigma\) can produce fragmented regions. For a computed XOR example, see \Cref{fig:rbf_boundary}.}
    \label{fig:rbf_sigma_sweep}
\end{figure}
\FloatBarrier

We return to this tuning picture again later with an XOR-style toy example, where the decision boundary is easy to visualize.

\paragraph{Notation note.} In this chapter we write radial basis functions as \(\varphi_i(\cdot)\) and use \(\boldsymbol{\Phi}\) for the associated design matrix. When we need a generic kernel feature map, we use \(\phi(\cdot)\) (consistent with \Cref{app:kernels}); when probability density functions are needed, we write them as \(p(\cdot)\). This avoids overloading a single symbol.

% Chapter 8

\subsection{Radial Basis Function Networks: Parameter Estimation and Training}
\label{sec:rbf_radial_basis_function_networks_parameter_estimation_and_training}

Recall that in Radial Basis Function (RBF) networks, the hidden layer neurons compute outputs based on radial basis functions centered at certain points \( \mathbf{v}_i \) with spread parameters \( \sigma_i \). The output is a linear combination of these nonlinear transformations. The key challenge is to determine the parameters:
\[
\{ \mathbf{v}_i, \sigma_i, w_i \}_{i=1}^M,
\]
where \( M \) is the number of hidden neurons.

\paragraph{Finding the Centers \(\mathbf{v}_i\):}
A natural approach to find the centers is to use clustering algorithms on the input data. For example, if we decide to have \( M \) hidden neurons, we run a clustering algorithm (e.g., K-means) to find \( M \) centroids:
\[
\mathbf{v}_1, \mathbf{v}_2, \ldots, \mathbf{v}_M.
\]
These centroids represent typical data points around which the radial basis functions are centered. This approach ensures that the radial basis functions cover the input space effectively.

\paragraph{Determining the Spread Parameters \(\sigma_i\):}
The spread parameters control the width of each radial basis function. One can initialize all \( \sigma_i \) to a common value or assign different values based on the data distribution. A practical rule-of-thumb is
\[
\sigma \approx \frac{d_{\max}}{\sqrt{2M}},
\]
where \(d_{\max}\) is the maximum pairwise distance between centers and \(M\) the number of RBF units; this ensures neighboring receptive fields overlap without collapsing to a constant function. After setting this global width, refine to per-center widths by setting each \( \sigma_i \) proportional to the average distance between the centroid \( \mathbf{v}_i \) and its nearest neighboring centroids. Anisotropic variants scale each dimension separately but follow the same principle of matching the local density of prototypes.

\paragraph{Training the Output Weights \( w_i \):}
Given fixed centers and spreads, the output weights \( w_i \) can be found by minimizing the squared error between the network output and the target values. The network output for an input \( \mathbf{x} \) is:
\[
\hat{y}(\mathbf{x}) = \sum_{i=1}^M w_i \varphi_i(\mathbf{x}).
\]
where
\[
\varphi_i(\mathbf{x}) = \exp\left(-\frac{\|\mathbf{x} - \mathbf{v}_i\|^2}{2\sigma_i^2}\right).
\]

    The training problem reduces to solving the linear system:
\begin{align}
\min_{\mathbf{w}} \| \mathbf{y} - \boldsymbol{\Phi} \mathbf{w} \|^2,
\label{eq:rbf_weight_training}
\end{align}
where \(\mathbf{y}\) is the vector of target outputs and \(\boldsymbol{\Phi}\) is the design matrix with entries \(\boldsymbol{\Phi}_{ji} = \varphi_i(\mathbf{x}_j)\).
When \(\boldsymbol{\Phi}^\top\boldsymbol{\Phi}\) is well-conditioned, the ordinary least-squares solution is
\[
    \mathbf{w}^\star = (\boldsymbol{\Phi}^\top \boldsymbol{\Phi})^{-1}\boldsymbol{\Phi}^\top \mathbf{y}.
\]

\begin{tcolorbox}[summarybox, title={Dual viewpoint: RBFN vs.\ kernel ridge regression}]
Fixing the RBF centers and widths makes the hidden layer a finite basis expansion. Training restricts itself to the \(M\) coefficients \(\mathbf{w}\) and resembles kernel ridge regression with a truncated basis. In the dual view, kernel ridge regression solves
\[
\min_{\boldsymbol{\alpha}} \|\mathbf{y} - \mathbf{K} \boldsymbol{\alpha}\|^2 + \lambda \boldsymbol{\alpha}^\top \mathbf{K} \boldsymbol{\alpha},
\]
where \(\mathbf{K}_{ij} = k(\mathbf{x}_i,\mathbf{x}_j)\) uses the same Gaussian kernel. Setting \(M=N\) and letting the RBF centers coincide with the training points recovers this dual form exactly. Finite \(M\) acts like Nystr\"om approximation: \(\boldsymbol{\Phi} \mathbf{w}\) projects onto a subset of kernel features.

Numerically, \(\boldsymbol{\Phi}^\top \boldsymbol{\Phi}\) can be ill-conditioned if the bases overlap excessively or if centers cluster tightly; kernel ridge has the same issue via \(\mathbf{K}\). Regularization is therefore essential: add \(\lambda I\) before inversion,
\[
\mathbf{w}^\star = (\boldsymbol{\Phi}^\top \boldsymbol{\Phi} + \lambda I)^{-1} \boldsymbol{\Phi}^\top \mathbf{y},
\]
mirroring the \(\lambda \boldsymbol{\alpha}^\top \mathbf{K} \boldsymbol{\alpha}\) term in the dual problem. Larger \(\lambda\) damps coefficients when \(\sigma\) is large (heavy overlap) or when data are noisy, while smaller \(\lambda\) preserves sharper fits at the cost of conditioning. Choosing \(\lambda\) via cross-validation keeps both primal (RBFN) and dual (kernel ridge) systems stable.
\end{tcolorbox}

\Cref{fig:rbf_primal_dual} summarizes the primal and dual viewpoints side by side and highlights where a finite basis acts like a Nystr\"om approximation.

\begin{figure}[t]
    \centering
    \begin{tikzpicture}[scale=0.95]
        \node at (-3,1.4) {\scriptsize Primal};
        \draw[gray!40] (-4,0) rectangle (-2,1);
        \node[font=\scriptsize] at (-3,0.5) {$\boldsymbol{\Phi}\in\mathbb{R}^{N\times M}$};
        \node[font=\scriptsize] at (-3,-0.1) {build features};
        \draw[->, thick] (-1.5,0.5) -- (-0.5,0.5);
        \node[font=\scriptsize] at (0.5,0.8) {solve};
        \node[font=\scriptsize] at (0.5,0.4) {$(\boldsymbol{\Phi}^\top\boldsymbol{\Phi}+\lambda I)w=\boldsymbol{\Phi}^\top y$};
        \draw[gray!40] (2,0) rectangle (4,1);
        \node[font=\scriptsize] at (3,0.5) {$w\in\mathbb{R}^{M}$};

        \begin{scope}[shift={(0,-2)}]
            \node at (-3,1.4) {\scriptsize Dual (kernel ridge/SVM)};
            \draw[gray!40] (-4,0) rectangle (-2,1);
        \node[font=\scriptsize] at (-3,0.5) {$K\in\mathbb{R}^{N\times N}$};
        \node[font=\scriptsize] at (-3,-0.1) {kernel matrix};
        \draw[->, thick] (-1.5,0.5) -- (-0.5,0.5);
        \node[font=\scriptsize] at (0.5,0.8) {solve};
        \node[font=\scriptsize] at (0.5,0.4) {$(K+\lambda I)\alpha=y$};
        \draw[gray!40] (2,0) rectangle (4,1);
        \node[font=\scriptsize] at (3,0.5) {$\alpha\in\mathbb{R}^{N}$};
    \end{scope}
    \draw[->, gray!60, dashed] (-1,-0.6) -- (1,-1.4) node[midway, sloped, above, gray!60]{Nystr\"om $M<N$};
\end{tikzpicture}
    % Avoid inline math in captions; it wraps poorly in some EPUB renderers.
    \caption{Primal (finite basis) vs.\ dual (kernel ridge) viewpoints. Using as many centers as data points recovers the dual form; using fewer centers corresponds to a Nystr\"om approximation. The same trade-off appears in kernel methods through the choice of kernel and effective rank.}
\label{fig:rbf_primal_dual}
\end{figure}
\FloatBarrier

To improve numerical stability or control model complexity, a Tikhonov (ridge) regulariser can be added,
\[
    \mathbf{w}_\lambda^\star = (\boldsymbol{\Phi}^\top \boldsymbol{\Phi} + \lambda I)^{-1}\boldsymbol{\Phi}^\top \mathbf{y}, \qquad \lambda>0,
\]
or more generally one can use the Moore--Penrose pseudoinverse \(\boldsymbol{\Phi}^{+}\) when \(\boldsymbol{\Phi}^\top\boldsymbol{\Phi}\) is singular, yielding \(\mathbf{w}^\star = \boldsymbol{\Phi}^{+}\mathbf{y}\). A quick dimensional sanity check is that \(\boldsymbol{\Phi}\in\mathbb{R}^{N\times M}\), \(\mathbf{w}\in\mathbb{R}^M\), and \(\mathbf{y}\in\mathbb{R}^N\); all matrix products above respect these shapes.

\paragraph{Iterative Optimization of \(\sigma_i\) and \( w_i \):}
Since both \( \sigma_i \) and \( w_i \) affect the network output, an alternating optimization procedure can be employed:
\begin{enumerate}
    \item Initialize \( \sigma_i \) (e.g., all equal or based on data heuristics).
    \item Fix \( \sigma_i \) and find \( w_i \) by solving the linear least squares problem \eqref{eq:rbf_weight_training}.
    \item Fix \( w_i \) and update \( \sigma_i \) to minimize the error, possibly using gradient-based methods or heuristics.
    \item Repeat steps 2 and 3 until convergence or error criteria are met.
\end{enumerate}

Note that the spreads \( \sigma_i \) can be scalar or vector-valued (anisotropic), allowing different widths in each input dimension:
\[
\sigma_i = [\sigma_{i1}, \sigma_{i2}, \ldots, \sigma_{id}],
\]
where \( d \) is the input dimension.

\paragraph{Summary of the Training Algorithm:}
\begin{enumerate}
    \item Use clustering (e.g., K-means) to find centers \( \mathbf{v}_i \) (or sample centers uniformly at random).
    \item Set widths \(\sigma_i\) via a rule-of-thumb (global \(\sigma\) from average center spacing or per-cluster covariance).
    \item Build \(\boldsymbol{\Phi}\) with entries \(\boldsymbol{\Phi}_{ji}=\varphi_i(\mathbf{x}_j)\); choose a small grid of \(\lambda\) values and solve \((\boldsymbol{\Phi}^\top\boldsymbol{\Phi}+\lambda I)\mathbf{w}=\boldsymbol{\Phi}^\top\mathbf{y}\).
    \item Evaluate on a validation set and pick \((\sigma,\lambda, M)\) that minimizes validation loss; for classification, CE/hinge losses are also feasible with the same design matrix \(\boldsymbol{\Phi}\).
\end{enumerate}

\begin{tcolorbox}[summarybox, title={Practical RBFN training (pseudocode)}]
\begin{verbatim}
Input: X, y, M, center_method=kmeans, sigma_rule, lambda_grid
Centers = center_method(X, M)
sigma = sigma_rule(Centers)
Phi = build_design_matrix(X, Centers, sigma)   # NxM
for lambda in lambda_grid:
    w_lambda = solve((Phi^T Phi + lambda I) w = Phi^T y)
    val_err[lambda] = validation_loss(Phi_val, y_val, w_lambda)
lambda_star = argmin val_err
Predict: yhat(x) = phi(x, Centers, sigma)^T w_lambda_star
\end{verbatim}
\end{tcolorbox}

\paragraph{Worked toy (classification, XOR-like).} Consider four points and XOR labels
\[
\mathbf{x}_1=(0,0),\;\mathbf{x}_2=(0,1),\;\mathbf{x}_3=(1,0),\;\mathbf{x}_4=(1,1),
\qquad \mathbf{t}=[0,1,1,0].
\]
Choose \(M=4\) centers at the data and set a global \(\sigma\) from the mean inter-center distance (here \(\sigma\approx 0.8\)). Build \(\boldsymbol{\Phi}\) with entries \(\boldsymbol{\Phi}_{ji}=\exp\!\big(-\|\mathbf{x}_j-\mathbf{c}_i\|^2/(2\sigma^2)\big)\) and solve \((\boldsymbol{\Phi}^\top\boldsymbol{\Phi}+\lambda I)\mathbf{w}=\boldsymbol{\Phi}^\top\mathbf{t}\) over a small grid \(\lambda\in\{10^{-4},10^{-3},10^{-2}\}\). The best \(\lambda\) yields a linear separator in the lifted \(\boldsymbol{\Phi}\)-space that classifies XOR. Widening \(\sigma\) makes bases overlap heavily and can wash out locality (very smooth boundaries and low margins), while shrinking \(\sigma\) makes the model extremely local and can create small islands that fit the training points but generalize poorly. Ridge helps whenever \(\boldsymbol{\Phi}^\top\boldsymbol{\Phi}\) is poorly conditioned (typically when bases overlap heavily or centers cluster).

\Cref{fig:rbf_boundary} shows one such boundary for a fixed choice of centers, width, and ridge regularization.

	    \begin{figure}[t]
	        \centering
	            % Reproduce the plots directly in PGFPlots so they remain grayscale/KDP-safe.
	            % Data tables are generated by `notes_output/scripts/gen_rbf_xor_sigma_sweep_tables.py`.
	            \begin{tikzpicture}
	            \pgfplotsset{colormap={rbfclass}{
	                % Light vs. darker tint: still distinct in grayscale, but more pleasant in color.
	                color(0cm)=(white);
	                color(1cm)=(cbBlue!35)
	            }}
                \begin{groupplot}[
                    group style={group size=3 by 1, horizontal sep=0.9cm},
                    width=0.31\linewidth,
                    height=0.31\linewidth,
                    xmin=-0.2, xmax=1.2,
                    ymin=-0.2, ymax=1.2,
                    grid=both,
                    grid style={gray!10},
                    axis on top,
                    clip=false,
                    colormap name=rbfclass,
                    point meta min=0,
                    point meta max=1,
                    xlabel={$x_1$},
                    xtick={0,1},
                    ytick={0,1},
                    tick label style={font=\scriptsize},
                    label style={font=\scriptsize},
                    title style={font=\scriptsize, align=center},
                ]
                \nextgroupplot[title={$\sigma=2.0$ (too broad)}, ylabel={$x_2$}]
                    \addplot[
                        matrix plot*,
                        mesh/cols=29,
                        mesh/rows=29,
                        point meta=explicit,
                        forget plot,
                    ] table [x=x, y=y, meta=cls] {rbf_xor_sigma2p0_boundary_class_table_with_breaks.dat};
                    \addplot[
                        black,
                        very thick,
                        unbounded coords=jump,
                        forget plot,
                    ] table [x=x, y=y] {rbf_xor_sigma2p0_contour_0p5.dat};
                    \addplot[only marks, mark=o, mark options={fill=white, draw=black}, mark size=2.6pt, forget plot]
                        coordinates {(0,0) (1,1)};
                    \addplot[only marks, mark=square*, mark options={fill=black, draw=black}, mark size=2.6pt, forget plot]
                        coordinates {(0,1) (1,0)};

                \nextgroupplot[title={$\sigma=0.8$ (balanced)}]
                    \addplot[
                        matrix plot*,
                        mesh/cols=29,
                        mesh/rows=29,
                        point meta=explicit,
                        forget plot,
                    ] table [x=x, y=y, meta=cls] {rbf_xor_sigma0p8_boundary_class_table_with_breaks.dat};
                    \addplot[
                        black,
                        very thick,
                        unbounded coords=jump,
                        forget plot,
                    ] table [x=x, y=y] {rbf_xor_sigma0p8_contour_0p5.dat};
                    \addplot[only marks, mark=o, mark options={fill=white, draw=black}, mark size=2.6pt, forget plot]
                        coordinates {(0,0) (1,1)};
                    \addplot[only marks, mark=square*, mark options={fill=black, draw=black}, mark size=2.6pt, forget plot]
                        coordinates {(0,1) (1,0)};

                \nextgroupplot[title={$\sigma=0.25$ (too local)}]
                    \addplot[
                        matrix plot*,
                        mesh/cols=29,
                        mesh/rows=29,
                        point meta=explicit,
                        forget plot,
                    ] table [x=x, y=y, meta=cls] {rbf_xor_sigma0p25_boundary_class_table_with_breaks.dat};
                    \addplot[
                        black,
                        very thick,
                        unbounded coords=jump,
                        forget plot,
                    ] table [x=x, y=y] {rbf_xor_sigma0p25_contour_0p5.dat};
                    \addplot[only marks, mark=o, mark options={fill=white, draw=black}, mark size=2.6pt, forget plot]
                        coordinates {(0,0) (1,1)};
                    \addplot[only marks, mark=square*, mark options={fill=black, draw=black}, mark size=2.6pt, forget plot]
                        coordinates {(0,1) (1,0)};
                \end{groupplot}
	        \end{tikzpicture}
	        % Avoid inline math in captions; it wraps poorly in some EPUB renderers.
	        \caption{Effect of \(\sigma\) on an RBFN XOR boundary (4 centers at the data corners, ridge \(\lambda=10^{-3}\), threshold 0.5). Too-large \(\sigma\) makes bases overlap heavily, producing a very smooth, low-contrast boundary; intermediate \(\sigma\) yields a cleaner separation; too-small \(\sigma\) makes the model extremely local, producing small ``islands'' around prototypes.}
	        \label{fig:rbf_boundary}
	    \end{figure}
\FloatBarrier


\subsection{Remarks on Radial Basis Function Networks}
\label{sec:rbf_remarks_on_radial_basis_function_networks}

\paragraph{Advantages:}
\begin{itemize}
    \item \textbf{Training speed:} Once centers and spreads are fixed, training reduces to a linear least squares problem with a closed-form solution, which is computationally efficient.
    \item \textbf{Universal approximation:} RBF networks can approximate any continuous function on a compact domain to arbitrary accuracy given sufficient neurons, provided the centers cover the domain and the widths are chosen to avoid degeneracy \citep{Micchelli1986,ParkSandberg1991}.
    \item \textbf{Interpretability:} Centers correspond to representative data points, making the network structure more interpretable.
    \item \textbf{Applications:} RBF networks have been successfully applied in control systems, communication systems, chaotic time series prediction (e.g., weather and power load forecasting), and decision-making tasks.
    \item \textbf{Flexible losses:} Squared loss is standard for regression; logistic or hinge losses pair naturally with the fixed design matrix for classification.
\end{itemize}

\paragraph{Disadvantages:}
\begin{itemize}
    \item \textbf{Parameter selection:} Choosing the number of neurons \( M \), centers \( \mathbf{v}_i \), and spreads \( \sigma_i \) is nontrivial and often requires heuristics or cross-validation.
    \item \textbf{Scalability:} The number of radial units required can grow quickly with input dimensionality, increasing computation and storage costs.
    \item \textbf{Center determination:} Identifying good centers (via clustering or other heuristics) can be computationally expensive and sensitive to noisy data.
\end{itemize}

% Chapter 8: Finalizing Derivations and Closure

\begin{tcolorbox}[summarybox, title={Sidebar (optional): Wiener filtering in one paragraph}]
The Wiener filter is the same least-squares projection story in signal-processing notation. In the Wiener case, you write the linear estimator as \(y(t)=\mathbf{w}^\top\mathbf{x}(t)\) and the normal equations involve second-order statistics \(\mathbf{R}=\mathbb{E}[\mathbf{x}\mathbf{x}^\top]\) and \(\mathbf{p}=\mathbb{E}[d\,\mathbf{x}]\), giving \(\mathbf{R}\mathbf{w}^\star=\mathbf{p}\). In an RBF network, you replace the raw input \(\mathbf{x}\) by fixed nonlinear features \(\boldsymbol{\Phi}\) (built from radial units), and the same idea becomes \((\boldsymbol{\Phi}^\top\boldsymbol{\Phi}+\lambda I)\mathbf{w}^\star=\boldsymbol{\Phi}^\top\mathbf{y}\). The point of the sidebar is just this mapping: both are ``fit a linear readout to fixed features,'' and conditioning/regularization control how stable that fit is.
\end{tcolorbox}

\subsection{Preview: Unsupervised and Localized Learning}
\label{sec:rbf_preview_unsupervised_and_localized_learning}

In \Cref{chap:som}, we move from supervised RBF models to unsupervised, self-organizing methods. Self-organizing maps (SOMs) and Hopfield-style associative memory discover structure in data (clusters, manifolds) without labeled targets, complementing the supervised architectures covered so far.

\begin{tcolorbox}[summarybox, title={Key takeaways}]
\textbf{Minimum viable mastery}
\begin{itemize}
    \item Localized Gaussian bases + linear readout give an interpretable nonlinear model; center/width/regularization choices control bias--variance.
    \item Primal RBFNs and kernel ridge regression are two views of the same estimator (full vs. truncated basis); regularization cures conditioning.
    \item RBFNs bridge learned-feature models (MLPs) and kernel methods (SVMs/GPs); they form a strong baseline for localized decision boundaries.
\end{itemize}
\medskip
\textbf{Common pitfalls}
\begin{itemize}
    \item Width selection: too small memorizes, too large collapses to a linear model; validate \(\sigma\) and \(\lambda\).
    \item Poor conditioning: without regularization, solves can be numerically unstable even when the math is correct.
    \item Confusing kernels with bases: a full kernel method scales differently than a truncated (primal) basis expansion.
\end{itemize}
\end{tcolorbox}

\begin{tcolorbox}[summarybox, title={Exercises and lab ideas}]
\begin{itemize}
    \item Train an RBFN on the two-moons dataset; sweep \(M\) and \(\sigma\), add a small \(\lambda\) grid, plot validation curves and decision boundaries; report the \((M,\sigma,\lambda)\) that minimizes validation error and discuss over/underfitting.
    \item Compare primal RBFN and kernel ridge regression with an RBF kernel on datasets of size \(N\in\{200,2000,20\,000\}\); measure accuracy and runtime; note when each approach is preferable.
    \item Show that setting centers at all data points with \(\lambda>0\) yields the same predictions as kernel ridge regression; derive the relationship between \(\mathbf{w}\) and \(\boldsymbol{\alpha}\).
    \item Plot how validation error moves with \((M,\sigma,\lambda)\) and link the curves back to the bias--variance discussion in \Cref{chap:supervised}.
\end{itemize}
\medskip
\noindent\textbf{If you are skipping ahead.} Keep the idea that ``nonlinear'' can still be linear-in-parameters once a basis is fixed. That perspective is reused in kernel methods and shows up again when we discuss architectural bias (CNNs) and representation choices.
\end{tcolorbox}

\medskip
\paragraph{Where we head next.} \Cref{chap:som} moves from supervised objectives to unlabeled competitive learning and prototype organization. \Cref{chap:hopfield} then revisits recurrence through an energy-based lens, setting up later sequence-model chapters.

\nocite{Haykin2013AdaptiveFilterTheory, WidrowStearns1985, SchreierScharf2010, Micchelli1986, ParkSandberg1991, PoggioGirosi1990, Bishop1995, HastieTibshiraniFriedman2009}

% Chapter 9
\section{Introduction to Self-Organizing Networks and Unsupervised Learning}\label{chap:som}
\graphicspath{{assets/lec5/}}

\Cref{chap:rbf} kept nonlinearity close to linear algebra: once you fix a set of basis functions, you lift the inputs into a feature space and (often) solve a regularized least-squares problem for the output weights. Now we switch the question. Instead of ``how do I fit targets?'', we ask: when labels are missing, expensive, or not even the right interface, what structure is already present in the inputs?

This chapter opens the unsupervised neural thread with \emph{Self-Organizing Maps} (SOMs), also known as Kohonen maps. You learn a set of prototypes and place them on a grid so that, as training progresses, units that are neighbors on the grid tend to respond to similar inputs. \Cref{chap:hopfield} then studies Hopfield networks as an energy\hyp{}based associative memory model. Both operate without explicit targets, but they use that freedom differently: SOMs emphasize organization and visualization, while Hopfield emphasizes retrieval dynamics.

\begin{tcolorbox}[summarybox, title={Learning Outcomes}]
\begin{itemize}
    \item Describe how competitive learning, cooperation, and annealing interact in SOM training.
    \item Monitor SOM quality via quantization error (QE), topographic error (TE), and interpret U-matrices (unified distance matrix plots).
    \item Connect SOMs to broader unsupervised techniques (clustering, dimensionality reduction) and know when to use each.
\end{itemize}
\end{tcolorbox}

\begin{tcolorbox}[summarybox, title={Design motif}]
Competition plus cooperation: pick a winner, then let its neighbors learn too, so the map becomes both a clustering device and a visualization.
\end{tcolorbox}

\subsection{Overview of Self-Organizing Networks}
\label{sec:som_overview_of_self_organizing_networks}

Self-organizing networks aim to discover structure in input data without labels. The most prominent example is the \emph{Self-Organizing Map} (SOM), introduced by Teuvo Kohonen. SOMs are used for clustering and for building a visualization-friendly map of a high-dimensional dataset.

The idea is fairly intuitive. If the data have a meaningful notion of similarity in the original space, then repeated \emph{competition} (pick the best-matching unit) and \emph{cooperation} (update its neighbors as well) can organize a fixed 2D \emph{grid} of prototype vectors so that nearby grid locations tend to represent nearby regions of the data. The map is not a perfect geometric embedding; it is an engineered bias that often produces a useful, inspectable picture.

SOMs are trained without labeled outputs; the only signal is input similarity under your chosen distance. Learning is \emph{competitive} (each input picks a best\hyp{}matching unit) but also \emph{cooperative} (neighbors of the winner move too). That neighborhood coupling is the deliberate bias: it encourages nearby grid locations to represent similar inputs, so the map can be read both as a set of prototypes and as a visualization.

\begin{tcolorbox}[summarybox, title={Historical intuition: two sheets and topographic neighborhoods}]
A useful way to picture SOMs is as two coupled ``sheets'': an input space and a fixed 2D grid of units. Each input is connected (in principle) to the whole grid, but learning makes some regions respond strongly (excitation) while others respond weakly (inhibition). The payoff is a \emph{topographic} map: inputs that are far apart in the original space can end up near one another on the grid if they are statistically similar under the features the SOM has learned.
\end{tcolorbox}

\begin{tcolorbox}[summarybox, title={Author's note: tie SOMs back to clustering and dimensionality reduction}]
SOMs live in the same ecosystem as clustering and dimensionality reduction: they learn prototypes without labels and simultaneously organize those prototypes on a low-dimensional grid. Treat the update rules as a carefully annealed clustering algorithm whose output just happens to be arranged on a grid for interpretability.
\end{tcolorbox}

The neighborhood influence is usually controlled by a kernel (often Gaussian) whose amplitude decays with grid distance and shrinks as training progresses, so early updates promote global organization while later updates refine only the closest units. \Cref{fig:lec5-learning-rate} juxtaposes these two time scales: the left panel shows why coarse early steps help traverse the energy landscape quickly, while the right panel compares two decaying learning-rate schedules commonly used when training SOMs.

\begin{figure}[t]
    \centering
    \begin{tikzpicture}
        \begin{groupplot}[
            group style={group size=2 by 1, horizontal sep=1.2cm},
            width=0.45\linewidth, height=0.36\linewidth
        ]
        % Left: coarse-to-fine steps on a convex bowl
        \nextgroupplot[
            title={Coarse $\rightarrow$ fine steps on $f(x, y)$},
            axis lines=middle, xmin=-2.2, xmax=2.2, ymin=-2.2, ymax=2.2,
            xlabel={$x$}, ylabel={$y$}
        ]
            % Simple elliptical contours of a convex quadratic
            \addplot[very thin, gray!50, samples=200, domain=0:6.283] ({1.9*cos(x)}, {0.9*sin(x)});
            \addplot[very thin, gray!50, samples=200, domain=0:6.283] ({1.3*cos(x)}, {0.6*sin(x)});
            \addplot[very thin, gray!50, samples=200, domain=0:6.283] ({0.8*cos(x)}, {0.4*sin(x)});
            % Trajectory: large steps then small steps with markers
            \addplot[cbOrange, thick, mark=*, mark options={scale=0.7}]
                coordinates {(-1.8,1.6) (-0.8,0.2) (0.2,-0.05) (0.4,-0.02) (0.5,0)};
            \node[cbOrange] at (-0.6,0.5) {large steps};
            \node[cbOrange] at (0.6,-0.1) {small steps};

        % Right: learning-rate schedule
        \nextgroupplot[
            title={Decay of $\alpha(t)$}, xmin=0, xmax=50, ymin=0, ymax=0.35,
            xlabel={$t$}, ylabel={$\alpha$}
        ]
            \addplot[cbBlue, thick, samples=100, domain=0:50] {0.3*exp(-x/15)};
            \addplot[cbGreen, dashed, thick, samples=100, domain=0:50] {0.25*(0.85)^(x)};
            \legend{$\alpha(t)=0.3\, e^{-t/15}$,$\alpha(t)=0.25\cdot0.85^{t}$}
        \end{groupplot}
    \end{tikzpicture}
    \caption{Learning-rate scheduling intuition (schematic). On a smooth objective (left), large initial steps quickly cover ground and roughly align prototypes, while a decaying step-size refines the solution near convergence. Right: common exponential and multiplicative decays used in SOM training.}
    \label{fig:lec5-learning-rate}
\end{figure}


\Cref{fig:lec5-som-component-planes} pairs feature-plane views with U-Matrix diagnostics for SOM audits.

Before delving into the mathematical formulation and algorithmic details of SOMs, it is important to review two foundational concepts that underpin their operation: \emph{clustering} and \emph{dimensionality reduction}.

\subsection{Clustering: Identifying Similarities and Dissimilarities}
\label{sec:som_clustering_identifying_similarities_and_dissimilarities}

Clustering is the process of grouping a set of objects such that objects within the same group (cluster) are more similar to each other than to those in other groups. Formally, given a dataset $\mathcal{X} = \{\mathbf{x}_1, \mathbf{x}_2, \ldots, \mathbf{x}_N\}$ where each $\mathbf{x}_i \in \mathbb{R}^d$ is represented by a feature vector, the goal is to partition the data into $K$ clusters $\{C_1, C_2, \ldots, C_K\}$ such that:
\begin{itemize}
    \item \textbf{Intra-cluster similarity} is maximized: points within the same cluster are close to each other.
    \item \textbf{Inter-cluster dissimilarity} is maximized: points in different clusters are far apart.
\end{itemize}
In the classical formulation used here (e.g., for K-means), the clusters form a partition of $\mathcal{X}$: they are disjoint and their union equals the entire dataset.

\paragraph{Example:} Think of clustering as ``discovering operating modes'' from measurements. Your \(\mathbf{x}_i\) could be a feature vector extracted from vibration spectra, network traffic, or sensor logs. Without labels, you still want the algorithm to separate one regime from another because the points inside a regime are genuinely similar in the feature space.

\paragraph{K-means Clustering:} A classical and widely used clustering algorithm is \emph{K-means}, which operates as follows:
\begin{enumerate}
    \item Initialize $K$ cluster centroids $\{\mathbf{m}_1, \mathbf{m}_2, \ldots, \mathbf{m}_K\}$ randomly.
    \item For each data point $\mathbf{x}_i$, assign it to the cluster with the nearest centroid:
    \begin{equation}
        c_i = \arg\min_{k} \|\mathbf{x}_i - \mathbf{m}_k\|_2,
        \label{eq:auto_som_a96d59060d}
    \end{equation}
    where $\|\cdot\|_2$ denotes the Euclidean norm.
    \item Update each centroid as the mean of all points assigned to it:
    \begin{equation}
        \mathbf{m}_k = \frac{1}{|C_k|} \sum_{\mathbf{x}_i \in C_k} \mathbf{x}_i,
        \label{eq:auto_som_5cd43ad14e}
    \end{equation}
    where $|C_k|$ is the number of points in cluster $C_k$.
    \item Repeat steps 2 and 3 until convergence (i.e., cluster assignments no longer change significantly).
\end{enumerate}

K-means is an unsupervised learning method because it does not require labeled data; it discovers clusters purely based on feature similarity.

Keep the K-means update in mind: SOMs reuse the same prototype-moving idea, but add a neighborhood on a fixed 2D grid so the prototypes are not only learned, but also \emph{organized} for inspection.

\subsection{Dimensionality Reduction: Simplifying High-Dimensional Data}
\label{sec:som_dimensionality_reduction_simplifying_high_dimensional_data}

Dimensionality reduction is what you do when the ambient dimension is too large to see and reason about directly. You accept some information loss, but you try to preserve the relationships you care about (variance, distances, neighborhoods) so the reduced view remains useful. This matters for:
\begin{itemize}
    \item \textbf{Visualization:} Humans can easily interpret data in two or three dimensions.
    \item \textbf{Computational efficiency:} Reducing dimensions can simplify subsequent processing.
    \item \textbf{Noise reduction:} Eliminating irrelevant or redundant features.
\end{itemize}

\paragraph{Example:} Consider a three-dimensional cube. Depending on its orientation, a linear projection (matrix multiplication by \(P: \mathbb{R}^3 \rightarrow \mathbb{R}^2\) with matrix representation in \(\mathbb{R}^{2 \times 3}\)) onto a two-dimensional plane can look like different shapes: a square arises from an orthogonal projection onto a face, whereas a hexagon appears under an oblique projection along a body-diagonal. This highlights that while the combinatorial adjacency (which vertices are connected) is preserved under such a projection, Euclidean lengths and angles are inevitably distorted. \Cref{fig:lec5-mds-projection} illustrates these two views.

\begin{figure}[t]
    \centering
    \begin{tikzpicture}
        \begin{groupplot}[
            group style={group size=2 by 1, horizontal sep=1.5cm},
            width=0.4\linewidth,
            height=0.32\linewidth,
            axis lines=middle,
            xmin=-1, xmax=1,
            ymin=-1, xmax=1,
            xtick={-1,0,1},
            ytick={-1,0,1},
            ticklabel style={font=\scriptsize},
            title style={font=\scriptsize}
        ]
        \nextgroupplot[title={Orthogonal projection}]
            \addplot[thick, cbBlue] coordinates {(-0.6,-0.6) (0.6,-0.6) (0.6,0.6) (-0.6,0.6) (-0.6,-0.6)};
            \addplot[only marks, mark=*, mark options={scale=0.6, fill=cbBlue}] coordinates {
                (-0.6,-0.6) (0.6,-0.6) (0.6,0.6) (-0.6,0.6)
            };
        \nextgroupplot[title={Oblique projection}]
            \addplot[thick, cbOrange] coordinates {
                (-0.7,0) (-0.3,0.55) (0.4,0.55) (0.7,0) (0.3,-0.55) (-0.4,-0.55) (-0.7,0)
            };
            \addplot[only marks, mark=*, mark options={scale=0.6, fill=cbOrange}] coordinates {
                (-0.7,0) (-0.3,0.55) (0.4,0.55) (0.7,0) (0.3,-0.55) (-0.4,-0.55)
            };
        \end{groupplot}
    \end{tikzpicture}
    \caption{Classical MDS intuition (schematic). Projecting a cube onto a plane via an orthogonal map yields a square (left), whereas an oblique projection along a body diagonal produces a hexagon (right). The local adjacency of vertices is preserved even though metric structure is distorted.}
    \label{fig:lec5-mds-projection}
\end{figure}


Common techniques include Principal Component Analysis (PCA), which preserves directions of maximum variance, and classical Multidimensional Scaling (MDS), which reconstructs a configuration whose pairwise distances match the original ones as closely as possible (via double-centering the squared-distance matrix and an eigen-decomposition). Nonlinear methods such as t\hyp{}distributed stochastic neighbor embedding (t\hyp{}SNE) or Uniform Manifold Approximation and Projection (UMAP) emphasize local neighborhoods but typically sacrifice global distance fidelity. SOMs will give us a different kind of reduction: a \emph{discrete} map built from learned prototypes on a 2D grid, not a continuous coordinate chart.

% Chapter 9 (continued)

\subsection{Dimensionality Reduction and Feature Mapping}
\label{sec:som_dimensionality_reduction_and_feature_mapping}

The dimensionality-reduction goals were covered in \Cref{sec:som_dimensionality_reduction_simplifying_high_dimensional_data}; here we make explicit what kind of ``mapping'' a SOM actually learns. A SOM does not return a continuous coordinate system like PCA or classical MDS. Instead, it learns a \emph{finite set} of prototype vectors \(\{\mathbf{w}_i\}\) arranged on a low-dimensional grid with fixed coordinates \(\{\mathbf{r}_i\}\). Each input \(\mathbf{x}\in\mathbb{R}^n\) is mapped to the grid by choosing its best matching unit (BMU),
\[
c(\mathbf{x})=\operatorname*{argmin}_i \|\mathbf{x}-\mathbf{w}_i\|_2^2,
\]
and then using \(\mathbf{r}_{c(\mathbf{x})}\) (or a neighborhood-smoothed variant) as the reduced representation. In that sense, the map \(f:\mathbb{R}^n \to \mathbb{R}^m\) is \emph{implicit}: it is realized by nearest-prototype assignment plus fixed grid coordinates.

This ``discrete embedding'' view is the reason SOMs are useful for visualization and clustering at the same time: nearby grid locations tend to represent nearby regions of the input space, but grid distance is only an \emph{approximate} proxy for geometry. Read the map primarily as neighborhood structure and qualitative ordering, not as a metric-preserving chart.

\begin{tcolorbox}[summarybox, title={How to read a SOM map as a feature representation}]
\begin{itemize}
    \item \textbf{Out-of-sample mapping:} a new point maps to its BMU index \(c(\mathbf{x})\) and grid coordinate \(\mathbf{r}_{c(\mathbf{x})}\).
    \item \textbf{What transfers from PCA/MDS:} you can still ask whether neighborhoods are preserved and whether the map is stable across runs.
    \item \textbf{What does not transfer:} do not treat grid distance as true Euclidean distance in data space; the representation is discrete and only approximately geometry-preserving.
\end{itemize}
\end{tcolorbox}

\subsection{Self-Organizing Maps (SOMs): Introduction}
\label{sec:som_self_organizing_maps_soms_introduction}

Self-Organizing Maps (SOMs), also known as Kohonen maps, sit at a practical intersection: they behave like a prototype-based clustering method, but the prototypes are arranged on a 2D grid so you can inspect how the dataset organizes itself. Unlike supervised neural networks, SOMs learn without explicit target outputs or labels. Instead, they organize a bank of prototype vectors so that nearby units on the grid tend to represent similar inputs.

\begin{tcolorbox}[summarybox, title={SOM at a glance}]
\textbf{What it learns:} Prototype vectors \(\mathbf{w}_i\) in input space, arranged on a fixed 2D grid so that neighboring units tend to represent neighboring regions of the data (topographic mapping).\\
\textbf{Knobs you actually tune:} Map size/topology, the learning-rate schedule \(\alpha(t)\), the neighborhood width \(\sigma(t)\) and its decay, and the distance metric (typically squared Euclidean).\\
\textbf{A practical starting point:} A 2D rectangular grid, squared Euclidean distance, exponential decays for \(\alpha(t)\) and \(\sigma(t)\), and a number of units comparable to (or slightly larger than) the number of clusters you expect to resolve.\\
\textbf{Common ways to fool yourself:} Using a map that is too small, shrinking \(\alpha(t)\) or \(\sigma(t)\) too quickly (the map freezes before it organizes), and reading grid distance as if it were a true metric in data space.
\end{tcolorbox}

\paragraph{Historical Context}

The concept of SOMs traces back to early models of self-organizing topographic maps, such as the two-sheet formulation of \citet{WillshawVonDerMalsburg1976}. Teuvo Kohonen later formalized and popularized the algorithmic framework in \citet{Kohonen1982} (see also \citealp{Kohonen2001}).

\paragraph{Basic Architecture}

Architecturally, a SOM pairs an input vector \(\mathbf{x}\in\mathbb{R}^n\) with a usually two-dimensional grid of units. Each unit \(i\) has (i) a fixed grid coordinate \(\mathbf{r}_i=[u_i,v_i]^\top\) with \(u_i,v_i\in\mathbb{Z}\), and (ii) a prototype (codebook) vector \(\mathbf{w}_i\in\mathbb{R}^n\). The coordinates \(\mathbf{r}_i\) define proximity \emph{on the grid} and enter the neighborhood function (\Cref{sec:som_neighborhood}); the prototypes \(\mathbf{w}_i\) are what you compare to data points when you pick the winner.

Each output neuron therefore possesses a weight vector of the same dimensionality as the input, so evaluating the match between an input and the map amounts to comparing the input against every stored prototype. The neurons then compete; the closest (best matching) unit "wins" and its neighbors are allowed to adapt by nudging their weight vectors toward the input, while distant units remain unchanged during that update. The resulting organization produces a discrete map that preserves qualitative ordering; it approximates the topology of the input space without providing a continuous Euclidean embedding.

\paragraph{Key Concept: Topographic Mapping}

The fundamental idea is simple: inputs that are similar in the original space should activate units that are close to each other on the map. In practice, that means if \(\mathbf{x}_1\) and \(\mathbf{x}_2\) are close in \(\mathbb{R}^n\), then their best matching units should end up near each other on the 2D grid. The neighborhood update is the mechanism that encourages this: when one unit wins, its neighbors are pulled in the same direction, so ``similar inputs'' repeatedly shape the same local region of the map.

If you want a compact formal statement, let \(\mathcal{N}_\epsilon(\mathbf{x})=\{\mathbf{z}\,|\,\|\mathbf{z}-\mathbf{x}\|_2<\epsilon\}\) be a small neighborhood in input space. SOM training aims (approximately) to map \(\mathcal{N}_\epsilon(\mathbf{x})\) into a small neighborhood around the BMU on the grid (see \citealp{Kohonen2001} for details and caveats). The guarantee is not exact; the engineering goal is a stable qualitative ordering that makes the map readable.

\subsection{Conceptual Description of SOM Operation}
\label{sec:som_conceptual_description_of_som_operation}

\begin{enumerate}
    \item \textbf{Initialization:} The weight vectors \(\mathbf{w}_i\) are initialized, often randomly or by sampling from the input space.

    \item \textbf{Competition:} For a given input \(\mathbf{x}\), find the best matching unit (BMU) or winning neuron:
    \begin{equation}
        c = \operatorname*{argmin}_{i} \|\mathbf{x} - \mathbf{w}_i\|_2^2, \label{eq:bmu}
    \end{equation}
    that is, the BMU index \(c\) minimizes the squared Euclidean distance between \(\mathbf{x}\) and the candidate prototype \(\mathbf{w}_i\).
    Minimizing the squared distance yields the same winner as minimizing the unsquared norm but streamlines gradient derivations, so we retain the squared form for consistency with later update rules.

    \item \textbf{Cooperation:} Define a neighborhood function \(h_{ci}(t)\) that determines the degree of influence the BMU has on its neighbors in the output grid. This function decreases with the distance between neurons \(c\) and \(i\) on the map and with time \(t\).

    \item \textbf{Adaptation:} Update the weight vectors of the BMU and its neighbors to move closer to the input vector:
    \begin{equation}
        \mathbf{w}_i(t+1) = \mathbf{w}_i(t) + \alpha(t) h_{ci}(t) \big(\mathbf{x} - \mathbf{w}_i(t)\big), \label{eq:som_update}
    \end{equation}
    where \(\alpha(t)\) is the learning rate, which decreases over time, and the effective width of \(h_{ci}(t)\) likewise shrinks so that large-scale ordering occurs early and fine-tuning occurs later (see \Cref{sec:som_neighborhood}).
\end{enumerate}

\begin{tcolorbox}[summarybox, title={Author's note: distance is part of the model}]
The BMU rule is only as sensible as your distance. Euclidean distance is the default because it makes the update rule simple, but it assumes features are commensurate. In practice you almost always normalize inputs (and sometimes whiten them) so that one coordinate does not dominate the match by sheer scale. If your notion of ``similar'' is directional rather than magnitude-based, cosine distance is often a better choice; if covariance is strongly anisotropic, a Mahalanobis distance can be justified.
\end{tcolorbox}

\begin{tcolorbox}[summarybox, title={Tiny numeric step (online update)}]
Input \(\mathbf{x}=[0.2,0.8]\), two map units with weights \(\mathbf{w}_1=[0.1,0.9]\), \(\mathbf{w}_2=[0.7,0.3]\), coordinates \(\mathbf{r}_1=[0,0]\), \(\mathbf{r}_2=[1,0]\), \(\alpha=0.5\), \(\sigma=1\). BMU \(c=1\) (closest to \(\mathbf{x}\)). Neighborhoods: \(h_{11}=1\), \(h_{21}=\exp(-1/2)\approx 0.607\). Updates:
\[
\mathbf{w}_1\leftarrow [0.15,\,0.85],\quad
\mathbf{w}_2\leftarrow [0.548,\,0.452].
\]
Even the neighbor moves toward \(\mathbf{x}\), illustrating cooperation.
\end{tcolorbox}

This iterative process causes the map to self-organize, with neurons specializing to represent clusters or features of the input space.

\subsection{Mathematical Formulation of SOM}
\label{sec:som_mathematical_formulation_of_som}

Let the input space be \(\mathcal{X} \subseteq \mathbb{R}^n\), and the output map be a grid of neurons indexed by \(i\), each with weight vector \(\mathbf{w}_i \in \mathbb{R}^n\).

\paragraph{Best Matching Unit (BMU)}

We reuse the BMU definition in \eqref{eq:bmu}; the same squared-distance criterion carries into the formal derivation here.

\paragraph{Neighborhood Function}

A common choice for the neighborhood kernel is the Gaussian function
\begin{equation}
    h_{ci}(t) = \exp\left(-\frac{\| \mathbf{r}_c - \mathbf{r}_i \|^2}{2\sigma^2(t)}\right),
    \label{eq:gaussian_neighborhood_short}
\end{equation}
where $\mathbf{r}_i$ denotes the grid coordinates of neuron $i$ and $\sigma(t)$ is the neighborhood radius that decreases monotonically with $t$. A common schedule is an exponential decay:
\begin{equation}
    \sigma(k) = \sigma_0 e^{-k/\tau},
    \label{eq:som_sigma_schedule}
\end{equation}
where $k$ counts updates and $\tau$ sets how quickly the neighborhood shrinks. Early in training $\sigma(t)$ is large, encouraging broad cooperation; as $\sigma(t)$ shrinks, only neurons near the BMU continue to adapt (\Cref{fig:lec5-gaussian-neighborhood}).
\begin{figure}[t]
    \centering
    \begin{tikzpicture}
        \begin{axis}[
            width=0.65\linewidth,
            height=5cm,
            xmin=0, xmax=4,
            ymin=0, ymax=1.05,
            xlabel={Grid distance $\|\mathbf{r}_c-\mathbf{r}_i\|_2$},
            ylabel={$h_{ci}(t)$},
            legend style={at={(0.02,0.98)}, anchor=north west, draw=none, fill=none}
        ]
            \addplot[cbBlue, thick, domain=0:4, samples=200]{exp(-0.5*(x/1.5)^2)};
            \addlegendentry{Early $\sigma(t)$ (broad)}
            \addplot[cbGreen, thick, dashed, domain=0:4, samples=200]{exp(-0.5*(x/0.7)^2)};
            \addlegendentry{Late $\sigma(t)$ (narrow)}
        \end{axis}
    \end{tikzpicture}
    % Avoid inline math in captions; it wraps poorly in some EPUB renderers.
    \caption{Gaussian neighborhood weights in SOM training (schematic). Early iterations use a broad kernel so many neighbors adapt; later iterations shrink the neighborhood width \(\sigma(t)\) so only units near the BMU update.}
    \label{fig:lec5-gaussian-neighborhood}
\end{figure}


% Chapter 9 (continued)

\subsection{Kohonen Self-Organizing Maps (SOMs): Network Architecture and Operation}
\label{sec:som_kohonen_self_organizing_maps_soms_network_architecture_and_operation}

So far we have described SOMs as an algorithmic loop: find the BMU, update the BMU and its neighbors, and anneal the step size and neighborhood width over time. If you prefer to think in terms of a network diagram, this subsection restates the same story as a concrete architecture you can trace and implement.

\paragraph{Network Structure}

The map is a fixed 2D grid of units. Each unit \(i\) stores a prototype vector \(\mathbf{w}_i\) in input space and has a fixed grid coordinate \(\mathbf{r}_i\). Every input is compared against every stored prototype.

\paragraph{Mapping and Competition}

For a given input \(\mathbf{x}\), the units compete by similarity (or distance) between \(\mathbf{x}\) and \(\mathbf{w}_i\). The closest unit wins; this is the BMU rule in \eqref{eq:bmu}.

\paragraph{Weight Update Rule}

Only the winning unit and its neighbors on the grid update; the update rule is \eqref{eq:som_update}. The important qualitative effect is that one input does not just move one prototype; it moves a small \emph{patch} of prototypes in the same direction, which is what makes the grid organize into a readable map rather than a bag of unrelated cluster centers.

\subsection{Example: SOM with a \texorpdfstring{\(3 \times 3\)}{3x3} Output Map and 4-Dimensional Input}
\label{sec:som_example_som_with_a_3_3_3x3_output_map_and_4_dimensional_input}

To make the symbols concrete, consider a SOM whose inputs live in \(\mathbb{R}^4\) and whose output map is a \(3\times 3\) grid. Each unit \(i\in\{1,\dots,9\}\) stores a prototype vector \(\mathbf{w}_i\in\mathbb{R}^4\). For a single input \(\mathbf{x}=[x_1,x_2,x_3,x_4]^\top\), the job of the map is to (i) decide which prototype matches best, then (ii) move that prototype \emph{and a small neighborhood around it} toward \(\mathbf{x}\).

\paragraph{Feedforward Computation}

For a given input \(\mathbf{x}\), each neuron computes a similarity score. Two common choices are:
\begin{align}
    y_i &= \mathbf{w}_i^\top \mathbf{x} && \text{(dot-product similarity)}, \label{eq:neuron_activation}\\
    d_i &= \|\mathbf{x} - \mathbf{w}_i\|_2^2 && \text{(squared Euclidean distance)}. \label{eq:neuron_distance}
\end{align}
In both expressions \(\mathbf{w}_i\) and \(\mathbf{x}\) are column vectors, so \( \mathbf{w}_i^\top \mathbf{x}\) is a scalar similarity score while \(d_i\) computes the squared Euclidean distance.

When using dot products we select the neuron with the maximum \(y_i\); when using distances we equivalently select the neuron with the minimum \(d_i\) (or the maximum of \(-d_i\)):
\[
c =
\begin{cases}
\arg \max_i y_i, & \text{if similarities are measured via } \eqref{eq:neuron_activation},\\
\arg \min_i d_i, & \text{if distances are used as in } \eqref{eq:neuron_distance}.
\end{cases}
\]
In practice the distance form is the most common in SOM code. If you do use dot-product similarity, normalize inputs (and often prototypes) so that ``largest dot product'' corresponds to your notion of ``closest.''

\paragraph{Weight Initialization and Update}

Weights \(\mathbf{w}_i\) are typically initialized randomly (often by sampling from the data) or by a PCA-style seeding if you want a stable orientation. One training step then has a predictable shape: compute the match scores (usually \(d_i\)), pick the BMU \(c\), compute neighborhood weights from the grid distances \(\|\mathbf{r}_i-\mathbf{r}_c\|\), and apply the update in \eqref{eq:som_update} to the BMU and a small neighborhood patch. For a fully worked numeric update with actual numbers, see the earlier \emph{Tiny numeric step (online update)} box; the point of the present \(3\times 3\) example is to keep the dimensions explicit so you can track what gets computed where.

This process repeats over many inputs, gradually organizing the map such that neighboring neurons respond to similar inputs, effectively performing a topology-preserving dimensionality reduction.

The grid coordinates \(\mathbf{r}_i \in \mathbb{Z}^2\) introduced for the neighborhood kernel serve as the geometry of the output map; distances such as \(\|\mathbf{r}_i - \mathbf{r}_c\|_2\) determine how strongly each neuron responds when \(c\) wins. Broad kernels (large \(\sigma(t)\)) encourage global ordering early in training, whereas shrinking \(\sigma(t)\) confines adaptation to local neighborhoods so that fine-grained structure emerges. Alternative kernel shapes (e.g., Epanechnikov, bubble) can be used, though Gaussians provide smooth decay and convenient derivatives.

SOM training is typically stochastic: each input triggers an update, so the map continuously refines prototypes as data arrive. Batch variants exist, but online updates capture streaming data and mirror Kohonen's original algorithm.

\subsection{Key Properties of Kohonen SOMs}
\label{sec:som_key_properties_of_kohonen_soms}

\begin{itemize}
    \item \textbf{Fixed output dimension:} The grid size is a design choice specified a priori and does not automatically scale with the input dimension.
    \item \textbf{Winner-takes-all competition:} Only the best matching unit and its neighbors adapt their weights, encouraging topological ordering.
    \item \textbf{Neighborhood cooperation:} Updating neighboring neurons enforces smooth transitions across the map.
\end{itemize}
% Chapter 9 (continued)

\subsection{Winner-Takes-All Learning and Weight Update Rules}
\label{sec:som_winner_takes_all_learning_and_weight_update_rules}

Recall that in competitive learning networks, the neuron with the highest discriminant value for a given input \(\mathbf{x}\) is declared the \emph{winner}. This subsection analyzes the classical \emph{winner-takes-all} (WTA) principle in which only the winning neuron updates its weights, while all others remain unchanged. In the SOM setting discussed earlier, a softened variant is used in which the winner and its grid neighbors update together.

\paragraph{Discriminant Function and Similarity Measures}

The discriminant measures in \eqref{eq:neuron_activation} (dot-product similarity) and \eqref{eq:neuron_distance} (squared Euclidean distance) are the same ones used in the earlier example; we reuse them here, favoring the distance form when deriving updates.

\paragraph{Weight Update Rule}

Once the winning neuron \(c\) is identified, WTA is the SOM update \eqref{eq:som_update} with a collapsed neighborhood: set \(h_{ci}(t)=1\) for \(i=c\) and \(h_{ci}(t)=0\) otherwise. The learning rate \(\alpha(t)\) still controls step size so the winner moves toward \(\mathbf{x}\) gradually rather than collapsing to it in a single update.

\paragraph{Learning Rate Schedule}

The learning rate \(\alpha(t)\) controls the magnitude of weight updates. It typically decreases over time to ensure convergence and stability:

\[
\alpha(t+1) \leq \alpha(t), \quad \lim_{t \to \infty} \alpha(t) = 0.
\]

This schedule allows large adjustments early in training (rapid learning) and fine-tuning later (stabilization).
Practitioners often start with \(\alpha(0)\) in the range \(0.05\)--\(0.5\) and decay it toward \(10^{-3}\) or smaller so that updates remain responsive initially but become conservative as the map stabilizes.

\paragraph{Summary of the Competitive Learning Algorithm}

\begin{enumerate}
    \item Initialize weights \(\mathbf{w}_j(0)\) randomly or heuristically.
    \item For each input \(\mathbf{x}\):
    \begin{enumerate}
        \item Compute discriminant functions \(g_j(\mathbf{x})\) or distances \(d_j(\mathbf{x})\).
        \item Select winning neuron:
        \[
        c = \arg \max_j g_j(\mathbf{x}) \quad \text{or} \quad c = \arg \min_j d_j(\mathbf{x})
        \]
        \item Update the winning neuron's weights using \eqref{eq:som_update} with \(h_{ci}(t)\) collapsed to a winner-only update.
    \end{enumerate}
    \item Decrease learning rate \(\alpha(t)\) according to schedule.
    \item Repeat until convergence or maximum iterations reached.
\end{enumerate}

\subsection{Numerical Example of Competitive Learning}
\label{sec:som_numerical_example_of_competitive_learning}

Consider a simple example with:

\begin{itemize}
    \item Four input vectors \(\mathbf{x}_1, \mathbf{x}_2, \mathbf{x}_3, \mathbf{x}_4 \in \mathbb{R}^4\).
    \item A competitive layer with three neurons (clusters).
    \item Initial learning rate \(\alpha(0) = 0.3\) with multiplicative decay \(\alpha(t) = 0.3 \times 0.5^{t}\) (ensuring \(\alpha(t) > 0\)).
    \item No neighborhood function (i.e., only the winner updates).
\end{itemize}

\paragraph{Initial Weights}

The initial weights \(\mathbf{w}_j(0)\) for neurons \(j=1,2,3\) are:

\[
\mathbf{W}(0) =
\begin{bmatrix}
0.2 & 0.3 & 0.5 & 0.1 \\
0.2 & 0.3 & 0.1 & 0.4 \\
0.3 & 0.5 & 0.2 & 0.3
\end{bmatrix}
\]

where row \(j\) contains the initial weight vector \(\mathbf{w}_j(0)\) for neuron \(j = 1,2,3\).

\paragraph{Input vectors}
Let the four inputs be
\[
\mathbf{x}_1=
\begin{bmatrix}
0.1\\ 0.3\\ 0.4\\ 0.2
\end{bmatrix},\quad
\mathbf{x}_2=
\begin{bmatrix}
0.8\\ 0.7\\ 0.2\\ 0.1
\end{bmatrix},\quad
\mathbf{x}_3=
\begin{bmatrix}
0.2\\ 0.2\\ 0.1\\ 0.9
\end{bmatrix},\quad
\mathbf{x}_4=
\begin{bmatrix}
0.4\\ 0.6\\ 0.5\\ 0.3
\end{bmatrix}.
\]
We walk through one update explicitly for \(\mathbf{x}_1\); the remaining inputs follow the same steps.

\paragraph{Step 1: pick the winner for \(\mathbf{x}_1\).}
Using squared Euclidean distance \(d_j=\|\mathbf{x}_1-\mathbf{w}_j(0)\|_2^2\),
\begin{align*}
d_1 &= (0.2-0.1)^2 + (0.3-0.3)^2 + (0.5-0.4)^2 + (0.1-0.2)^2 = 0.03,\\
d_2 &= \|\mathbf{x}_1-\mathbf{w}_2(0)\|_2^2 = 0.14,\qquad
d_3 = \|\mathbf{x}_1-\mathbf{w}_3(0)\|_2^2 = 0.13.
\end{align*}
So the winner is \(c=\arg\min_j d_j = 1\).

\paragraph{Step 2: update the winner.}
At \(t=0\) the learning rate is \(\alpha(0)=0.3\), and only the winner updates:
\[
\mathbf{w}_1(1)=\mathbf{w}_1(0)+\alpha(0)\big(\mathbf{x}_1-\mathbf{w}_1(0)\big)
=
\begin{bmatrix}
0.17\\ 0.30\\ 0.47\\ 0.13
\end{bmatrix}.
\]
For the next input, \(t=1\) so \(\alpha(1)=0.3\times 0.5 = 0.15\), and the same computation repeats with the new winner.

% QC-BEGIN: som_competitive_learning_example
% d 0.03 0.14 0.13 winner 1
% w1_new 0.17 0.30 0.47 0.13
% alpha0 0.30 alpha1 0.15
% QC-END: som_competitive_learning_example

\subsection{Winner-Takes-All Learning Recap}
\label{sec:som_winner_takes_all_learning_recap}

The WTA selection and update were defined in \Cref{sec:som_winner_takes_all_learning_and_weight_update_rules}; the numerical example above uses the same BMU criterion \eqref{eq:bmu} and the winner-only form of the update in \eqref{eq:som_update}.

\paragraph{How WTA relates to SOMs.}
You can view WTA learning as a limiting case of SOM training where the neighborhood collapses to a single unit: \(h_{ci}(t)=\mathbf{1}[i=c]\). Adding a nontrivial neighborhood \(h_{ci}(t)\) is what turns pure prototype learning into a \emph{topographic} map: neighbors are encouraged to represent similar inputs, and the grid becomes a structured visualization rather than an unstructured codebook.

\paragraph{Practical considerations.}
In both SOMs and WTA networks, input vectors are commonly normalized (e.g., zero mean and unit variance) so that distance comparisons are meaningful. Training is typically terminated when weight changes fall below a small threshold or after a prescribed number of epochs.

\subsection{Regularization and Monitoring During SOM Training}
\label{sec:som_regularization_and_monitoring_during_som_training}

Even though SOMs are inherently unsupervised, their training dynamics still benefit from the same regularization heuristics used in supervised settings. Two complementary diagnostics are especially useful in practice.

\paragraph{Bias--variance view.} Increasing the grid resolution or keeping the kernel width large for too long can overfit local noise. \Cref{fig:lec5-bias-variance} visualizes the familiar \(U\)-shaped trade-off: the left regime underfits (high bias), whereas the right regime yields jagged maps (high variance).

\begin{figure}[t]
    \centering
    \begin{tikzpicture}
        \begin{axis}[
            width=0.82\linewidth,
            height=5cm,
            xlabel={Model capacity},
            ylabel={Error},
            xmin=0, xmax=1,
            ymin=0, ymax=0.9,
            legend style={at={(0.5,1.04)}, anchor=south, legend columns=3}
        ]
            \addplot[cbBlue, thick, domain=0:1, samples=200]{0.55*exp(-4*x) + 0.08};
            \addlegendentry{Bias}
            \addplot[cbOrange, thick, domain=0:1, samples=200]{0.12 + 0.65*x^1.8};
            \addlegendentry{Variance}
            \addplot[cbGreen, thick, domain=0:1, samples=200]{0.55*exp(-4*x) + 0.65*x^1.8 + 0.08};
            \addlegendentry{Total error}
            \addplot[cbPink, dashed, domain=0:1, samples=2]{0.3};
            \node[anchor=west, font=\scriptsize, text=cbPink] at (axis cs:0.45,0.32){validation floor};
        \end{axis}
    \end{tikzpicture}
    \caption{Illustrative bias--variance trade-off when sweeping SOM capacity (number of units or kernel width). The optimum appears near the knee where bias and variance intersect.}
    \label{fig:lec5-bias-variance}
\end{figure}


\paragraph{Loss-landscape smoothing.} Adding small cooperative penalties (e.g., weight decay between neighbors) produces smoother loss contours and accelerates convergence, as sketched in \Cref{fig:lec5-regularization}. The penalty discourages neighboring prototypes from diverging and keeps the map topologically ordered.

\begin{figure}[t]
    \centering
    \begin{tikzpicture}
        \begin{groupplot}[
            group style={group size=2 by 1, horizontal sep=2.2cm},
            width=0.42\linewidth,
            height=4.8cm,
            xmin=-2, xmax=2,
            ymin=-2, ymax=2,
            xtick=\empty,
            ytick=\empty,
            xlabel={$w_1$},
            ylabel={$w_2$},
            xlabel style={at={(axis description cs:0.5,-0.08)}, anchor=north},
            ylabel style={at={(axis description cs:-0.06,0.5)}, anchor=south},
            title style={yshift=0.2em},
            view={35}{25},
            samples=35,
            samples y=35,
            ztick=\empty
        ]
            \nextgroupplot[title={Unregularized}]
                \ifdefined\HCode
                    \addplot3[surf, shader=flat, opacity=0.92, domain=-2:2, y domain=-2:2]{0.4*x^2 + 0.1*y^2 + 0.8*x*y};
                \else
                    \addplot3[surf, shader=interp, opacity=0.92, domain=-2:2, y domain=-2:2]{0.4*x^2 + 0.1*y^2 + 0.8*x*y};
                \fi
            \nextgroupplot[title={Neighbor-coupled}]
                \ifdefined\HCode
                    \addplot3[surf, shader=flat, opacity=0.92, domain=-2:2, y domain=-2:2]{0.25*x^2 + 0.25*y^2 + 0.2*(x-y)^2};
                \else
                    \addplot3[surf, shader=interp, opacity=0.92, domain=-2:2, y domain=-2:2]{0.25*x^2 + 0.25*y^2 + 0.2*(x-y)^2};
                \fi
        \end{groupplot}
    \end{tikzpicture}
    \caption{Regularization smooths an objective surface (schematic). Coupling neighboring prototypes (right) yields wider, flatter basins than the jagged unregularized landscape (left).}
    \label{fig:lec5-regularization}
\end{figure}


\paragraph{Quantization vs. information preservation.} Classical SOM optimizes a topology-preserving vector quantization objective; it does not include cross\hyp{}entropy terms. Modern variants sometimes introduce \emph{auxiliary} regularizers to encourage codebook utilization (e.g., entropy penalties on assignment histograms) or draw analogies to VQ\hyp{}VAE. Monitoring both quantization error and an entropy-style regularizer, as in \Cref{fig:lec5-crossentropy}, helps reveal when the map is collapsing to a few units or when density variations are no longer represented faithfully.

\begin{figure}[t]
    \centering
    \begin{tikzpicture}
        \begin{axis}[
            width=0.75\linewidth,
            height=5.2cm,
            xlabel={Quantization error},
            ylabel={Entropy penalty},
            zlabel={Objective},
            view={40}{30},
            domain=0:1,
            y domain=0:1,
            samples=31,
            samples y=31,
            colormap/viridis
        ]
            \ifdefined\HCode
                \addplot3[surf, shader=flat, opacity=0.95]{0.6*(x-0.35)^2 + 0.4*(y-0.55)^2 + 0.25*x*(1-y)};
            \else
                \addplot3[surf, shader=interp, opacity=0.95]{0.6*(x-0.35)^2 + 0.4*(y-0.55)^2 + 0.25*x*(1-y)};
            \fi
        \end{axis}
    \end{tikzpicture}
    % Avoid inline math in captions; it wraps poorly in some EPUB renderers.
    \caption{Illustrative objective surface combining quantization error and an entropy-style regularizer (a modern SOM variant; for example, a negative sum of \(p\log p\) over unit usage). Valleys arise when prototypes cover the space evenly; ridges highlight collapse or poor topological preservation.}
    \label{fig:lec5-crossentropy}
\end{figure}


\paragraph{Quantization vs. topographic error.} Given data points \(\{\mathbf{x}_i\}\) and best-matching units \(b_i = \operatorname{BMU}(\mathbf{x}_i)\), the \emph{quantization error} is
\[
    \text{QE} = \frac{1}{N}\sum_{i=1}^N \bigl\|\mathbf{x}_i - \mathbf{w}_{b_i}\bigr\|_2,
\]
which measures reconstruction fidelity. The \emph{topographic error} is the fraction of inputs whose first- and second-best BMUs are not adjacent on the grid (default: 4-neighbor connectivity), capturing topology preservation. Both metrics reappear in later figures; we monitor QE for representation quality and TE for magnification distortions.

\begin{tcolorbox}[summarybox, title={Tiny numeric check: QE vs.\ TE (they can disagree)}]
To see what these metrics measure, consider a \(2\times 2\) SOM grid with fixed coordinates
\(\mathbf{r}_1=[0,0]\), \(\mathbf{r}_2=[1,0]\), \(\mathbf{r}_3=[0,1]\), \(\mathbf{r}_4=[1,1]\) (4-neighbor adjacency). Let the prototypes in input space be
\[
\mathbf{w}_1=[0,0],\quad \mathbf{w}_2=[2,0],\quad \mathbf{w}_3=[0,2],\quad \mathbf{w}_4=[0.25,0.25].
\]
Notice that \(\mathbf{w}_4\) is close to \(\mathbf{w}_1\) in input space, even though units 4 and 1 are diagonal neighbors on the grid.

Now evaluate four inputs, using squared Euclidean distance to pick the best and second-best matches:
\[
\mathbf{x}_1=[0.30,0.30],\ \mathbf{x}_2=[1.80,0.20],\ \mathbf{x}_3=[0.20,1.70],\ \mathbf{x}_4=[0.05,0.05].
\]
The BMU/second-BMU pairs are \((4,1)\), \((2,4)\), \((3,4)\), \((1,4)\). Under 4-neighbor adjacency, pairs \((4,1)\) and \((1,4)\) are \emph{not} adjacent (diagonal), so \(\text{TE}=2/4=0.5\).
Meanwhile the average distance to the BMU is \(\text{QE}\approx 0.1962\).

The lesson is that QE is a ``prototype fit'' score, while TE is an ``ordering'' score. You can improve QE by moving prototypes toward the data while still tearing the map if nearby grid units do not represent nearby regions of the input space.
\end{tcolorbox}

% QC-BEGIN: som_qe_te_example
% w1 0.00 0.00
% w2 2.00 0.00
% w3 0.00 2.00
% w4 0.25 0.25
% x1 0.30 0.30 bmu 4 sbmu 1 adj 0
% x2 1.80 0.20 bmu 2 sbmu 4 adj 1
% x3 0.20 1.70 bmu 3 sbmu 4 adj 1
% x4 0.05 0.05 bmu 1 sbmu 4 adj 0
% QE 0.196205 TE 0.500000
% QC-END: som_qe_te_example

\begin{tcolorbox}[summarybox, title={Batch SOM in practice}, breakable]
Online SOM updates one sample at a time: pick a best-matching unit (BMU), nudge it and its neighbors, move on. Batch SOM instead aggregates responsibilities across a dataset (or mini\hyp{}batch) before shifting prototypes:
\begin{align*}
h_{j, i}(t) &= \kappa\big(\text{dist}(j, b(i)); \sigma_t\big),\\
\mathbf{w}_j^{(t+1)} &= \frac{\sum_i h_{j, i}(t)\,\mathbf{x}_i}{\sum_i h_{j, i}(t)}.
\end{align*}
Key differences:
\begin{itemize}
    \item \textbf{Deterministic passes.} Batch updates remove stochastic noise and converge in fewer epochs on static datasets, making results reproducible (useful for dashboards/visual analytics).
    \item \textbf{Parallelism.} Computations collapse to matrix ops (compute BMUs, accumulate weighted sums), so GPUs/CPUs can process large mini\hyp{}batches efficiently.
    \item \textbf{Streaming trade-off.} Online updates remain preferable when data arrive continuously or when you need the map to adapt mid-stream; batch SOM suits offline datasets.
\end{itemize}
Most modern SOM libraries expose both modes, so choose the update rule that matches your data pipeline and stability requirements.
\end{tcolorbox}

\paragraph{Stopping criteria.} Because stochastic updates can eventually increase topographic error, it is standard to stop training once a moving-average validation curve plateaus. \Cref{fig:lec5-early-stopping} shows the canonical trend: fast initial improvement followed by saturation.

\begin{figure}[t]
    \centering
    \begin{tikzpicture}
        \begin{axis}[
            width=0.82\linewidth,
            height=5cm,
            xlabel={Epoch},
            ylabel={Error},
            xmin=0, xmax=60,
            ymin=0, ymax=1,
            legend style={at={(0.5,1.03)}, anchor=south, legend columns=2}
        ]
            \addplot[cbBlue, thick, smooth] table {
                epoch quant
                0 0.95
                5 0.78
                10 0.6
                15 0.48
                20 0.39
                25 0.33
                30 0.29
                35 0.27
                40 0.26
                45 0.26
                50 0.27
                55 0.29
                60 0.32
            };
            \addlegendentry{Quantization error}
            \addplot[cbOrange, thick, smooth, dashed] table {
                epoch topo
                0 0.9
                5 0.72
                10 0.55
                15 0.43
                20 0.35
                25 0.3
                30 0.27
                35 0.25
                40 0.245
                45 0.245
                50 0.25
                55 0.27
                60 0.3
            };
            \addlegendentry{Topographic error}
            \addplot[cbPink, very thick, domain=0:60]{0.25} node[pos=0.62, anchor=south west, font=\scriptsize, text=cbPink]{stop window};
        \end{axis}
    \end{tikzpicture}
    \caption{Illustrative validation curves used to identify an early\hyp{}stopping knee. When both quantization and topographic errors flatten (shaded band), further training risks map drift.}
    \label{fig:lec5-early-stopping}
\end{figure}


\subsection{Limitations of Winner-Takes-All and Motivation for Cooperation}
\label{sec:som_limitations_of_winner_takes_all_and_motivation_for_cooperation}

While WTA is simple and effective for clustering, it has some limitations:
\begin{itemize}
    \item Only one neuron updates per input, which can lead to slow convergence.
    \item The hard competition ignores relationships among neighboring neurons.
    \item The resulting clusters correspond to hard assignments, so boundaries between codebook vectors are sharp with little smoothing across neighboring neurons.
\end{itemize}
There is also a very practical failure mode: you can drive the prototypes toward the data (so QE improves) while the \emph{grid organization} never really forms. If you start with \(\sigma(t)\approx 1\) from the very beginning (or you drop neighborhood updates entirely), nearby grid locations can end up representing unrelated regions of the input space. The map looks like a shuffled deck: the U-Matrix becomes speckled, and TE stays stubbornly high because the second-best match for a point often lives far away on the grid. The fix is not a new objective; it is simply the training schedule: use a broad neighborhood early so large-scale ordering can settle, then shrink \(\sigma(t)\) so the map can refine without smearing away detail.
More importantly for SOMs, plain WTA gives you \emph{prototypes} but not necessarily a readable \emph{map}. If only the winner moves, nothing forces neighboring grid units to represent neighboring regions of the data; the prototypes can end up arranged arbitrarily across the grid, and small changes in initialization can produce very different-looking maps. Cooperation is the fix: early in training you use a broad neighborhood (large \(\sigma(t)\)) so an input shapes a local patch rather than a single unit, which encourages global ordering; later you shrink \(\sigma(t)\) so the map can refine without smearing away detail.
The geometric effect of these limitations is easiest to see in \Cref{fig:lec5-softmax-regions}: the left panel shows the brittle Voronoi partitions created by a strict winner-takes-all rule, whereas the right panel demonstrates how shrinking the neighborhood kernel produces softer responsibilities and smoother maps.

\pgfmathdeclarefunction{somindex}{2}{%
    \pgfmathsetmacro{\da}{(#1-0.2)^2 + (#2-0.2)^2}%
    \pgfmathsetmacro{\db}{(#1-0.8)^2 + (#2-0.25)^2}%
    \pgfmathsetmacro{\dc}{(#1-0.28)^2 + (#2-0.78)^2}%
    \pgfmathsetmacro{\dd}{(#1-0.78)^2 + (#2-0.78)^2}%
    \pgfmathparse{%
        (\da<=\db) && (\da<=\dc) && (\da<=\dd)? 0:
        ((\db<=\da) && (\db<=\dc) && (\db<=\dd)? 1:
        ((\dc<=\da) && (\dc<=\db) && (\dc<=\dd)? 2: 3))}%
}
\pgfmathdeclarefunction{somsoftpeak}{2}{%
    \pgfmathsetmacro{\beta}{8}%
    \pgfmathsetmacro{\ta}{exp(-\beta*((#1-0.2)^2 + (#2-0.2)^2))}%
    \pgfmathsetmacro{\tb}{exp(-\beta*((#1-0.8)^2 + (#2-0.25)^2))}%
    \pgfmathsetmacro{\tc}{exp(-\beta*((#1-0.28)^2 + (#2-0.78)^2))}%
    \pgfmathsetmacro{\td}{exp(-\beta*((#1-0.78)^2 + (#2-0.78)^2))}%
    \pgfmathsetmacro{\sumw}{\ta+\tb+\tc+\td}%
    \pgfmathsetmacro{\maxw}{max(max(\ta,\tb), max(\tc,\td))}%
    \pgfmathparse{\maxw/\sumw}%
}
\pgfplotsset{colormap={somregions}{
        color(0cm)=(cbBlue);
        color(0.33cm)=(cbOrange);
        color(0.66cm)=(cbGreen);
        color(1cm)=(cbPink)
}}
\begin{figure}[t]
    \centering
    \begin{tikzpicture}
        \begin{groupplot}[
            group style={group size=2 by 1, horizontal sep=1.1cm},
            width=0.4\linewidth,
            height=0.42\linewidth,
            xmin=0, xmax=1,
            ymin=0, ymax=1,
            xlabel={$x_1$},
            ylabel={$x_2$},
            axis on top,
            enlargelimits=false
        ]
            \nextgroupplot[
                title={Hard BMU regions},
                view={0}{90},
                colorbar style={title={Prototype}, yshift=-2ex},
                zmin=0, zmax=3,
                colormap name=somregions,
                ticks=none
            ]
                \addplot3[surf, shader=flat, point meta=rawz,
                    domain=0:1, y domain=0:1, samples=45, samples y=45]
                    {somindex(x, y)};
                \addplot+[only marks, mark=*, mark size=1.8pt, color=cbBlue] coordinates {(0.2,0.2)};
                \addplot+[only marks, mark=*, mark size=1.8pt, color=cbOrange] coordinates {(0.8,0.25)};
                \addplot+[only marks, mark=*, mark size=1.8pt, color=cbGreen] coordinates {(0.28,0.78)};
                \addplot+[only marks, mark=*, mark size=1.8pt, color=cbPink] coordinates {(0.78,0.78)};
                \node[font=\scriptsize, anchor=west, cbBlue] at (axis cs:0.22,0.22){$\mathbf{w}_1$};
                \node[font=\scriptsize, anchor=west, cbOrange] at (axis cs:0.82,0.25){$\mathbf{w}_2$};
                \node[font=\scriptsize, anchor=south west, cbGreen] at (axis cs:0.3,0.82){$\mathbf{w}_3$};
                \node[font=\scriptsize, anchor=south east, cbPink] at (axis cs:0.78,0.82){$\mathbf{w}_4$};
            \nextgroupplot[
                title={Soft assignments},
                view={0}{90},
                colormap/viridis,
                colorbar style={title={Max softmax prob}, yshift=-2ex},
                ticks=none
            ]
                \addplot3[surf, shader=flat, point meta=rawz,
                    domain=0:1, y domain=0:1, samples=45, samples y=45]
                    {somsoftpeak(x, y)};
                \addplot+[only marks, mark=*, mark size=1.8pt, color=black, fill=white] coordinates {(0.2,0.2)(0.8,0.25)(0.28,0.78)(0.78,0.78)};
                \node[font=\scriptsize, anchor=south east, fill=white, inner sep=1pt] at (axis cs:0.98,0.05){$\sigma(t)\ \text{small}$};
            \end{groupplot}
    \end{tikzpicture}
    \caption{Voronoi-like regions induced by fixed prototypes in input space (left) and the corresponding soft assignments after sharpening the neighborhood kernel (right). Softer updates blur the decision frontiers and reduce jagged mappings between adjacent units (schematic illustration).}
    \label{fig:lec5-softmax-regions}
\end{figure}


To address these issues, the concept of \emph{cooperation} among neurons is introduced. Instead of a single winner neuron updating its weights, a neighborhood of neurons around the winner also update their weights, albeit to a lesser extent. This idea leads to smoother mappings and better topological ordering.

\subsection{Cooperation in Competitive Learning}
\label{sec:som_neighborhood}

\paragraph{Neighborhood Concept}

Consider the output layer arranged in a 2D grid of neurons. For each input \(\mathbf{x}\), after determining the winning neuron \(c\), we define a neighborhood \(\mathcal{N}(c)\) consisting of neurons close to \(c\) on the grid. In practice the neighborhood weight is supplied by the kernel \(h_{jc}(t)\) of \eqref{eq:gaussian_neighborhood_short}, which is positive for units inside the neighborhood (and decays with the grid distance \(\|\mathbf{r}_j - \mathbf{r}_c\|\)) and zero for units far away.

The neighborhood size typically shrinks over time during training, starting large to encourage global ordering and gradually reducing to fine-tune local details.

\paragraph{Weight Update with Neighborhood Cooperation}
The grid structure and how the best matching unit (BMU) influences nearby neurons are visualized in \Cref{fig:lec5-som-lattice-umatrix}. The U-Matrix on the right provides a quick diagnostic for cluster boundaries during training.

\begin{figure}[t]
    \centering
    \begin{tikzpicture}[scale=0.9]
        % Left panel: grid
        \begin{scope}
            % draw grid of neurons
            \foreach \i in {0,...,4} {
                \foreach \j in {0,...,4} {
                    \filldraw[fill=white, draw=gray!60] (\i,\j) circle (2.2pt);
                }
            }
            % BMU and neighbors
            \filldraw[fill=cbBlue, draw=cbBlue] (2,2) circle (2.4pt);
            \foreach \p in {(1,2),(3,2),(2,1),(2,3),(1,1),(1,3),(3,1),(3,3)} {
                \filldraw[fill=cbGreen!60, draw=cbGreen!60] \p circle (2.2pt);
            }
            \draw[cbBlue, thick] (2,2) circle (1.2);
            % lattice distance scale (one-step neighbor)
            \draw[<->, gray!70] (2,2.2) -- (3,2.2);
            \node[font=\scriptsize, gray!70] at (2.5,2.45) {$\|\mathbf{r}_j-\mathbf{r}_c\|=1$};
            \node at (2,-0.5) {$\mathbf{r}_c$};
            \node[align=center] at (2,-1.1) {SOM grid and BMU neighborhood};
        \end{scope}

        % Right panel: U-Matrix (toy heatmap; colored but grayscale-friendly)
        \begin{scope}[shift={(7,0)}]
            \begin{axis}[
                width=0.40\linewidth,
                height=0.40\linewidth,
                view={0}{90},
                xmin=-0.5, xmax=4.5,
                ymin=-0.5, ymax=4.5,
                xtick=\empty, ytick=\empty,
                axis lines=none,
                colormap/viridis,
                point meta min=0.30,
                point meta max=0.90,
                colorbar,
                colorbar style={
                    title={avg.\ distance},
                    ytick={0.30,0.60,0.90},
                    yticklabel style={font=\scriptsize},
                    title style={font=\scriptsize},
                },
                title={U-Matrix (neighbor distances)},
                title style={font=\scriptsize},
            ]
                \addplot[
                    matrix plot*,
                    mesh/cols=5,
                    mesh/rows=5,
                    point meta=explicit,
                ] table [meta=z] {
                    x y z
                    0 4 0.85
                    1 4 0.65
                    2 4 0.60
                    3 4 0.70
                    4 4 0.90
                    0 3 0.75
                    1 3 0.55
                    2 3 0.50
                    3 3 0.65
                    4 3 0.85
                    0 2 0.70
                    1 2 0.55
                    2 2 0.45
                    3 2 0.60
                    4 2 0.80
                    0 1 0.65
                    1 1 0.35
                    2 1 0.30
                    3 1 0.40
                    4 1 0.70
                    0 0 0.80
                    1 0 0.60
                    2 0 0.55
                    3 0 0.60
                    4 0 0.85
                };
            \end{axis}
        \end{scope}
    \end{tikzpicture}
    % Avoid inline math in captions; it wraps poorly in some EPUB renderers.
    \caption{Left: a 5-by-5 SOM grid with the best matching unit (blue) and neighbors inside the Gaussian-kernel radius (green). Right: a toy U-Matrix (grayscale-safe colormap) showing average distances between neighboring codebook vectors; larger distances suggest likely cluster boundaries. Treat a U-Matrix as a qualitative boundary hint unless preprocessing and scaling are fixed.}
    \label{fig:lec5-som-lattice-umatrix}
\end{figure}

\begin{figure}[t]
    \centering
            \begin{tikzpicture}
                    \begin{groupplot}[
                        group style={group size=3 by 1, horizontal sep=0.55cm},
                        width=0.30\linewidth,
                        height=0.30\linewidth,
                        view={0}{90},
                        xmin=-0.5, xmax=4.5,
                        ymin=-0.5, ymax=4.5,
                        xtick=\empty, ytick=\empty,
                        colormap/viridis,
                        point meta min=0,
                        point meta max=0.7,
                        nodes near coords,
                        nodes near coords style={font=\scriptsize},
                        every node near coord/.append style={fill=white, fill opacity=0.8, text opacity=1, inner sep=0.6pt},
                        mesh/check=false
                    ]
                \nextgroupplot[title={Plane: pixel 1}]
                    \addplot[matrix plot*, mesh/cols=5, mesh/rows=5, point meta=explicit] table [meta=z] {
                        x y z
                        0 0 0.2
                    1 0 0.3
                    2 0 0.4
                    3 0 0.6
                    4 0 0.7
                    0 1 0.1
                    1 1 0.2
                    2 1 0.3
                    3 1 0.5
                    4 1 0.6
                    0 2 0.0
                    1 2 0.1
                    2 2 0.2
                    3 2 0.4
                    4 2 0.5
                    0 3 0.0
                    1 3 0.1
                    2 3 0.2
                    3 3 0.3
                    4 3 0.4
                    0 4 0.0
                    1 4 0.1
                    2 4 0.1
                    3 4 0.2
                        4 4 0.3
                    };
                \nextgroupplot[title={Plane: pixel 2}]
                    \addplot[matrix plot*, mesh/cols=5, mesh/rows=5, point meta=explicit] table [meta=z] {
                        x y z
                        0 0 0.7
                    1 0 0.5
                    2 0 0.4
                    3 0 0.3
                    4 0 0.2
                    0 1 0.6
                    1 1 0.4
                    2 1 0.3
                    3 1 0.2
                    4 1 0.1
                    0 2 0.5
                    1 2 0.3
                    2 2 0.2
                    3 2 0.1
                    4 2 0.0
                    0 3 0.4
                    1 3 0.2
                    2 3 0.1
                    3 3 0.0
                    4 3 0.0
                    0 4 0.3
                    1 4 0.1
                    2 4 0.0
                    3 4 0.0
                        4 4 0.0
                    };
                \nextgroupplot[
                        title={Plane: pixel 3},
                        colorbar,
                        colorbar style={
                            ytick={0,0.35,0.7},
                            yticklabel style={font=\scriptsize},
                        },
                    ]
                    \addplot[matrix plot*, mesh/cols=5, mesh/rows=5, point meta=explicit] table [meta=z] {
                        x y z
                        0 0 0.1
                    1 0 0.2
                    2 0 0.3
                    3 0 0.4
                    4 0 0.5
                    0 1 0.1
                    1 1 0.2
                    2 1 0.3
                    3 1 0.4
                    4 1 0.5
                    0 2 0.1
                    1 2 0.2
                    2 2 0.3
                    3 2 0.4
                    4 2 0.5
                    0 3 0.1
                    1 3 0.2
                    2 3 0.3
                    3 3 0.4
                    4 3 0.5
                    0 4 0.1
                    1 4 0.2
                    2 4 0.3
                    3 4 0.4
                    4 4 0.5
                };
        \end{groupplot}
    \end{tikzpicture}
\caption{Component planes for three features on a trained SOM (toy data). Each plane maps one feature across the grid; aligned bright/dark regions reveal correlated features and complement the U-Matrix in \Cref{fig:lec5-som-lattice-umatrix}. Interpret brightness comparatively within a plane rather than as an absolute scale.}
    \label{fig:lec5-som-component-planes}
\end{figure}

The weight update rule generalizes to:
\begin{align}
\mathbf{w}_j(t+1) = \mathbf{w}_j(t) + \alpha(t) \, h_{j c}(t) \left( \mathbf{x} - \mathbf{w}_j(t) \right),
\label{eq:coop_weight_update}
\end{align}
where
\begin{itemize}
    \item \(h_{j c}(t)\) is the \emph{neighborhood function} that quantifies the degree of cooperation between neuron \(j\) and the winner \(c\).
    \item \(\alpha(t)\) is the learning rate at time \(t\).
\end{itemize}

The neighborhood function satisfies:
\[
h_{j c}(t) = \begin{cases}
1, & j = c \\
\in (0,1), & j \in \mathcal{N}(c), j \neq c \\
0, & \text{otherwise}
\end{cases}
\]

\paragraph{Gaussian Neighborhood Function}

A common choice for \(h_{j c}(t)\) is a Gaussian function based on the distance between neurons \(j\) and \(c\) on the output grid:
\begin{align}
h_{j c}(t) = \exp \left( - \frac{\| \mathbf{r}_j - \mathbf{r}_{c} \|^2}{2 \sigma^2(t)} \right),
\label{eq:gaussian_neighborhood}
\end{align}
where
\begin{itemize}
    \item \(\mathbf{r}_j\) and \(\mathbf{r}_{c}\) are the coordinates of neurons \(j\) and \(c\) on the output grid.
    \item \(\sigma(t)\) is the neighborhood radius (width) at time \(t\), which decreases over training.
\end{itemize}

This function ensures that neurons closer to the winner receive larger updates, while distant neurons are updated less or not at all.

\paragraph{Interpretation}

The cooperative update encourages neighboring neurons to become sensitive to similar inputs, thereby preserving topological relationships in the input space. This is the key principle behind Self-Organizing Maps (SOMs).

\subsection{Example: Neighborhood Update Illustration}
\label{sec:som_example_neighborhood_update_illustration}

Suppose the output neurons are arranged in a 2D grid as shown schematically in \Cref{fig:som_neighborhood}, where each neuron is indexed by its grid coordinates. For an input \(\mathbf{x}\), neuron \(c\) wins. The neighborhood \(\mathcal{N}(c)\) might include neurons within a radius \(\sigma\) around \(c\).

\begin{figure}[t]
    \centering
    \begin{tikzpicture}[scale=0.8]
        \foreach \x in {0,...,3} {
            \foreach \y in {0,...,3} {
                \node[draw, circle, minimum size=0.5cm, fill=gray!5] (n\x\y) at (\x,\y) {};
            }
        }
        \node[draw, circle, minimum size=0.55cm, fill=cbBlue!40] at (2,1) {};
        \draw[cbGreen, dashed, thick] (2,1) circle (1.5);
        \node at (2,1.5) {\scriptsize BMU $c$};
        \node[cbGreen!70!black] at (3.6,1) {\scriptsize radius $\sigma(t)$};
    \end{tikzpicture}
    \caption{SOM grid with the best-matching unit (BMU) highlighted in blue and a dashed neighborhood radius indicating which prototype vectors receive cooperative updates (schematic).}
    \label{fig:som_neighborhood}
\end{figure}


Each neuron \(j\) in \(\mathcal{N}(c)\) updates its weight vector according to \eqref{eq:coop_weight_update}, with the magnitude of update modulated by \(h_{j c}(t)\).

\subsection{Summary of Cooperative Competitive Learning Algorithm}
\label{sec:som_summary_of_cooperative_competitive_learning_algorithm}

\begin{enumerate}
    \item Present an input vector and identify the winning neuron using the discriminant function.
    \item Update the winning neuron's weights and those of its neighbors according to the cooperative rule.
    \item Decrease the learning rate and neighborhood radius according to the annealing schedule.
    \item Repeat for all inputs until the map stabilizes or a maximum number of epochs is reached.
\end{enumerate}
\subsection{Wrapping Up the Kohonen Self-Organizing Map (SOM) Derivations}
\label{sec:som_wrapping_up_the_kohonen_self_organizing_map_som_derivations}

At this point the SOM training story is complete: you have a similarity rule (to pick a winner), a neighborhood rule (to decide who else learns), and schedules that anneal both over time. The only subtlety is that these pieces interact. A large neighborhood early is what gives you global organization; a small neighborhood late is what lets prototypes settle into fine detail without tearing the map.

Recall the weight update rule for neuron \( j \) at time step \( t \):
\begin{align}
    \Delta \mathbf{w}_j(t) = \alpha(t) \, h_{j, c}(t) \left[ \mathbf{x}(t) - \mathbf{w}_j(t) \right].
    \label{eq:auto:lecture_5_part_i:2}
\end{align}
where:
\begin{itemize}
    \item \(\mathbf{x}(t)\) is the input vector at time \( t \).
    \item \(\mathbf{w}_j(t)\) is the weight vector of neuron \( j \) at time \( t \).
    \item \(c\) is the index of the winning neuron (best matching unit) for input \(\mathbf{x}(t)\).
    \item \(\alpha(t)\) is the learning rate, a monotonically decreasing function of time.
    \item \(h_{j, c}(t)\) is the neighborhood function centered on the winning neuron \( c \), also decreasing over time.
\end{itemize}

\paragraph{Neighborhood Function and Its Role}

The neighborhood function \( h_{j, c}(t) \) typically takes a Gaussian form:
\begin{align}
    h_{j, c}(t) = \exp\left(-\frac{\| \mathbf{r}_j - \mathbf{r}_{c} \|^2}{2\sigma^2(t)}\right).
    \label{eq:neighborhood_function}
\end{align}
where:
\begin{itemize}
    \item \(\mathbf{r}_j\) and \(\mathbf{r}_{c}\) are the positions of neurons \( j \) and \( c \) on the SOM grid.
    \item \(\sigma(t)\) is the neighborhood radius, which decreases over time.
\end{itemize}

This function ensures that neurons closer to the winning neuron receive larger updates, while those farther away receive smaller or zero updates. Initially, \(\sigma(t)\) is large, allowing broad neighborhood cooperation, but it shrinks as training progresses, focusing updates increasingly on the winning neuron itself.

\paragraph{Time-Dependent Parameters}

Both the learning rate \(\alpha(t)\) and neighborhood radius \(\sigma(t)\) decrease over time, typically following exponential decay laws:
\begin{align}
    \alpha(t) &= \alpha_0 \exp\left(-\frac{t}{\tau_\alpha}\right), \\
    \sigma(t) &= \sigma_0 \exp\left(-\frac{t}{\tau_\sigma}\right),
    \label{eq:decay_parameters}
\end{align}
where \(\alpha_0\) and \(\sigma_0\) are initial values, and \(\tau_\alpha, \tau_\sigma\) are time constants controlling the decay rates.

\paragraph{Summary of the Six Learning Steps}
\label{sec:som_training_steps}

One run of SOM training is repetitive in a good way: you keep presenting inputs, keep finding the BMU, and keep nudging a neighborhood patch until the map stops moving. The following pseudocode is the same loop written in a way you can implement directly.

\begin{tcolorbox}[summarybox, title={Self-Organizing Map (SOM) training pseudocode}]
\begin{enumerate}
    \item \textbf{Initialize} weight vectors \(\mathbf{w}_j(0)\) randomly or from samples.
    \item For iteration \(t=0,\ldots, T\):
    \begin{enumerate}
        \item Sample an input \(\mathbf{x}(t)\).
        \item Find the best matching unit (BMU) \(c = \arg\min_{j} \|\mathbf{x}(t)-\mathbf{w}_j(t)\|_2^2\).
        \item Compute neighborhood coefficients \(h_{j, c}(t)\).
        \item Update every neuron:
        \[
            \mathbf{w}_j(t+1) = \mathbf{w}_j(t) + \alpha(t)\, h_{j, c}(t)\Bigl(\mathbf{x}(t)-\mathbf{w}_j(t)\Bigr).
        \]
        \item Decay learning-rate \(\alpha(t)\) and neighborhood radius \(\sigma(t)\) (e.g., exponentially).
    \end{enumerate}
\end{enumerate}
\end{tcolorbox}

\begin{tcolorbox}[summarybox, title={Batch SOM (deterministic pass)}]
\begin{enumerate}
    \item Fix centers \(\mathbf{r}_j\) on the grid; initialize \(\mathbf{w}_j\) (random data or along PCA directions).
    \item Repeat until convergence or max epochs:
    \begin{enumerate}
        \item Assign each \(\mathbf{x}_n\) to its BMU \(c_n\).
        \item For each neuron \(j\), update
        \[
        \mathbf{w}_j \leftarrow \frac{\sum_n h_{j, c_n}(t)\,\mathbf{x}_n}{\sum_n h_{j, c_n}(t)}.
        \]
        \item Decay \(\alpha(t),\sigma(t)\).
    \end{enumerate}
\end{enumerate}
Batch SOM (one pass per epoch) is deterministic given the assignments; it often stabilizes faster than purely online updates.
\end{tcolorbox}

\noindent I find it useful to tell the SOM story in \emph{three stages}, even though the code is often written as a longer checklist. The stages are: \textbf{Initialization} (seed the prototype grid), \textbf{Competition} (pick a best-matching unit for each input), and \textbf{Cooperation} (update a neighborhood and decay \(\alpha(t)\) and \(\sigma(t)\)). The six-step procedure in \Cref{sec:som_training_steps} is simply one way to operationalize those stages when you implement the algorithm and decide when to stop.

\subsection{Applications of Kohonen Self-Organizing Maps}
\label{sec:som_applications_of_kohonen_self_organizing_maps}

Kohonen SOMs show up when you want \emph{both} a prototype model and a picture you can reason about. In practice they tend to be used in three roles:

\begin{itemize}
    \item \textbf{Clustering / regime discovery:} Group similar points without supervision and summarize each region by a prototype vector (e.g., operating modes in telemetry features).
    \item \textbf{Exploratory reduction:} Map high-dimensional data onto a 2D grid index for visualization and exploratory analysis. You do not get a continuous coordinate system; you get a discrete map that is often easier to inspect.
    \item \textbf{Visualization diagnostics:} Use U-Matrices and component planes as ``instrument panels'' that reveal boundaries, smoothness, and which input features drive the organization.
\end{itemize}

\begin{tcolorbox}[summarybox, title={Author's note: when a SOM is the wrong tool}]
If your only goal is to assign points to \(K\) groups and summarize each group by a center, k-means is usually the simpler baseline. If you want a continuous low-dimensional chart with a clear linear story, start with PCA; if you need a global distance-preserving embedding, MDS is a better conceptual match. If your goal is a local-neighborhood visualization and you accept distortion of global geometry, methods like UMAP/t-SNE are often more visually dramatic.

I reach for a SOM when I want three things at once: (i) prototypes I can inspect, (ii) a fixed 2D organization that is stable enough to compare across runs, and (iii) diagnostics (U-Matrix/component planes, QE/TE) that tell me whether the picture is trustworthy. If you cannot define a sensible distance (or you cannot normalize features so distance means something), a SOM will happily organize noise.
\end{tcolorbox}

\begin{tcolorbox}[summarybox, title={Application example: website sessions as behavioral modes}]
Consider a website where the system you are trying to understand is not a motor or a circuit, but a stream of user sessions. Each session becomes a data point \(\mathbf{x}\): dwell time, number of pages, scroll depth, time-to-first-interaction, device class, referrer type, returning vs.\ new, and a few content signals (e.g., the category of the first page visited, or an embedding of the landing-page text). You may not have labels for ``intent,'' but you do expect recurring modes: quick bounce, comparison shopping, documentation lookup, checkout flow, support troubleshooting, and so on.

A SOM is useful here because it gives you prototypes \(\mathbf{w}_i\) (representative session fingerprints) \emph{and} a 2D grid you can inspect. After training, each session maps to a best-matching unit (BMU). Dense patches of mapped sessions indicate common behavior; ridges in the U-Matrix suggest boundaries where neighboring prototypes are far apart, which often corresponds to genuinely different behaviors under your feature choices. Component planes then turn the picture into an explanation: you can see, for example, that one boundary is mostly driven by dwell time and scroll depth, while another is driven by referrer and device. In practice you read these plots as heat maps over the 2D grid: different behaviors show up as different regions you can name, compare, and inspect.

\textbf{Recommendation:} treat preprocessing as part of the model. Normalize continuous features, log-scale heavy-tailed counts, and be deliberate about how you represent categorical variables (one-hot, grouped buckets, or a small embedding) so Euclidean distance matches your notion of similarity.

\textbf{Audit hook:} rerun with a few seeds and a time-based split. If the map reorganizes completely across runs or across weeks, you are seeing instability or seasonality rather than durable structure.
\end{tcolorbox}

\begin{tcolorbox}[summarybox, title={Application example: mapping page archetypes and performance drivers}]
A second, complementary use is to map \emph{pages} (or queries) rather than sessions. Here each data point \(\mathbf{x}\) describes a URL or content item: a text embedding of the page, topic/category tags, layout signals (word count, number of images), and aggregate behavior metrics (click-through rate (CTR) from search, median dwell time, bounce rate, conversion rate, and the fraction of visits from mobile). You are effectively asking: which pages are similar in content and behavior, and where do the sharp breaks live?

A SOM turns this into a visual inventory. Pages with similar content and similar usage patterns compete for the same region of the grid, and the U-Matrix highlights boundaries between page families (e.g., ``documentation-like'' pages versus ``marketing-like'' pages). Component planes are the practical payoff: they show which variables line up with those boundaries. If one region lights up on ``mobile fraction'' and ``bounce rate,'' you immediately get a hypothesis about responsiveness or page speed. If another region is coherent in ``topic embedding'' but not in conversion, you have found a content family that is consistent but not effective.

\textbf{Recommendation:} separate what you want to \emph{organize by} (content similarity) from what you want to \emph{diagnose} (performance metrics). A simple approach is to train the SOM on content-heavy features, then overlay performance as color/slices; otherwise the map can end up organizing primarily by whatever has the largest numeric range.

\textbf{Audit hook:} overlay slices (language, device, traffic source). If the map's dominant structure is ``source domain'' or ``mobile vs.\ desktop,'' that may be a useful finding, but it is also a warning that your feature design is driving the story.
\end{tcolorbox}

\paragraph{Relation to k-means and modern variants}

When the neighborhood is collapsed to a single winner (no neighbors), the update reduces to a k-means-style prototype move without any notion of grid organization \citep{MacQueen1967}; SOMs therefore sit between pure clustering (k-means) and manifold learning, adding a topographic prior that encourages neighboring units to represent similar inputs. Recent ``neural-SOM'' hybrids embed SOM-like updates inside deep networks, but still rely on the same BMU search and neighborhood-weighted updates described above.

\begin{tcolorbox}[summarybox, title={Theory notes and recipes}]
\textbf{Convergence/magnification (theory lens):} With decays \(\alpha(t)\to 0\), \(\sigma(t)\to 0\) and \(\sum_t \alpha(t)=\infty\), \(\sum_t \alpha^2(t)<\infty\), SOM updates converge under mild assumptions \citep{ErwinObermayerSchulten1992,CottrellFort1986}. One qualitative takeaway is \emph{magnification}: dense regions of the data tend to attract more units of the map.\\
\textbf{A practical recipe (engineering lens):} If I have no prior, I treat these as starting points, not rules. Use \(5\sqrt{N}\)--\(10\sqrt{N}\) units when unsure; hex grids reduce anisotropy; wrap-around (toroidal) grids reduce edge effects. Initialize \(\mathbf{w}_j\) from data or along the first two PCs; pick \(\alpha_0\in[0.1,0.5]\) and decay toward \(\alpha_{\min}\approx 10^{-3}\); set \(\sigma_0\) to roughly the map radius and decay toward 1--1.5 cells.
\end{tcolorbox}

\begin{tcolorbox}[summarybox, title={Related and growing variants}]
\textbf{Neural Gas / Growing Neural Gas} \citep{MartinetzBerkovichSchulten1993,Fritzke1994GrowingNeuralGas} drop the fixed grid and instead learn neighborhood structure (and, in the growing variants, add units dynamically). \textbf{Generative topographic mapping (GTM)} (Bishop et al.) provides a probabilistic, topographic embedding with likelihoods/uncertainty. I like to remember the landscape this way: k-means gives prototypes with no organization; SOMs add a fixed organizing prior; these variants let the organization itself adapt.
\end{tcolorbox}

\paragraph{Complexity and out-of-sample mapping.} A full online epoch costs \(O(NM)\) (N data, M units); batch passes cost similar but fewer epochs. For large M, use approximate nearest-neighbor BMU search (k-d trees or vector-index libraries such as FAISS). New points map via their BMU; optional soft responsibilities use \(h_{ci}\) for smoothing.
\paragraph{Theory link to other chapters.} SOMs learn prototypes like the centers in \Cref{chap:rbf} but add a topographic prior; quality diagnostics (QE/TE) can be tracked with the validation-curve diagnostics in \Cref{chap:supervised}. For task-driven embeddings, see \Cref{chap:cnn,chap:nlp}; Hopfield (\Cref{chap:hopfield}) contrasts with energy-based associative recall.

\paragraph{Quality measures and magnification.} Two diagnostics are standard when reporting SOM quality:
\begin{itemize}
    \item \textbf{Quantization error (QE):} average Euclidean distance between each input and its BMU. Lower QE indicates prototypes that better represent the data manifold.
    \item \textbf{Topographic error (TE):} fraction of inputs whose first- and second-best BMUs are not adjacent, quantifying topology preservation (magnification factor).
\end{itemize}
Tracking both metrics reveals whether the neighborhood decay is too slow (over-smoothing) or too aggressive (tearing the topology). \Cref{chap:supervised}'s learning-curve plots suggest early\hyp{}stopping heuristics: stop when QE/TE on a validation split flatten.

\begin{tcolorbox}[summarybox, title={Key takeaways}]
\begin{itemize}
    \item SOMs perform topology-preserving vector quantization on a discrete grid.
    \item A shrinking neighborhood and decaying learning rate drive coarse-to-fine organization.
    \item U-Matrices and quantization/topographic errors are practical diagnostics for convergence.
\end{itemize}

\medskip
\noindent\textbf{Minimum viable mastery.}
\begin{itemize}
    \item Given \(x\), compute the BMU, write the weight update \(w_i \leftarrow w_i + \alpha(t) h_{ci}(t) (x-w_i)\), and explain how \(h_{ci}\) enforces cooperation.
    \item Interpret a U-Matrix and component planes as diagnostics for neighborhood structure and convergence.
    \item Choose sensible \(\alpha(t)\) and \(\sigma(t)\) schedules and justify them using QE/TE validation curves.
\end{itemize}

\noindent\textbf{Common pitfalls.}
\begin{itemize}
    \item Using a map that is too small or a neighborhood decay that is too aggressive (tearing the topology).
    \item Skipping input normalization, causing prototypes to track scale rather than structure.
    \item Treating grid distance as a true metric in data space (SOM topology is approximate, not exact geometry).
    \item Over-interpreting a single run without checking QE/TE plateaus and sensitivity to initialization.
\end{itemize}
\end{tcolorbox}

\begin{tcolorbox}[summarybox, title={Exercises and lab ideas}]
\begin{itemize}
    \item Train a \(10\times 10\) SOM on handwritten digits (MNIST) and plot component planes; report quantization/topographic errors as training progresses.
    \item Implement the six-step SOM procedure with both Gaussian and rectangular neighborhood functions and compare convergence speed.
    \item Visualize the effect of annealing schedules by freezing the learning rate and neighborhood radius at different epochs and observing the resulting U-Matrix.
    \item Compare SOM prototypes to K-means centers on the same dataset; sweep \((\sigma(t))\) schedules and map sizes; report QE/TE and a trustworthiness@k measure.
\end{itemize}

\medskip
\noindent\textbf{If you are skipping ahead.} Keep the notion of an ``energy landscape'' from this chapter in mind: \Cref{chap:hopfield} makes that idea explicit with a Lyapunov energy, and later attention models (\Cref{chap:transformers}) can be read as learned, content-based neighborhood selection.
\end{tcolorbox}

\medskip
\paragraph{Where we head next.} \Cref{chap:hopfield} extends the unsupervised thread with energy-based associative memory, complementing SOM topology learning with retrieval dynamics that foreshadow later attention-style mechanisms.

% Chapter 10
\section{Hopfield Networks: Introduction and Context}\label{chap:hopfield}

\Cref{chap:som} focused on self-organizing maps and unsupervised feature maps; we now transition to another unsupervised/energy-based model: the \emph{Hopfield network}, a recurrent system that stores patterns as attractors. \Cref{fig:roadmap} marks this as the energy-based branch. The debug mindset carries over, but the diagnostics change: SOMs are easiest to audit through map geometry (U-matrix/component planes), while Hopfield recall is easiest to audit through an energy trace and overlap with the intended memory.

\begin{tcolorbox}[summarybox, title={Learning Outcomes}]
\begin{itemize}
    \item Interpret Hopfield networks as energy-minimizing recurrent systems and derive their asynchronous update rule.
    \item Quantify capacity, recall dynamics, and pitfalls (spurious memories, bias encodings) using simple analytical bounds.
    \item Relate Hopfield updates to modern energy-based models and attention mechanisms to build intuition for later chapters.
\end{itemize}
\end{tcolorbox}

\begin{tcolorbox}[summarybox, title={Design motif}]
Constrain recurrence so the dynamics become a descent process: symmetric weights and an energy function turn ``feedback'' into ``stable memory.''
\end{tcolorbox}

\subsection{From Feedforward to Recurrent Neural Networks}
\label{sec:hopfield_from_feedforward_to_recurrent_neural_networks}

Feedforward networks compute a single forward map: once an input has passed through the layers, the computation ends. That is the right abstraction for static input\(\rightarrow\)output tasks, but it is not a natural model of memory: nothing inside the network persists unless you explicitly feed it back in.

Recurrent networks add that feedback. Cycles let the current state influence the next state, which is exactly what you want for sequences and recall---and also exactly what makes the dynamics harder to reason about. Hopfield's contribution is to keep the recurrence, but constrain it so the system settles instead of wandering.

\paragraph{Challenges with general recurrent networks}
The moment you add feedback, you inherit a dynamical system. That buys you memory, but it also means the network can (i) fail to settle and instead oscillate, (ii) end up in different states from tiny changes in the starting point, and (iii) become harder to train because learning signals must propagate through time (and that path can vanish or explode). Historically, these issues pushed many applications toward feedforward models unless recurrence was the cleanest abstraction for the task.

\begin{tcolorbox}[summarybox, title={Then vs.\ now: energy-based memory in context}]
Classical Hopfield networks are the clean case: binary states, symmetric weights, and an explicit Lyapunov energy give you a convergence guarantee.

The intuition that survives is the one we care about: memory is an energy landscape, and recall is descent toward an attractor (pattern completion).

Modern systems borrow the geometry without necessarily inheriting the guarantee. Many ``memory'' mechanisms are differentiable, learned from data, and content-addressable at scale; the symmetry that makes Hopfield proofs work is not always present. Keep the attractor picture, but do not over-assume the math.
\end{tcolorbox}

\begin{tcolorbox}[summarybox, title={Author's note: stabilizing recurrence}]
General recurrent networks can behave unpredictably because feedback can create cycles that oscillate or amplify small differences. Hopfield's key move was to restrict the architecture so the dynamics become a descent process: symmetric weights and no self-loops allow the network to be assigned an energy function that decreases under asynchronous updates. That single design choice turns ``recurrent'' from ``chaotic'' into ``stable memory.''
\end{tcolorbox}

\subsection{Hopfield's breakthrough}
\label{sec:hopfield_hopfield_s_breakthrough_1982}

In 1982, John Hopfield introduced a special class of recurrent networks by putting engineering constraints on the feedback loop \citep{Hopfield1982}. The constraints are simple enough to implement, and each one buys you a piece of the convergence story.

First, make the weights symmetric and remove self-loops:
\begin{equation}
    w_{ij} = w_{ji} \quad \forall i, j,
    \label{eq:auto_hopfield_591d3daa63}
\end{equation}
\begin{equation}
    w_{ii} = 0 \quad \forall i.
    \label{eq:auto_hopfield_bd57392190}
\end{equation}
Second, keep neuron states binary, \(s_i \in \{+1,-1\}\), so the network evolves by flipping bits rather than flowing through a continuous activation curve.

Under these choices you can define a Lyapunov energy \(E(\mathbf{s})\) (a scalar function that decreases under the update) and choose an asynchronous (one\hyp{}neuron\hyp{}at\hyp{}a\hyp{}time) update rule so that each flip never increases \(E\). That is the key move: the network becomes a descent process in a finite state space, so it must settle at a fixed point. Those fixed points are the stored patterns (and their complements), plus whatever spurious attractors appear when the network is heavily loaded.

\subsection{Network Architecture and Dynamics}
\label{sec:hopfield_network_architecture_and_dynamics}

Consider a Hopfield network with \(N\) neurons. The state vector is \(\mathbf{s} = (s_1, s_2, \ldots, s_N)^T\), where each \(s_i \in \{+1, -1\}\). The symmetric weight matrix \(\mathbf{W} = [w_{ij}]\) satisfies \(w_{ij} = w_{ji}\) and \(w_{ii} = 0\).
Throughout this discussion \(w_{ij}\) denotes the weight applied to state \(s_j\) when computing the input to neuron \(i\), so column indices correspond to presynaptic neurons.

The \emph{local field} or \emph{input energy} to neuron \(i\) is defined as
\begin{equation}
    h_i(t) = \sum_{j=1}^N w_{ij} s_j(t).
\label{eq:auto_hopfield_5a6eb676df}
\end{equation}
The scalar \(h_i(t)\) therefore represents the total input (or \emph{local field}) accumulated at neuron \(i\) before thresholding during iteration \(t\).

The neuron updates its state according to the sign of \(h_i(t)\) relative to a threshold \(\theta_i\):
\begin{equation}
    s_i(t+1) =
    \begin{cases}
        +1, & h_i(t) \geq \theta_i, \\
        -1, & h_i(t) < \theta_i,
    \end{cases}
    \label{eq:hopfield_update_rule}
\end{equation}

Typically, thresholds \(\theta_i\) are set to zero or learned as part of the model.

\paragraph{Interpretation:} The neuron "fires" (state \(+1\)) if the weighted sum of inputs exceeds the threshold; otherwise, it remains "off" (state \(-1\)). This binary update rule contrasts with the continuous activation functions used in feedforward networks.

\subsection{Encoding conventions}
\label{sec:hopfield_encoding_conventions}

Two binary encodings are common. We primarily use \(s_i \in \{-1,+1\}\) because it simplifies the energy function, but many software libraries work with \(x_i \in \{0,1\}\). Define \(\mathbf{s} = 2\mathbf{x} - \mathbf{1}\) and \(\mathbf{x} = (\mathbf{s}+\mathbf{1})/2\); then
\[
E_{\pm1}(\mathbf{s}) = -\frac{1}{2}\sum_{i\neq j} w_{ij} s_i s_j + \sum_i \theta_i s_i
\]
and
\[
E_{01}(\mathbf{x}) = -\frac{1}{2}\sum_{i\neq j} w'_{ij} x_i x_j + \sum_i \theta'_i x_i + \text{const},
\]
with \(w'_{ij}=4w_{ij}\) and \(\theta'_i = 2\theta_i + 2\sum_{j\neq i} w_{ij}\) under the sign convention in \(E_{\pm1}\). The additive constant does not affect which states minimize the energy or the update dynamics. This table summarizes the correspondence:

\begin{center}
\begin{tabular}{@{}p{0.32\linewidth}p{0.28\linewidth}p{0.28\linewidth}@{}}
\toprule
 & \(\{-1,+1\}\) encoding & \(\{0,1\}\) encoding \\
\midrule
State variable & \(s_i \in \{-1,+1\}\) & \(x_i = (s_i+1)/2\) \\
Energy & \(E_{\pm1}(\mathbf{s})\) & \(E_{01}(\mathbf{x}) = E_{\pm1}(2\mathbf{x}-\mathbf{1})\) \\
Update rule & \(s_i \leftarrow \operatorname{sign}(h_i - \theta_i)\) & \(x_i \leftarrow \mathbf{1}[h'_i - \theta'_i > 0]\) \\
\bottomrule
\end{tabular}
\end{center}

Whenever an equation later in the chapter uses \(s_i\) you can translate it to \(x_i\) via this affine mapping; we call out both forms only when the distinction matters. As a concrete example, the pattern \(\mathbf{x}=[1,0,1,0]\) in the \(\{0,1\}\) encoding maps to \(\mathbf{s}=2\mathbf{x}-\mathbf{1}=[+1,-1,+1,-1]\); conversely \(\mathbf{s}=[-1,+1,+1]\) corresponds to \(\mathbf{x}=[0,1,1]\).

\subsection{Energy Function and Stability}
\label{sec:hopfield_energy_function_and_stability}

Hopfield defined an energy function \(E: \{-1, +1\}^N \to \mathbb{R}\) associated with the network state \(\mathbf{s}\):
\begin{equation}
    E(\mathbf{s}) = -\frac{1}{2} \sum_{i=1}^N \sum_{j=1}^N w_{ij} s_i s_j + \sum_{i=1}^N \theta_i s_i.
    \label{eq:energy_function}
\end{equation}
Because the weights are symmetric and satisfy \(w_{ii}=0\), the double sum may equivalently be written as \(\sum_{i<j} w_{ij} s_i s_j\); the \(\tfrac{1}{2}\) factor explicitly prevents counting each unordered pair twice, so removing it would scale the energy by two. Thresholds \(\theta_i\) act like biases; many texts write the second term as \(-\sum_i b_i s_i\) with \(b_i = \theta_i\).
% Chapter 10 (continued)

\begin{tcolorbox}[summarybox, title={Engineering lens: energy as an objective (not a metaphor)}]
It helps to read \(E(\mathbf{s})\) the same way you read a loss in supervised learning: a scalar objective defined over a space of states. Here the states are discrete (\(\mathbf{s}\in\{-1,+1\}^N\)), so the ``optimization'' is not gradient descent; it is a sequence of coordinate updates (flip one bit at a time) that never increase \(E\) under Hopfield's symmetry constraints.

This is why the update rule matters. The weights define the energy landscape, and the asynchronous update is a descent procedure on that landscape. Local minima are stable memories; other minima (spurious attractors) are the price of capacity.
\end{tcolorbox}

\paragraph{Hopfield network states and the energy view}
\label{sec:hopfield_hopfield_network_states_and_energy_function}

Encoding choices and the corresponding energy forms were laid out in \Cref{sec:hopfield_encoding_conventions} and \Cref{sec:hopfield_energy_function_and_stability}. The key continuity idea is that Hopfield dynamics are not ``mysterious recurrence'': with symmetric weights, the update rule becomes a descent process on \(E(\cdot)\). We make this concrete in three passes: define what ``stable'' means, walk one tiny numeric trace, then prove the one-step claim (\(\Delta E\le 0\)) that guarantees convergence under asynchronous updates.

\subsection{Energy Minimization and Stable States}
\label{sec:hopfield_energy_minimization_and_stable_states}

The fundamental goal in Hopfield networks is to find a state \(\mathbf{s}\) that minimizes the energy \(E\). Such states correspond to stable equilibria or attractors of the network dynamics.

\paragraph{State update dynamics:}
The network updates neuron states using the sign rule in \eqref{eq:hopfield_update_rule}; for \(\{0,1\}\) encodings use the corresponding thresholded version. The pseudo-code below makes the asynchronous sweep explicit for later convergence arguments.

\begin{tcolorbox}[summarybox, title={Asynchronous Hopfield update (pseudo-code)}]
\begin{enumerate}
    \item Initialize \(\mathbf{s}\) (e.g., noisy probe), set max sweeps \(T\).
    \item For \(t=1,\dots, T\):
    \begin{enumerate}
        \item Pick a neuron index \(i\) (random order or cyclic sweep).
        \item Compute \(h_i = \sum_j w_{ij} s_j - \theta_i\).
        \item Update \(s_i \leftarrow \operatorname{sign}(h_i)\).
    \end{enumerate}
    \item Stop early if a full sweep causes no flips; else continue.
\end{enumerate}
Each single-neuron update satisfies \(\Delta E \le 0\) by \eqref{eq:deltaE}, so the loop converges to a local minimum of \eqref{eq:energy_function}.
\end{tcolorbox}

For the formal monotonicity proof and the synchronous-update caveat, see \Cref{sec:hopfield_async_vs_sync}.

\begin{tcolorbox}[summarybox, title={Implementation checklist (debugging a Hopfield net)}]
\begin{itemize}
    \item \textbf{Weight constraints:} enforce \(W=W^\top\) and \(w_{ii}=0\). If either is violated, the classical energy argument does not apply.
    \item \textbf{Energy definition:} implement \eqref{eq:energy_function} with the \(\tfrac{1}{2}\) factor (or equivalently sum over \(i<j\)). Use one sign convention for thresholds/biases and stick to it.
    \item \textbf{Update scheme:} test with asynchronous single-neuron updates first. If you switch to synchronous updates, short cycles are possible (see \Cref{sec:hopfield_async_vs_sync}).
    \item \textbf{Sign at zero:} choose a deterministic convention for \(\operatorname{sign}(0)\) (e.g., keep the old state, or return \(+1\)) so runs are reproducible.
    \item \textbf{Diagnostics:} on a tiny network where you know the stored patterns, plot \(E(\mathbf{s}(t))\) and overlap \(m^{(\mu)}(\mathbf{s}(t))\). Under asynchronous updates and symmetric weights, \(E\) should never increase.
\end{itemize}
\end{tcolorbox}

\subsection{Example: Energy Calculation and State Updates}
\label{sec:hopfield_example_energy_calculation_and_state_updates}

Consider a Hopfield network with three neurons, bipolar states \(s_i \in \{-1,+1\}\), zero thresholds, and the symmetric weight matrix
\[
W = \begin{bmatrix}
0 & 3 & -4 \\
3 & 0 & 2 \\
-4 & 2 & 0
\end{bmatrix}.
\]

Let the initial state be \(\mathbf{s} = (1,\,1,\,-1)\). Using the energy definition with the \(\frac{1}{2}\) factor to avoid double counting, we obtain
\begin{align}
    E(\mathbf{s}) &= -\frac{1}{2} \sum_{i=1}^3 \sum_{j=1}^3 w_{ij} s_i s_j \nonumber \\
    &= -\frac{1}{2} \Big[ 2 \cdot 3 \cdot (1)(1) + 2 \cdot (-4) \cdot (1)(-1) + 2 \cdot 2 \cdot (1)(-1) \Big] = -5.
    \label{eq:auto:lecture_5_part_ii:1}
\end{align}

\paragraph{State update attempts:}
One way to check whether \(\mathbf{s}=(1,1,-1)\) is stable is to try each single-neuron flip and recompute \(E\). Flipping \(s_1\) gives \(E(-1,1,-1)=9\); flipping \(s_2\) gives \(E(1,-1,-1)=-3\); and flipping \(s_3\) gives \(E(1,1,1)=-1\). Each move raises the energy relative to \(-5\), so no asynchronous update would accept it: the state is a local minimum.

For clarity, \Cref{tab:hopfield-deltaE} reports the energy change for each single flip relative to the current state.
\begin{table}[h]
    \centering
    \begin{tabular}{lccc}
        \toprule
        Flip & New state & \(\Delta E\) & Accept? \\
        \midrule
        \(s_1 \leftarrow -1\) & \((-1,\,1,\,-1)\) & \(+14\) & No \\
        \(s_2 \leftarrow -1\) & \((1,\,-1,\,-1)\) & \(+2\) & No \\
        \(s_3 \leftarrow +1\) & \((1,\,1,\,1)\) & \(+4\) & No \\
        \bottomrule
    \end{tabular}
    % Avoid inline math in captions; it wraps poorly in some EPUB renderers.
    \caption{Single-neuron flips from \((1,1,-1)\); all raise the energy, so the state is a local minimum.}
    \label{tab:hopfield-deltaE}
\end{table}

% QC-BEGIN: hopfield_energy_example
% W 0 3 -4
% W 3 0 2
% W -4 2 0
% theta 0 0 0
% s0 1 1 -1 E -5
% flip_s1 -1 1 -1 E 9
% flip_s2 1 -1 -1 E -3
% flip_s3 1 1 1 E -1
% noisy 1 1 1 update_i 3 s1 1 1 -1 E -5
% QC-END: hopfield_energy_example


Because every single-neuron flip raises the energy, the state \((1,1,-1)\) is a stable local minimum for this network.
If the network is perturbed slightly (for instance, by flipping \(s_3\) to \(+1\) to create the noisy pattern \((1,1,1)\)), the next asynchronous update flips it back. In this state the local field at neuron 3 is \(h_3=-4(1)+2(1)=-2\), so the update rule sets \(s_3\leftarrow -1\), returning to \((1,1,-1)\). The energy drops from \(-1\) to \(-5\), which is the basic content-addressable recall behavior.

\begin{figure}[t]
    \centering
    \begin{tikzpicture}
        \begin{axis}[
            width=0.72\linewidth,
            height=4.2cm,
            ymin=-6.5, ymax=0.5,
            ytick={-6,-4,-2,0},
            symbolic x coords={s0, s1},
            xticklabels={$\,\mathbf{s}^{(0)}$,$\,\mathbf{s}^{(1)}$},
            xtick=data,
            xlabel={Update step},
            ylabel={$E(\mathbf{s})$},
            axis lines=left,
            enlarge x limits=0.1
        ]
            \addplot+[cbBlue, thick, mark=*, mark size=2pt] coordinates {(s0,-1) (s1,-5)};
            \addplot[cbPink, -{Latex[length=2mm]}, thick] coordinates {(s0,-1) (s1,-5)};
            \node[cbPink, font=\scriptsize, anchor=west] at (rel axis cs:0.42,0.75) {one asynchronous flip};
            \node[cbBlue, font=\scriptsize, anchor=south west] at (axis cs: s0,-1) {noisy start $\mathbf{s}^{(0)}$};
        \end{axis}
    \end{tikzpicture}
    % Avoid inline math in captions; it wraps poorly in some EPUB renderers.
    \caption{Hopfield energy decreases monotonically under asynchronous updates. Starting from a noisy probe \(\mathbf{s}^{(0)}\), single-neuron flips move downhill until a stable memory is recovered.}
    \label{fig:hopfield-energy-descent}
\end{figure}


% Chapter 10 (continued)

As \Cref{fig:hopfield-energy-descent} shows, the energy trace never increases when single neurons flip asynchronously, so the descent depicted here is exactly what the formal convergence proof below captures.

\subsection{Energy Function and Convergence of Hopfield Networks}
\label{sec:hopfield_energy_function_and_convergence_of_hopfield_networks}

The figure and pseudo-code above already suggest the main claim: under the symmetry constraints that make \(E(\cdot)\) well defined, asynchronous updates are guaranteed to move ``downhill'' (or stay flat) in energy. This is the formal reason Hopfield networks converge to a fixed point: the state space is finite, and you cannot decrease a bounded quantity forever.

In this subsection we prove the monotonicity step that the rest of the convergence story rests on. The only assumptions you should keep in view are the ones you would implement anyway: symmetric weights with zero diagonal, and an asynchronous update rule (one neuron at a time). When these assumptions are violated (e.g., synchronous updates or asymmetric weights), the energy argument no longer guarantees convergence.

We use the energy in \eqref{eq:energy_function}; the goal is to show that each single-neuron update never increases it.

\paragraph{Goal:} Show that asynchronous updates of neuron states always decrease (or leave unchanged) the energy \( E \), guaranteeing convergence to a local minimum.

\subsubsection{Energy Change Upon Updating a Single Neuron}
\label{sec:hopfield_energy_change_upon_updating_a_single_neuron_sub}

Consider updating neuron \( i \) from old state \( s_i^{\text{old}} \) to new state \( s_i^{\text{new}} \). All other neuron states \( s_j \) for \( j \neq i \) remain fixed. The change in energy is
\begin{equation}
    \Delta E = E_{\text{new}} - E_{\text{old}}.
\label{eq:auto_hopfield_927711ec51}
\end{equation}

Using \eqref{eq:energy_function}, write out the energies explicitly:
\begin{align}
    E_{\text{old}} &= -\frac{1}{2} \sum_{k=1}^N \sum_{l=1}^N w_{kl} s_k^{\text{old}} s_l^{\text{old}} + \sum_{k=1}^N \theta_k s_k^{\text{old}}, \\
    E_{\text{new}} &= -\frac{1}{2} \sum_{k=1}^N \sum_{l=1}^N w_{kl} s_k^{\text{new}} s_l^{\text{new}} + \sum_{k=1}^N \theta_k s_k^{\text{new}}.
    \label{eq:auto:lecture_5_part_ii:2}
\end{align}

Since only \( s_i \) changes, and weights are symmetric with zero diagonal \( w_{ii} = 0 \), the difference simplifies to
\begin{align}
    \Delta E &= E_{\text{new}} - E_{\text{old}} \nonumber \\
    &= -\frac{1}{2} \sum_{j=1}^N \left( w_{ij} s_i^{\text{new}} s_j + w_{ji} s_j s_i^{\text{new}} \right) + \theta_i s_i^{\text{new}} \nonumber \\
    &\quad + \frac{1}{2} \sum_{j=1}^N \left( w_{ij} s_i^{\text{old}} s_j + w_{ji} s_j s_i^{\text{old}} \right) - \theta_i s_i^{\text{old}} \nonumber \\
    &= - \sum_{j=1}^N w_{ij} s_j \left( s_i^{\text{new}} - s_i^{\text{old}} \right) + \theta_i \left( s_i^{\text{new}} - s_i^{\text{old}} \right) \nonumber \\
    &= - \left( s_i^{\text{new}} - s_i^{\text{old}} \right) \left( \sum_{j=1}^N w_{ij} s_j - \theta_i \right).
    \label{eq:deltaE}
\end{align}

Define the \emph{local field} \( h_i \) at neuron \( i \) as
\begin{equation}
    h_i = \sum_{j=1}^N w_{ij} s_j - \theta_i.
    \label{eq:local_field}
\end{equation}

Then,
\[
    \Delta E = - (s_i^{\text{new}} - s_i^{\text{old}}) h_i.
\]

\paragraph{Numeric check (single flip).} With two neurons, weights \(w_{12}=w_{21}=1\), thresholds \( \theta_i=0\), and current state \((s_1, s_2) = (1,-1)\), the local field at neuron 1 is \(h_1 = 1\cdot (-1) = -1\). The update rule sets \(s_1^{\text{new}} = \operatorname{sign}(h_1) = -1\), so \(s_1^{\text{new}} - s_1^{\text{old}} = -2\) and \(\Delta E = -(-2)(-1) = -2 < 0\), confirming the energy drop predicted by \eqref{eq:deltaE}.
\begin{tcolorbox}[summarybox, title={Modern Hopfield views and attention}]
Recent work \citep{Krotov2016,Krotov2020,Ramsauer2021} revisits Hopfield networks as dense
associative memories with continuous states and softmax interactions that are closely related to
Transformer attention. In this view the stored patterns play the role of keys and values, the
current state or query probes the landscape, and the update rule resembles a softmax-weighted
average over memories, minimizing an energy of the form \(\log \sum_\mu \exp(\beta \mathbf{s}^\top \mathbf{m}^\mu)\) with inverse temperature \(\beta\). The useful connection for us is the geometry: content-addressable lookup as motion in an energy landscape. Do not confuse that bridge with the classical guarantee above: the Lyapunov proof in this chapter relies on binary states, symmetric weights, and asynchronous flips.
\end{tcolorbox}

\paragraph{Continuous Hopfield networks (bridge).}
Modern extensions replace the binary sign activation with a smooth, often softmax-like update that keeps neuron states continuous. Storing \(P\) real-valued patterns \(\{\mathbf{m}^\mu\}\), a common update takes the form
\[
\mathbf{s}^{(t+1)} = \operatorname{softmax}\!\left(\beta M^\top \mathbf{s}^{(t)}\right) M,
\]
where \(M\) stacks the memory vectors and \(\beta\) controls sharpness. Read this as ``soft'' associative recall: the update produces a weighted combination of stored patterns, and that picture lines up closely with attention in \Cref{chap:transformers}. Unlike the classical binary case, you should not assume strict monotone descent unless the model is built to have a Lyapunov function.

Combining the sign update in \eqref{eq:hopfield_update_rule} with \eqref{eq:deltaE} yields \(\Delta E \le 0\) for each asynchronous flip, so we omit the redundant case split here.
% Chapter 10: Hopfield Network Weight Update and Capacity

\subsection{Asynchronous vs. Synchronous Updates in Hopfield Networks}
\label{sec:hopfield_async_vs_sync}

Recall from the previous discussion that the Hopfield network energy function decreases monotonically with each asynchronous update of a single neuron state. This guarantees convergence to a local minimum of the energy landscape. In contrast, fully synchronous updates (flipping all neurons at once) can lead to oscillations or short cycles rather than convergence, which is why the classical convergence proofs assume asynchronous updates.

\paragraph{Why asynchronous updates?}
With symmetric weights you can still write the same energy \(E(\mathbf{s})\), but the classic monotonicity proof is for \emph{single-neuron} (asynchronous) updates. If you update all neurons simultaneously, two-cycles can appear.

For a concrete example, take two neurons with \(\theta_1=\theta_2=0\) and \(w_{12}=w_{21}=1\). Start from \((s_1,s_2)=(1,-1)\). A synchronous update computes both new states from the old pair:
\[
s_1 \leftarrow \operatorname{sign}(w_{12}s_2)=\operatorname{sign}(-1)=-1,\qquad
s_2 \leftarrow \operatorname{sign}(w_{21}s_1)=\operatorname{sign}(1)=1,
\]
so the state becomes \((-1,1)\). Repeating the synchronous update maps \((-1,1)\) back to \((1,-1)\). The network therefore oscillates in a 2-cycle, even though the energy
\[
E(s_1,s_2) = -\frac{1}{2}\sum_{i,j} w_{ij}s_is_j = -w_{12}s_1s_2
\]
stays constant across the two states.

To ensure the classical convergence guarantee, use \emph{asynchronous} updates (one neuron at a time) under symmetric weights with zero diagonal. Synchronous updates can still work in practice, but the energy argument no longer rules out short cycles.

\subsection{Storage Capacity of Hopfield Networks}
\label{sec:hopfield_storage_capacity_of_hopfield_networks}

A key question is: \emph{How many memories can a Hopfield network reliably store and recall?}
\[
p \approx 0.138\, n,
\]
is the classical rule-of-thumb for the number of random patterns that can be stored with low error in a network of \(n\) neurons \citep{McEliece1987}. The important point is not the constant itself; it is that the capacity is only a small fraction of \(n\). Both \(\mathbf{\xi}^\mu\) and its complement \(-\mathbf{\xi}^\mu\) are fixed points, and odd mixtures of stored patterns become spurious states as \(p/n\) grows. This low capacity is why Hopfield networks are not used as practical storage devices despite their elegant associative-memory behavior.

\paragraph{Stochastic updates (bridge to Boltzmann machines).} A stochastic variant replaces the hard sign in \eqref{eq:hopfield_update_rule} with probabilistic flips (e.g., Gibbs sampling). With symmetric weights this defines a Boltzmann distribution whose energy matches \eqref{eq:energy_function}, linking Hopfield recall to the Boltzmann/energy-based models that underlie modern probabilistic neural networks.

\subsection{Improving Storage Capacity via Weight Updates}
\label{sec:hopfield_improving_storage_capacity_via_weight_updates}

Is it possible to improve storage by choosing the weights to ``bake in'' the memories directly? That is exactly what the classical Hebbian construction does: use stored patterns to set \(W\), then let the state updates perform recall.

\paragraph{Hebbian learning rule:}
Given \( p \) stored patterns \(\{\mathbf{\xi}^1, \mathbf{\xi}^2, \ldots, \mathbf{\xi}^p\}\), each \(\mathbf{\xi}^\mu = (\xi_1^\mu, \xi_2^\mu, \ldots, \xi_n^\mu)\) with \( \xi_i^\mu \in \{+1, -1\} \), the weights are set by:
\begin{equation}
w_{ij} = \frac{1}{n} \sum_{\mu=1}^p \xi_i^\mu \xi_j^\mu, \quad w_{ii} = 0.
\label{eq:hebbian_weights}
\end{equation}

This is the classical Hebbian learning rule for Hopfield networks.

\paragraph{What this formula is doing:}
Setting \(w_{ii}=0\) removes self-feedback. The factor \(1/n\) keeps weight magnitudes \(O(1)\) as the network grows. The sum itself is an outer-product accumulation: it encodes pairwise correlations across the stored patterns.

\subsection{Example: Weight Calculation for a Single Pattern}
\label{sec:hopfield_example_weight_calculation_for_a_single_pattern}

Consider a fundamental memory pattern:
\[
\mathbf{\xi} = (1, 1, 1, -1),
\]
with no thresholds (\( \theta_i = 0 \)).

\paragraph{Step 1: Compute outer product}
Form the matrix \( \mathbf{B} = \mathbf{\xi} \mathbf{\xi}^\top \). Each entry \(B_{ij} = \xi_i \xi_j\) captures the pairwise correlation between neurons \(i\) and \(j\).
\[
\mathbf{B}
=
\mathbf{\xi}\mathbf{\xi}^\top
=
\begin{bmatrix}
1 & 1 & 1 & -1 \\
1 & 1 & 1 & -1 \\
1 & 1 & 1 & -1 \\
-1 & -1 & -1 & 1
\end{bmatrix}.
\]

\paragraph{Step 2: Remove diagonal terms}
Zero the diagonal entries to obtain the weight matrix \( \mathbf{W} \) with \(w_{ii} = 0\). The off-diagonal values remain the same as in \(\mathbf{B}\), encoding the pairwise interactions required to store the memory pattern.
Including the \(1/n\) normalization from \eqref{eq:hebbian_weights} (here \(n=4\)) gives
\[
\mathbf{W}
=
\frac{1}{4}\Big(\mathbf{B}-\operatorname{diag}(\mathbf{B})\Big)
=
\frac{1}{4}
\begin{bmatrix}
0 & 1 & 1 & -1 \\
1 & 0 & 1 & -1 \\
1 & 1 & 0 & -1 \\
-1 & -1 & -1 & 0
\end{bmatrix}.
\]

\paragraph{Sanity check: the stored pattern is a fixed point}
With zero thresholds, the local field is \(\mathbf{h}=\mathbf{W}\mathbf{\xi}\). For this one-pattern example,
\[
\mathbf{h}
=
\begin{bmatrix}
3/4 \\
3/4 \\
3/4 \\
-3/4
\end{bmatrix},
\]
so \(\operatorname{sign}(\mathbf{h})=\mathbf{\xi}\). That is the basic content-addressable ``snap-back'': if you start near \(\mathbf{\xi}\), asynchronous updates push you toward it.

\begin{tcolorbox}[summarybox, title={Numeric recall trace (two asynchronous flips)}]
Start from a probe with two wrong bits, \(\mathbf{s}^{(0)}=(1,-1,-1,-1)\). With \(\theta_i=0\), the local field is \(h_i=\sum_j w_{ij}s_j\), and the update rule \eqref{eq:hopfield_update_rule} sets \(s_i\leftarrow \operatorname{sign}(h_i)\). One possible asynchronous update order (neuron 2, then neuron 3) gives:
\begin{center}
\begin{tabular}{@{}lccc@{}}
\toprule
Step & Update & Local field & New state \\
\midrule
0 & -- & -- & \((1,-1,-1,-1)\) \\
1 & \(i=2\) & \(h_2=0.25\) & \((1,1,-1,-1)\) \\
2 & \(i=3\) & \(h_3=0.75\) & \((1,1,1,-1)=\mathbf{\xi}\) \\
\bottomrule
\end{tabular}
\end{center}
The energy drops along the way: \(E=0.5 \rightarrow 0 \rightarrow -1.5\).
\end{tcolorbox}

% QC-BEGIN: hopfield_single_pattern_weights
% xi 1 1 1 -1
% W 0 0.25 0.25 -0.25
% W 0.25 0 0.25 -0.25
% W 0.25 0.25 0 -0.25
% W -0.25 -0.25 -0.25 0
% h 0.75 0.75 0.75 -0.75
% recall_s0 1 -1 -1 -1 E 0.5
% recall_step 1 update_i 2 h 0.25 s 1 1 -1 -1 E 0.0
% recall_step 2 update_i 3 h 0.75 s 1 1 1 -1 E -1.5
% QC-END: hopfield_single_pattern_weights

\subsection{Finalizing the Hopfield Network Derivation and Discussion}
\label{sec:hopfield_finalizing_the_hopfield_network_derivation_and_discussion}

We have already defined the Hebbian weights in \eqref{eq:hebbian_weights}, the update rule in \eqref{eq:hopfield_update_rule}, and the energy in \eqref{eq:energy_function}. Here we focus on what those ingredients imply for retrieval basins and limitations in practice.

\paragraph{Memory Retrieval and Basins of Attraction}

The stored patterns \(\{\mathbf{\xi}^\mu\}\) correspond to local minima of the energy landscape. Starting from an initial state \(\mathbf{s}(0)\) that is a noisy or partial version of a stored pattern, the network dynamics converge to the closest attractor, ideally retrieving the original memory or its complement \(-\mathbf{\xi}^\mu\).

For example, if the initial state is corrupted, the network will iteratively update states to reduce energy until it reaches a stable point:
\[
\mathbf{s}(\infty) \in \{\mathbf{\xi}^\mu, -\mathbf{\xi}^\mu\}.
\]
The same story can be read visually as motion toward a basin of attraction in an energy landscape (\Cref{fig:hopfield-basins-schematic}).

\begin{tcolorbox}[summarybox, title={Terminology and diagnostics}]
\textbf{Attractor.} A stable fixed point of the update rule: once you reach it, further asynchronous updates do not change the state.

\textbf{Basin of attraction.} The set of initial states that converge to a given attractor under the chosen update scheme.

\textbf{Spurious state.} An attractor that is not one of the intended stored patterns (nor its complement). Spurious states are real local minima of \(E(\cdot)\) created by interference among memories.

\textbf{Overlap (a simple health check).} For a stored pattern \(\mathbf{\xi}^\mu\), define
\[
m^{(\mu)}(\mathbf{s}) = \frac{1}{N} (\mathbf{\xi}^\mu)^\top \mathbf{s}.
\]
It ranges from \(+1\) (exact match) to \(-1\) (exact complement). Plot \(E(\mathbf{s}(t))\) and \(m^{(\mu)}(\mathbf{s}(t))\) versus update step to see whether recall is behaving or drifting into a spurious basin.
\end{tcolorbox}

\begin{figure}[t]
    \centering
    \begin{tikzpicture}
        \begin{axis}[
            width=0.78\linewidth,
            height=4.2cm,
            xmin=-2.4, xmax=2.4,
            ymin=-1.2, ymax=1.8,
            axis lines=left,
            xlabel={State coordinate (schematic)},
            ylabel={Energy},
            ytick=\empty,
            xtick=\empty,
        ]
            \addplot[cbBlue, thick, samples=200, domain=-2.4:2.4]
                {0.25*(x^4) - 0.9*(x^2) + 0.4*sin(deg(2*x)) + 0.4};
            \node[font=\scriptsize, anchor=north] at (axis cs:-1.25,-0.5) {memory basin};
            \node[font=\scriptsize, anchor=north] at (axis cs:1.2,-0.55) {spurious basin};
            \addplot[cbPink, -{Latex[length=2mm]}, thick] coordinates {(-0.2,1.1) (-1.05,-0.35)};
            \node[cbPink, font=\scriptsize, anchor=west] at (axis cs:-0.15,1.1) {noisy probe};
        \end{axis}
    \end{tikzpicture}
    \caption{Energy-landscape intuition (schematic). Hopfield recall is a descent process toward a nearby basin (a stored memory), but other minima can exist and act as spurious attractors when the network is heavily loaded.}
    \label{fig:hopfield-basins-schematic}
\end{figure}

\paragraph{Limitations (what breaks first)}
Hopfield networks are a beautiful model to reason about, but several limitations show up quickly in practice:
\begin{itemize}
    \item \textbf{Capacity:} with the classical Hebbian construction, reliable storage for random patterns scales like \(P_{\max}\approx 0.138\,N\) for large \(N\). Past that loading level, retrieval errors and spurious minima become common.
    \item \textbf{Spurious attractors:} even below capacity, the energy landscape can contain minima that are not stored patterns (or their complements). A noisy probe can ``snap'' to one of these unintended fixed points.
    \item \textbf{Classification mismatch:} for discriminative tasks (say, digit recognition), the ``nearest minimum'' behavior is not the same thing as a calibrated class decision. A corrupted input can converge to the wrong stored pattern, and low energy does not imply correct label.
    \item \textbf{When not to use:} heavy loading (\(P/N\) large) produces a glassy landscape; scaling is \(O(N^2)\) in connectivity; and the low\hyp{}capacity regime makes classical Hopfield networks a poor fit for high\hyp{}dimensional supervised problems compared with the ERM models in \Cref{chap:logistic} or the deep models in \Cref{chap:transformers}.
\end{itemize}

\begin{tcolorbox}[summarybox, title={Applications lens: memory retrieval and optimization}]
In the classical framing, Hopfield networks are about associative memory: you store patterns, then recover a full pattern from a partial or noisy cue (pattern completion). That is the same idea behind many information-retrieval and recognition stories: the input is a corrupted probe, and the system ``snaps'' to a nearby clean prototype.

There is also an optimization lens. Because \(E(\mathbf{s})\) is a scalar objective over binary states, you can sometimes encode a cost function as an energy and run asynchronous updates as a heuristic descent method. This is one way Hopfield-style dynamics have been discussed for combinatorial optimization tasks (for example, variants of traveling-salesman-style objectives). The warning is the same one we give throughout this book: the algorithm will happily converge to a local minimum. Your real work is in deciding whether the energy you wrote down matches the engineering goal and whether the resulting minima are meaningful.
\end{tcolorbox}

\paragraph{Example: Memory recovery (one flip)}

Store one pattern
\[
\mathbf{\xi} = (-1, -1, 1, -1)^T.
\]
With \(n=4\), the Hebbian weights from \eqref{eq:hebbian_weights} are
\[
\mathbf{W} = \frac{1}{4} \mathbf{\xi} \mathbf{\xi}^\top, \qquad w_{ii} = 0,
\]
which numerically becomes a single symmetric matrix
\[
\mathbf{W} = \frac{1}{4}
\begin{pmatrix}
0 & 1 & -1 & 1 \\
1 & 0 & -1 & 1 \\
-1 & -1 & 0 & -1 \\
1 & 1 & -1 & 0
\end{pmatrix}.
\]
The off-diagonal entries are therefore the scaled products of pattern components (e.g., $w_{12}=w_{21}=0.25$ and $w_{13}=w_{31}=-0.25$).
More generally, every off-diagonal weight is the scaled product of the corresponding pattern entries.

Now start from a one-bit-corrupted probe \(\mathbf{s}^{(0)}=[-1,-1,1,1]^T\), with \(\theta_i=0\). If we update neuron \(4\),
\[
h_4=\sum_j w_{4j}s_j = -0.75 \quad\Rightarrow\quad s_4\leftarrow -1,
\]
so in one asynchronous flip we recover \(\mathbf{s}^{(1)}=\mathbf{\xi}\). The energy drops from \(E=0\) to \(E=-1.5\) under \eqref{eq:energy_function}. The appearance of \(-\mathbf{\xi}\) as a fixed point is expected: the energy only depends on products \(s_i s_j\), so negating all bits leaves every term unchanged.

% QC-BEGIN: hopfield_memory_recovery_4n
% xi -1 -1 1 -1
% W 0 0.25 -0.25 0.25
% W 0.25 0 -0.25 0.25
% W -0.25 -0.25 0 -0.25
% W 0.25 0.25 -0.25 0
% s0 -1 -1 1 1 E 0.0
% step 1 update_i 4 h -0.75 s -1 -1 1 -1 E -1.5
% QC-END: hopfield_memory_recovery_4n

\paragraph{Spurious attractors}

Beyond the intended memories \(\{\pm \mathbf{\xi}^\mu\}\), Hopfield networks can converge to \emph{spurious attractors}: stable states formed by mixtures of stored patterns. These unintended minima become increasingly common as the loading factor \(P/N\) grows (here \(P\) denotes the number of stored patterns); for random patterns the practical capacity is roughly \(0.138\, N\). The possibility of converging to a spurious state, or to the complemented memory rather than the original, explains why Hopfield networks are better viewed as associative memories than as discriminative classifiers.

\paragraph{Historical and practical significance}
I like Hopfield networks because they show a recurring engineering move: constrain a system until you can prove something useful, then use that proof as a debugging lens. Symmetry + no self-loops gives you an energy you can compute; asynchronous updates give you a monotone progress measure.

You will almost never deploy a classical Hopfield net as-is (capacity and spurious minima show up fast). But the idea ``memory as an objective landscape'' survives, and it is the right mental model to carry into later energy-based and content-addressable mechanisms.

\paragraph{Connections to other chapters.} Hopfield networks sit next to SOMs in spirit: both are unsupervised, but the diagnostic tools differ. SOMs give you map geometry (U-matrix/component planes); Hopfield gives you overlap and an energy trace to check whether recall is behaving. Later, attention revisits content-addressable lookup in a differentiable form.

\begin{tcolorbox}[summarybox, title={Key takeaways}]
\begin{itemize}
    \item Hopfield networks store binary patterns as attractors in an energy landscape defined by symmetric weights.
    \item Asynchronous updates monotonically reduce the Lyapunov energy, ensuring convergence to a fixed point.
    \item Capacity is limited and spurious attractors appear as the load \(P/N\) grows; treat Hopfield nets primarily as associative memories.
\end{itemize}

\medskip
\noindent\textbf{Minimum viable mastery.}
\begin{itemize}
    \item Write the energy \(E(s)\), state the symmetry condition \(W=W^\top\) with zero diagonal, and explain why asynchronous updates decrease \(E\).
    \item Distinguish true memories from spurious attractors, and connect failure modes to loading \(P/N\) and correlation among stored patterns.
    \item Use overlap and energy traces as diagnostics when demonstrating recall under corruption.
\end{itemize}

\noindent\textbf{Common pitfalls.}
\begin{itemize}
    \item Using asymmetric weights or nonzero self-connections (breaks the standard energy argument).
    \item Overloading the network and expecting clean recall (spurious minima dominate).
    \item Reporting a single cherry-picked recall trajectory instead of multiple corruption levels and multiple stored sets.
\end{itemize}
\end{tcolorbox}

\begin{tcolorbox}[summarybox, title={Exercises and lab ideas}]
\begin{itemize}
    \item Implement asynchronous recall for bipolar states with symmetric \(W\) and \(w_{ii}=0\). Plot energy vs.\ update step and verify it never increases.
    \item Store \(P\) random patterns with Hebbian weights. Use a fixed random seed and a fixed corruption protocol: pick one pattern \(\mathbf{\xi}^\mu\), flip \(k\) bits uniformly at random, then run asynchronous updates for a fixed budget (or until no flips occur). Sweep \(P/N\) and corruption level \(k\); plot both energy \(E(\mathbf{s}(t))\) and overlap \(m^{(\mu)}(\mathbf{s}(t))\) versus step, and report when spurious attractors become common.
    \item Construct (or search for) a small example where synchronous updates enter a 2-cycle; compare with asynchronous updates from the same initial state.
\end{itemize}

\medskip
\noindent\textbf{If you are skipping ahead.} The key transferable idea is the combination of a state update rule with a scalar quantity that tracks progress (energy, loss, or validation curves). That mindset carries into deep training in \Crefrange{chap:mlp}{chap:cnn}.
\end{tcolorbox}

\medskip
\paragraph{Where we head next.} \Cref{chap:cnn} turns from associative memory to deep feedforward perception, where convolution and pooling form hierarchical features. The optimization workflow is unchanged: the ERM toolkit from \Cref{chap:supervised} and the training mechanics from \Crefrange{chap:mlp}{chap:backprop} remain central. The analogy to keep in mind is the diagnostic habit: in Hopfield networks we watched an energy trace and overlap; in deep CNNs you will watch training/validation curves and slice errors. Recurrence and attention return in \Cref{chap:rnn} and \Cref{chap:transformers}.

\nocite{Hopfield1982, AmitGutfreundSompolinsky1985}

% Chapter 11
\section{Introduction to Deep Learning and Neural Networks}\label{chap:cnn}
\graphicspath{{assets/lec6/}{assets/lec8/}}

\begin{tcolorbox}[summarybox,title={Learning Outcomes}]
After this chapter, you should be able to:
\begin{itemize}
  \item Derive convolution/cross-correlation with stride and padding in 1D/2D.
  \item Explain receptive-field growth across layers and pooling effects.
  \item Compare loss choices for classification vs. regression and evaluation metrics.
  \item Connect the hinge-loss/soft-margin ideas from \Cref{chap:supervised} to kernels and CNN features.
  \item Describe practical optimizers and regularizers (BN, dropout, weight decay).
\end{itemize}
\end{tcolorbox}

\Cref{chap:hopfield} used energy to tame recurrence and create stable memories. Here we pivot back to deep feedforward models for perception, where convolutions and pooling impose a spatial inductive bias that improves sample efficiency and robustness. The roadmap in \Cref{fig:roadmap} shows this as the deep feedforward branch.

\begin{tcolorbox}[summarybox,title={Design motif}]
Keep the same statistical learning loop from \Crefrange{chap:supervised}{chap:logistic}, but move the ``bias'' into architecture via weight sharing and locality.
\end{tcolorbox}

\begin{tcolorbox}[summarybox,title={Shape reminder}]
Throughout this chapter we use the row-major (deep-learning) convention for batches: inputs \(X\in\mathbb{R}^{B\times d_{\text{in}}}\), weights \(W\in\mathbb{R}^{d_{\text{in}}\times d_{\text{out}}}\), and biases \(b\in\mathbb{R}^{d_{\text{out}}}\), with forward map \(Z=XW+\mathbf{1}b^\top\). When we write single-example equations, you can read them as the same convention with \(B=1\). For convolution/cross-correlation we follow the standard deep-learning tensor convention (channels and spatial axes); the same shape logic applies once the tensors are flattened into matrix form.
\end{tcolorbox}

\subsection{Historical Context and Motivation}

Artificial neural networks (ANNs) have a long history dating back to the 1940s, with the seminal work of McCulloch and Pitts in 1943. Despite this early start, it took several decades before deep learning models became widely successful and practical. Neural networks experienced waves of interest, notably in the 1980s and 1990s, but it was only in the last 10--15 years that deep architectures have become dominant.

Understanding why deep learning took so long to mature is crucial. Several challenges hindered progress for many years:

\begin{itemize}
    \item \textbf{Optimization hurdles:} Early neural networks were shallow (few layers) and suffered from problems such as vanishing or exploding gradients, making it hard to train deep models effectively.
    \item \textbf{Computational resources:} Deep networks require significant computational power and memory, which were not readily available until recent advances in hardware (e.g., GPUs).
    \item \textbf{Data availability:} Large labeled datasets, essential for training deep models, were scarce until the advent of big data.
    \item \textbf{Algorithmic improvements:} Innovations such as better activation functions, initialization schemes, and optimization algorithms were necessary to enable deep learning.
\end{itemize}

These factors combined to delay the widespread adoption of deep learning despite its theoretical potential.

\subsection{Overview of Neural Network Architectures}

Before delving into deep learning, it is important to review the basic building blocks of neural networks.

\subsubsection{Feedforward Neural Networks (Multi-Layer Perceptrons)}

A feedforward neural network consists of an input layer, one or more hidden layers, and an output layer. Data flows in one direction from input to output without cycles.

Consider a simple network with an input layer of dimension $d$ and a single hidden layer with $h$ neurons. We write a single input as a row vector
\[
\mathbf{x} = [x_1, x_2, \ldots, x_d],
\]
and the weight matrix connecting the input to the hidden layer is
\[
\mathbf{W} \in \mathbb{R}^{d \times h}.
\]

The pre-activation input to the hidden layer neurons is
\begin{align}
\mathbf{z} = \mathbf{x}\mathbf{W} + \mathbf{b}, \label{eq:preactivation}
\end{align}
where $\mathbf{b} \in \mathbb{R}^h$ is the bias vector.

Applying a nonlinear activation function $\sigma(\cdot)$ element-wise yields the hidden layer output
\[
\mathbf{h} = \sigma(\mathbf{z}).
\]

The output layer then produces the final output, often via another linear transformation and activation.

\paragraph{Fully Connected Layers and Feature Transformation}

Each neuron in the hidden layer is connected to every input feature, making the layer \emph{fully connected}. The weights $\mathbf{W}$ serve two main purposes:

\begin{itemize}
    \item \textbf{Feature extraction:} Each neuron computes a weighted combination of input features, effectively extracting new features.
\item \textbf{Attenuation of irrelevant inputs:} Weights with small magnitude suppress the contribution of certain input features, although genuine feature selection usually requires explicit regularization (e.g., L1 penalties) or pruning.
\end{itemize}

Thus, each layer transforms the input features into a new representation, which subsequent layers can further process.

\subsection{Why Shallow Networks Are Insufficient}

In theory, shallow networks with a single hidden layer are universal function approximators \citep{Cybenko1989}. However, in practice, they have several limitations:

\begin{itemize}
    \item \textbf{Large number of neurons required:} To approximate complex functions, shallow networks often need exponentially many neurons.
    \item \textbf{Overfitting:} Large networks with many parameters tend to overfit training data, especially with limited data.
\item \textbf{Limited expressivity:} Although universal approximators, shallow networks often require exponentially many neurons to capture rich structure, so depth provides a far more parameter-efficient representation.
\end{itemize}

Deep networks, with multiple hidden layers, can represent complex functions more compactly by learning hierarchical feature representations. This hierarchical structure is key to the success of deep learning.

\subsection{Limitations of Traditional Feedforward Neural Networks}

\paragraph{Requirement for large datasets}

Feedforward networks typically require large amounts of labeled data to generalize well. For small datasets (e.g., Titanic survival data, movie ratings), simpler models like logistic regression or decision trees may outperform neural networks due to overfitting risks.

\paragraph{High-dimensional inputs and flattening}

Consider image data, which is naturally represented as a 2D matrix (or 3D tensor for color images). For example, a single-channel (grayscale) image of size $256 \times 276$ pixels can be represented as a matrix:
\[
X \in \mathbb{R}^{256 \times 276}.
\]

To input this into a traditional feedforward network, the image must be \textit{flattened} into a vector:
\[
\mathbf{x} = \mathrm{vec}(X) \in \mathbb{R}^{70,656},
\]
where $70,656 = 256 \times 276$ is the total number of pixels.

This flattening process has two major drawbacks:

\begin{itemize}
    \item \textbf{Loss of spatial structure:} The 2D spatial relationships between pixels are ignored, which is critical for tasks like image recognition.
    \item \textbf{High dimensionality:} The input vector becomes very large, increasing the number of parameters and computational cost, and requiring more data to train effectively.
\end{itemize}

\paragraph{Implications}

These limitations motivate the development of specialized architectures, such as convolutional neural networks (CNNs), which exploit spatial locality and reduce parameter count by sharing weights.

\subsection{Challenges in Training Large Fully Connected Networks}

Consider a fully connected neural network where the input layer has 70,656 neurons (flattened from a $256 \times 276$ grayscale image), connected to a hidden layer with 100 neurons, which in turn connects to an output layer for classification. Although this is a simplified example, it illustrates key challenges in training large networks.

\paragraph{Parameter Explosion}

The number of weights between the input and hidden layer is:
\begin{equation}
70,656 \times 100 = 7,065,600,
\end{equation}
and between the hidden and output layer (assuming 4 output classes) is:
\begin{equation}
100 \times 4 = 400.
\end{equation}

Thus, the first layer alone requires learning just over 7 million parameters before we even consider deeper architectures. Coupled with the additional 400 output weights (plus biases), the optimization problem quickly becomes data-hungry and computationally expensive.

\paragraph{Data Requirements}

To reliably learn these parameters, the amount of training data must be sufficiently large. A common heuristic is that the number of training samples should be at least 10 times the number of parameters:
\begin{equation}
N_{\text{samples}} \geq 10 \times N_{\text{parameters}}.
\end{equation}

This rule-of-thumb is intentionally conservative and should be read as guidance rather than a hard requirement; in practice, regularization, data augmentation, and strong inductive biases often permit useful models with fewer samples.

For the first layer, this implies roughly:
\begin{equation}
N_{\text{samples}} \gtrsim 10 \times 7,000,000 \approx 70,000,000,
\end{equation}
meaning on the order of seventy million labeled images. This is an impractical requirement for most projects.

\paragraph{Computational and Storage Constraints}

Storing and processing such a large dataset requires enormous storage and computational resources. Training on hundreds of millions of images is typically infeasible for most research groups or applications without specialized infrastructure.

\paragraph{Overfitting Risk}

With millions of parameters, the model has high capacity and can easily memorize the training data, leading to overfitting. This means the network may not generalize well to unseen data, as it learns to fit noise or irrelevant details rather than meaningful features.

\subsection{Sparse connectivity and parameter sharing}

CNNs replace dense connections with \emph{local receptive fields}. Each output unit connects to a small neighborhood of pixels, not the entire image. If a \(k \times k\) filter scans a \(H\times W\) input, the number of learned weights is \(k^2\) (per channel), not \(HW\). The same filter is reused at every spatial location, so a single set of parameters detects a pattern anywhere in the image. This parameter sharing is the key to scalability: it cuts the parameter count and preserves translation equivariance.

Historically, dense MLPs on image benchmarks struggled to compete with classical pipelines because the parameter count and data requirements were prohibitive. CNNs reversed that trend by tying weights and focusing on local patterns while keeping the same gradient-based training loop, so capacity grows with depth and channel count rather than with the raw pixel grid.

\subsection{Convolution and pooling mechanics}

For an input feature map \(X\) and filter \(K\), the (cross-correlation) output at spatial location \((i,j)\) is
\[
(X * K)_{ij} = \sum_{u=0}^{k-1} \sum_{v=0}^{k-1} K_{uv}\,X_{i+u,\,j+v}.
\]
With stride \(s\) and padding \(p\), the output size along one axis is
\[
\left\lfloor \frac{n + 2p - k}{s} \right\rfloor + 1,
\]
so padding controls resolution while stride controls downsampling. Pooling then aggregates local neighborhoods (typically max or average) to build invariance and further reduce spatial size.

Convolution itself is a linear map; the nonlinearity enters when we apply an activation after each convolutional block. Stacking these linear--nonlinear stages yields hierarchical feature detectors (edges \(\rightarrow\) textures \(\rightarrow\) parts \(\rightarrow\) objects) without abandoning the backprop training machinery.

Without padding (often called \emph{valid} convolution), boundary pixels participate in fewer receptive fields and the spatial grid shrinks each layer. \emph{Same} padding chooses \(p=(k-1)/2\) for odd \(k\) to preserve spatial size when \(s=1\) and give edge pixels comparable influence. Larger strides reduce resolution and compute but can skip fine detail, so the padding/stride combination is a deliberate trade-off rather than a fixed rule.

\subsubsection{Worked stride and padding example}

Suppose an input is \(6\times6\) and the filter is \(3\times3\).
\begin{itemize}
    \item \textbf{Stride \(s=1\), valid padding (\(p=0\)).} Output size is \((6-3)/1 + 1 = 4\), so you get a \(4\times4\) feature map.
    \item \textbf{Stride \(s=2\), valid padding.} Output size is \(\left\lfloor(6-3)/2\right\rfloor + 1 = 2\), so you get a \(2\times2\) feature map.
    \item \textbf{Stride \(s=1\), same padding.} With \(k=3\), choose \(p=1\) so the output stays \(6\times6\).
\end{itemize}
The floor in the formula is important: if \((n + 2p - k)\) is not divisible by \(s\), the last partial window is dropped.

\subsection{Pooling as nonparametric downsampling}

Pooling reduces spatial size without learning new parameters: a max-pooling window keeps the strongest activation; average pooling keeps the mean. This is a nonparametric operation, so it can feel like ``cheating'' compared to learned filters, yet it often improves robustness by discarding small shifts and noise. Max pooling is the most common because it preserves the strongest feature response, but average and even median pooling appear in specialized settings. In modern CNNs, aggressive pooling is used sparingly; strided convolutions are a common alternative when you want learned downsampling instead of a fixed aggregation.

\begin{tcolorbox}[summarybox,title={Author's note: pooling is a design choice}]
Pooling is not ``more correct'' than strided convolutions; it is a trade-off. If you want downsampling with learned weights, use stride. If you want a fixed local summary (often more stable early in training), max pooling is a reasonable default. Avoid padding in pooling unless you need spatial alignment with a parallel branch.
\end{tcolorbox}

\begin{tcolorbox}[summarybox,title={Author's note: ``convolution'' vs.\ cross-correlation}]
In most deep-learning libraries, the operation called ``convolution'' is technically \emph{cross-correlation}: the kernel is not flipped before sliding. The name stuck because the two operations differ only by a reversal of the kernel, and the kernel is learned anyway. What matters in practice is that the ``filter'' is a small learned weight matrix that is applied repeatedly across space (weight sharing). Learn the shapes, the stride/padding bookkeeping, and the inductive bias; the terminology is imperfect but the idea is powerful.
\end{tcolorbox}

\subsection{Channels and feature maps}

Real inputs are multi-channel. An RGB image has three channels, so a \(k\times k\) filter is really \(k\times k\times C_{\text{in}}\) and produces one output map. A convolutional layer applies \(C_{\text{out}}\) such filters, yielding a volume of feature maps with shape \(H\times W\times C_{\text{out}}\). This is why ``same'' padding refers to spatial dimensions only: channel depth is set by the number of filters, not by padding.

\begin{tcolorbox}[summarybox,title={Dimensionality bookkeeping example}]
Start with an input tensor of size \(50\times50\times30\). Apply \(10\) filters of size \(3\times3\) with stride \(1\) and valid padding. The spatial size becomes \(50-3+1=48\), so the output is \(48\times48\times10\). Apply \(2\times2\) max pooling with stride \(2\): the spatial size becomes \(\left\lfloor(48-2)/2\right\rfloor + 1 = 24\), so the pooled output is \(24\times24\times10\). Flattening that volume yields \(24\cdot24\cdot10=5760\) features for a dense classifier.
\end{tcolorbox}

\subsection{Convolutional hyperparameters (what you choose up front)}
\begin{itemize}
    \item \textbf{Filter size} (\(k\times k\)) and \textbf{number of filters} (\(C_{\text{out}}\)).
    \item \textbf{Stride} \(s\) and \textbf{padding} \(p\) (valid vs.\ same).
    \item \textbf{Activation} function after each convolution.
    \item \textbf{Pooling} type (max/average), window size, and stride (if pooling is used).
\end{itemize}

\subsection{Multi-branch convolution blocks (Inception idea)}
One practical variant is to run multiple filter sizes in parallel (for example \(1\times1\), \(3\times3\), and \(5\times5\)) and concatenate the resulting feature maps along the channel axis. This allows the network to capture multi-scale patterns without committing to a single kernel size. When branches must line up spatially, \emph{same} padding is used to keep all outputs the same height and width; pooling branches often pad for this alignment even though pooling by itself is usually unpadded.

\subsection{From feature maps to classifiers}

Stacks of convolutional and pooling layers produce a hierarchy of feature maps. A common design is to flatten the final maps into a vector and pass them to a small dense classifier (often a softmax layer) that predicts the class label. Backpropagation updates both the dense weights and the shared convolutional filters, so the feature extractor and classifier are learned jointly.

\subsection{Historical Context and the 2012 Breakthrough}

Before 2012, neural networks were often dismissed in many academic circles due to their poor performance on large-scale problems and the dominance of other methods such as Support Vector Machines (SVMs). The sentiment was that neural networks were "fancy" but not practical or well-understood.

\paragraph{SVM geometry refresher.} The classical soft-margin picture views an SVM as balancing a wide margin against slack variables \(\xi_i\) that widen the feasible tube so that mislabeled points incur linear penalties instead of rendering the optimization infeasible. The geometric intuition is to maximize the margin while tolerating limited violations. This highlights why SVMs were attractive when data were scarce; the hinge-loss curves in \Cref{chap:supervised} supply the loss-level view of the same trade-off.

\paragraph{Stack depth versus receptive field.} Stacking identical \(3\times3\) filters still grows the effective receptive field: each stage wraps a thicker box around the original pixels, which explains why deep-but-narrow CNNs can capture wide spatial context without enormous kernels even when individual kernels remain small.

With the architectural motivation in place, we now focus on how these models are trained and why deep optimization remains delicate.

\subsection{Training Neural Networks: Gradient-Based Optimization}

Training neural networks involves minimizing a loss function $\mathcal{L}$ that measures the discrepancy between the network output and the target. The parameters (weights and biases) are updated iteratively using gradient descent or its variants.

For a weight $w$, the update rule is
\begin{align}
w \leftarrow w - \eta \frac{\partial \mathcal{L}}{\partial w}.
\end{align}
where $\eta$ is the learning rate.

\paragraph{Backpropagation and Gradient Computation}

The gradient $\frac{\partial \mathcal{L}}{\partial w}$ is computed efficiently using the backpropagation algorithm, which applies the chain rule to propagate errors backward through the network layers.

\paragraph{Challenges in Deep Networks}

In deep networks, gradients can vanish or explode as they propagate through many layers, making training unstable or slow. This problem was a major obstacle until solutions such as better activation functions (e.g., ReLU), normalization techniques, and initialization methods were developed.

\subsection{Deep Network Optimization \mbox{Challenges}}

Deep networks are difficult to optimize because the objective is highly nonconvex and the
gradient signal can be distorted as it flows through many layers. We begin with the most
basic pathology---vanishing and exploding gradients---and then summarize practical
mitigations.

% Chapter 11 (continued)

\subsection{Vanishing and Exploding Gradients in Deep Networks}

Recall from the previous discussion that when training deep neural networks, the backpropagation algorithm involves repeated multiplication of gradients through many layers. This repeated multiplication can cause gradients to either vanish (approach zero) or explode (grow exponentially large), leading to significant training difficulties.

\paragraph{Mathematical intuition}

Consider a deep network with $L$ layers. Let $\delta \mathbf{W}^{(\ell)} = \nabla_{\mathbf{W}^{(\ell)}} \mathcal{L}$ denote the gradient of the loss with respect to the weights at layer $\ell$. If we assume the weights are initialized identically and the derivative of the activation function is approximately constant, then the gradient at the first layer can be expressed schematically as:
\begin{equation}
    \delta \mathbf{W}^{(1)} \approx \left( W^{(2)} D^{(2)} \right) \left( W^{(3)} D^{(3)} \right) \cdots \left( W^{(L)} D^{(L)} \right) \delta \mathbf{W}^{(L)}.
    \label{eq:gradient_chain}
\end{equation}
where $W$ represents the weight matrix and $f'$ is the derivative of the activation function.

Here $W^{(\ell)}$ denotes the weight matrix connecting layers $\ell-1$ and $\ell$, while $D^{(\ell)} = \operatorname{diag}\!\big(f'(z^{(\ell)})\big)$ collects the activation derivatives at layer $\ell$. The product therefore chains together Jacobians from layers $2$ through $L$. If the spectral norm (largest singular value) of each factor \(W^{(\ell)} D^{(\ell)}\) exceeds one, then $\|\delta \mathbf{W}^{(1)}\|$ grows exponentially with $L$, causing \textbf{exploding gradients}. Conversely, norms less than one cause $\|\delta \mathbf{W}^{(1)}\|$ to shrink exponentially, leading to \textbf{vanishing gradients}.

\paragraph{Consequences}

\begin{itemize}
    \item \textbf{Exploding gradients:} The gradient values become extremely large, causing numerical instability and making the network parameters diverge during training.
    \item \textbf{Vanishing gradients:} The gradient values approach zero, especially in early layers, preventing those weights from updating effectively. This stalls learning in the initial layers, limiting the network's ability to learn hierarchical features.
\end{itemize}

\paragraph{Example: Activation function derivatives}

Consider the sigmoid activation function $\sigma(x) = \frac{1}{1 + e^{-x}}$. Its derivative is:
\[
\sigma'(x) = \sigma(x)(1 - \sigma(x)).
\]
Note that $\sigma'(x)$ approaches zero when $\sigma(x)$ is near 0 or 1, i.e., when the neuron output saturates. This saturation leads to very small gradients, exacerbating the vanishing gradient problem.
The derivative is maximized at $\sigma(x)=0.5$, where $\sigma'(x)=0.25$, so repeatedly multiplying sigmoid derivatives can shrink gradients roughly like $0.25^L$ across $L$ layers.

\paragraph{Notation note.} In this chapter $\sigma(\cdot)$ always denotes the logistic sigmoid nonlinearity, whereas symbols such as $\sigma^2$ are reserved for variances in earlier statistical chapters; context (function of an argument vs. squared scalar) distinguishes them.

\subsection{Strategies to Mitigate Vanishing and Exploding Gradients}

\paragraph{Weight initialization}

\begin{sloppypar}
Initializing weights carefully can help maintain gradient magnitudes within a reasonable range.
Set var $\approx\,1/n$ (fan\hyp{}in $n$).\\
This stabilizes signals across layers. It underlies Xavier and He.
\end{sloppypar}

\paragraph{Choice of activation function}

Selecting activation functions whose derivatives do not vanish easily is crucial. For example:

\begin{itemize}
    \item \textbf{ReLU (Rectified Linear Unit):} Defined as
    \[
    \mathrm{ReLU}(x) = \max(0, x),
    \]
    its derivative is 1 for positive inputs and 0 otherwise. This avoids saturation in the positive regime and helps maintain gradient flow.

    \item \textbf{Leaky ReLU and variants:} These allow a small, non-zero gradient when the input is negative, further mitigating dead neurons and keeping derivatives away from exact zero.
\end{itemize}

\paragraph{Batch normalization}

Batch normalization normalizes layer inputs during training, reducing the effective internal covariate shift and helping gradients maintain stable magnitudes.

\paragraph{Gradient clipping}

For exploding gradients, gradient clipping limits the maximum gradient norm during backpropagation, preventing excessively large updates.

Taken together, these tools stabilize optimization; the figures below highlight dropout, normalization, and optimizer behavior in practice.


\begin{figure}[t]
    \centering
    \includegraphics[width=0.65\linewidth]{lec6_dropout_curves}
    \caption[Dropout effect on training/validation curves (validation flattening)]{Schematic: Dropout effect on training/validation curves. Compared to a no-dropout baseline, validation curves flatten and generalization improves.}
    \label{fig:lec6-dropout}
\end{figure}

\paragraph{Batch normalization.} BN accelerates convergence by normalizing mini\hyp{}batch statistics and learning scale/shift parameters. \Cref{fig:lec6-bn} contrasts the pre- and post\hyp{}normalization activation distributions; whitening the distribution keeps gradients in a well-behaved range and reduces covariate shift. In deep Transformer stacks, a closely related design choice is \emph{pre-LN} versus \emph{post-LN}: modern architectures typically place layer normalization \emph{before} the residual block (pre-LN) to improve training stability on very deep networks.

\begin{figure}[t]
    \centering
    \begin{tikzpicture}
        \begin{groupplot}[
            group style={group size=2 by 1, horizontal sep=1.2cm},
            width=0.42\linewidth, height=0.34\linewidth,
            xlabel={$x$}, ylabel={density}, xmin=-6, xmax=6, ymin=0, ymax=0.5
        ]
        \nextgroupplot[title={Pre-BN activations}]
            \addplot[cbBlue, thick, samples=200, domain=-6:6] {1/sqrt(2*pi*1.5)*exp(-((x-2)^2)/(2*1.5))};
            \addplot[cbOrange, thick, samples=200, domain=-6:6] {1/sqrt(2*pi*0.8)*exp(-((x+1)^2)/(2*0.8))};
        \nextgroupplot[title={Post-BN (per-channel)}]
            \addplot[cbBlue, thick, samples=200, domain=-6:6] {1/sqrt(2*pi*1.0)*exp(-(x^2)/2)}; \addlegendentry{channel 1}
            \addplot[cbOrange, thick, samples=200, domain=-6:6] {1/sqrt(2*pi*1.0)*exp(-(x^2)/2)}; \addlegendentry{channel 2}
        \end{groupplot}
    \end{tikzpicture}
    \caption{Schematic: Batch normalization transforms per-channel activations toward zero mean and unit variance prior to the learned affine re-scaling, stabilizing training.}
    \label{fig:lec6-bn}
\end{figure}

\paragraph{Adaptive optimizers.} While vanilla SGD remains a workhorse, Adam and related methods adapt learning rates per-parameter \citep{Kingma2015}. \Cref{fig:lec6-optimizers} summarizes the typical loss trajectories; Adam converges faster initially, whereas SGD+momentum often attains a slightly lower asymptote after fine-tuning.

\begin{figure}[t]
    \centering
    \begin{tikzpicture}
        \begin{axis}[
            width=0.72\linewidth, height=0.36\linewidth,
            xlabel={epoch}, ylabel={loss}, xmin=0, xmax=60, ymin=0, ymax=1.1,
            legend style={at={(0.5,1.05)},anchor=south,draw=none,fill=none}
        ]
            \addplot[cbBlue, thick, samples=120, domain=0:60] {0.2 + 0.9*exp(-x/20)}; \addlegendentry{SGD+momentum}
            \addplot[cbOrange, thick, samples=120, domain=0:60] {0.15 + 1.0*exp(-x/10)}; \addlegendentry{Adam}
        \end{axis}
    \end{tikzpicture}
    \caption{Schematic: Representative training curves for SGD with momentum versus Adam on the same CNN.}
    \label{fig:lec6-optimizers}
\end{figure}

\begin{tcolorbox}[summarybox,title={Practical optimizer notes}]
\textbf{Mixed precision.} Modern CNN stacks often run activations/gradients in FP16 or BF16 while keeping master weights in FP32. Frameworks such as PyTorch AMP/TF mixed precision insert dynamic loss scaling so gradients do not underflow; the reward is higher throughput and lower memory pressure on recent GPUs/TPUs.\\
\textbf{AdamW vs.\ Adam.} Decoupled weight decay (AdamW) subtracts \(\eta\lambda W\) outside the adaptive-moment step, avoiding the ``L2-as-gradient-scaling'' behavior of classical Adam and producing more predictable regularization \citep{Loshchilov2019}. In code:
\begin{verbatim}
m = beta1 * m + (1-beta1) * grad
v = beta2 * v + (1-beta2) * grad**2
W -= eta * (m_hat / (sqrt(v_hat) + eps) + lambda * W)
\end{verbatim}
Use AdamW (or SGD+momentum) when you want clean weight-decay semantics; reserve plain Adam for rapid prototyping or when adaptive steps dominate regularization.
\end{tcolorbox}

\begin{tcolorbox}[summarybox,title={MLP/CNN block pseudocode (schematic)}]
\small
\begin{verbatim}
function ForwardBackward(params, x, y):
  # forward
  caches = []
  a = x
  for (W, b, f) in params.layers:
    z = W @ a + b
    caches.append((a, z))
    a = f(z)
  loss, delta = params.loss(a, y)

  # backward
  grads = []
  for (W, b, f), (a_prev, z_prev) in
      reversed(list(zip(params.layers, caches))):
    grads.append((delta @ a_prev.T, delta))
    delta = (W.T @ delta) * f'(z_prev)

  params.update(grads[::-1])
  return loss
\end{verbatim}
\normalsize
This omits batching, convolution strides, and optimizer detail but highlights the cache-then-backprop pattern reused throughout the deep-learning chapters.
\end{tcolorbox}

\begin{tcolorbox}[summarybox,title={Key takeaways}]
\begin{itemize}
    \item Convolutions introduce sparse connectivity and parameter sharing, dramatically reducing parameters vs. fully connected layers.
    \item Padding and stride control spatial resolution; pooling aggregates features to build invariances.
    \item Batch normalization, dropout, and optimizer choice strongly influence training stability and generalization.
    \item Stacking small kernels expands the effective receptive field across depth.
\end{itemize}
\end{tcolorbox}

\begin{tcolorbox}[summarybox,title={Exercises and lab ideas}]
\begin{itemize}
    \item Hand-compute a 1D cross-correlation with several \((n,f,s,p)\) tuples and verify the shapes and values against a small script.
    \item Compare max-pool + stride 1 vs.\ stride-2 convolutions on a toy dataset; report accuracy and FLOPs.
    \item Equivariance sanity check: translate an image and confirm intermediate feature maps translate accordingly.
    \item Train two depth-10 CNNs on a tiny dataset: one plain, one with identity skips; compare convergence and accuracy.
\end{itemize}
\end{tcolorbox}

\medskip
\paragraph{Where we head next.} \Cref{chap:rnn} returns to sequence modeling: recurrent networks and attention mechanisms inherit many of the optimization tools discussed here, but add temporal dependencies and stateful computation.

\paragraph{References.} Full citations for works mentioned in this chapter appear in the book-wide bibliography.

% Chapter 12
\section{Introduction to Recurrent Neural Networks}\label{chap:rnn}
\graphicspath{{assets/lec7/}{assets/lec9/}{assets/lec14/}}

\begin{tcolorbox}[summarybox,title={Learning Outcomes}]
\begin{itemize}
    \item Explain why recurrent structures are needed for sequence modeling and contrast them with feedforward nets.
    \item Derive the forward dynamics and backpropagation-through-time (BPTT) updates for vanilla RNN cells.
    \item Recognize practical stabilization techniques (gradient clipping, gating, normalization) that motivate later LSTM/Transformer chapters.
\end{itemize}
\end{tcolorbox}

\Cref{chap:cnn} showed how architectural bias (convolution/pooling) can replace some data demands while keeping the same optimization loop. This chapter turns to sequences, where the missing ingredient is memory: the model must remember enough of the past to act in the present. The roadmap in \Cref{fig:roadmap} flags this as the sequential branch of the neural strand.

\begin{tcolorbox}[summarybox,title={Design motif}]
Add recurrence, then train it by unrolling time and reusing the backprop machinery from \Cref{chap:backprop}.
\end{tcolorbox}

\subsection{Quick recap: padding in CNNs}

For a 1D or 2D convolution with input size \(n\), kernel size \(k\), stride \(s\), and padding \(p\), the output size is
\[
\left\lfloor \frac{n + 2p - k}{s} \right\rfloor + 1.
\]
When \(s=1\) and you want to preserve the original size, choose \(p=(k-1)/2\) for odd \(k\); padding \(p\) means adding \(p\) zeros on each side (left/right in 1D, all borders in 2D). This bookkeeping matters later when we compare sequence padding to spatial padding in CNNs.

\subsection{Autoencoders and latent representations}

Autoencoders map an input \(\mathbf{x}\) through an \emph{encoder} to a latent vector \(\mathbf{z}\), then reconstruct \(\hat{\mathbf{x}}\) with a \emph{decoder}. Training uses the input itself as the target, so the loss penalizes reconstruction error. An \emph{undercomplete} bottleneck (\(\dim \mathbf{z} < \dim \mathbf{x}\)) behaves like compression; an \emph{overcomplete} latent space can still learn useful structure when paired with regularization. We mention autoencoders here because the idea of encoding a long input into a compact representation will reappear in sequence models.

The statistical learning chapters (\Crefrange{chap:supervised}{chap:logistic}) established the basic training loop: choose a model class, define a loss, optimize it, and then audit generalization and calibration. The feedforward neural chapters then extended the model class from linear predictors to multilayer networks (MLPs, RBF networks, and CNNs). Those architectures have proven effective for tasks such as classification, regression, and feature extraction, but they share a common structural limitation: information flows strictly from input to output, without an internal state that can store context over time.

\subsection{Motivation for Recurrent Neural Networks}

Before delving into the architecture and mathematics of RNNs, it is important to understand why feedforward networks are insufficient for certain applications. Consider the following scenario:

\begin{quote}
\textit{You want to predict an output at time $t$ based not only on the input at time $t$, but also on inputs from previous time steps $t-1, t-2, \ldots, t-k$.}
\end{quote}

This is a common situation in many real-world problems, such as:

\begin{itemize}
    \item Time series forecasting (e.g., stock prices, weather data)
    \item Natural language processing (e.g., predicting the next word in a sentence)
    \item Speech recognition and synthesis
    \item Control systems with memory of past states
\end{itemize}

Order carries meaning even in simple settings: if today is Saturday, the next day is Sunday. In language, ``out of the blue'' means something sudden, whereas ``the ball was blue'' refers to color; the sequence makes the difference. In predictive text, ``I want to buy ...'' favors a different continuation than ``Write a book about Teddy ...'' These examples also highlight variable-length inputs: a review can be three words or three hundred. You can always build a fixed window of past inputs and feed it to a standard MLP, but that approach scales poorly as the history grows and does not share parameters across time.

Feedforward networks treat each input independently and do not have an inherent mechanism to remember or utilize past inputs. To incorporate past information, one might consider explicitly including previous inputs as part of the current input vector, but this approach quickly becomes impractical as the history length grows.

\subsection{Key Idea: State and Memory in RNNs}

Recurrent neural networks address this limitation by introducing a \emph{state vector} $\mathbf{h}_t$ that summarizes information from all previous inputs up to time $t$. The state is updated recursively as new inputs arrive, allowing the network to maintain a form of memory.

Formally, at each time step $t$, the RNN receives an input vector $\mathbf{x}_t$ and updates its hidden state $\mathbf{h}_t$ according to a function $f$ parameterized by weights $\theta$:

\begin{align}
    \mathbf{h}_t = f(\mathbf{h}_{t-1}, \mathbf{x}_t; \theta)
    \label{eq:rnn_state_update}
\end{align}

The output $\mathbf{y}_t$ at time $t$ is then computed as a function of the current state:

\begin{align}
    \mathbf{y}_t = g(\mathbf{h}_t; \theta')
    \label{eq:rnn_output}
\end{align}

Here, $f$ and $g$ are typically nonlinear functions implemented by neural network layers, and $\theta, \theta'$ are learned parameters.

\paragraph{Interpretation:} The hidden state $\mathbf{h}_t$ acts as a \emph{summary} or \emph{encoding} of the entire input history $\{\mathbf{x}_1, \mathbf{x}_2, \ldots, \mathbf{x}_t\}$. This allows the network to make predictions that depend on the temporal context, not just the current input.

\paragraph{Parameter sharing across time.} The same weights are reused at each time step. In the simplest formulation there are three learned matrices: \(W_{xh}\) (input to state), \(W_{hh}\) (state to state), and \(W_{hy}\) (state to output). When you unroll the recurrence, these matrices appear at every step, so the model learns a single transition rule rather than a separate set of parameters for each position in the sequence.

Unrolling makes the training graph \(T\) steps deep. Gradients from each time step accumulate onto the shared weights, and the repeated Jacobian products are precisely why long sequences revive the vanishing/exploding gradient issues from deep feedforward networks.
Training this unrolled graph with standard backpropagation is known as \emph{backpropagation through time} (BPTT).

Recurrent neural networks (RNNs) were among the first practical sequence models \citep{Elman1990,Bengio1994}. CNNs from \Cref{chap:cnn} trade recurrence for parallel, spatially shared filters, while \Cref{chap:transformers} will revisit sequence modeling without recurrence. \Cref{chap:nlp} supplies the embeddings and perplexity metrics commonly paired with RNNs.

\subsection{Comparison with Feedforward Networks}

To contrast, a feedforward network computes the output at time $t$ as:

\begin{align}
    \mathbf{y}_t = \psi(\mathbf{x}_t; \theta)
    \label{eq:ffn_output}
\end{align}

where $\psi$ is a nonlinear function without any dependence on past inputs. This limits the ability of feedforward networks to model temporal dependencies unless the input vector $\mathbf{x}_t$ explicitly contains past information.

\paragraph{Summary:} RNNs extend feedforward networks by incorporating a recurrent connection that allows information to persist across time steps, enabling modeling of sequences and temporal dynamics.

\begin{tcolorbox}[summarybox,title={Shape reminder}]
We keep the row-major (deep-learning) convention: \(\mathbf{x}_t \in \mathbb{R}^{d_x}\), \(\mathbf{h}_t \in \mathbb{R}^{d_h}\), pre-activation \(\mathbf{a}_t = \mathbf{x}_t W_{xh} + \mathbf{h}_{t-1} W_{hh} + \mathbf{b}_h\), output \(\mathbf{y}_t = \mathbf{h}_t W_{hy} + \mathbf{b}_y\). Column-vector formulations simply transpose the order of factors; all stability conclusions (spectral norms of Jacobian factors) carry over.
\end{tcolorbox}

\begin{tcolorbox}[summarybox,title={Simple RNN at a glance}]
\begin{itemize}
    \item \textbf{Objective:} Minimize cross\hyp{}entropy (or another sequence loss) between targets and \(p_\theta(y_t\mid h_t)\), with \(h_t = f(x_t W_{xh} + h_{t-1} W_{hh} + b_h)\).
    \item \textbf{Key hyperparameters:} Hidden state size, BPTT truncation window, optimizer/learning rate, regularization (dropout, weight decay), gradient-clipping threshold.
    \item \textbf{Defaults:} \(\tanh\) or ReLU activations, hidden size 128--512, Adam or SGD+momentum with clipping, layer norm or gating (LSTM/GRU) for long sequences.
    \item \textbf{Common pitfalls:} Vanishing/exploding gradients on long sequences, too-small hidden states, and train/test mismatch when teacher forcing is not reflected at inference.
\end{itemize}
\end{tcolorbox}

\subsection{Outline of this chapter}

In this chapter, we will:

\begin{itemize}
    \item Revisit CNN padding and autoencoders as a bridge to sequence encoders.
    \item Formally define the architecture of recurrent neural networks.
    \item Derive the forward and backward passes for training RNNs.
    \item Discuss challenges such as vanishing and exploding gradients.
    \item Introduce variants of RNNs designed to mitigate these challenges.
    \item Explore applications where RNNs provide significant advantages over feedforward networks.
\end{itemize}

\begin{tcolorbox}[summarybox,title={Vanilla RNN cell (forward + BPTT)}]
\begin{verbatim}
# Forward for a sequence {x_t, y_t}
h_0 = 0
for t = 1..T:
    pre_h = h_{t-1} W_hh + x_t W_xh + b_h
    h_t  = tanh(pre_h)
    yhat_t = h_t W_hy + b_y

# Backward (BPTT with optional truncation K)
delta_pre_next = 0
for t = T..1:
    delta_y = grad_loss(yhat_t, y_t)
    grad_W_hy += h_t^T delta_y
    grad_b_y += delta_y
    delta_h = delta_y W_hy^T + delta_pre_next W_hh^T
    delta_pre = delta_h .* (1 - h_t^2)
    grad_W_hh += h_{t-1}^T delta_pre
    grad_W_xh += x_t^T delta_pre
    grad_b_h += delta_pre
    delta_pre_next = delta_pre
    if t < T-K: break  # truncated BPTT
\end{verbatim}
Use gradient clipping (e.g., clip the global norm of parameter gradients) and layer/batch normalization when sequences are long to avoid exploding/vanishing gradients.
\end{tcolorbox}
The element-wise product in the backward step corresponds to the Hadamard factors described in the derivation in \Cref{chap:mlp}; we write it as \texttt{.*} to align with NumPy/Matlab notation.

\paragraph{Code--math dictionary.} In code blocks we use ASCII identifiers such as \texttt{h\_t}, \texttt{W\_hh}, and \texttt{b\_h}; in equations the same objects appear as \(\mathbf{h}_t\), \(\mathbf{W}_{hh}\), and \(\mathbf{b}_h\) (boldface for vectors/matrices, subscripts for time and role).

Detailed algebraic derivations (forward/backward passes and gradient expressions) appear in \Crefrange{eq:simple_rnn_hidden}{eq:simple_rnn_output}; readers are encouraged to work through the accompanying examples to solidify intuition.

\subsection{Recap: Feedforward Building Blocks}

RNNs reuse the same ingredients as multilayer perceptrons (activations, nonlinear decision boundaries, loss functions, and training heuristics) but wrap them around a temporal axis. \Cref{fig:backprop-computational-graph} from \Cref{chap:backprop} highlights the canonical MLP dataflow along with common activation choices and derivatives that govern gradient flow.

Two-dimensional toy datasets remain useful for reasoning about inductive biases. \Cref{fig:lec7-boundaries} contrasts logistic regression and a shallow MLP on the moons dataset, illustrating how additional hidden units carve nonlinear boundaries that RNN readouts later rely on when decoding the final state.

\begin{figure}[t]
    \centering
    \begin{tikzpicture}
        \begin{groupplot}[
            group style={group size=2 by 1, horizontal sep=1cm},
            width=0.43\linewidth,
            height=0.38\linewidth,
            xmin=-1.2, xmax=1.6,
            ymin=-0.8, ymax=1.4,
            xlabel={$x_1$},
            ylabel={$x_2$},
            axis equal,
            legend style={at={(0.03,0.03)},anchor=south west,legend columns=2,fill=white,draw=none}
        ]
            \nextgroupplot[title={Logistic regression}]
                \addplot+[only marks, mark=*, color=cbBlue, mark size=1.6pt] coordinates {
                    (0.1,0.4) (0.3,0.8) (-0.1,0.9) (-0.4,0.6) (-0.6,0.2)
                    (-0.3,0.1) (-0.7,0.55) (-0.2,0.35) (0.2,1.0) (0.5,0.6)
                };
                \addlegendentry{Class 1}
                \addplot+[only marks, mark=square*, color=cbOrange, mark size=1.6pt] coordinates {
                    (0.4,-0.2) (0.8,-0.3) (1.0,0.1) (0.6,-0.5) (0.2,-0.4)
                    (-0.2,-0.3) (-0.5,-0.4) (0.9,0.35) (1.2,0.2) (0.6,0.2)
                };
                \addlegendentry{Class 2}
                \addplot[thick, color=black] coordinates {(-1.2,-0.2) (1.6,0.6)};
                \node[anchor=west, font=\scriptsize] at (axis cs:0.9,0.65){$w^\top x + b = 0$};
            \nextgroupplot[title={Shallow MLP}]
                \addplot+[only marks, mark=*, color=cbBlue, mark size=1.6pt] coordinates {
                    (0.1,0.4) (0.3,0.8) (-0.1,0.9) (-0.4,0.6) (-0.6,0.2)
                    (-0.3,0.1) (-0.7,0.55) (-0.2,0.35) (0.2,1.0) (0.5,0.6)
                };
                \addplot+[only marks, mark=square*, color=cbOrange, mark size=1.6pt] coordinates {
                    (0.4,-0.2) (0.8,-0.3) (1.0,0.1) (0.6,-0.5) (0.2,-0.4)
                    (-0.2,-0.3) (-0.5,-0.4) (0.9,0.35) (1.2,0.2) (0.6,0.2)
                };
                \addplot[cbGreen, thick, domain=-1.2:1.4, samples=200]
                    {0.35 + 0.55*sin(deg(2.1*x)) - 0.1*x};
                \node[anchor=west, font=\scriptsize, cbGreen] at (axis cs:0.9,0.95){Nonlinear separator};
        \end{groupplot}
    \end{tikzpicture}
    \caption{Schematic: Decision boundaries for logistic regression (left) versus a shallow MLP (right). Linear models carve a single hyperplane, whereas hidden units can warp the boundary to follow non-convex manifolds such as the moons dataset.}
    \label{fig:lec7-boundaries}
\end{figure}

Finally, \Cref{fig:lec7-loss-hyperparams} summarizes two diagnostics: BCE geometry and the effect of learning\hyp{}rate schedules/early stopping.
Here BCE (binary cross\hyp{}entropy) for a binary target $y\in\{0,1\}$ and logit $z$ is $\mathcal{L}(z,y)=\log(1+e^{-z})$ for $y{=}1$ and $\log(1+e^{z})$ for $y{=}0$; the \emph{logit} $z$ is the pre\hyp{}sigmoid score so that $\sigma(z)$ yields the predicted probability. The middle panel contrasts a conservative schedule (smooth decay) with a more aggressive one (faster initial drop but risk of oscillation), and the right panel shows early stopping triggered when validation loss ceases to improve while training loss continues decreasing.
We will reuse these when tuning sequence models, where overfitting appears as a divergence between per\hyp{}token training and validation likelihood.

\begin{tcolorbox}[summarybox,title={Author's note: treat early stopping as the default brake}]
Unless there is a compelling reason to run to numerical convergence, stop as soon as the validation curve flattens while the training curve keeps dropping. Checkpoint the best weights and resume only if new data or regularization changes warrant it. That simple rule prevents most runaway experiments without elaborate hyperparameter sweeps.
\end{tcolorbox}

\begin{figure}[t]
    \centering
    \begin{tikzpicture}
        \begin{groupplot}[
            group style={group size=3 by 1, horizontal sep=1.2cm},
            width=0.32\linewidth, height=0.36\linewidth
        ]
        % BCE vs logit
        \nextgroupplot[
            title={BCE vs logit $z$},
            xlabel={$z$},
            ylabel={$\mathcal{L}$},
            xmin=-6, xmax=6,
            ymin=0, ymax=6,
            legend style={at={(0.97,0.97)},anchor=north east,fill=white,draw=none}
        ]
            \addplot[cbBlue, thick, samples=200, domain=-6:6] {ln(1+exp(-x))}; \addlegendentry{$y{=}1$}
            \addplot[cbOrange, dashed, thick, samples=200, domain=-6:6] {ln(1+exp(x))}; \addlegendentry{$y{=}0$}
        % Learning-rate effect (schematic loss curves)
        \nextgroupplot[
            title={Learning rate effect},
            xlabel={epoch},
            ylabel={loss},
            xmin=0, xmax=50,
            ymin=0, ymax=1.2,
            legend style={at={(0.97,0.97)},anchor=north east,fill=white,draw=none}
        ]
            \addplot[cbBlue, thick, samples=100, domain=0:50] {0.2 + 0.9*exp(-x/12)}; \addlegendentry{conservative}
            \addplot[cbPink, dashed, thick, samples=100, domain=0:50] {0.15 + 1.0*exp(-x/6) + 0.05*sin(0.6*x)}; \addlegendentry{aggressive}
        % Early stopping (train vs val)
        \nextgroupplot[
            title={Early stopping},
            xlabel={epoch},
            ylabel={loss},
            xmin=0, xmax=50,
            ymin=0, ymax=1.2,
            legend style={at={(0.97,0.97)},anchor=north east,fill=white,draw=none}
        ]
            \addplot[cbGreen, thick, samples=100, domain=0:50] {0.1 + 0.9*exp(-x/10)}; \addlegendentry{train}
            % U-shaped validation curve: improves early, then degrades (overfitting).
            \addplot[cbOrange, dashed, thick, samples=200, domain=0:50] {0.25 + 0.7*exp(-x/12) + 0.0004*(x-20)^2}; \addlegendentry{val}
            \addplot[gray!70, dashed] coordinates {(20,0) (20,1.2)};
            \node[font=\scriptsize, anchor=south west, gray!70] at (axis cs:20,0.08) {stop};
        \end{groupplot}
    \end{tikzpicture}
    \ifdefined\HCode
        \caption{Schematic: Binary cross-entropy geometry (left), effect of learning-rate schedules on loss (middle), and the typical training/validation divergence that motivates early stopping (right).}
    \else
        % Avoid inline math in captions; it wraps poorly in some EPUB renderers.
        \caption{Schematic: Binary cross-entropy geometry (left), effect of learning-rate schedules on loss (middle), and the typical training/validation divergence that motivates early stopping (right).}
    \fi
    \label{fig:lec7-loss-hyperparams}
\end{figure}

\begin{tcolorbox}[summarybox,title={LayerNorm and residual RNN tips}]
Layer Normalization~\citep{Ba2016} stabilizes recurrent dynamics by normalizing each hidden vector \(h_t\) across features before applying the nonlinearity; unlike BatchNorm it works with batch size 1 and handles variable-length sequences gracefully. Residual RNN stacks (adding the input of a layer back to its output) keep gradients flowing even when depth increases, mirroring the skip-connections that make deep CNNs trainable. Together, LayerNorm + residual links curb exploding/vanishing gradients and are the default when building multi-layer RNN/LSTM stacks.
\end{tcolorbox}

\paragraph{Historical Note: Hopfield Networks}

An early influential recurrent network is the Hopfield network \citep{Hopfield1982}, which is a form of associative memory. Unlike modern RNNs, Hopfield networks have symmetric weights and are designed to converge to stable states representing stored patterns. While Hopfield networks are not directly used for sequence modeling, they helped establish the energy-based viewpoint that reappears in later recurrent and attention-based models.

Bidirectional extensions run two RNNs in opposite directions and concatenate their states; they are widely used in encoders for labeling tasks when the full context is available.

\subsection{Input--output configurations and mathematical formulation}

RNNs can map sequences to sequences in several ways:
\begin{itemize}
    \item Many-to-one (e.g., sentiment classification): consume \(x_{1:T}\), emit one label after the final state.
    \item One-to-many (e.g., conditional generation): condition on a context vector, then autoregressively emit a sequence.
    \item Many-to-many (e.g., tagging, ASR): emit \(y_t\) at every step; encoder--decoder variants compress \(x_{1:T}\) then decode.
    \item Bidirectional encoders: run a forward and backward RNN and concatenate the states for sequence labeling or as encoder context.
    \end{itemize}

Consider an input sequence \(\{x_1, x_2, \ldots, x_T\}\), where each \(x_t \in \mathbb{R}^d\). The RNN computes hidden states \(\{h_1, h_2, \ldots, h_T\}\) and outputs \(\{y_1, y_2, \ldots, y_T\}\) as follows:
\begin{align}
h_0 &= \mathbf{0} \quad \text{(initial hidden state)} \\
h_t &= f(x_t W_{xh} + h_{t-1} W_{hh} + b_h), \quad t=1,\ldots,T \label{eq:rnn_hidden_state}\\
y_t &= g(h_t W_{hy} + b_y), \quad t=1,\ldots,T \label{eq:rnn_output_state}
\end{align}

\begin{tcolorbox}[summarybox,title={Shapes and masks (batch $B$, time $T$)}]
Inputs \(X \in \mathbb{R}^{B\times T \times d_x}\); hidden states \(H \in \mathbb{R}^{B\times T \times d_h}\); logits \(Y \in \mathbb{R}^{B\times T \times d_o}\). Parameters (row-major): \(W_{xh}\in\mathbb{R}^{d_x\times d_h}\), \(W_{hh}\in\mathbb{R}^{d_h\times d_h}\), \(W_{hy}\in\mathbb{R}^{d_h\times d_o}\), biases \(b_h\in\mathbb{R}^{d_h}\), \(b_y\in\mathbb{R}^{d_o}\).\\
Padding mask \(M\in\{0,1\}^{B\times T}\): loss \(L=\sum_{b,t} M_{b,t}\,\text{CE}(\hat y_{b,t}, y_{b,t})/\sum_{b,t} M_{b,t}\). Masks preview the padding/causal masks detailed in \Cref{chap:transformers}.
\end{tcolorbox}
\subsection{Recurrent Neural Networks: Historical Context and Motivation}

Recall from our earlier discussion on Hopfield networks that the configuration of the network states significantly impacts the overall energy landscape. The sequence of states, or more precisely, their spatial arrangement within the network, determines the energy and thus the network's behavior. This property endowed Hopfield networks with associative memory capabilities, as the weights were constructed to "remember" specific patterns.

However, Hopfield networks were primarily designed for storage and retrieval of static patterns rather than for dynamic prediction or forecasting tasks. Despite their introduction in 1982, their practical utility beyond research was limited.

\subsection{The 1986 Breakthrough: Backpropagation and Trainable Multi-Layer Networks}

In 1986, Rumelhart, Hinton, and Williams popularized the use of backpropagation as a practical training method for multilayer networks \citep{Rumelhart1986}. That contribution is not an RNN architecture; it is a training procedure that made it feasible to fit large parametric models by propagating error derivatives through compositions of linear maps and nonlinearities.

For recurrent networks, the conceptual move is to view the computation graph as a \emph{shared} module repeated across time steps. Training then becomes standard backpropagation applied to the unrolled graph, with gradients accumulated across each copy of the shared parameters; this is precisely what ``backpropagation through time'' (BPTT) implements in the sections that follow.

\subsection{State Dynamics in Recurrent Neural Networks}

The 1986 RNN formulation introduced the concept of a \emph{state} that evolves over time as a function of the previous state and the current input. Formally, the state update can be expressed as:
\begin{equation}
    \mathbf{h}_t = f(\mathbf{h}_{t-1}, \mathbf{x}_t; \theta).
\end{equation}
where
\begin{itemize}
    \item \(\mathbf{h}_t\) is the hidden state at time \(t\),
    \item \(\mathbf{x}_t\) is the input at time \(t\),
    \item \(f\) is a nonlinear function parameterized by \(\theta\) (e.g., weights and biases),
    \item \(\mathbf{h}_{t-1}\) is the hidden state at the previous time step.
\end{itemize}

The output at time \(t\), denoted \(\mathbf{y}_t\), is typically computed as a function of the hidden state:
\begin{equation}
    \mathbf{y}_t = g(\mathbf{h}_t; \phi).
\end{equation}
where \(g\) is another nonlinear function parameterized by \(\phi\).

\paragraph{Interpretation}

- The hidden state \(\mathbf{h}_t\) acts as a memory that summarizes information from all previous inputs up to time \(t\).
- The recurrence allows the network to maintain context and model temporal dependencies.

\subsection{Unfolding the Recurrent Neural Network}

To better understand and implement RNNs, it is common to \emph{unfold} the network through time. Unfolding transforms the recurrent structure into a feedforward network with shared weights across time steps.

\paragraph{Process}

- Start with an initial hidden state \(\mathbf{h}_0\), which may be initialized to zero or learned.
- At each time step \(t\), compute \(\mathbf{h}_t\) using \Cref{eq:rnn_state_update}.
- Compute output \(\mathbf{y}_t\) using \Cref{eq:rnn_output}.
- The parameters \(\theta\) and \(\phi\) are shared across all time steps, enabling the network to generalize across sequences of varying lengths.

\paragraph{Significance}

- Unfolding clarifies the flow of information and dependencies across time.
- It facilitates the application of backpropagation through time (BPTT) for training.

\subsection{Mathematical Formulation of a Simple RNN Cell}

Consider a simple RNN cell with the following update equations:
\begin{align}
    \mathbf{h}_t &= \sigma_h \left( \mathbf{x}_t \mathbf{W}_{xh} + \mathbf{h}_{t-1} \mathbf{W}_{hh} + \mathbf{b}_h \right), \label{eq:simple_rnn_hidden} \\
    \mathbf{y}_t &= \sigma_y \left( \mathbf{h}_t \mathbf{W}_{hy} + \mathbf{b}_y \right). \label{eq:simple_rnn_output}
\end{align}
\begin{figure}[t]
    \centering
    \begin{tikzpicture}[
        font=\small\sffamily,
        >=Latex,
        node distance=2.4cm,
        box/.style={draw, rounded corners, minimum width=1.8cm, minimum height=0.9cm},
        arrow/.style={->, thick}
    ]
        % Nodes
        \node[box] (h1) at (0,0) {$h_{t-1}$};
        \node[box, right=of h1] (h2) {$h_{t}$};
        \node[box, right=of h2] (h3) {$h_{t+1}$};
        \node (x1) at (0,-1.2) {$x_{t-1}$};
        \node (x2) at (h2|-x1) {$x_{t}$};
        \node (x3) at (h3|-x1) {$x_{t+1}$};
        \node (y1) at (0,1.2) {$y_{t-1}$};
        \node (y2) at (h2|-y1) {$y_{t}$};
        \node (y3) at (h3|-y1) {$y_{t+1}$};
        % Arrows
        \draw[arrow] (h1) -- node[above, font=\scriptsize] {$\mathbf{W}_{hh}$} (h2);
        \draw[arrow] (h2) -- (h3);
        \draw[arrow] (x1) -- node[midway, right, font=\scriptsize, xshift=1mm] {$\mathbf{W}_{xh}$} (h1);
        \draw[arrow] (x2) -- node[midway, right, font=\scriptsize, xshift=1mm] {$\mathbf{W}_{xh}$} (h2);
        \draw[arrow] (x3) -- node[midway, right, font=\scriptsize, xshift=1mm] {$\mathbf{W}_{xh}$} (h3);
        \draw[arrow] (h1) -- node[midway, right, font=\scriptsize, xshift=1mm] {$\mathbf{W}_{hy}$} (y1);
        \draw[arrow] (h2) -- node[midway, right, font=\scriptsize, xshift=1mm] {$\mathbf{W}_{hy}$} (y2);
        \draw[arrow] (h3) -- node[midway, right, font=\scriptsize, xshift=1mm] {$\mathbf{W}_{hy}$} (y3);

        % Shared-parameter annotation: brace anchored to the left/right cells so it
        % stays aligned if the figure spacing changes.
        \coordinate (braceL) at ($(h1.south west)+(0,-12mm)$);
        \coordinate (braceR) at ($(h3.south east)+(0,-12mm)$);
        \draw[decorate, decoration={brace, amplitude=5pt}] (braceL) -- node[midway, below=10pt]{\scriptsize shared parameters across the unrolled sequence} (braceR);
    \end{tikzpicture}
    \caption{Schematic: Unrolling an RNN reveals repeated application of the same parameters across time steps. This view motivates backpropagation through time (BPTT), which accumulates gradients through every copy before updating the shared weights.}
    \label{fig:lec7-rnn-unrolled}
\end{figure}
\subsection{Recurrent Neural Network (RNN) Unfolding and Parameter Sharing}

Recall that a recurrent neural network (RNN) processes sequential data by maintaining a hidden state that evolves over time. At each time step \( t \), the network receives an input \( \mathbf{x}_t \) and updates its hidden state \( \mathbf{h}_t \), which in turn produces an output \( \mathbf{y}_t \). \Cref{fig:lec7-rnn-unrolled} depicts this unfolding explicitly so we can reason about the repeated weight matrices across time.

\paragraph{Unfolding the RNN}
Unfolding the RNN across time steps transforms the recurrent structure into a deep feedforward network with shared weights across layers (time steps). This unrolled network looks like a chain where each hidden state depends on the previous hidden state and the current input:

\[
\mathbf{h}_t = f(\mathbf{h}_{t-1}, \mathbf{x}_t; \Theta), \quad \mathbf{y}_t = g(\mathbf{h}_t; \Theta_y)
\]

where \( f \) and \( g \) are nonlinear functions parameterized by weights \(\Theta\) and \(\Theta_y\), respectively.

\paragraph{Parameter Sharing}
A key property of RNNs is \emph{parameter sharing} across time steps. Specifically:

\begin{itemize}
    \item The weights connecting the previous hidden state \(\mathbf{h}_{t-1}\) to the current hidden state \(\mathbf{h}_t\) are the same for all \( t \).
    \item The weights connecting the input \(\mathbf{x}_t\) to the hidden state \(\mathbf{h}_t\) are also shared across all time steps.
    \item The weights mapping the hidden state \(\mathbf{h}_t\) to the output \(\mathbf{y}_t\) are shared as well.
\end{itemize}

This parameter sharing reduces the number of parameters to learn and enables the network to generalize across different positions in the sequence.

\subsection{Mathematical Formulation of the RNN}

We formalize the RNN update equations as follows. Let the hidden state at time \( t \) be \(\mathbf{h}_t \in \mathbb{R}^{1\times h}\), the input at time \( t \) be \(\mathbf{x}_t \in \mathbb{R}^{1\times d}\), and the output at time \( t \) be \(\mathbf{y}_t \in \mathbb{R}^{1\times o}\) (row-vector convention). Let \(\mathbf{a}_t\in\mathbb{R}^{1\times h}\) denote the pre-activation (affine) hidden update.

\begin{align}
    \mathbf{a}_t &= \mathbf{h}_{t-1}\mathbf{W}_a + \mathbf{x}_t\mathbf{W}_x + \mathbf{b}_h, \\
    \mathbf{h}_t &= f\!\left(\mathbf{a}_t\right), \\
    \mathbf{y}_t &= g\left( \mathbf{h}_t\mathbf{W}_y + \mathbf{b}_y \right),
\end{align}

where:
\begin{itemize}
    \item \(\mathbf{W}_a \in \mathbb{R}^{h \times h}\) is the recurrent weight matrix (hidden-to-hidden).
    \item \(\mathbf{W}_x \in \mathbb{R}^{d \times h}\) is the input-to-hidden weight matrix.
    \item \(\mathbf{W}_y \in \mathbb{R}^{h \times o}\) is the hidden-to-output weight matrix.
    \item \(\mathbf{b}_h \in \mathbb{R}^h\) and \(\mathbf{b}_y \in \mathbb{R}^o\) are bias vectors.
    \item \(f(\cdot)\) is the activation function for the hidden state (commonly \(\tanh\) or \(\mathrm{ReLU}\)).
    \item \(g(\cdot)\) is the output activation function (e.g., softmax for classification).
\end{itemize}
Checking dimensions: \(\mathbf{h}_{t-1}\mathbf{W}_a\in\mathbb{R}^{1\times h}\), \(\mathbf{x}_t\mathbf{W}_x\in\mathbb{R}^{1\times h}\), and \(\mathbf{h}_t\mathbf{W}_y\in\mathbb{R}^{1\times o}\), so the shapes are consistent with the compact form \eqref{eq:rnn_compact}.

\paragraph{Interpretation}
\Cref{eq:rnn_hidden_state} shows that the current hidden state \(\mathbf{h}_t\) is a nonlinear transformation of the previous hidden state \(\mathbf{h}_{t-1}\) and the current input \(\mathbf{x}_t\). \Cref{eq:rnn_output} maps the hidden state to the output at time \( t \).

\paragraph{Reusability of the Hidden State}
The hidden state \(\mathbf{h}_t\) serves as a summary of all previous inputs up to time \( t \). This recursive formulation allows the network to capture temporal dependencies of arbitrary length.

\subsection{Generalized Notation}

To simplify notation, define the concatenated input vector at time \( t \):

\[
\mathbf{z}_t = \begin{bmatrix} \mathbf{h}_{t-1} & \mathbf{x}_t \end{bmatrix} \in \mathbb{R}^{1\times (h + d)}.
\]

Correspondingly, define the combined weight matrix:

\[
\mathbf{W} = \begin{bmatrix} \mathbf{W}_a \\ \mathbf{W}_x \end{bmatrix} \in \mathbb{R}^{(h + d)\times h}.
\]

Then the hidden state update can be written compactly as:

\begin{equation}
    \mathbf{h}_t = \sigma\left( \mathbf{z}_t \mathbf{W} + \mathbf{b}_h \right). \label{eq:rnn_compact}
\end{equation}
This stacks the recurrent and input weights into a single matrix.
\subsection{Recurrent Neural Network (RNN) Architectures and Loss Computation}

Recall from previous discussions that the loss function for classification tasks often involves cross\hyp{}entropy terms of the form:
\begin{equation}
\mathcal{L} = - \sum_{i} y_i \log \hat{y}_i,
\end{equation}
where \( y_i \) is the true label (often one-hot encoded) and \( \hat{y}_i \) is the predicted probability for class \( i \). When \( \hat{y} = y \), the loss is zero, indicating perfect prediction.

In the context of RNNs, the total loss over a sequence is typically the sum of losses at each time step:
\begin{equation}
\mathcal{L}_{\text{total}} = \sum_{t=1}^T \mathcal{L}_t,
\end{equation}
where \( T \) is the sequence length.

\paragraph{Forward and Backward Passes in RNNs}

The forward pass involves propagating inputs through the network over time steps \( t = 1, \ldots, T \), producing outputs \( \hat{y}_t \) at each step. After computing the loss, the backward pass computes gradients with respect to parameters by backpropagating errors through time, a process known as \emph{Backpropagation Through Time} (BPTT).

BPTT unfolds the RNN across time steps and applies standard backpropagation through this unrolled network. The key insight is that parameters are shared across time steps, so gradients accumulate contributions from all time steps; \cref{fig:lec7-bptt} highlights the simultaneous forward flow (black) and backward gradients (pink) that piggyback across every copy.

\begin{tcolorbox}[summarybox,title={Truncated BPTT in practice}]
Unroll \(K\) steps, accumulate loss, backprop through those \(K\) steps, then detach the hidden state to stop graph growth:
\begin{verbatim}
h = h0
for t in range(T):
    h, yhat = rnn(x[t], h)
    loss += mask[t] * CE(yhat, y[t])
    if (t+1) % K == 0:
        loss.backward()
        clip_grad_norm_(model.parameters(), tau)
        opt.step(); opt.zero_grad()
        h = h.detach()  # carry state, drop graph
\end{verbatim}
Choose \(K\) to balance memory and credit assignment (common range: 20--100 steps).
\end{tcolorbox}

\begin{figure}[t]
    \centering
    \begin{tikzpicture}[
        font=\small\sffamily,
        >=Latex,
        arrow/.style={->, thick}
    ]
        % forward chain
        \foreach \i in {0,...,4} {
            \node[draw, rounded corners, minimum width=1.2cm, minimum height=0.7cm] (h\i) at (2*\i,0) {$h_{t+\i}$};
            \node (x\i) at (2*\i,-1.0) {$x_{t+\i}$};
            \draw[arrow] (x\i) -- (h\i);
        }
        \foreach \i/\j in {0/1,1/2,2/3,3/4} {\draw[arrow] (h\i) -- (h\j);}
        % backward (gradient) arrows
        \foreach \i/\k in {1/0,2/1,3/2,4/3} {
            \draw[cbPink, <-, thick] (h\i.north) .. controls +(0,1.0) and +(0,1.0) .. node[above,pos=0.5]{\scriptsize $\partial\mathcal{L}/\partial h_{t+\i}$} (h\k.north);
        }
        % shared parameters note
        \node at (4,-2.0) {\scriptsize shared $(\mathbf{W}_{hh}, \mathbf{W}_{xh}, \mathbf{W}_{hy})$ across time};
    \end{tikzpicture}
    \caption{Schematic: Backpropagation through time (BPTT): unrolled forward pass (black) and backward gradients (pink) through time.}
    \label{fig:lec7-bptt}
\end{figure}

\paragraph{Vanishing and Exploding Gradients}

Because each gradient term contains products of Jacobians such as
\[
\frac{\partial \mathbf{h}_t}{\partial \mathbf{h}_{t-1}} = \mathrm{diag}\!\big( f'(\mathbf{a}_t) \big)\,\mathbf{W}_{hh}^{\top},
\]
with pre\hyp{}activation \(\mathbf{a}_t=\mathbf{x}_t\mathbf{W}_{xh}+\mathbf{h}_{t-1}\mathbf{W}_{hh}+\mathbf{b}_h\) and elementwise nonlinearity \(f\), long sequences multiply many such factors. Here \(\mathbf{W}_{hh}\in\mathbb{R}^{d_h\times d_h}\) and \(\mathrm{diag}\big(f'(\mathbf{a}_t)\big)\in\mathbb{R}^{d_h\times d_h}\), so the Jacobian \(\partial \mathbf{h}_t/\partial \mathbf{h}_{t-1}\) is a \(d_h\times d_h\) matrix. If the spectral norm of each factor is below one the product decays exponentially (vanishing); norms above one cause growth (exploding). \Cref{fig:lec7-vanishing} illustrates both behaviors across time. Practical remedies include gradient clipping, orthogonal or unitary recurrent matrices, layer normalization, and gated architectures (LSTM/GRU) that introduce additive memory paths.
Empirically, vanilla RNNs often fail to preserve dependencies beyond roughly 5--10 steps in language tasks, which is why gated cells became the default for longer sequences.
\begin{figure}[t]
    \centering
    \includegraphics[width=0.72\linewidth]{lec7_vanish_explode}
    \caption{Schematic: Vanishing (blue) versus exploding (orange) gradients on a log scale. The gray strip highlights the stability band; the inset reminds readers that repeated Jacobian products either shrink gradients (thin blue arrows) or amplify them (thick orange arrows).}
    \label{fig:lec7-vanishing}
\end{figure}

\paragraph{Parameter Updates}

At each time step, the gradient of the loss with respect to parameters (e.g., weights \( W \)) depends on the chain of partial derivatives through the network states:
\begin{equation}
\frac{\partial \mathcal{L}}{\partial W} = \sum_{t=1}^T \frac{\partial \mathcal{L}_t}{\partial W}.
\end{equation}
Because of parameter sharing, the same \( W \) influences multiple time steps, and the total gradient is the sum over these contributions.

\subsection{Stabilizing Recurrent Training}

\paragraph{Gradient clipping.} A practical safeguard is to clip the global norm of the gradient when it exceeds a threshold. \Cref{fig:lec7-clipping} shows how clipping prevents the exploding case from destabilizing optimization while leaving the vanishing regime untouched. Orthogonal or unitary initializations for the recurrent weight matrix \(W_{hh}\) are another common trick: because orthogonal matrices preserve Euclidean norms, gradients neither explode nor vanish in the very early stages of training (before nonlinearities and data\hyp{}dependent effects accumulate).

\begin{tcolorbox}[summarybox,title={Rule of thumb: keep gradients in the safe band}]
Spectral norm \(\|\mathbf{W}_{hh}\|_2 \approx 1\) keeps gradients roughly stable over tens of steps; clipping thresholds \(\tau \in [0.5, 5]\) are sensible defaults when sequences are long. BatchNorm inside the recurrent loop is rarely helpful; prefer LayerNorm or gating to stabilize dynamics.
\end{tcolorbox}

\begin{tcolorbox}[summarybox,title={Author's note: do not fight vanilla RNNs}]
If your task needs dependencies longer than a handful of steps, do not over-tune a plain RNN. Start with clipping and a sensible truncation window, but move to GRU/LSTM once you see long-range information vanish.
\end{tcolorbox}

\paragraph{Dropout in RNNs.} Variational/recurrent dropout applies the same dropout mask at every time step to avoid injecting temporal noise \citep{Gal2016,Semeniuta2016}; zoneout preserves hidden units stochastically instead of zeroing them \citep{Krueger2016}. Standard per-time-step dropout often harms sequence retention.

\begin{figure}[t]
    \centering
    \begin{tikzpicture}
        \begin{groupplot}[
            group style={group size=2 by 1, horizontal sep=1.6cm},
            width=0.44\linewidth,
            height=0.35\linewidth,
            xmin=0,xmax=80,
            xlabel={Iteration},
            grid=both,
            minor tick num=1,
            xlabel style={yshift=4pt}
        ]
            \nextgroupplot[
                ylabel={$\|\mathbf{g}\|$},
                ymin=0,ymax=20,
                legend style={at={(0.02,0.95)},anchor=north west},
                ytick={0,5,10,15,20}
            ]
                \addplot[cbOrange, thick, smooth] table {
                x y
                0 4
                10 5
                20 7
                30 10
                40 13
                50 16
                60 18
                70 17
                80 19
                };
                    \addlegendentry{Unclipped}
                    \addplot[cbGreen, thick, smooth] table {
                    x y
                    0 4
                    10 4.5
                    20 5
                    30 5.7
                    40 6.5
                    50 6.5
                    60 6.5
                    70 6.5
                    80 6.5
                    };
                    \addlegendentry{Clipped at $\tau$}
                    \draw[gray!40, thick, dashed] (axis cs:0,6.5) -- (axis cs:80,6.5)
                        node[pos=0.98, anchor=south east, font=\scriptsize]{threshold $\tau$};
            \nextgroupplot[
                ylabel={Training loss},
                ymin=0.2,ymax=1.2,
                ytick={0.2,0.4,0.6,0.8,1.0,1.2}
            ]
                \addplot[cbOrange, thick, smooth] table {
                x y
                0 1.05
                10 0.9
                20 0.85
                30 0.92
                40 0.95
                50 1.0
                60 1.05
                70 1.1
                80 1.12
                };
                \addplot[cbGreen, thick, smooth] table {
                x y
                0 1.05
                10 0.92
                20 0.8
                30 0.68
                40 0.6
                50 0.55
                60 0.5
                70 0.48
                80 0.46
                };
        \end{groupplot}
    \end{tikzpicture}
    % Avoid inline math in captions; it wraps poorly in some EPUB renderers.
    \caption{Schematic: Gradient norms (left) explode without clipping (orange) but remain bounded when the global norm is clipped at tau (green). Training loss (right) stabilizes as a result.}
    \label{fig:lec7-clipping}
\end{figure}

\paragraph{Teacher forcing and scheduled sampling.} Sequence-to-sequence models frequently feed the ground-truth token back into the decoder during training (teacher forcing) to accelerate convergence. \Cref{fig:lec7-teacher-forcing} contrasts this regime with free-running inference: teacher forcing injects gold tokens at every step, whereas inference conditions the decoder on its own predictions. This mismatch is precisely what scheduled-sampling curricula aim to mitigate \citep{Bengio2015}.

\begin{figure}[t]
\centering
\resizebox{0.92\linewidth}{!}{%
\begin{tikzpicture}[
    x=2.4cm,
    y=1cm,
    font=\small\sffamily,
    >=Latex,
    dec/.style={
        draw, rounded corners=4pt, thick, fill=cbBlue!15,
        minimum width=1.9cm, minimum height=1.0cm,
        align=center
    },
    arrow/.style={->, thick},
    gold/.style={->, thick, cbGreen},
    orange/.style={->, thick, cbOrange}
]

% ==== (a) Teacher forcing ====
\node at (-0.8,2.1) {(a) Teacher forcing};

\node[dec] (d1) at (0,1.2) {$f_\theta$};
\node[dec, right=1 of d1] (d2) {$f_\theta$};
\node[dec, right=1 of d2] (d3) {$f_\theta$};
\node[dec, right=1 of d3, minimum width=1.6cm] (d4) {$f_\theta$};

\node at (-0.8,1.55) {$h_{t-1}$};
\draw[arrow] (-0.55,1.2) -- (d1.west);

\draw[arrow] (d1) -- (d2) node[midway, above=3pt] {\scriptsize distribution};
\draw[arrow] (d2) -- (d3) node[midway, above=3pt] {\scriptsize distribution};
\draw[arrow] (d3) -- (d4);

% gold targets injected upward
\draw[gold] ($(d1.south)+(0,-0.75)$) node[below] {$y_{t-1}$} -- (d1.south);
\draw[gold] ($(d2.south)+(0,-0.75)$) node[below] {$y_t$} -- (d2.south);
\draw[gold] ($(d3.south)+(0,-0.75)$) node[below] {$y_{t+1}$} -- (d3.south);

% ellipsis after fourth block
\node at ($(d4)+(0.6,0)$) {\color{gray}$\cdots$};

% ==== (b) Inference ====
\node at (-0.8,0.1) {(b) Inference};

\node[dec] (s1) at (0,-0.6) {$f_\theta$};
\node[dec, right=1 of s1] (s2) {$f_\theta$};
\node[dec, right=1 of s2] (s3) {$f_\theta$};
\node[dec, right=1 of s3, minimum width=1.6cm] (s4) {$f_\theta$};

\node at (-0.8,-0.25) {$h_{t-1}$};
\draw[arrow] (-0.55,-0.6) -- (s1.west);

\draw[arrow] (s1) -- (s2) node[midway, above=3pt] {\scriptsize sampled token};
\draw[arrow] (s2) -- (s3) node[midway, above=3pt] {\scriptsize sampled token};
\draw[arrow] (s3) -- (s4);

% autoregressive tokens injected upward
\draw[orange] ($(s1.south)+(0,-0.75)$) node[below] {$\langle \text{START}\rangle$} -- (s1.south);
\draw[orange] ($(s2.south)+(0,-0.75)$) node[below] {$\hat y_t$} -- (s2.south);
\draw[orange] ($(s3.south)+(0,-0.75)$) node[below] {$\hat y_{t+1}$} -- (s3.south);

\node at ($(s4)+(0.6,0)$) {\color{gray}$\cdots$};

\end{tikzpicture}
}%

\caption{Schematic: Teacher forcing vs.\ inference in a sequence-to-sequence decoder. Gold arrows show supervised targets; orange arrows highlight autoregressive feedback that motivates scheduled sampling.}
\label{fig:lec7-teacher-forcing}
\end{figure}

\paragraph{Gated cells.} LSTMs \citep{Hochreiter1997,Gers2000} and GRUs \citep{Cho2014} alleviate vanishing gradients by introducing additive memory paths guarded by gates. \Cref{fig:lec7-lstm,fig:lec7-gru} present the canonical cell diagrams used later in the chapter when deriving the update equations. Intuitively, LSTM forget/input/output gates control retention, writing, and exposure of the memory \(c_t\); GRU update/reset gates interpolate between \(h_{t-1}\) and a candidate \(\tilde h_t\) and decide how much past context influences the candidate. In an LSTM, the state updates as \(c_t = f_t \odot c_{t-1} + i_t \odot \tilde c_t\) and \(h_t = o_t \odot \tanh(c_t)\). In a GRU, the hidden state updates as \(h_t = (1-z_t)\odot h_{t-1} + z_t \odot \tilde h_t\), with the reset gate \(r_t\) shaping the candidate computation. In both figures, \(x_t\) denotes the input at time \(t\), and \(h_{t-1}\) / \(c_{t-1}\) denote the carried state(s).

% LSTM cell: clean routing (no overlaps/crossings; no ambiguous "buses")
% Use an aggressive top-float spec so this can stack with the following GRU cell
% before the summary boxes (avoids the "box jumps ahead, GRU spills to next page" regression).
\begin{figure}[!t]
    \centering
    % Local palette for this diagram (match Figure 52's GRU palette).
    \definecolor{lstmCbBlue}{RGB}{218, 232, 252}
    \definecolor{lstmCbOrange}{RGB}{255, 230, 204}
    \definecolor{lstmCbGray}{RGB}{235, 235, 235}
    \definecolor{lstmArrowGray}{RGB}{100, 100, 100}
    \noindent\resizebox{0.85\textwidth}{!}{%
        \begin{tikzpicture}[
            x=1cm, y=1cm,
            font=\small\sffamily,
            >={Stealth[round, length=2.5mm, width=1.75mm]},
            state/.style={circle, draw, line width=1pt, minimum size=11mm, inner sep=0pt},
            gate/.style={rectangle, rounded corners=3mm, draw, line width=0.8pt,
                fill=lstmCbBlue, minimum width=18mm, minimum height=9mm, align=center},
            cand/.style={rectangle, rounded corners=3mm, draw, line width=0.8pt,
                fill=lstmCbOrange, minimum width=22mm, minimum height=10mm, align=center},
            func/.style={rectangle, rounded corners=2mm, draw, line width=0.8pt,
                fill=lstmCbGray, minimum width=15mm, minimum height=8mm, align=center},
            op/.style={circle, draw, line width=0.8pt, minimum size=7mm, inner sep=0pt, fill=white},
            flow/.style={->, line width=1.2pt, black, rounded corners=5pt, shorten >=1pt},
            lane/.style={->, line width=1.0pt, lstmArrowGray, rounded corners=5pt, shorten >=1pt},
            connect/.style={circle, fill=black, inner sep=0pt, minimum size=4pt}
        ]

            % --- 1. THE HIGHWAY (Top Cell State) ---
            \node[state] (cprev) at (-1.5, 4.0) {$c_{t-1}$};
            \node[op]    (mulF)  at (4.5, 4.0)  {$\odot$};
            \node[op]    (addI)  at (8.0, 4.0)  {$+$};
            \node[state] (ct)    at (13.0, 4.0) {$c_t$};

            % --- 2. INPUTS (Bottom) ---
            \node[state] (hprev) at (-1.5, 0.5) {$h_{t-1}$};
            \node[state] (xt)    at (-1.5,-3.5) {$x_t$};

            % --- 3. GATES STACK (Define positions first) ---
            \node[gate] (ft) at (4.5, 1.5) {$f_t$\\\scriptsize $\sigma$};
            \node[gate] (it) at (4.5, -0.5) {$i_t$\\\scriptsize $\sigma$};
            \node[cand] (ctilde) at (4.5, -2.5) {$\tilde c_t$\\[-2pt]\scriptsize $\tanh$};
            \node[gate] (ot) at (4.5, -4.5) {$o_t$\\\scriptsize $\sigma$};

            % --- 4. AFFINE BLOCK ---
            \node[func, minimum height=7.2cm, minimum width=1.8cm] (affine) at (1.5, -1.5) {\scriptsize Affine};

            % --- 5. LOGIC OPERATORS ---
            \node[op] (mulI) at (8.0, -1.5) {$\odot$};
            \node[cand] (tanhC) at (11.0, 0.5) {$\tanh$};
            \node[op]   (mulO)  at (11.0, -4.5) {$\odot$};
            \node[state] (ht)    at (13.0, -4.5) {$h_t$};
            \node[connect] (cSplit) at (11.0, 4.0) {};

            % ================= CONNECTIONS =================

            % --- INPUTS -> AFFINE ---
            \draw[flow] (hprev) -- (hprev -| affine.west);
            \draw[flow] (xt) -- (xt -| affine.west);

            % --- AFFINE -> GATES ---
            \draw[lane] (affine.east |- ft) -- (ft.west);
            \draw[lane] (affine.east |- it) -- (it.west);
            \draw[lane] (affine.east |- ctilde) -- (ctilde.west);
            \draw[lane] (affine.east |- ot) -- (ot.west);

            % --- GATE LOGIC ---
            \draw[lane] (ft.north) -- (mulF.south);

            \draw[lane] (it.east) -| (mulI.north);
            \draw[flow] (ctilde.east) -| (mulI.south);
            \draw[flow] (mulI.north) -- (addI.south);

            \draw[flow] (cprev) -- (mulF);
            \draw[flow] (mulF) -- (addI);
            \draw[flow, -] (addI) -- (cSplit);
            \draw[flow] (cSplit) -- (ct);

            % --- OUTPUT LOGIC ---
            \draw[flow] (cSplit) -- (tanhC.north);
            \draw[flow] (tanhC.south) -- (mulO.north);
            \draw[lane] (ot.east) -- (mulO.west);
            \draw[flow] (mulO.east) -- (ht.west);

        \end{tikzpicture}%
    }
        \ifdefined\HCode
            \caption{Schematic: Long Short-Term Memory (LSTM) cell \citep{Hochreiter1997,Gers2000}.}
        \else
            \caption{Schematic: Long Short-Term Memory (LSTM) cell \citep{Hochreiter1997,Gers2000}.}
        \fi
    \label{fig:lec7-lstm}
\end{figure}

% GRU cell (based on the provided reference TikZ, with explicit (1-z_t) block).
% Matches: h_t = (1-z_t)\odot h_{t-1} + z_t \odot \tilde h_t.
\begin{figure}[!t]
    \centering
    % Local palette for this diagram (avoid clobbering global cbBlue/cbOrange used elsewhere).
    \definecolor{gruCbBlue}{RGB}{218, 232, 252}
    \definecolor{gruCbOrange}{RGB}{255, 230, 204}
    \definecolor{gruCbGray}{RGB}{235, 235, 235}
    \definecolor{gruArrowGray}{RGB}{100, 100, 100}
    \noindent\resizebox{0.85\textwidth}{!}{%
        \begin{tikzpicture}[
            x=1cm, y=1cm,
            font=\small\sffamily,
            >={Stealth[round, length=2.5mm, width=1.75mm]},
            state/.style={circle, draw, line width=1pt, minimum size=11mm, inner sep=0pt},
            gate/.style={rectangle, rounded corners=3mm, draw, line width=0.8pt,
                fill=gruCbBlue, minimum width=18mm, minimum height=9mm, align=center},
            cand/.style={rectangle, rounded corners=3mm, draw, line width=0.8pt,
                fill=gruCbOrange, minimum width=22mm, minimum height=10mm, align=center},
            func/.style={rectangle, rounded corners=2mm, draw, line width=0.8pt,
                fill=gruCbGray, minimum width=15mm, minimum height=8mm, align=center},
            op/.style={circle, draw, line width=0.8pt, minimum size=7mm, inner sep=0pt, fill=white},
            flow/.style={->, line width=1.2pt, black, rounded corners=5pt, shorten >=1pt},
            lane/.style={->, line width=1.0pt, gruArrowGray, rounded corners=5pt, shorten >=1pt},
            connect/.style={circle, fill=black, inner sep=0pt, minimum size=4pt}
        ]

            % --- Nodes ---
            \node[state] (hprev) at (-1.5, 2.0) {$h_{t-1}$};
            \node[state] (x)     at (-1.5,-2.0) {$x_t$};

            \node[func] (affg) at (2.5, 0) {\scriptsize affine};

            \node[gate] (zt) at (5.5, 1.2) {$z_t$\\\scriptsize $\sigma$};
            \node[gate] (rt) at (5.5,-1.2) {$r_t$\\\scriptsize $\sigma$};

            \node[op] (mulr) at (8.0, -1.2) {$\odot$};

            \node[func] (affh)   at (9.8, -1.2) {\scriptsize affine};
            \node[cand] (htilde) at (12.5, -1.2) {$\tilde h_t$\\[-2pt]\scriptsize $\tanh$};

            \node[gate] (oneMinusZ) at (8.0, 3.2) {$1-z_t$};

            \node[op] (mulPrev) at (12.5, 3.2) {$\odot$};
            \node[op] (mulNew)  at (12.5, 0.5) {$\odot$};

            \node[op] (addh)    at (15.0, 0.5) {$+$};

            \node[state] (ht) at (17.0, 0.5) {$h_t$};
            \node[state] (yt) at (17.0,-1.5) {$y_t$};

            % --- SPLIT POINTS ---
            \node[connect] (hSplit) at (0.5, 2.0) {};
            \node[connect] (xSplit) at (0.5, -2.0) {};
            \node[connect] (zSplit) at (7.0, 1.2) {};

            % --- ROUTING ---
            \draw[flow, -] (hprev) -- (hSplit);
            \draw[flow, -] (x) -- (xSplit);

            % Control Lines (Inputs to gates)
            \draw[lane] (hSplit) |- ($(affg.west)+(0,0.15)$);
            \draw[lane] (xSplit) |- ($(affg.west)+(0,-0.15)$);

            \draw[lane] (affg.east) -- ++(0.5,0) |- (zt.west);
            \draw[lane] (affg.east) -- ++(0.5,0) |- (rt.west);

            % Gate Logic Outputs
            \draw[lane, -] (zt) -- (zSplit);
            \draw[lane] (zSplit) -| (oneMinusZ.south);
            \draw[lane] (zSplit) |- (mulNew.west);
            \draw[lane] (rt) -- (mulr);

            % --- MAIN DATA PATHS ---

            % 1. H_prev to Reset Mixer
            % Route: Right to x=4.0 (gap between affine and gates) -> down to y=0 -> right -> down
            \draw[flow] (hSplit) -- (4.0, 2.0) -- (4.0, 0.0) -- (8.0, 0.0) -- (mulr.north);

            % 2. X to Candidate Mixer
            \draw[flow] (xSplit) -| (affh.south);

            % 3. H_prev to Preservation Mixer (High road)
            \draw[flow] (hSplit) -- ++(0, 2.5) -| (mulPrev.north);

            % --- OPERATIONS ---
            \draw[lane] (oneMinusZ.east) -- (mulPrev.west);
            \draw[flow] (mulr) -- (affh);
            \draw[flow] (affh.east) -- (htilde.west);

            \draw[flow] (htilde.north) -- (mulNew.south);

            \draw[flow] (mulPrev.east) -| (addh.north);
            \draw[flow] (mulNew.east) -- (addh.west);

            \draw[flow] (addh) -- (ht);
            \draw[flow] (ht) -- (yt);

        \end{tikzpicture}%
    }
    \ifdefined\HCode
    \caption{Schematic: Gated Recurrent Unit (GRU) cell \citep{Cho2014}.}
    \else
    \caption{Schematic: Gated Recurrent Unit (GRU) cell \citep{Cho2014}.}
    \fi
    \label{fig:lec7-gru}
\end{figure}

        % Keep the cell diagrams together and prevent the following non-float boxes from
        % appearing before a deferred figure.
            \FloatBarrier
            \clearpage

            \begin{tcolorbox}[summarybox,title={Vanilla RNN vs.\ GRU vs.\ LSTM}]
\textbf{Vanilla RNN:} Single hidden state \(h_t\) updated via \(h_t=f(x_tW_{xh}+h_{t-1}W_{hh}+b_h)\); simplest and parameter\hyp{}efficient but most prone to vanishing/exploding gradients on long sequences.\\
\textbf{GRU:} Uses update and reset gates to interpolate between \(h_{t-1}\) and a candidate state; fewer parameters than LSTM, often a good default when sequences are moderately long.\\
\textbf{LSTM:} Maintains a separate cell state \(c_t\) and uses input/forget/output gates; highest parameter count but most robust on very long-range dependencies and widely used in legacy sequence models.
\end{tcolorbox}

\begin{tcolorbox}[summarybox,title={Minimal training loop with masks and clipping}]
\begin{verbatim}
h = torch.zeros(B, d_h)
for x, y, mask in loader:         # [B,T,dx], [B,T], [B,T]
    h = h.detach()                # carry state, drop graph
    logits, h = rnn(x, h)
    loss = (mask * CE(logits, y)).sum() / mask.sum()
    loss.backward()
    torch.nn.utils.clip_grad_norm_(model.parameters(), tau)
    opt.step(); opt.zero_grad()
\end{verbatim}
Variable-length sequences are padded; the mask zeros out pads in the loss. Reset \(h\) between sequences that should not share state.
\end{tcolorbox}

\begin{tcolorbox}[summarybox,title={Pitfalls checklist}]
Mis-handled pads (loss on padded tokens); forgetting to detach hidden state across mini\hyp{}batches; no clipping on long sequences; BatchNorm inside recurrence (prefer LayerNorm); teacher-forcing train/test mismatch without scheduled sampling; no masking leads to biased gradients; dropout applied independently each time step instead of variational/recurrent dropout.
\end{tcolorbox}

\paragraph{Attention mechanisms.} Even with gating, long sequences can challenge fixed-size hidden states. Attention augments the decoder with a content-based lookup into the encoder states, as visualized in \Cref{fig:lec7-attention}. Bright entries correspond to encoder positions that most influence each generated token.

\begin{figure}[t]
    \centering
    \begin{tikzpicture}
    \begin{axis}[
        width=0.75\linewidth,
        height=6.5cm,
        xlabel={Source tokens (encoder positions)},
        ylabel={Target tokens (decoder steps)},
        xtick={1,2,3,4,5},
        xticklabels={je,veux,acheter,un,livre},
        ytick={1,2,3,4,5},
        yticklabels={I,want,to,buy,.},
        y dir=reverse,
        colormap/viridis,
        colorbar,
        colorbar style={title={$\alpha_{t,s}$}, height=3.0cm},
        nodes near coords={\pgfmathprintnumber[fixed,precision=2]{\pgfplotspointmeta}},
        nodes near coords align={center},
        every node near coord/.append style={
            font=\tiny,
            fill=white,
            fill opacity=0.75,
            text opacity=1,
            inner sep=1pt
        },
        point meta min=0, point meta max=1,
        every axis/.append style={font=\small},
        enlargelimits=false,
        axis on top,
    ]
    % rows = target tokens, columns = source tokens
    \addplot[
        matrix plot*,
        mesh/cols=5,
        mesh/rows=5,
        point meta=explicit,
    ] table[row sep=\\, meta=z]{
    x y z\\
    % I
    1 1 0.75\\ 2 1 0.10\\ 3 1 0.05\\ 4 1 0.05\\ 5 1 0.05\\
    % want
    1 2 0.10\\ 2 2 0.70\\ 3 2 0.10\\ 4 2 0.05\\ 5 2 0.05\\
    % to
    1 3 0.10\\ 2 3 0.15\\ 3 3 0.55\\ 4 3 0.10\\ 5 3 0.10\\
    % buy
    1 4 0.05\\ 2 4 0.10\\ 3 4 0.70\\ 4 4 0.10\\ 5 4 0.05\\
    % .
    1 5 0.05\\ 2 5 0.05\\ 3 5 0.10\\ 4 5 0.20\\ 5 5 0.60\\
    };
    % highlight argmax per target token
    \addplot[only marks, mark=*, mark size=1.8pt] coordinates {
        (1,1)  % I -> je
        (2,2)  % want -> veux
        (3,3)  % to -> acheter
        (3,4)  % buy -> acheter
        (5,5)  % . -> livre
    };
    \end{axis}
    \end{tikzpicture}
    % Avoid inline math in captions; it wraps poorly in some EPUB renderers.
    \caption{Schematic: Attention heatmap for a translation model. Rows are target tokens (decoder steps) and columns are source tokens (encoder positions). Each cell is an attention weight; the dot in each row marks the source position receiving the most attention.}
    \label{fig:lec7-attention}
\end{figure}

\subsection{RNN Input-Output Configurations}

RNNs can be configured in several ways depending on the task:

\begin{itemize}
    \item \textbf{Many-to-Many (Equal Length):} Input and output sequences have the same length \( T \). For example, sequence labeling tasks.
    \item \textbf{Many-to-One:} Input is a sequence of length \( T \), output is a single prediction. Example: sentiment analysis where a sentence maps to a sentiment score.
    \item \textbf{Many-to-Many (Unequal Length):} Input and output sequences have different lengths. Example: machine translation where input and output sentences differ in length.
    \item \textbf{One-to-Many:} Single input produces a sequence output. Less common, but applicable in tasks like image captioning where one image input generates a sequence of words.
\end{itemize}

The main difference lies in how the loss is computed and how outputs are generated, but the underlying backpropagation principles remain consistent.

\subsection{Representing Words for RNN Inputs}

Natural language processing (NLP) requires converting words into numerical representations that RNNs can process. Since machines operate on numbers, words must be encoded appropriately.

\paragraph{Vocabulary Size and Word Representation}

Natural language has a large but finite vocabulary at any chosen granularity. Depending on whether you count words, inflections, or subword units, the effective inventory can range from tens of thousands (common token vocabularies) to far larger sets of distinct surface forms.

This finite vocabulary allows us to define a fixed-size dictionary \( V \) of words.

\paragraph{One-Hot Encoding}

A simple method to represent words is \emph{one-hot encoding}:
\begin{itemize}
    \item Assign each word in the vocabulary a unique index \( i \in \{1, \ldots, |V|\} \).
    \item Represent each word as a vector \( \mathbf{w} \in \mathbb{R}^{|V|} \) where all entries are zero except the \( i \)-th entry, which is 1.
\end{itemize}

For example, if \( |V| = 10,000 \), the word "house" might be represented as:
\[
\mathbf{w}_{\text{house}} = [0, 0, \ldots, 1, \ldots, 0],
\]
with the 1 in the position corresponding to "house".

This representation is sparse and high-dimensional. Conceptually the one-hot basis vectors correspond to the rows of the identity matrix \( I_{|V|} \), but in practice modern models replace that fixed basis with a \emph{learned} embedding table whose rows are trainable parameters. One convenient view is:
\[
\text{one-hot words} \;\leftrightarrow\; \text{rows of } I_{|V|},\qquad
\mathbf{E} \in \mathbb{R}^{|V|\times d}\ \text{trainable},\ \ \mathbf{e}_t = \mathbf{x}_t \mathbf{E},
\]
so a one-hot input \(\mathbf{x}_t\) simply selects the corresponding row of \(\mathbf{E}\) as its dense embedding \(\mathbf{e}_t\).

\paragraph{Limitations of One-Hot Encoding}

One-hot vectors do not capture semantic similarity between words (e.g., "king" and "queen" are orthogonal). Indeed, the cosine similarity between any two distinct one-hot vectors is exactly zero because their non-zero entries never overlap. They also lead to very high-dimensional inputs, which can be computationally costly to store and process.

To address these limitations we introduce distributed word representations (e.g., Word2Vec, GloVe, fastText) that map words to dense vectors where geometric relationships encode semantic similarity.

\subsection{Example: Sentiment Analysis with RNNs}

Consider the sentence:
\[
\text{"This place is great."}
\]

Each word is first converted into a numerical vector (e.g., one-hot encoded). The sequence of vectors is fed into the RNN, which processes them sequentially.

For a \emph{many-to-one} RNN (e.g., sentiment classification), we are interested in the hidden state after processing the entire sentence. This final hidden state summarizes the contextual information and can be fed into a classifier to predict the sentiment label.

\subsection{Limitations of One-Hot Encoding in Natural Language Processing}

Recall that one-hot encoding represents each word in the vocabulary as a unique vector with a single 1 and zeros elsewhere. While this approach guarantees uniqueness, it fails to capture any semantic or syntactic relationships between words.

\paragraph{Example:} Consider the sentences:
\begin{itemize}
    \item ``This place is great.''
    \item ``This place is awesome.''
    \item ``This place is good.''
\end{itemize}
Using one-hot encoding, the words \textit{great}, \textit{awesome}, and \textit{good} are represented as orthogonal vectors. Thus, a model trained to associate ``great'' with a five-star rating may not generalize to ``awesome'' or ``good,'' despite their similar meanings.

\paragraph{Document similarity:} Suppose we have two documents:
\begin{align*}
    D_1 &: \text{``I enjoyed talking to the monarchs.''} \\
    D_2 &: \text{``I loved conversing with the Royals.''}
\end{align*}
Semantically, these sentences convey the same meaning. However, one-hot encoding treats \textit{monarchs} and \textit{Royals} as distinct tokens, as well as \textit{talking} and \textit{conversing}. Consequently, simple word-count based similarity metrics (e.g., cosine similarity on bag-of-words vectors) would yield a low similarity score, failing to capture the semantic equivalence.

\paragraph{Summary:} One-hot encoding:
\begin{itemize}
    \item Ignores semantic similarity between words.
    \item Treats synonyms and related words as completely unrelated.
    \item Does not capture contextual or syntactic information.
\end{itemize}

This motivates the need for richer \textbf{feature representations} of words that encode their meanings and relationships.

\subsection{Feature-Based Word Representations}

To encode the meaning of words, we can represent each word as a vector of \emph{features} that capture semantic properties. These features can be handcrafted or learned, and aim to reflect qualities such as sentiment, category, or other linguistic attributes.

\paragraph{Example:} Consider the following words:
\[
\text{man, woman, king, queen, orange, apple, monarch, royal}
\]
We can define features such as:
\begin{itemize}
    \item \textbf{Gender}: male, female, neutral
    \item \textbf{Royalty status}: commoner, royalty
    \item \textbf{Age}: adult, child
    \item \textbf{Category}: animal, fruit, person, abstract
    \item \textbf{Edibility}: edible, inedible
    \item \textbf{Sweetness}: sweet, not sweet
\end{itemize}

Assigning numerical values to these features for each word yields a vector representation that encodes semantic information. For example:

\begin{center}
\small
\begin{tabular}{l|cccccccc}
\textbf{Word} & Gender & Royalty & Age & Person & Fruit & Title & Abstract & Sweet \\
\hline
man & 1 & 0 & 1 & 1 & 0 & 0 & 0 & 0 \\
woman & 0 & 0 & 1 & 1 & 0 & 0 & 0 & 0 \\
king & 1 & 1 & 1 & 1 & 0 & 1 & 0 & 0 \\
queen & 0 & 1 & 1 & 1 & 0 & 1 & 0 & 0 \\
orange & 0 & 0 & 0 & 0 & 1 & 0 & 0 & 1 \\
apple & 0 & 0 & 0 & 0 & 1 & 0 & 0 & 1 \\
monarch & 0.5 & 1 & 0.5 & 1 & 0 & 1 & 0 & 0 \\
royal & 0 & 1 & 0.5 & 0 & 0 & 1 & 1 & 0 \\
\end{tabular}
\end{center}

\paragraph{Notes:}
\begin{itemize}
    \item The values can be binary or continuous, reflecting degrees or uncertainty (e.g., ``monarch'' receives a gender value of 0.5 to indicate that the term is used for multiple genders).
    \item High-level categories are often represented with several binary indicators (person, fruit, title, abstract) rather than a single categorical feature.
    \item Some features may be language- or culture-specific, and this approach requires domain knowledge and manual feature engineering.
\end{itemize}

\paragraph{Advantages:}
\begin{itemize}
    \item Captures semantic similarity: words with similar features have similar representations.
    \item Enables reasoning about relationships (e.g., gender, royalty).
    \item Provides interpretable dimensions.
\end{itemize}

\paragraph{Limitations:}
\begin{itemize}
    \item Requires extensive manual effort to define and annotate features.
    \item May not scale well to large vocabularies or complex semantics.
    \item Difficult to capture contextual nuances and polysemy.
\end{itemize}

\subsection{Towards Distributed Word Representations}

The feature-based approach motivates the idea of \textbf{distributed representations}, where each word is represented as a dense vector in a continuous space. These vectors encode semantic and syntactic properties implicitly, often learned from large corpora.

\paragraph{Key idea:} Instead of one-hot vectors, represent each word \( w \) as a vector \(\mathbf{v}_w \in \mathbb{R}^d\), where \(d \ll |V|\) (vocabulary size), such that:
\[
\text{similarity}(\mathbf{v}_w, \mathbf{v}_{w'}) \approx \text{semantic similarity}(w, w')
\]

\paragraph{Methods to obtain distributed representations}
Several approaches learn such embeddings automatically from corpora, including neural language models (Word2Vec CBOW and Skip-gram), matrix factorization methods (GloVe), and contextual models (ELMo, BERT). These methods leverage co-occurrence statistics to place semantically similar words nearby in the embedding space.

\subsection{Semantic Relationships in Word Embeddings}

We continue our exploration of word embeddings by examining how semantic relationships between words can be captured in vector space. The key insight, as demonstrated by \citet{Mikolov2013}, is that certain linguistic regularities and patterns manifest as linear relationships between word vectors.
\paragraph{Subword tokenization and OOV handling.} Modern NLP systems rarely operate on raw word types alone. To reduce vocabulary size and handle out-of-vocabulary (OOV) words, they tokenize text into \emph{subword units}. Byte Pair Encoding (BPE) and WordPiece learn a compact inventory of frequent character sequences; words are segmented into a small number of subwords that can be re-composed by the model. FastText instead augments word vectors with character \(n\)-gram embeddings, so the representation of an unseen word is the sum of its subword vectors. Subword methods improve data efficiency, model morphology, and eliminate true OOVs while keeping sequence lengths manageable.

\paragraph{Example: Gender and Royalty Analogies}

Consider the analogy involving gender and royalty:

\[
\text{king} - \text{man} + \text{woman} \approx \text{queen}.
\]

This relationship suggests that the vector difference between \textit{king} and \textit{man} encodes the concept of "royal masculinity," and adding the vector for \textit{woman} shifts this to "royal femininity," yielding a vector close to \textit{queen}.

More formally, if we denote the embedding of a word \( w \) as \(\mathbf{v}_w\), then the analogy can be expressed as:

\begin{equation}
\mathbf{v}_{\text{king}} - \mathbf{v}_{\text{man}} + \mathbf{v}_{\text{woman}} \approx \mathbf{v}_{\text{queen}}.
\label{eq:king-queen-analogy}
\end{equation}

This vector arithmetic captures semantic relationships and can be used to find words that best complete analogies by maximizing cosine similarity:

\[
\arg\max_{w} \cos\big(\mathbf{v}_w, \mathbf{v}_{\text{king}} - \mathbf{v}_{\text{man}} + \mathbf{v}_{\text{woman}}\big).
\]

Here \(\cos(\mathbf{a}, \mathbf{b}) = \frac{\mathbf{a}^\top \mathbf{b}}{\|\mathbf{a}\|\,\|\mathbf{b}\|}\) denotes cosine similarity between vectors \(\mathbf{a}\) and \(\mathbf{b}\).

\paragraph{Empirical Validation}

Mikolov et al. showed that these relationships hold not only for gender and royalty but also for other semantic categories such as family relations (e.g., \textit{uncle} to \textit{aunt}), geographical locations (e.g., \textit{Portugal} to \textit{Lisbon}), and cultural concepts. The distances between word vectors reflect meaningful semantic distances, such as:

\[
\|\mathbf{v}_{\text{man}} - \mathbf{v}_{\text{woman}}\|_2 \approx \|\mathbf{v}_{\text{king}} - \mathbf{v}_{\text{queen}}\|_2,
\]

and similarly for other pairs.

\paragraph{Geographical and Cultural Clustering}

Word embeddings also often (empirically) capture geographic and cultural proximity. For example, the embeddings for countries and their capitals frequently cluster together:

\[
\mathbf{v}_{\text{Portugal}} \approx \mathbf{v}_{\text{Lisbon}}, \quad \mathbf{v}_{\text{Spain}} \approx \mathbf{v}_{\text{Madrid}}, \quad \mathbf{v}_{\text{France}} \approx \mathbf{v}_{\text{Paris}},
\]

and countries that are geographically close tend to have embeddings closer in vector space (e.g., \textit{China} is closer to \textit{Russia} and \textit{Japan} than to \textit{Portugal}), although the strength of this effect depends on the corpus used for training. Throughout this chapter, statements such as $\mathbf{v}_{\text{Portugal}} \approx \mathbf{v}_{\text{Lisbon}}$ are shorthand for ``the cosine similarity between the vectors exceeds a data-dependent threshold (typically $>0.8$)'' or, equivalently, that the two vectors lie in each other's $k$-nearest-neighbor list under cosine distance. These relations are empirical regularities rather than hard equalities, and the precise neighborhood structure depends on the corpus, training objective, and dimensionality of the embedding space.
\Cref{fig:lec7-embedding-clusters} illustrates such neighborhoods after projecting embeddings to two principal components; the visual clusters make the relational analogies immediately apparent.
\begin{figure}[t]
    \centering
    \includegraphics[width=\textwidth]{lec14_embeddings_clusters}
    % Avoid inline math in captions; it wraps poorly in some EPUB renderers.
    \caption{Schematic: Toy 2D projection of word embeddings showing neighboring clusters (countries vs. capitals). Light hulls highlight clusters; arrows show that country-to-capital displacement vectors align, a visual check on analogy structure.}
    \label{fig:lec7-embedding-clusters}
\end{figure}

\subsection{Feature-Based Representation vs. One-Hot Encoding}

The success of word embeddings lies in their ability to represent words as dense vectors encoding multiple latent features, as opposed to sparse one-hot vectors.

\paragraph{One-Hot Encoding}

One-hot encoding represents each word as a vector with a single 1 and zeros elsewhere. This representation is:

\begin{itemize}
    \item \textbf{Sparse}: High-dimensional with mostly zeros (in the one-hot representation used here, the dimensionality equals the vocabulary size and only one entry is non-zero for each word).
    \item \textbf{Uninformative}: No notion of similarity between words.
\end{itemize}

\paragraph{Feature-Based Embeddings}

In contrast, word embeddings are dense vectors in \(\mathbb{R}^d\) (typically \(d=100\) to \(300\)) where each dimension can be interpreted as a latent feature capturing semantic or syntactic properties. These features emerge from the training process rather than being explicitly defined. The term ``feature-based embedding'' is non-standard in the literature; we use it here simply to stress that the coordinates behave like automatically discovered features. Most papers instead refer to these objects as \emph{dense distributed representations}, and we always mean that same concept.
Unlike the hand-crafted example below, the latent dimensions of distributed embeddings are not usually interpretable in isolation. They capture statistical regularities uncovered automatically during training. Interpretability can sometimes be probed post hoc (e.g., via probing classifiers or dimension alignment), but there is no guarantee that any single axis corresponds cleanly to a human-understandable attribute.

\paragraph{Context Window Convention}

When we refer to the ``context'' of a word \(w_t\) we mean the multiset of tokens that fall within a symmetric sliding window of radius \(c\) around position \(t\). Formally,
\[
    \mathcal{C}_t = \{\,w_{t-c},\ldots,w_{t-1},\; w_{t+1},\ldots,w_{t+c}\,\}.
\]
Directional variants sometimes use only the preceding words. The co-occurrence matrix in the next section corresponds to the special case \(c=1\), where we only count the following token. Making the window definition explicit removes ambiguity about which neighboring words contribute counts to \(C_{ij}\).

\subsection{Open Questions: Feature Discovery and Representation}

Two natural questions arise regarding the nature of these features:

\begin{enumerate}
    \item \textbf{Who decides the features?}
    Unlike manually engineered features, the features in word embeddings are \emph{discovered automatically} during training. There is no explicit human selection of features such as "gender" or "age." Instead, the training algorithm uncovers latent dimensions that best capture word co-occurrence statistics.

    \item \textbf{How are the feature values determined?}
The feature values (vector components) are learned by optimizing an objective function that encourages words appearing in similar contexts to have similar embeddings. This is typically done via self-supervised learning on large corpora. In a self-supervised setting the model creates its own supervision signal (future tokens, masked tokens, or neighboring sentences), so that no external labels are required.
\end{enumerate}

\paragraph{Self-supervised learning of embeddings}

Although this learning is sometimes described informally as ``unsupervised,'' it is more accurately \emph{self-supervised} because the training objective uses the structure of the data itself (e.g., predicting context words) to create targets. In self-supervised setups the model manufactures its own targets from the input (for example, masking a word and asking the network to predict it), eliminating the need for manually annotated labels.

\paragraph{Summary}

Thus, the embedding process can be viewed as a function:

\[
f: \text{Vocabulary} \to \mathbb{R}^d,
\]

where \(f\) is learned to encode semantic and syntactic properties implicitly, without explicit feature engineering. In matrix form we implement \(f\) by selecting the \emph{row} of the learned embedding matrix \(\mathbf{E}\) corresponding to the word of interest (row\hyp{}embedding convention).

In practice we optimize objectives such as the continuous bag-of-words (CBOW) likelihood (predicts a center word from its surrounding context) and the skip-gram with negative sampling (SGNS) loss (predicts context words given a center word). These training regimes are typically optimized with SGD variants (SGD, Adam) on large corpora; \Cref{chap:nlp} spells out the CBOW/skip-gram objectives in detail.

% References:
% \begin{thebibliography}{9}
% \bibitem{mikolov2013distributed}
% Tomas Mik

\noindent\textbf{Forward pointer.} We cover Word2Vec/CBOW, skip-gram, and GloVe in \Cref{chap:nlp}; keep the self-supervised framing in mind as we shift from RNNs to dedicated embedding objectives.

\subsection{Wrapping Up the Derivations}

In this chapter, we have explored the foundational concepts behind modeling sequences in natural language processing (NLP) using recurrent neural networks (RNNs). We began by considering the problem of predicting the probability of a word given its preceding context, which is central to language modeling.

Recall that the goal is to estimate the conditional probability of a word \( w_t \) given the sequence of previous words \( w_1, w_2, \ldots, w_{t-1} \):
\begin{equation}
    P(w_t \mid w_1, w_2, \ldots, w_{t-1}).
\end{equation}

This probability can be modeled using an RNN, which maintains a hidden state \( \mathbf{h}_t \) that summarizes the history up to time \( t \). A common indexing choice is: consume the current token \(w_t\) (as an embedding \(\mathbf{x}_t\)) and predict the \emph{next} token \(w_{t+1}\). This is equivalent to modeling \(P(w_t\mid w_{1:t-1})\) after a one-step shift.
\begin{align}
    \mathbf{h}_t &= f(\mathbf{h}_{t-1}, \mathbf{x}_t; \theta), \\
    P(w_{t+1} \mid w_1, \ldots, w_t) &= g(\mathbf{h}_t; \theta), \label{eq:rnn_output_prob}
\end{align}
where \( \mathbf{x}_t \) is the input representation (e.g., word embedding) of the word \( w_t \), \( f \) is the recurrent update function parameterized by \(\theta\), and \( g \) maps the hidden state to a probability distribution over the vocabulary. Because the hidden state is computed recursively, \(\mathbf{h}_t\) already aggregates information about the entire prefix \((w_1,\ldots,w_t)\); predicting \(w_{t+1}\) from \(\mathbf{h}_t\) therefore reflects the Markovian summary that RNNs maintain. Explicitly, repeatedly substituting \Cref{eq:rnn_hidden_state} reveals that \(\mathbf{h}_t=f(f(\cdots f(\mathbf{h}_0,\mathbf{x}_1),\ldots),\mathbf{x}_t)\), so no information is lost other than the compression inherent to the finite-dimensional state vector.

\paragraph{Training Objective}

The network is trained to maximize the likelihood of the observed sequences in a large corpus of text. Given a training sequence \( (w_1, w_2, \ldots, w_T) \), the log-likelihood is:
\begin{equation}
    \mathcal{L}(\theta) = \sum_{t=1}^{T-1} \log P(w_{t+1} \mid w_1, \ldots, w_t; \theta).
\end{equation}

This objective encourages the model to assign high probability to the actual next word in the sequence, effectively learning the statistical structure of the language without explicit labeling of word relationships.

\paragraph{Self-supervised nature of language modeling}

A key insight is that no explicit labeling is required to train such models. The natural co-occurrence statistics of words in large corpora serve as implicit supervision. For example, the model learns that the word "juice" often follows "apple" because this pattern frequently appears in the training data. This is the essence of \emph{self-supervised} learning in NLP, where the prediction targets are created directly from the input sequence.

\paragraph{Feature Representations}

The input to the RNN is typically a dense vector representation of words, known as \emph{word embeddings}. These embeddings capture semantic and syntactic properties of words and are learned jointly with the model parameters. The embedding matrix \( \mathbf{E} \in \mathbb{R}^{V \times d} \), where \( V \) is the vocabulary size and \( d \) is the embedding dimension, maps each word index to a vector. We denote by \(\mathbf{e}_{w_t}\in\{0,1\}^{1\times V}\) the one-hot row indicator of word \(w_t\). The embedding lookup can then be written compactly as
\begin{equation}
    \mathbf{x}_t = \mathbf{e}_{w_t}\mathbf{E},
\end{equation}
so \(\mathbf{E}[w_t]\) simply selects the row of \(\mathbf{E}\) associated with \(w_t\). The boldface \(\mathbf{E}[\,\cdot\,]\) notation is intentional: it denotes array indexing into the learnable embedding matrix rather than an expectation operator \( \mathbb{E}[\cdot] \). Whenever expectations appear later in the book we write them explicitly as \( \mathbb{E}[\cdot] \) to avoid overload.

\paragraph{Summary of the Modeling Pipeline}

\begin{enumerate}
    \item Collect a large corpus of text data.
    \item Tokenize the text into sequences of words.
    \item Represent words as embeddings (initialized from a lookup table that is \emph{learned jointly} with the network parameters).
    \item Use an RNN to process sequences and produce hidden states.
    \item Predict the next word probability distribution from the hidden state.
    \item Train the model by maximizing the likelihood of the observed sequences.
\end{enumerate}

\begin{tcolorbox}[summarybox,title={LSTM and GRU equations (compact)}]
\textbf{LSTM} (single layer):
\[
\begin{aligned}
    \mathbf{i}_t &= \sigma(\mathbf{x}_t\mathbf{W}_i+\mathbf{h}_{t-1}\mathbf{U}_i+\mathbf{b}_i), &
    \mathbf{f}_t &= \sigma(\mathbf{x}_t\mathbf{W}_f+\mathbf{h}_{t-1}\mathbf{U}_f+\mathbf{b}_f), \\
    \tilde{\mathbf{c}}_t &= \tanh(\mathbf{x}_t\mathbf{W}_c+\mathbf{h}_{t-1}\mathbf{U}_c+\mathbf{b}_c), &
    \mathbf{o}_t &= \sigma(\mathbf{x}_t\mathbf{W}_o+\mathbf{h}_{t-1}\mathbf{U}_o+\mathbf{b}_o), \\
    \mathbf{c}_t &= \mathbf{f}_t\odot \mathbf{c}_{t-1} + \mathbf{i}_t\odot \tilde{\mathbf{c}}_t, &
    \mathbf{h}_t &= \mathbf{o}_t\odot \tanh(\mathbf{c}_t).
\end{aligned}
\]
\textbf{GRU}:
\[
\begin{aligned}
    \mathbf{z}_t &= \sigma(\mathbf{x}_t\mathbf{W}_z+\mathbf{h}_{t-1}\mathbf{U}_z+\mathbf{b}_z), &
    \mathbf{r}_t &= \sigma(\mathbf{x}_t\mathbf{W}_r+\mathbf{h}_{t-1}\mathbf{U}_r+\mathbf{b}_r), \\
    \tilde{\mathbf{h}}_t &= \tanh(\mathbf{x}_t\mathbf{W}_h+(\mathbf{r}_t\odot\mathbf{h}_{t-1})\mathbf{U}_h+\mathbf{b}_h), &
    \mathbf{h}_t &= (1-\mathbf{z}_t)\odot \mathbf{h}_{t-1} + \mathbf{z}_t\odot \tilde{\mathbf{h}}_t.
\end{aligned}
\]
All gates are elementwise; $\sigma$ denotes the logistic sigmoid and $\odot$ the Hadamard product.

\paragraph{Notation note.} In the sequence\hyp{}model chapters, $\sigma(\cdot)$ always denotes the logistic sigmoid gate nonlinearity; when $\sigma$ is used without an argument (e.g., in earlier chapters for noise scales $\sigma^2$) it refers to a standard deviation. Context distinguishes these roles.
\end{tcolorbox}

\begin{tcolorbox}[summarybox,title={Key takeaways}]
\begin{itemize}
    \item Language modeling is trained with self-supervision by maximizing next-token likelihood.
    \item Embeddings provide dense, learned word features; RNN hidden states encode context.
    \item Stability tools (clipping, gating, attention) enable long-range dependency modeling.
\end{itemize}
\end{tcolorbox}

\begin{tcolorbox}[summarybox,title={Exercises and lab ideas}]
\begin{itemize}
    \item Train a many-to-one sentiment classifier; plot gradient norms with and without clipping.
    \item Train a small LSTM language model; compare perplexity with/without weight tying and scheduled sampling.
    \item Empirically sweep \(\|\mathbf{W}_{hh}\|\) (spectral scaling) and reproduce vanishing/exploding behavior.
    \item Implement the masked, truncated-BPTT loop above and verify that pads do not affect the loss.
\end{itemize}
\end{tcolorbox}

\medskip
\paragraph{Where we head next.} RNNs excel at streaming and moderate-context tasks but struggle with very long dependencies and parallel hardware utilization. \Cref{chap:transformers} introduces attention/Transformers to address these limits; \Cref{chap:nlp} revisits embeddings, perplexity, and weight tying \citep{Press2016} that pair naturally with RNN language models.

\paragraph{References.} Full citations for works mentioned in this chapter appear in the book-wide bibliography.
\nocite{JurafskyMartin2023,Goodfellow2016,Mikolov2010,LevyGoldberg2014}

% Chapter 13
\section{Neural Network Applications in Natural Language Processing}\label{chap:nlp}
\graphicspath{{assets/lec8/}{assets/lec14/}}

\begin{tcolorbox}[summarybox,title={Learning Outcomes}]
\begin{itemize}
    \item Describe distributional semantics and the motivation for dense word embeddings.
    \item Derive and implement common embedding objectives (CBOW/skip-gram, negative sampling) and evaluate them via analogy tasks.
    \item Connect embedding quality to downstream architectures (RNNs, Transformers) and fairness considerations.
\end{itemize}
\end{tcolorbox}

\Cref{chap:rnn} framed language as prediction under context: the model must assign probability to the next token using a compact state that summarizes the past. To make that workable in practice, we need representations that turn discrete symbols into geometry. This chapter builds those representations (embeddings and their training objectives) and shows how to evaluate and audit them; \Cref{chap:transformers} then uses the same ingredients inside attention-based architectures for long-context modeling and efficient deployment. The roadmap in \Cref{fig:roadmap} places this as the application/deployment tail of the neural strand.

\begin{tcolorbox}[summarybox,title={Design motif}]
Representation learning as a contract between data and objective---when you train on co-occurrence, you get both useful structure (analogies, clusters) and the biases present in the corpus.
\end{tcolorbox}

\begin{tcolorbox}[summarybox,title={Risk \& audit}]
\begin{itemize}
    \item \textbf{Evaluation leakage:} similarity/analogy benchmarks can overlap training sources; keep a truly held-out evaluation set and treat ``standard'' datasets as potentially contaminated.
    \item \textbf{Tokenization debt:} preprocessing and vocabulary choices (case, subwords, cutoffs) change what the model can represent; version tokenizers and report them with results.
    \item \textbf{Frequency bias:} rare words get unstable vectors; audit neighborhoods by frequency and use subsampling/regularization so geometry is not just Zipf effects.
    \item \textbf{Social bias:} co-occurrence reflects social structure and stereotypes; probe for bias before using embeddings in decisions and document mitigations.
    \item \textbf{Privacy/memorization:} large corpora can contain sensitive strings; treat training data as a security boundary and audit downstream systems for memorization.
\end{itemize}
\end{tcolorbox}

\begin{tcolorbox}[summarybox,title={Chapter map (core vs.\ detours)}]
\begin{itemize}
    \item \textbf{Core path:} distributional hypothesis $\rightarrow$ Word2Vec (CBOW/skip-gram) objectives $\rightarrow$ negative sampling $\rightarrow$ how to evaluate embeddings (analogies, neighbors) $\rightarrow$ how embeddings plug into RNNs/Transformers.
    \item \textbf{Optional detours:} deeper objective derivations, historical notes, and the deployment/fairness discussion (important when embeddings are used in decisions rather than as a pure feature extractor).
\end{itemize}
\end{tcolorbox}

\subsection{Context and Motivation}
\label{sec:nlp_context_and_motivation}

In this chapter, we focus on neural methods for natural language processing (NLP) through the lens of representation learning. Previously, we introduced the idea of representing words as inputs to a neural network, typically encoded as one-hot vectors, and obtaining as output a feature representation of these words. This feature representation captures semantic and syntactic properties of words in a continuous vector space.

A classic example illustrating the power of such representations is the analogy:
\[
\text{king} - \text{man} + \text{woman} \approx \text{queen}.
\]
This demonstrates that vector arithmetic on word embeddings can capture meaningful relationships between words. The goal is to find a vector space embedding where semantic similarity corresponds to geometric closeness.

\subsection{Problem Statement}
\label{sec:nlp_problem_statement}

Given a vocabulary (corpus) of approximately 10,000 words, we want to learn a mapping from each word to a dense vector representation in a feature space of dimension \(d\), where \(d\) is typically between 200 and 500. Formally, if the vocabulary size is \(V\), each word \(w_i\) is initially represented as a one-hot vector \(\mathbf{x}_i \in \mathbb{R}^V\), where
\[
x_{ij} = \begin{cases}
1 & \text{if } j = i, \\
0 & \text{otherwise}.
\end{cases}
\]
Here the row index \(i\) selects the word and the column index \(j\) specifies the position within the \(V\)-dimensional one-hot vector, so each word is associated with a unique canonical basis vector.
Our objective is to learn an embedding function
\[
f: \{1, \ldots, V\} \to \mathbb{R}^d,
\]
such that semantic and syntactic properties of words are preserved in the embedding space.

\subsection{Key Insight: Distributional Hypothesis}
\label{sec:nlp_key_insight_distributional_hypothesis}

The foundational linguistic principle underlying word embeddings is the \emph{distributional hypothesis}, often summarized by the phrase:
\begin{quote}
\textit{You shall know a word by the company it keeps.}
\end{quote}
This idea, attributed to the linguist John Robert Firth, states that the meaning of a word can be inferred from the contexts in which it appears.

\paragraph{Example:} The word \emph{pretty} can have different meanings depending on context:
\begin{itemize}
    \item In the collocation ``pretty good,'' \emph{pretty} functions as an adverb meaning ``very'' and modifies an adjective.
    \item In phrases such as ``pretty image'' or ``pretty optics,'' \emph{pretty} is an adjective meaning ``attractive.''
\end{itemize}
By explicitly examining the surrounding words (context windows of a few tokens to the left and right), we can infer the intended meaning: instances co-occurring with evaluative adjectives like ``good'' teach the ``intensifier'' sense, whereas contexts rich in nouns like ``image'' teach the ``aesthetic'' sense.

\subsection{Contextual Meaning and Feature Extraction}
\label{sec:nlp_contextual_meaning_and_feature_extraction}

Words appear in many different contexts, and by aggregating information from these contexts, we can infer intrinsic features of the word. For example, the contexts in which \emph{pretty} appears with \emph{good} or \emph{image} help us understand its different senses.

This motivates the use of statistical models that learn word embeddings by analyzing large corpora and capturing co-occurrence patterns.

\begin{tcolorbox}[summarybox,title={Author's note: who chooses the features?}]
A natural student question is: if embeddings represent ``features,'' who decides what the features are? In modern embedding learning, the answer is: nobody writes them down explicitly. The features emerge from the training objective. By training a model to predict nearby words (or to distinguish real context pairs from random ones), the optimization process forces the hidden representation to encode whatever properties are useful for prediction. This is best viewed as \emph{self-supervised} learning: targets come from the text itself via context windows, rather than from human labels.
\end{tcolorbox}

\subsection{Word2Vec: Two Architectures}
\label{sec:nlp_word2vec_two_architectures}

The Word2Vec framework, introduced by \citet{Mikolov2013}, operationalizes the distributional hypothesis through two main architectures:

\begin{enumerate}
    \item \textbf{Continuous Bag of Words (CBOW):} Predicts the target word given its surrounding context words.
    \item \textbf{skip-gram:} Predicts the surrounding context words given the target word.
\end{enumerate}

Both architectures learn word embeddings as a byproduct of solving these prediction tasks.

\subsubsection{Continuous Bag of Words (CBOW)}
\label{sec:nlp_continuous_bag_of_words_cbow_sub}

In CBOW, the model takes as input the context words surrounding a target word and tries to predict the target word itself. Formally, given a sequence of words \(\{w_1, w_2, \ldots, w_T\}\), and a context window size \(n\), the context for word \(w_t\) is
\[
\mathcal{C}_t = \{w_{t-n}, \ldots, w_{t-1}, w_{t+1}, \ldots, w_{t+n}\}.
\]

The CBOW model maximizes the probability
\[
p(w_t \mid \mathcal{C}_t),
\]
where the context words \(\mathcal{C}_t\) are represented as one-hot vectors and combined (e.g., averaged) to form the input.

\paragraph{Example:} Consider the sentence
\[
\text{``to buy an automatic car''}.
\]
If we want to learn the embedding for the word \emph{automatic}, the context might be \(\{\text{to}, \text{buy}, \text{an}, \text{car}\}\). The CBOW model uses these context words to predict \emph{automatic}.

\subsubsection{skip-gram}
\label{sec:nlp_skip_gram_sub}

Conversely, the skip-gram model takes the target word as input and tries to predict each of the context words. It maximizes
\[
\prod_{w_c \in \mathcal{C}_t} p(w_c \mid w_t).
\]
The product makes the modeling assumption explicit: every context word within the window contributes a likelihood factor.
In practice we maximize the sum of log-probabilities
\(\sum_{w_c \in \mathcal{C}_t} \log p(w_c \mid w_t)\) so that each neighboring prediction provides an additive gradient signal.

This approach tends to perform better on infrequent words and captures more detailed semantic relationships.

\subsection{Mathematical Formulation of CBOW}
\label{sec:nlp_mathematical_formulation_of_cbow}

Let the vocabulary size be \(V\), and embedding dimension be \(d\). Define the embedding matrix \(\mathbf{W} \in \mathbb{R}^{V \times d}\), where the \(i\)-th row \(\mathbf{v}_i\) is the embedding vector for word \(w_i\). \emph{Convention: we treat embeddings as \textbf{rows}; one-hot words index rows via $\mathbf{x}\mathbf{W}$ (row lookup).}

% Chapter 13 (continued)

\subsection{Neural Network Architecture for Word Embeddings}
\label{sec:nlp_neural_network_architecture_for_word_embeddings}

Consider a corpus with vocabulary size \( V = 10,000 \) words. Our goal is to learn a dense vector representation (embedding) for each word in this vocabulary. We denote the dimensionality of the embedding space as \( d = 300 \).

\paragraph{Input Representation}

Each input word is represented as a one-hot row vector \(\mathbf{x} \in \mathbb{R}^{1\times V}\), where only one element is 1 (corresponding to the word index) and the rest are 0. For example, if the word "want" is the \(i\)-th word in the vocabulary, then \(\mathbf{x}_i = 1\) and \(\mathbf{x}_j = 0\) for \(j \neq i\).

\paragraph{Network Structure}

We consider a simple feedforward neural network with:

\begin{itemize}
    \item An input layer of size \(V\) (one-hot encoded words).
    \item A hidden layer of size \(d = 300\), which will serve as the embedding layer.
    \item An output layer of size \(V\), which predicts the target word.
\end{itemize}

The weight matrix between the input and hidden layer is denoted as
\[
W \in \mathbb{R}^{V \times d}.
\]
Each row \(W_{i,:}\) corresponds to the embedding vector of the \(i\)-th word.

\paragraph{Forward Pass}

Given an input word represented by \(\mathbf{x}\), the hidden layer output \(\mathbf{h} \in \mathbb{R}^d\) is computed as:
\begin{align}
    \mathbf{h} &= \mathbf{x} W, \label{eq:hidden_layer}
\end{align}
where \(\mathbf{x}\) is a \(1 \times V\) vector and \(W\) is \(V \times d\), resulting in \(\mathbf{h}\) of size \(1 \times d\).

Because \(\mathbf{x}\) is one-hot, this operation simply selects the row of \(W\) corresponding to the input word, i.e., the embedding vector for that word.

\paragraph{Output Layer}

The hidden layer output \(\mathbf{h}\) is then multiplied by an output matrix \(W_{\text{out}} \in \mathbb{R}^{d \times V}\) to produce the output logits \(\mathbf{z} \in \mathbb{R}^V\):
\begin{align}
    \mathbf{z} &= \mathbf{h} W_{\text{out}}. \label{eq:output_logits}
\end{align}

These logits are then passed through a softmax function to produce a probability distribution over the vocabulary:
\begin{align}
    \hat{\mathbf{y}}_j = \frac{\exp(z_j)}{\sum_{k=1}^V \exp(z_k)}, \quad j=1,\ldots,V. \label{eq:softmax}
\end{align}

\paragraph{Training Objective}

The target output \(\mathbf{y}\) is also a one-hot vector corresponding to the word we want to predict (e.g., the word "automatic"). The training objective is to minimize the cross\hyp{}entropy loss between the predicted distribution \(\hat{\mathbf{y}}\) and the target \(\mathbf{y}\):
\begin{align}
    \mathcal{L} = - \sum_{j=1}^V y_j \log \hat{y}_j. \label{eq:cross_entropy}
\end{align}

\paragraph{Backpropagation and Weight Updates}

During training, the weights \(\mathbf{W}\) and \(\mathbf{W}_{\text{out}}\) are updated via backpropagation to minimize \(\mathcal{L}\). This process adjusts the embeddings in \(\mathbf{W}\) so that words appearing in similar contexts have similar vector representations.

\subsection{Context window and sequential input}
\label{sec:nlp_context_window_and_sequential_input}

Suppose we use a context window of size 4 words surrounding the target word. For example, to predict the word ``automatic'' in the phrase ``to buy an automatic car,'' the context words are
\[
\text{to}, \quad \text{buy}, \quad \text{an}, \quad \text{car}.
\]

Each context word is represented as a one-hot vector and fed into the network. Each one-hot vector shares the same embedding matrix \(\mathbf{W}\); multiplying \(\mathbf{x}\mathbf{W}\) is an efficient row lookup because \(\mathbf{x}\) is one-hot.

\paragraph{Input Sequence Processing}

The same embedding lookup is applied to each context token. If \(\mathbf{x}_{t+j}\) is the one-hot vector for the word at position \(t+j\), then its embedding is
\[
\mathbf{h}_{t+j} = \mathbf{x}_{t+j}\mathbf{W}.
\]

The hidden representations \(\mathbf{h}^{(i)}\) for each context word can be combined (e.g., concatenated or averaged) before passing to the output layer to predict the target word.

\paragraph{Dimensionality and Sparsity}

Note that the input vectors \(\mathbf{x}_{t+j}\) are extremely sparse (one-hot), and the embedding matrix \(\mathbf{W}\) is large (\(10{,}000 \times 300\), for example). However, the multiplication \(\mathbf{x}_{t+j}\mathbf{W}\) is efficient because it selects a single row of \(\mathbf{W}\) per input word.

\subsection{Interpretation of the Weight Matrix \texorpdfstring{\(W\)}{W}}
\label{sec:nlp_interpretation_of_the_weight_matrix_w_w}

The matrix \(\mathbf{W}\) can be interpreted as a lookup table: the \(i\)-th row is the embedding for word \(w_i\), and \(\mathbf{x}\mathbf{W}\) (with one-hot \(\mathbf{x}\)) selects that row directly.

% Chapter 13 (continued)

\subsection{Word Embeddings: Continuous Bag of Words (CBOW) and skip-gram models}
\label{sec:nlp_word_embeddings_continuous_bag_of_words_cbow_and_skip_gram_models}

Recall from the previous discussion that word embeddings are dense vector representations of words learned from large corpora, capturing semantic and syntactic properties. Two foundational models for learning such embeddings are the Continuous Bag of Words (CBOW) and skip-gram models, both introduced in the Word2Vec framework.

\subsubsection{Continuous Bag of Words (CBOW)}
\label{sec:nlp_continuous_bag_of_words_cbow_sub_2}

In CBOW, the objective is to predict a target word given its surrounding context words. Formally, given a sequence of words \( w_1, w_2, \ldots, w_T \), and a context window of size \( c \), the model predicts the word \( w_t \) based on the context words \( \{ w_{t-c}, \ldots, w_{t-1}, w_{t+1}, \ldots, w_{t+c} \} \).

The input to the model is a one-hot encoded vector representing the context words. Since each word is represented as a one-hot vector of dimension \( V \) (the vocabulary size), the input is a sparse vector with a single 1 and zeros elsewhere. The embedding matrix \( \mathbf{W} \in \mathbb{R}^{V \times d} \) maps each word to a \( d \)-dimensional dense vector (embedding).

The CBOW model computes the average of the embeddings of the context words using an \emph{input} embedding matrix \(\mathbf{W}\) and predicts with a separate \emph{output} embedding matrix \(\mathbf{W}_{\text{out}}\):

\begin{align}
\mathbf{h} = \frac{1}{2c} \sum_{\substack{-c \leq j \leq c \\ j \neq 0}} \mathbf{x}_{t+j}\mathbf{W}
    \label{eq:auto:lecture_8_part_i:1}
\end{align}

where \( \mathbf{x}_{t+j} \) is the one-hot vector for the context word at position \( t+j \), and \(c\) denotes the \emph{half-window} size (there are \(2c\) context words around \(w_t\) when the document is long enough).

This hidden representation \( \mathbf{h} \) is then used to predict the target word \( w_t \) via a softmax layer:

\begin{align}
P(w_t \mid \text{context}) = \frac{\exp(\mathbf{h}\,\mathbf{u}_{w_t})}{\sum_{w=1}^V \exp(\mathbf{h}\,\mathbf{u}_w)}
\label{eq:cbow-softmax}
\end{align}

where \( \mathbf{u}_w \) is the output vector corresponding to word \( w \).
It is useful to think of the set of output vectors as the \emph{columns} of a second matrix \(\mathbf{W}_{\text{out}} \in \mathbb{R}^{d \times V}\); although \(\mathbf{W}_{\text{out}}\) often starts as a copy of \(\mathbf{W}^\top\), the two sets of embeddings are optimized independently during training. Many modern implementations optionally \emph{tie} these matrices so that \(\mathbf{W}_{\text{out}}=\mathbf{W}^\top\), reducing parameters and encouraging symmetry between input and output spaces.

Training proceeds by maximizing the log-likelihood over the corpus, adjusting the embedding matrix \( \mathbf{W} \) and output vectors \( \mathbf{u}_w \) to improve prediction accuracy. After sufficient training, the rows of \( \mathbf{W} \) serve as the learned word embeddings.

\paragraph{Key Insight:} Because the input vectors are one-hot encoded, the multiplication \( \mathbf{x}_{t+j}\mathbf{W} \) simply selects the \emph{row} of $\mathbf{W}$ corresponding to the context word \( w_{t+j} \). This makes the embedding matrix \( \mathbf{W} \) a lookup table of word features.

\subsubsection{skip-gram model}
\label{sec:nlp_skip_gram_model_sub}

The skip-gram model reverses the CBOW objective: it uses the current word to predict its surrounding context words. Given a center word \( w_t \), the model aims to maximize the probability of each context word \( w_{t+j} \) within a window \( c \):

\begin{align}
\prod_{\substack{-c \leq j \leq c \\ j \neq 0}} P(w_{t+j} \mid w_t)
    \label{eq:auto:lecture_8_part_i:2}
\end{align}

The input is the one-hot vector \( \mathbf{x}_t \) representing the center word, which is projected into the embedding space via the same input embedding matrix \( \mathbf{W} \in \mathbb{R}^{V \times d} \):

\begin{align}
\mathbf{h} = \mathbf{x}_t\mathbf{W}
    \label{eq:auto:lecture_8_part_i:3}
\end{align}

Each context word \( w_{t+j} \) is predicted by applying a softmax over the output vectors (again using the output-embedding matrix \(\mathbf{W}_{\text{out}}\)):

\begin{align}
P(w_{t+j} \mid w_t) = \frac{\exp(\mathbf{h}\,\mathbf{u}_{w_{t+j}})}{\sum_{w=1}^V \exp(\mathbf{h}\,\mathbf{u}_w)}
\label{eq:skipgram-softmax}
\end{align}

where \( \mathbf{u}_w \) are the output vectors as before.

\paragraph{Training Objective:} Maximize the log-likelihood of the context words given the center word over the entire corpus. Compare to the RNN language-model objective in \Cref{chap:rnn}: both predict nearby tokens, but here the context is a fixed sliding window rather than a learned recurrent state.

\paragraph{Interpretation:} The skip-gram model learns embeddings such that words appearing in similar contexts have similar vector representations.

\subsubsection{Computational Challenges: Softmax Normalization}
\label{sec:nlp_computational_challenges_softmax_normalization_sub}

Both CBOW and skip-gram models require computing the softmax normalization over the entire vocabulary \( V \), which can be very large (e.g., \( V = 10,000 \) or more). The denominator in equations \eqref{eq:cbow-softmax} and \eqref{eq:skipgram-softmax} involves summing exponentials over all vocabulary words:

\begin{align}
Z = \sum_{w=1}^V \exp(\mathbf{h}\,\mathbf{u}_w)
    \label{eq:auto:lecture_8_part_i:4}
\end{align}

\begin{tcolorbox}[summarybox,title={Recipe: skip-gram with negative sampling}]
Preprocess: tokenize (BPE/WordPiece or whitespace), lowercase if appropriate, drop rare words below a cutoff, subsample frequent words with \(t\approx 10^{-5}\).\\
Hyperparameters: window \(c=2\)--\(5\) (often dynamic/symmetric), embedding dim \(d=100\)--\(300\), negatives \(k=5\)--\(20\), unigram noise \(P_n(w)\propto f(w)^{0.75}\), LR on the order of \(10^{-3}\)--\(10^{-2}\).\\
Per-positive loss (one context word): \(-\log \sigma(\mathbf{h}\,\mathbf{u}_{\text{pos}}) - \sum_{i=1}^k \log \sigma(-\mathbf{h}\,\mathbf{u}_{\text{neg},i})\); complexity \(O(k)\) vs. \(O(V)\) for full softmax.
\end{tcolorbox}

This is computationally expensive, especially when training on large corpora.

\paragraph{Approximate Solutions:} To address this, several approximation techniques have been proposed:

\begin{itemize}
    \item \textbf{Hierarchical softmax:} factor the softmax into a tree so each update touches only a \(\log V\) path.
    \item \textbf{Negative sampling:} replace the full softmax with \(k\) binary logistic losses against sampled ``noise'' words.
\end{itemize}
\subsection{Efficient Training of Word Embeddings: Hierarchical Softmax and Negative Sampling}
\label{sec:nlp_efficient_training_of_word_embeddings_hierarchical_softmax_and_negative_sampling}

Recall from the previous discussion that computing the full softmax over a large vocabulary is computationally expensive. Specifically, given an input word, calculating the probability distribution over all possible output words in the vocabulary requires a normalization over potentially millions of terms, which is prohibitive in practice.

There are two primary strategies to address this computational bottleneck:

\paragraph{1. Hierarchical Softmax}
Hierarchical softmax replaces the flat softmax layer with a binary tree representation of the vocabulary. Each word corresponds to a leaf node, and the probability of a word is decomposed into the probabilities of traversing the path from the root to that leaf. This reduces the computational complexity from \(O(V)\) to \(O(\log V)\), where \(V\) is the vocabulary size.

The key idea is to organize words so that frequent words have shorter paths, thus further improving efficiency. During training, only the nodes along the path to the target word are updated, avoiding the need to compute scores for all words.

\paragraph{2. Negative Sampling}
Negative sampling is an alternative approximation that simplifies the objective by transforming the multi-class classification problem into multiple binary classification problems.

\begin{itemize}
    \item For each observed word-context pair \((w, c)\), the model aims to distinguish the true pair from randomly sampled \emph{negative} pairs \((w, c')\), where \(c'\) is drawn from a noise distribution.
    \item Instead of computing probabilities over the entire vocabulary, the model only updates parameters for the positive pair and a small number of negative samples.
\end{itemize}

\paragraph{Example:} Consider the sentence:
\[
\text{``I want to buy a big brick house in the city.''}
\]
Suppose the context word is \texttt{brick}. The true target word is \texttt{house}. Negative samples might be \texttt{lion}, \texttt{bake}, or \texttt{big} (although \texttt{big} appears in the sentence, it can still be sampled as a negative example depending on the sampling strategy).
\noindent Negative draws occasionally colliding with real context words is harmless. The associated losses simply push the model to separate the sampled pair unless the data provide strong evidence to the contrary.

\paragraph{Training Objective with Negative Sampling}

Define the logistic regression classifier that, given an input word vector \(\mathbf{v}_w\) and an output word vector \(\mathbf{v}'_c\), predicts whether the pair \((w, c)\) is observed (label 1) or a negative sample (label 0).

The probability that the pair is observed is modeled as:
\begin{equation}
    p(D=1 \mid w, c) = \sigma(\mathbf{v}_w\,\mathbf{v}'_c) \label{eq:neg-sample-prob}
\end{equation}
where \(\sigma(x) = \frac{1}{1 + e^{-x}}\) is the sigmoid function.

The training objective for one positive pair \((w, c)\) and \(k\) negative samples \(\{c_1', \ldots, c_k'\}\) is:
\begin{equation}
    \log \sigma(\mathbf{v}_w\,\mathbf{v}'_c) + \sum_{i=1}^k \log \sigma\big(-\mathbf{v}_w\,\mathbf{v}'_{c_i'}\big) \label{eq:neg-sample-loss}
\end{equation}
where each \(c_i'\) is drawn independently from the noise distribution \(P_n(c)\). A widely used practical choice is \(P_n(w)\propto f(w)^{0.75}\), where \(f(w)\) is the empirical unigram frequency; this slightly downweights extremely frequent words while still sampling them often enough to learn robust embeddings.

\paragraph{Tiny worked example (skip-gram with \(k=2\)).} Suppose the center word is \texttt{brick} with embedding \(\mathbf{v}_{\text{brick}}\), the true context is \texttt{house} (\(\mathbf{v}'_{\text{house}}\)), and we sample two negatives \texttt{lion}, \texttt{bake}. We compute:
\[
L = \log\sigma(\mathbf{v}_{\text{brick}}\,\mathbf{v}'_{\text{house}}) + \log\sigma(-\mathbf{v}_{\text{brick}}\,\mathbf{v}'_{\text{lion}}) + \log\sigma(-\mathbf{v}_{\text{brick}}\,\mathbf{v}'_{\text{bake}}).
\]
Gradients push \(\mathbf{v}_{\text{brick}}\) closer to \(\mathbf{v}'_{\text{house}}\) (if the dot product is too small) and simultaneously push it away from \(\mathbf{v}'_{\text{lion}}\) and \(\mathbf{v}'_{\text{bake}}\). Only these three context vectors update this step, so the cost stays \(O(k)\) regardless of vocabulary size.

\paragraph{Interpretation:} The model learns to assign high similarity scores to true word-context pairs and low similarity scores to randomly sampled pairs, effectively learning meaningful embeddings without computing the full softmax. The expectation over the noise distribution is estimated by the empirical average across the \(k\) sampled negatives in \eqref{eq:neg-sample-loss}. Unlike noise-contrastive estimation (NCE), negative sampling is not a consistent estimator of the normalized softmax probabilities; it is best viewed as a task-specific approximation that yields high-quality embeddings rather than calibrated class posteriors.

\paragraph{Backpropagation:} The gradients are computed only for the positive pair and the sampled negative pairs, drastically reducing computation.

\paragraph{Connection to PMI (Levy \& Goldberg).} A useful theoretical lens relates skip-gram with negative sampling (SGNS) to pointwise mutual information (PMI). Under common choices of windowing and negative sampling distribution, SGNS implicitly factorizes a \emph{shifted} PMI matrix such that inner products approximate:
\[
    \mathbf{v}_i\,\mathbf{u}_k \;\approx\; \operatorname{PMI}(i,k) - \log k,\quad \text{where}\quad \operatorname{PMI}(i,k)=\log\frac{P(i,k)}{P(i)P(k)}.
\]
This connection helps explain why SGNS and GloVe often yield similar geometric regularities despite different training objectives: both methods recover statistics of co-occurrence up to monotone transformations and weighting.

\subsection{Local Context vs. Global Matrix Factorization Approaches}
\label{sec:nlp_local_context_vs_global_matrix_factorization_approaches}

Word embedding methods can be broadly categorized into two classes based on how they utilize context information:

\paragraph{1. Local Context Window Methods}

These methods focus on the immediate context of a word within a fixed-size window. Examples include:

\begin{itemize}
    \item Continuous Bag-of-Words (CBOW)
    \item skip-gram
\end{itemize}

They learn embeddings by predicting a word given its neighbors (CBOW) or predicting neighbors given a word (skip-gram). These methods are computationally efficient and capture syntactic and semantic relationships based on local co-occurrence patterns.

\paragraph{2. Global Matrix Factorization Methods}

These methods consider the entire corpus to build a global co-occurrence matrix \(X\), where each entry \(X_{ij}\) counts how often word \(i\) co-occurs with word \(j\) across the corpus.

\begin{itemize}
    \item Latent Semantic Analysis (LSA) is an early example, which applies singular value decomposition (SVD) to the co-occurrence matrix.
    \item More recent methods include GloVe (Global Vectors), which factorizes a weighted log-count matrix \citep{Pennington2014}.
\end{itemize}

\paragraph{Example: Co-occurrence Matrix}

Suppose the vocabulary size is \(V\). The co-occurrence matrix \(X \in \mathbb{R}^{V \times V}\) is defined as:
\[
X_{ij} = \text{number of times word } i \text{ appears in the context of word } j
\]

This matrix is \emph{sparse} and \emph{large} (especially when $V$ runs into the hundreds of thousands), so storing it explicitly or factorizing it naively can be computationally expensive.

\subsection{Global Word Vector Representations via Co-occurrence Statistics}
\label{sec:nlp_global_word_vector_representations_via_co_occurrence_statistics}

Recall that our goal is to obtain a global vector representation for words, capturing semantic relationships beyond simple one-hot encodings. Instead of encoding words individually, we leverage \emph{co-occurrence} statistics of word pairs within a corpus to build richer embeddings.

\paragraph{Setup:} Consider two words \( w_i \) and \( w_j \) appearing in some context window within a text corpus. We are interested in modeling the \emph{co-occurrence} of these words, possibly mediated by a third \emph{context} word \( w_k \). For example, in the phrase ``big historic castle,'' the words ``big'' and ``historic'' are targets, and ``castle'' can be a context word connecting them.

\paragraph{Notation:}
\begin{itemize}
    \item Plain symbols \(w_i, w_j, w_k\) denote lexical items drawn from the vocabulary.
    \item Bold symbols denote vectors: \(\mathbf{v}_i\) is the embedding of target word \(w_i\) and \(\mathbf{u}_k\) the embedding of context word \(w_k\).
    \item \( X_{ik} \) counts how often \(w_i\) and \(w_k\) co-occur within the chosen context window, and \( X_i = \sum_k X_{ik} \) is the total number of context observations for \(w_i\).
\end{itemize}

\paragraph{Goal:} Define a function \( f \) that relates the co-occurrence statistics of the word pairs and context words to a scalar quantity representing their semantic association.

\paragraph{Visualization.} Projecting the learned vectors onto two principal components typically reveals well-separated semantic clusters. \Cref{fig:lec8-embedding-clusters} highlights how gendered titles, fruits, and locations occupy distinct regions, reinforcing that co-occurrence-driven training captures rich lexical structure.

\begin{figure}[t]
    \centering
    \includegraphics[width=0.82\linewidth]{lec14_analogy_geometry}
    % Avoid inline math in captions; it wraps poorly in some EPUB renderers.
    \caption{Analogy geometry in embedding space. The classic offset ``v(king) - v(man) + v(woman) approx v(queen)'' forms a parallelogram; a similar gender direction also moves ``doctor'' toward ``nurse.'' Visualizing these displacement vectors (solid vs. dashed) makes the shared relational direction explicit. Points are shown after a 2D PCA projection, so directions are approximate rather than exact. Use this when sanity-checking whether embedding offsets capture a consistent relation (and when they expose bias directions).}
    \label{fig:lec8-embedding-clusters}
\end{figure}

\subsubsection{Modeling Co-occurrence Probabilities}
\label{sec:nlp_modeling_co_occurrence_probabilities_sub}

We start by considering the conditional probability of observing a context word \( w_k \) given a target word \( w_i \):
\begin{equation}
    P(k|i) = \frac{X_{ik}}{X_i}.
    \label{eq:cond_prob}
\end{equation}

This probability captures how likely the context word \( w_k \) appears near the target word \( w_i \).

\paragraph{Relating to word vectors:} Suppose each word \( w_i \) is represented by a vector \( \mathbf{v}_i \in \mathbb{R}^d \). We want to model the relationship between \( \mathbf{v}_i \), \( \mathbf{u}_k \), and the co-occurrence probability \( P(k|i) \).

A natural assumption is that the co-occurrence probability can be modeled as an exponential function of the inner product of the corresponding word vectors:
\begin{equation}
    P(k|i) \propto \exp\left( \mathbf{v}_i\,\mathbf{u}_k \right).
    \label{eq:exp_model}
\end{equation}
More explicitly, we can write a normalized model
\[
    P(k|i) \approx \frac{1}{Z_i} \exp\big( \mathbf{v}_i\,\mathbf{u}_k + b_i + b_k \big),
\]
with partition function
\[
    Z_i = \sum_{k'} \exp\big( \mathbf{v}_i\,\mathbf{u}_{k'} + b_i + b_{k'} \big).
\]

Taking logarithms on both sides and absorbing the (word-specific) normalizer into the biases gives the approximate relation
\begin{equation}
    \log P(k|i) \approx \mathbf{v}_i\,\mathbf{u}_k + b_i + b_k,
    \label{eq:log_prob}
\end{equation}
where \( b_i \) and \( b_k \) are bias terms associated with words \( w_i \) and \( w_k \), respectively. These biases account for the overall frequency or importance of each word while implicitly capturing the effect of \(Z_i\).

\paragraph{Derivation:} Starting from the co-occurrence counts,
\begin{align}
    \log X_{ik} - \log X_i &= \log \frac{X_{ik}}{X_i} = \log P(k|i) \\
    &\approx \mathbf{v}_i\,\mathbf{u}_k + b_i + b_k.
    \label{eq:log_xik}
\end{align}

This equation suggests that the log co-occurrence counts can be approximated by a bilinear form plus biases.

\subsubsection{Optimization Objective}
\label{sec:nlp_optimization_objective_sub}

Given the corpus co-occurrence matrix \( X = [X_{ik}] \), our goal is to find word vectors \( \mathbf{v}_i \), \( \mathbf{u}_k \) and biases \( b_i, b_k \) that minimize the reconstruction error:
\begin{equation}
    J = \sum_{i,k} f(X_{ik}) \left( \mathbf{v}_i\,\mathbf{u}_k + b_i + b_k - \log X_{ik} \right)^2,
\label{eq:auto_nlp_be1df77f53}
\end{equation}
where \( f \) is a weighting function that controls the influence of each co-occurrence pair.

\paragraph{Why weighting?} Many entries \( X_{ik} \) are zero or very small, which can cause numerical instability or dominate the objective. The function \( f \) is designed to:
\begin{itemize}
    \item Downweight rare co-occurrences (small \( X_{ik} \)) to avoid overfitting noise.
    \item Possibly cap the influence of very frequent co-occurrences to prevent them from dominating.
\end{itemize}

A typical choice for \( f \) is:
\begin{equation}
    f(x) = \begin{cases}
    \left(\frac{x}{x_{\max}}\right)^\alpha & \text{if } x < x_{\max}, \\
    1 & \text{otherwise},
    \end{cases}
    \label{eq:weighting_function}
\end{equation}
where \( \alpha \in (0,1) \) and \( x_{\max} \) is a cutoff parameter.

\subsubsection{Interpretation and Remarks}
\label{sec:nlp_interpretation_and_remarks_sub}

\subsection{Finalizing the Word Embedding Derivations}
\label{sec:nlp_finalizing_the_word_embedding_derivations}

In the previous sections, we explored the formulation of word embeddings through co-occurrence statistics and matrix factorization approaches. We now conclude the derivations and clarify the role of bias terms and optimization strategies.

Recall the key equation relating the word vectors \( \mathbf{v}_i \) and context vectors \( \mathbf{u}_k \) to the co-occurrence counts \( x_{ik} \):
\begin{equation}
    \mathbf{v}_i\,\mathbf{u}_k + b_i + b_k = \log x_{ik},
    \label{eq:cooccurrence_log}
\end{equation}
where \( b_i \) and \( b_k \) are bias terms associated with the word and context, respectively.

\paragraph{Symmetry and Bias Terms}

Initially, two separate bias terms \( b_i \) and \( b_k \) were introduced to account for asymmetries in the data. However, it is often possible to simplify the model by combining or eliminating one of the biases without loss of generality. This is because the biases can absorb constant shifts in the embeddings, and the key information lies in the relative positions of the vectors. In practice we keep both biases so that very frequent terms (e.g., stop words) can learn large offsets while rarer words keep their dot products numerically stable.

Hence, the equation can be rewritten as
\begin{equation}
    \mathbf{v}_i\,\mathbf{u}_k = \log x_{ik} - b_i - b_k.
    \label{eq:cooccurrence_log_bias}
\end{equation}

In practice, the biases \( b_i \) and \( b_k \) are learned jointly with the embeddings to best fit the observed co-occurrence statistics.

\paragraph{Objective Function and Optimization}

Let the target embeddings be rows \( \mathbf{v}_i \in \mathbb{R}^{1\times d} \), the context embeddings be columns \( \mathbf{u}_k \in \mathbb{R}^{d\times 1} \), and the biases scalars \( b_i, b_k \in \mathbb{R} \). The scalar score \( \mathbf{v}_i\,\mathbf{u}_k \) therefore measures the alignment between the target and context embeddings. The goal is to find \( \{\mathbf{v}_i, \mathbf{u}_k, b_i, b_k\} \) that minimize the reconstruction error of the log co-occurrence matrix. Because raw counts span several orders of magnitude, the loss must behave like plain least squares for large \(x_{ik}\) yet dampen the influence of very small counts. Enforcing the limits \(f(x) \to 0\) as \(x \to 0\) and \(f(x) \to 1\) for \(x \ge x_{\max}\) yields the weighting scheme used by GloVe. The final weighted least-squares loss is
\begin{equation}
    J = \sum_{i=1}^{V}\sum_{k=1}^{V} f(x_{ik}) \left( \mathbf{v}_i\,\mathbf{u}_k + b_i + b_k - \log x_{ik} \right)^2,
    \label{eq:objective}
\end{equation}
where \( f(x) \) is the weighting function that downweights rare (or extremely common) co-occurrences to improve robustness. GloVe, for instance, uses the piecewise definition
\begin{equation}
    f(x) = \begin{cases}
        \left(\frac{x}{x_{\max}}\right)^\alpha & \text{if } x < x_{\max}, \\
        1 & \text{otherwise},
    \end{cases}
    \qquad 0 < \alpha \leq 1,
    \label{eq:glove_weight}
\end{equation}
so that very small counts contribute little to the loss while still allowing moderately frequent pairs to influence the fit. In the original paper, practical defaults such as \(\alpha \approx 0.75\) and \(x_{\max}\approx 100\) were found to work well across a range of corpora \citep{Pennington2014}; keeping those guardrails explicit also explains why the same weighting recipe keeps reappearing in derived models.

\paragraph{Singular Value Decomposition (SVD) Connection}

One approach to solving this problem is to perform a low-rank approximation of the matrix \( \log X \), where \( X = [x_{ik}] \) is the co-occurrence matrix and the logarithm is applied elementwise (with small smoothing constants, e.g., \(\epsilon = 10^{-8}\), added to avoid \(\log 0\)). The singular value decomposition (SVD) provides a principled method to find such a factorization:
\begin{equation}
    \log X \approx U_r \Sigma_r V_r^\top,
    \label{eq:svd}
\end{equation}
where \(U_r \in \mathbb{R}^{V \times r}\) and \(V_r \in \mathbb{R}^{V \times r}\) contain the top-\(r\) singular vectors (for the desired embedding dimension \(d=r\)), and \(\Sigma_r \in \mathbb{R}^{r \times r}\) is a diagonal matrix of the corresponding singular values. The truncation rank \(r\), often between 100 and 300 in practice, acts exactly like the embedding dimensionality knob in neural models.

By setting
\[
\mathbf{v}_i = (U_r)_i \Sigma_r^{1/2}, \quad \mathbf{u}_k = (V_r)_k \Sigma_r^{1/2},
\]
we obtain embeddings that approximate the log co-occurrence matrix in a least-squares sense.

\paragraph{Interpretation and Limitations}

While SVD provides a closed-form solution, it does not explicitly model the bias terms \( b_i, b_k \) or the weighting function \( f(x) \). Those additional degrees of freedom allow gradient-based methods such as GloVe to better match empirical co-occurrence ratios. Biases soak up unigram frequency effects while the weighting function prevents very noisy counts from dominating the fit.

\subsection{Bias in Natural Language Processing}
\label{sec:nlp_bias_in_natural_language_processing}

An important consideration in word embedding models is the presence of bias inherited from the training corpora. Since embeddings are learned from co-occurrence patterns in text, they reflect the statistical properties of the language data, including cultural and societal biases.

\paragraph{Sources of Bias}

- \textbf{Cultural Bias:} Text corpora often contain stereotypes or skewed representations of gender, ethnicity, and other social categories (e.g., news archives that associate ``nurse'' more frequently with women than men).
- \textbf{Historical Bias:} Older texts may reflect outdated or prejudiced views. Digitized literature from the 19th century, for instance, over-represents colonial perspectives.
- \textbf{Language-Specific Bias:} Different languages and dialects encode different cultural norms and connotations, such as grammatical gender or honorifics that privilege particular groups.

\paragraph{Impact on Embeddings}

For example, the well-known analogy
\[
\text{king} - \text{man} + \text{woman} \approx \text{queen}
\]
illustrates that many embeddings support approximately linear semantic relationships. However, these same linear structures can also reveal problematic biases, such as associating certain professions or attributes disproportionately with one gender or group.

\paragraph{Debiasing Techniques}

Addressing bias in embeddings is an active area of research. Techniques include:
- \textit{Post-processing} embeddings to remove bias directions (e.g., Hard Debiasing by \citet{Bolukbasi2016}).
- \textit{Data augmentation} to balance training corpora or swap gendered terms.
- \textit{Regularization} during training to penalize biased associations or enforce equality constraints.

\paragraph{Cross-Lingual Challenges}

When extending embeddings to multiple languages, biases can manifest differently due to linguistic and cultural variations. For example, gender is grammatically encoded in Romance languages, so direct projection of English debiasing techniques may still leave gendered artifacts in Spanish or French embeddings. Careful consideration is required to ensure fairness and robustness across languages.

\begin{tcolorbox}[summarybox,title={Practical bias checks}]
\begin{itemize}
    \item \textbf{Dataset audit:} Inspect class balance, label sources, and sensitive attributes; check for under-represented groups and spurious correlations (e.g., profession \(\leftrightarrow\) gender cues).
    \item \textbf{Calibration and reliability:} Evaluate calibration (ECE, reliability diagrams) overall and for key subgroups; severely miscalibrated models magnify harm when used for decision support.
    \item \textbf{Disaggregated evaluation:} Report accuracy, ROC/PR, and calibration metrics by subgroup rather than only aggregate scores; look for systematic performance gaps.
    \item \textbf{Mitigation loop:} Combine data interventions (rebalancing, augmentation) with model-side debiasing and re-evaluation; treat mitigation as an iterative, experiment-driven process.
\end{itemize}
\end{tcolorbox}

\begin{tcolorbox}[summarybox,title={Author's note: embeddings mix geometry and bias}]
Embedding spaces faithfully capture geometry (analogies, clusters) precisely because they also capture the biases present in the data. Treat every downstream use as a combination of those two facets: audit the geometry you need, but also audit the offset directions you would rather suppress. Vector arithmetic makes biases quantifiable, so put that ability to work before shipping a model.
\end{tcolorbox}

\subsection{Responsible deployment checklist}
\label{sec:nlp_responsible_deployment_checklist_appendix}
\begin{enumerate}
    \item \textbf{Purpose \& consent.} Document the use-case, decision stakes, and where humans remain in the loop; distinguish exploratory prototypes from production decision aids.
    \item \textbf{Data lineage \& licensing.} Track licenses for each corpus (newswire, Common Crawl, proprietary logs) and state whether downstream users may redistribute the embeddings or derived models.
    \item \textbf{Privacy \& security.} Scan corpora for PII, redact when necessary, and restrict raw-data access. When embeddings leave the lab, accompany them with an acceptable-use policy and redaction guarantees.
    \item \textbf{Monitoring.} Deploy subgroup-aware metrics, calibration checks, and toxicity filters in production; log drifts and institute retraining/rollback thresholds.
    \item \textbf{Documentation.} Ship a short ``model card'' summarizing intended uses, failure modes, and evaluation data so downstream teams can reason about fit-for-purpose decisions.
\end{enumerate}

\subsection{Contextual embeddings and transformers}
\label{sec:nlp_contextual_embeddings_and_transformers}

Static embeddings assign a single vector per word type, so polysemous words such as ``bank'' cannot adapt to their context. Transformer-based language models (e.g., BERT; \citealp{Devlin2019}) compute token representations conditioned on the entire sentence via multi-head self-attention, allowing each occurrence to carry a context-specific vector. The techniques developed in this chapter remain useful for lightweight models and as initialization, but modern NLP pipelines increasingly fine-tune contextual models to capture sentence-level nuance.

\subsection*{Wrap-up}

In this chapter, we concluded the derivation of word embedding models based on co-occurrence statistics, emphasizing the role of bias terms and optimization strategies such as singular value decomposition. We highlighted the importance of understanding and mitigating bias in natural language processing, as embeddings inherently reflect the cultural and societal context of their training data. These considerations are crucial for developing fair and effective language models.

\begin{tcolorbox}[summarybox,title={Key takeaways}]
\begin{itemize}
    \item Word embeddings are dense vectors learned from co-occurrence statistics (local windows or global matrices).
    \item Analogies and clustering arise from linear geometry in the embedding space.
    \item Bias in corpora propagates to embeddings; debiasing and careful datasets are important.
\end{itemize}

\medskip
\noindent\textbf{Minimum viable mastery.}
\begin{itemize}
    \item Explain how co-occurrence statistics induce geometry (dot products, cosine similarity, and linear offsets).
    \item Distinguish local-window objectives from global matrix factorization views and state when each is a good approximation.
    \item Identify at least one concrete bias test and one mitigation strategy, and articulate their limitations.
\end{itemize}

\noindent\textbf{Common pitfalls.}
\begin{itemize}
    \item Treating nearest neighbors as meaning rather than distributional evidence (polysemy and domain shift).
    \item Over-interpreting analogy accuracy without controlling for frequency and evaluation set construction.
    \item Applying debiasing as a post-hoc patch while ignoring corpus composition and labeling practices.
\end{itemize}
\end{tcolorbox}

\begin{tcolorbox}[summarybox,title={Exercises and lab ideas}]
\begin{itemize}
    \item Implement a minimal example from this chapter and visualize intermediate quantities (plots or diagnostics) to match the pseudocode.
    \item Stress-test a key hyperparameter or design choice discussed here and report the effect on validation performance or stability.
    \item Re-derive one core equation or update rule by hand and check it numerically against your implementation.
\end{itemize}

\medskip
\noindent\textbf{If you are skipping ahead.} You can treat embeddings as feature vectors for any downstream model, but the audit mindset matters: log your corpus choices and evaluation splits so that later ``fairness'' or calibration conclusions have context.
\end{tcolorbox}

\medskip
\paragraph{Where we head next.} \Cref{chap:transformers} scales the sequence toolkit by replacing recurrence with attention, so context can be accessed directly rather than compressed into a single state. Keep the same audit instincts as you read: tokenization choices, masking correctness, and evaluation protocol matter as much as architecture. After that, \Cref{chap:softcomp} pivots to soft computing (fuzzy logic and evolutionary ideas) as an alternative paradigm for reasoning under uncertainty.

\paragraph{References.} Full citations for works mentioned in this chapter appear in the book-wide bibliography.
\nocite{Mikolov2013,Pennington2014,LevyGoldberg2014,Devlin2019}

% Modern sequence modeling beyond RNNs
% Chapter 14
\section{Transformers: Attention-Based Sequence Modeling}\label{chap:transformers}

\Cref{chap:rnn} made the sequence problem explicit: stateful computation plus gradients that must flow across time. \Cref{chap:nlp} then supplied the representation layer that makes those sequence objectives practical. This chapter keeps those representations but loosens the ``one step at a time'' constraint: instead of marching through time, we let each position look around and pull in what it needs through attention.

\begin{tcolorbox}[summarybox, title={Learning Outcomes}]
After this chapter, you should be able to:
\begin{itemize}
  \item Explain the encoder--decoder bottleneck and how attention fixes it in sequence-to-sequence problems.
  \item Write scaled dot-product attention and multi-head attention, and interpret them as weighted averages.
  \item Distinguish self-attention from cross-attention and describe where each appears in a Transformer.
  \item Explain positional encodings and masking (padding/causal) in training and decoding.
  \item Describe a Transformer block (residual paths, layer norm, FFN) and common objectives (MLM/CLM) and families (BERT/GPT/encoder--decoder).
\end{itemize}
\end{tcolorbox}

\begin{tcolorbox}[summarybox, title={Design motif}]
Make information flow explicit. Attention is a controlled mixing operation; masks enforce which interactions are allowed; residual paths and normalization keep optimization stable as models deepen and context windows grow.
\end{tcolorbox}

\begin{tcolorbox}[perspectivebox, title={Author's note: models, world models, and language models (an opinionated lens)}]
A \emph{model} predicts outcomes from captured information. In the chapters so far, that ``outcome'' ranged
from a class label, to a time-step forecast, to a retrieved memory pattern.

A \emph{world model}, as I use the phrase here, is aspirational: it is the idea of a general-purpose internal
model that can represent situations and dynamics well enough to support planning, counterplanning, and
``what-if'' reasoning. Whether we can build such systems reliably is part of the larger research program.

Language models are a simplified and very constrained training interface to this aspiration: at training time
the next action is a token predicted from previous tokens. My opinion is that this setup can \emph{suggest}
a kind of internal world structure (because the output must remain internally and externally consistent to be
useful), but it does not \emph{guarantee} that a faithful world model has formed. Fluency is not proof of
understanding; it is a behavior you still have to audit.
\end{tcolorbox}

\subsection{From encoder--decoder bottlenecks to attention}
\label{sec:transformers_why_transformers_after_rnns}
Sequence-to-sequence (seq2seq) problems have a natural story: read an input sequence and produce an output sequence. Translation is the canonical example. In the classical encoder--decoder picture, the encoder reads the source sentence and produces a representation; the decoder then uses that representation to generate the target sentence token by token.

We will keep the term \emph{token} from \Cref{chap:nlp}: a token is a discrete symbol index that gets mapped to a vector. In text it might be a word or subword; in other sequence problems it can be any event ID you embed into a feature vector.

The practical bottleneck is the fixed-vector squeeze. If the decoder is only given one summary vector, it is forced to carry \emph{everything} about the source through time, even when the next output token only depends on a small part of the input. Attention is the engineering fix: at each decoding step, compute a weighted average of the encoder states and let that mixture act as the context for the next prediction. In other words, the decoder is allowed to look back at the encoder's ``memory'' and ask which source positions matter \emph{right now}.

Transformers \citep{Vaswani2017} take that idea seriously and push it further. They remove recurrence and make ``look around and mix'' the core operation inside a layer. This allows parallel computation across positions and makes long-range interactions a first-class design choice rather than a side effect of how well information survives through many recurrent steps.

\paragraph{Seq2seq with attention (cross-attention).}
In an encoder--decoder RNN, the encoder produces a sequence of hidden states \(\{\mathbf{h}_j\}_{j=1}^{S}\) from the input \(\{\mathbf{x}_j\}\), and the decoder produces a sequence of states \(\{\mathbf{s}_t\}_{t=1}^{T}\) while generating the output \(\{y_t\}\). A convenient probabilistic view of translation is
\begin{equation}
    p(\mathbf{y}\mid \mathbf{x})
    = \prod_{t=1}^{T} p\!\left(y_t \mid y_{<t}, \mathbf{x}\right).
    \label{eq:transformers_seq2seq_factorization}
\end{equation}
The ``one-vector bottleneck'' appears when the decoder is forced to rely on a single summary of the entire input. Attention replaces that with a \emph{per-step context}:
\begin{align}
    e_{t j} &= a(\mathbf{s}_{t-1}, \mathbf{h}_j), \label{eq:transformers_attn_score_seq2seq}\\
    \alpha_{t j} &= \operatorname{softmax}_j(e_{t j}), \label{eq:transformers_attn_weights_seq2seq}\\
    \mathbf{c}_t &= \sum_{j=1}^{S} \alpha_{t j}\,\mathbf{h}_j. \label{eq:transformers_context_seq2seq}
\end{align}
Here \(a(\cdot,\cdot)\) is any differentiable scoring function (a dot product, a bilinear form, or a small MLP) and \(\mathbf{c}_t\) is the context the decoder uses when predicting \(y_t\). That is the core move: different output positions can ``look back'' at different parts of the source. This is the bridge from encoder--decoder RNNs to Transformers: the attention math stays; what changes is that Transformers build the states \(\mathbf{h}_j\) and \(\mathbf{s}_t\) without recurrence.

\subsection{Scaled Dot-Product Attention}
\label{sec:transformers_scaled_dot_product_attention}
\begin{tcolorbox}[summarybox, title={Author's note: attention is a weighted average}]
I like to read attention as a weighted average over candidate pieces of information. A query asks a question (what do I need?), keys advertise what each candidate is about (what do I contain?), and values carry the content that gets mixed. Similarity scores between queries and keys become nonnegative weights that sum to one; the output is the weighted sum of the value vectors.
\end{tcolorbox}

\begin{tcolorbox}[summarybox, title={A tiny analogy: keys, values, and weighted retrieval}]
Think of a tiny ``database'' with keys \(\{70,80\}\) and values \(\{1000,1500\}\). If your query is exactly \(70\), you retrieve \(1000\). If your query is \(75\), there is no exact match, so you can do a soft retrieval: score each key by a similarity such as
\[
s_i = \frac{1}{|k_i-q|+\epsilon},
\]
then normalize those scores so they sum to one and use them as weights on the values. Here \(|70-75|=|80-75|=5\), so \(s_1\approx s_2\approx 0.2\), the normalized weights are \((0.5,0.5)\), and the weighted average gives \(0.5\cdot 1000 + 0.5\cdot 1500 = 1250\).
\par\smallskip
This is only an intuition pump: Transformers do not use scalar keys like ``70.'' They learn vector keys and queries and score them in a learned feature space. But the weighted-average mechanism is exactly the same.
\end{tcolorbox}

At the per-position level, the computation is
\[
\mathbf{z}_i = \sum_{j} \alpha_{ij}\,\mathbf{v}_j,\qquad
\alpha_{ij} \ge 0,\quad \sum_j \alpha_{ij} = 1,
\]
Given query, key, value matrices \(\mathbf{Q} \in \mathbb{R}^{n_q \times d_k}\), \(\mathbf{K} \in \mathbb{R}^{n_k \times d_k}\), and \(\mathbf{V} \in \mathbb{R}^{n_k \times d_v}\), the basic attention operation is
\begin{equation}
    \operatorname{Attn}(\mathbf{Q}, \mathbf{K}, \mathbf{V})
    = \operatorname{softmax}\!\left(\frac{\mathbf{Q}\mathbf{K}^\top}{\sqrt{d_k}}\right) \mathbf{V}.
    \label{eq:transformers_scaled_dot_product_attention}
\end{equation}
Here \(n_q\) is the sequence length of the queries and \(n_k\) the sequence length of keys/values. We keep the same sequence-first convention used in \Cref{chap:rnn,chap:nlp}: rows index time positions (a ``token dimension'') and columns index features, while batch elements are processed independently. The \(1/\sqrt{d_k}\) factor stabilizes gradients by keeping logits in a reasonable range.

\paragraph{Where do \(Q,K,V\) come from?}
Start from token vectors stacked into a matrix \(\mathbf{X}\in\mathbb{R}^{n\times d_{\text{model}}}\) (one row per position). For a \emph{single head}, learned projections produce
\begin{equation}
    \mathbf{Q}=\mathbf{X}\mathbf{W}^Q,\qquad
    \mathbf{K}=\mathbf{X}\mathbf{W}^K,\qquad
    \mathbf{V}=\mathbf{X}\mathbf{W}^V,
    \label{eq:transformers_qkv_projections}
\end{equation}
with \(\mathbf{W}^Q,\mathbf{W}^K\in\mathbb{R}^{d_{\text{model}}\times d_k}\) and \(\mathbf{W}^V\in\mathbb{R}^{d_{\text{model}}\times d_v}\).
In self-attention, \(\mathbf{X}\) is the same sequence for all three. In cross-attention (seq2seq), \(\mathbf{Q}\) typically comes from decoder states while \(\mathbf{K},\mathbf{V}\) come from encoder states; the math is unchanged. Multi-head attention simply runs this projection-and-attend pattern several times in parallel with separate \(\mathbf{W}_i^Q,\mathbf{W}_i^K,\mathbf{W}_i^V\).

\begin{tcolorbox}[summarybox, title={Worked example: 2-token causal self-attention (one head)}]
Let \(d_k=d_v=2\) and \(Q=K=V=\mathbf{I}_2\). Without masking, the scaled scores are
\[
S=\frac{QK^\top}{\sqrt{d_k}}=\frac{1}{\sqrt{2}}\mathbf{I}_2.
\]
A \emph{causal mask} forbids looking into the future: for query index \(i\), mask out all keys with index \(j>i\) by setting those logits to \(-\infty\) (so they vanish after the softmax).

Row 1 can only attend to itself, so \(\alpha_{1\cdot}=[1,0]\). Row 2 sees logits \([0,\,1/\sqrt{2}]\), so
\[
\begin{aligned}
u &= e^{1/\sqrt{2}},\\
\alpha_{2\cdot} &= \operatorname{softmax}\!\left([0,\,1/\sqrt{2}]\right)
= \left[\frac{1}{1+u},\;\frac{u}{1+u}\right]\\
&\approx[0.330238,\;0.669762].
\end{aligned}
\]
Because \(V=\mathbf{I}_2\), the attention output equals the weight matrix: \(\operatorname{Attn}(Q,K,V)=\alpha\).
\end{tcolorbox}

\begin{tcolorbox}[summarybox, title={Shape ledger}]
We treat mini\hyp{}batches as \(\mathbf{X}\in\mathbb{R}^{B\times n \times d_{\text{model}}}\) (batch, sequence, features). After the linear projections each head carries \(\mathbf{Q},\mathbf{K}\in\mathbb{R}^{B\times h \times n \times d_k}\), \(\mathbf{V}\in\mathbb{R}^{B\times h \times n \times d_v}\), and the attention weights live in \(\mathbb{R}^{B\times h \times n \times n}\). Reading dimensions in this order avoids confusion when mixing frameworks; \(h\cdot d_k = d_{\text{model}}\) (often \(d_v=d_k\)). FFN inner widths typically 2--4$\times d_{\text{model}}$.
\end{tcolorbox}

\begin{tcolorbox}[summarybox, title={Complexity and memory}]
Naive attention is \(O(n^2 d_{\text{model}})\) compute and \(O(n^2)\) memory per head/layer for the attention map; this dominates long sequences. FlashAttention reduces activation I/O but keeps the quadratic arithmetic; sparse/linear variants reduce the \(n^2\) factor by trading exactness for structure (see Longformer/BigBird/Reformer/Performer/Linformer). Causal/padding masks do not change complexity, only which entries participate.
\end{tcolorbox}

\subsection{Self-attention vs.\ cross-attention}
\label{sec:transformers_self_attention_vs_cross_attention}
The same equations serve two distinct roles.
\emph{Self-attention} means \(Q,K,V\) come from the same sequence. This is how a token in a sentence can incorporate information from other tokens in that sentence. In decoder-only generation, self-attention is typically \emph{causal}: the mask enforces that position \(t\) can only use positions \(\le t\).
Self-attention by itself does not know what ``before'' and ``after'' mean: without positional information the operation is permutation-equivariant. That is why we explicitly add positional encodings in \Cref{sec:transformers_positional_information}.

\emph{Cross-attention} is the seq2seq bridge. Here the queries come from the decoder states, but the keys and values come from the encoder outputs. In translation terms: each output position asks a question (query) and then pulls a weighted average over the source-side memory (keys/values) to decide what to emit next.

\subsection{Multi-Head Attention (MHA)}
\label{sec:transformers_multi_head_attention_mha}
One attention head gives you one similarity space. Multi-head attention gives you several in parallel: each head learns its own projections and can focus on different relations at the same time (local vs.\ global cues, syntactic vs.\ semantic signals, or different parts of an image-like grid). You should not read heads as guaranteed ``modules'' with fixed roles; the point is capacity and parallel views under the same weighted-average mechanism.

Multiple heads attend in parallel after learned linear projections:
\begin{align}
    \mathrm{head}_i &= \operatorname{Attn}(\mathbf{X}\mathbf{W}_i^Q,\, \mathbf{X}\mathbf{W}_i^K,\, \mathbf{X}\mathbf{W}_i^V),\\
    \operatorname{MHA}(\mathbf{X}) &= [\mathrm{head}_1;\ldots;\mathrm{head}_h]\, \mathbf{W}^O,
    \label{eq:transformers_mha}
\end{align}
with \(\mathbf{W}_i^Q \in \mathbb{R}^{d_{\text{model}} \times d_k}\), \(\mathbf{W}_i^K \in \mathbb{R}^{d_{\text{model}} \times d_k}\), \(\mathbf{W}_i^V \in \mathbb{R}^{d_{\text{model}} \times d_v}\), and output projection \(\mathbf{W}^O\in\mathbb{R}^{(h d_v)\times d_{\text{model}}}\).
For cross-attention, the formula is the same but keys/values come from encoder states: \(\operatorname{Attn}(\mathbf{X}_{\text{dec}}\mathbf{W}_i^Q,\,\mathbf{X}_{\text{enc}}\mathbf{W}_i^K,\,\mathbf{X}_{\text{enc}}\mathbf{W}_i^V)\).
\Cref{fig:lec13_transformer_block} bundles scaled dot-product attention, multi-head concatenation, and the residual pre-LN block so the entire signal path is visible at a glance.

\begin{figure}[h]
    \centering
    \resizebox{\linewidth}{!}{%
    \begin{tikzpicture}[
        >={Latex[length=4pt, width=3pt]},
        font=\small\sffamily,
        % Panels
        panel/.style={draw=gray!40, rounded corners=5pt, fill=gray!3, line width=0.6pt},
        panel_label/.style={anchor=north west, font=\bfseries\footnotesize, text=gray!80, inner sep=4pt},
        % Functional blocks
        func/.style={draw=cbBlue!80, fill=cbBlue!10, rounded corners=2pt, line width=0.7pt, minimum height=1.8em, align=center, inner sep=4pt},
        % Operation blocks
        proc/.style={draw=cbGreen!80, fill=cbGreen!10, rounded corners=2pt, line width=0.7pt, minimum height=1.8em, align=center},
        % Warning/mask blocks
        maskblock/.style={draw=cbOrange!80, fill=cbOrange!10, rounded corners=2pt, line width=0.7pt, minimum height=1.6em, align=center},
        % Merge/add/concat
        sum/.style={circle, draw=cbPink!80, fill=cbPink!10, line width=0.7pt, inner sep=1pt, minimum size=1.2em},
        merge/.style={draw=cbPink!80, fill=cbPink!10, rounded corners=2pt, line width=0.7pt, minimum height=1.8em, align=center},
        % Wires + annotation
        wire/.style={->, draw=gray!70, line width=0.8pt, rounded corners=3pt},
        skip/.style={->, draw=gray!55, line width=0.8pt, rounded corners=6pt},
        annot/.style={font=\scriptsize, text=gray!60, align=left}
    ]
        % ==========================
        % Panel (a): Scaled Dot-Product
        % ==========================
        \node[panel, minimum width=4.2cm, minimum height=6.2cm] (p1) at (0,0) {};
        \node[panel_label] at (p1.north west) {(a) Scaled Dot-Product};

        \node[func, minimum width=0.8cm] (q) at ([yshift=-1.2cm, xshift=-1.2cm]p1.north) {\(\mathbf{Q}\)};
        \node[func, minimum width=0.8cm] (k) at ([yshift=-1.2cm]p1.north) {\(\mathbf{K}\)};
        \node[func, minimum width=0.8cm] (v) at ([yshift=-1.2cm, xshift=1.2cm]p1.north) {\(\mathbf{V}\)};

        \node[proc, minimum width=2.4cm] (matmul1) at ([yshift=-1.0cm]k.south) {MatMul};
        \node[annot, right] at (matmul1.east) {\(\frac{\mathbf{Q}\mathbf{K}^\top}{\sqrt{d_k}}\)};

        \node[maskblock, minimum width=2.4cm] (masknode) at ([yshift=-0.7cm]matmul1.south) {Mask (opt.)};
        \node[proc, minimum width=2.4cm] (softmax) at ([yshift=-0.7cm]masknode.south) {Softmax};
        \node[proc, minimum width=2.4cm] (matmul2) at ([yshift=-0.9cm]softmax.south) {MatMul};
        \node[annot, right] at (matmul2.east) {Weighted\\Sum};

        \draw[wire] (q.south) -- ++(0,-0.3) -| ([xshift=-4pt]matmul1.north);
        \draw[wire] (k.south) -- (matmul1.north);
        \draw[wire] (matmul1.south) -- (masknode.north);
        \draw[wire] (masknode.south) -- (softmax.north);
        \draw[wire] (softmax.south) -- (matmul2.north);
        % Route V -> MatMul2 with an "outside" elbow so it avoids the right-side annotations.
        \coordinate (vRouteX) at ($(p1.east)+(0.55,0)$);
        \draw[wire] (v.east) -- (vRouteX |- v.east) |- (matmul2.north east);
        \draw[wire] (matmul2.south) -- ++(0,-0.4);

        % ==========================
        % Panel (b): Multi-Head
        % ==========================
        \node[panel, minimum width=4.2cm, minimum height=6.2cm] (p2) at (5,0) {};
        \node[panel_label] at (p2.north west) {(b) Multi-Head};

        \node (input_b) at ([yshift=-0.8cm]p2.north) {};
        \node[func, fill=white, draw=gray!30, minimum width=2.8cm, yshift=2pt, xshift=2pt] at ([yshift=-0.6cm]input_b.south) {};
        \node[func, minimum width=2.8cm] (proj) at ([yshift=-0.6cm]input_b.south) {Projections\\\(W_i^Q, W_i^K, W_i^V\)};

        \node[proc, fill=white, draw=gray!30, minimum width=2.2cm, yshift=2pt, xshift=2pt] at ([yshift=-0.9cm]proj.south) {};
        \node[proc, minimum width=2.2cm] (head) at ([yshift=-0.9cm]proj.south) {Attention\\Head \(i\)};
        \node[annot, right] at (head.east) {\(\times h\)};

        \node[merge, minimum width=2.8cm] (concat) at ([yshift=-0.8cm]head.south) {Concat};
        \node[func, minimum width=2.8cm] (out_proj) at ([yshift=-0.8cm]concat.south) {Output \(W^O\)};

        \draw[wire] (proj.south) -- (head.north);
        \draw[wire] (head.south) -- (concat.north);
        \draw[wire] (concat.south) -- (out_proj.north);
        \draw[wire] (out_proj.south) -- ++(0,-0.4);

        % ==========================
        % Panel (c): Pre-LN Encoder
        % ==========================
        \node[panel, minimum width=4.2cm, minimum height=6.2cm] (p3) at (10,0) {};
        \node[panel_label] at (p3.north west) {(c) Pre-LN Block};

        \node[coordinate] (in_c) at ([yshift=-1.0cm]p3.north) {};

        \node[func, minimum width=2.4cm] (ln1_real) at ([yshift=-0.6cm]in_c) {LayerNorm};
        \node[proc, minimum width=2.4cm] (mha_real) at ([yshift=-0.6cm]ln1_real.south) {Multi-Head Attn};
        \node[sum] (add1_real) at ([yshift=-0.5cm]mha_real.south) {+};

        \node[func, minimum width=2.4cm] (ln2_real) at ([yshift=-0.6cm]add1_real.south) {LayerNorm};
        \node[proc, minimum width=2.4cm] (ffn_real) at ([yshift=-0.6cm]ln2_real.south) {FFN (GELU)};
        \node[sum] (add2_real) at ([yshift=-0.5cm]ffn_real.south) {+};

        \draw[wire] (in_c) -- (ln1_real.north);
        \draw[wire] (ln1_real.south) -- (mha_real.north);
        \draw[wire] (mha_real.south) -- (add1_real.north);
        \draw[wire] (add1_real.south) -- (ln2_real.north);
        \draw[wire] (ln2_real.south) -- (ffn_real.north);
        \draw[wire] (ffn_real.south) -- (add2_real.north);
        \draw[wire] (add2_real.south) -- ++(0,-0.4);

        % Residual skips (explicit bypass into each residual add)
        \draw[skip] ([yshift=0.2cm]ln1_real.north) -- ++(-1.6,0) |- node[pos=0.2, above, font=\scriptsize, text=gray!55]{residual} (add1_real.west);
        \draw[skip] ([yshift=0.2cm]ln2_real.north) -- ++(-1.6,0) |- node[pos=0.2, above, font=\scriptsize, text=gray!55]{residual} (add2_real.west);

        \node[draw=gray!50, dotted, fill=white, rounded corners, font=\tiny, align=left, inner sep=2pt] (postln) at ([xshift=1.3cm]add2_real.east) {Post-LN:\\Norm here};
        \draw[->, dotted, gray!60] (postln.west) -- (add2_real.east);
    \end{tikzpicture}}
    % Avoid dense inline math in captions; it wraps poorly in EPUB renderers.
    \caption{Reference schematic for the Transformer. Left: scaled dot-product attention. Center: multi-head concatenation with an output projection. Right: pre-LN encoder block combining attention, FFN, and residual connections; a post-LN variant simply moves each LayerNorm after its residual add (dotted alternative, not shown).}
    \label{fig:lec13_transformer_block}
\end{figure}
\FloatBarrier

\Cref{fig:lec13_micro_figures} provides a quick visual summary of positional encoding, KV cache reuse, and LoRA adapters.

\begin{figure}[t]
    \centering
    \resizebox{\linewidth}{!}{%
    \begin{tikzpicture}[
        font=\small\sffamily,
        enc/.style={thick, smooth},
        panel/.style={draw=gray!55, rounded corners=4pt, thick, fill=gray!6, inner sep=6pt, minimum width=4.6cm, minimum height=3.6cm},
        box/.style={draw=gray!60, rounded corners=3pt, thick, fill=gray!5, inner sep=6pt},
        subbox/.style={draw=gray!60, rounded corners=3pt, thick, fill=cbOrange!20, inner sep=4pt},
        mat/.style={draw=gray!60, rounded corners=3pt, thick, fill=cbBlue!12, minimum width=2cm, minimum height=1cm},
        smallmat/.style={draw=gray!60, rounded corners=3pt, thick, fill=cbGreen!25, minimum width=1.4cm, minimum height=0.7cm}
    ]

    % Left panel: positional encodings
    \begin{scope}[xshift=0cm]
        \node[panel] (p1) {};
        \begin{axis}[
            at={(p1.center)},
            anchor=center,
            width=3.9cm,
            height=2.8cm,
            axis lines=none,
            xmin=0, xmax=2*pi,
            ymin=-1.2, ymax=1.2,
            clip=false
        ]
            \addplot[enc, cbBlue, samples=200, domain=0:2*pi] {sin(deg(x))};
            \addplot[enc, cbOrange, samples=200, domain=0:2*pi] {sin(deg(2*x))};
        \end{axis}
        \node[anchor=south, font=\small] at ([yshift=12pt]p1.south) {Positional encodings};
        \node[anchor=north, font=\small] at ([yshift=2pt]p1.south) {sinusoidal / RoPE};
        \node[anchor=north west, font=\footnotesize] at ([xshift=4pt, yshift=-4pt]p1.north west) {(a)};
    \end{scope}

    % Middle panel: Decoder + KV Cache
    \begin{scope}[xshift=5.8cm]
        \node[panel] (p2) {};
        \node[box, minimum width=2.6cm, font=\small] (dec) at (p2.center|-0,0.65) {Decoder block};
        \node[subbox, minimum width=2.6cm, font=\small] (kv) at (p2.center|-0,-0.65) {K/V cache};
        \draw[->, line width=0.9pt] (dec) -- (kv);
        \node[anchor=north west, font=\footnotesize] at ([xshift=4pt, yshift=-4pt]p2.north west) {(b)};
    \end{scope}

    % Right panel: LoRA adapters
    \begin{scope}[xshift=11.6cm]
        \node[panel] (p3) {};
        % horizontal BA feeding W
        \node[mat, font=\small] (W) at (p3.center|-0,1.35) {$\mathbf{W}$};
        \node[smallmat, font=\small] (B) at ([xshift=-1.7cm]p3.center|-0,-0.15) {$\mathbf{B}$};
        \node[smallmat, font=\small] (A) at ([xshift=1.7cm]p3.center|-0,-0.15) {$\mathbf{A}$};
        \draw[->, line width=0.9pt] (B.north) -- ($(W.south)!0.55!(W.south west)$);
        \draw[->, line width=0.9pt] (A.north) -- ($(W.south)!0.55!(W.south east)$);
        \node[font=\scriptsize, anchor=south] at ([yshift=2pt]W.south) {rank $r$};
        \node[font=\small] at ([yshift=-12pt]p3.south) {LoRA adapters};
        \node[anchor=north west, font=\footnotesize] at ([xshift=4pt, yshift=-4pt]p3.north west) {(c)};
    \end{scope}

    % Cross-panel connectivity arrows
    % Token flowing from positional encodings into the decoder block
    \draw[->, line width=0.9pt, draw=gray!65]
        ([xshift=0.25cm]p1.east) --
        ([xshift=-0.25cm]p2.west)
        node[midway, above, font=\small] {token};
    % Reuse of cached K/V feeding the following decoding step
    \draw[->, line width=0.9pt, draw=gray!65]
        ([xshift=0.25cm]kv.east) --
        ([xshift=-0.25cm]p3.west|-kv.east)
        node[midway, above, font=\small] {reuse};

    \end{tikzpicture}
    }%
    % Avoid inline math in captions; it wraps poorly in some EPUB renderers.
    \caption{Transformer micro-views. Left: positional encodings (sinusoidal/rotary) add order information. Center: KV cache stores past keys/values so decoding a new token reuses prior context. Right: LoRA inserts low-rank adapters (B A) on top of a frozen weight matrix W for parameter-efficient tuning.}
    \label{fig:lec13_micro_figures}
\end{figure}
\FloatBarrier

\subsection{Positional Information}
\label{sec:transformers_positional_information}
Transformers lack recurrence, so order has to be injected explicitly. A simple baseline (from \citet{Vaswani2017}) is a sinusoidal positional encoding: for position \(\mathrm{pos}\) and feature index \(i\),
\begin{align}
    \mathrm{PE}(\mathrm{pos},\,2i)   &= \sin\!\left(\frac{\mathrm{pos}}{10000^{2i/d_{\text{model}}}}\right),\\
    \mathrm{PE}(\mathrm{pos},\,2i+1) &= \cos\!\left(\frac{\mathrm{pos}}{10000^{2i/d_{\text{model}}}}\right).
    \label{eq:transformers_sinusoidal_pe}
\end{align}
You add this vector (or a learned alternative) to the token embedding at each position. The engineering point is not the specific constant; it is that different dimensions oscillate at different frequencies, so nearby positions look similar in some coordinates and far positions look different in others.

In modern practice you will also see learned positional embeddings, relative-position schemes, and rotary position embeddings (RoPE). The chapter keeps sinusoidal PE as the clean reference, and we treat other variants as drop-in replacements that mainly change how well models extrapolate to longer contexts or shifting windows.

\subsection{Objectives, masks, and model families}
\label{sec:transformers_masks_and_training_objectives}
Two masks show up so often that it is worth naming them early.
A \emph{causal mask} forbids attention to future positions; this is what makes next-token generation well-defined.
A \emph{padding mask} forbids attention to padded positions inserted only to make batches rectangular.
Both are easy to get wrong: a single missing mask can leak information from the future or silently change what the model is allowed to use.
See \Cref{fig:lec13_masks} for a concrete picture of both patterns (queries on rows, keys on columns).

\begin{figure}[t]
    \centering
    \begin{tikzpicture}
            \begin{groupplot}[
                group style={group size=2 by 1, horizontal sep=1.2cm},
                width=0.42\linewidth,
                height=0.36\linewidth,
                view={0}{90},
            xmin=-0.5, xmax=4.5,
            ymin=-0.5, ymax=4.5,
            xtick={0,...,4},
            ytick={0,...,4},
            xticklabels={t\(_0\), t\(_1\), t\(_2\), t\(_3\), t\(_4\)},
            yticklabels={t\(_0\), t\(_1\), t\(_2\), t\(_3\), t\(_4\)},
                xticklabel style={font=\scriptsize},
                yticklabel style={font=\scriptsize},
                colorbar,
                colorbar style={height=3.0cm},
                colormap/viridis,
                point meta min=0, point meta max=1,
                nodes near coords,
                nodes near coords align={center},
                every node near coord/.append style={
                    font=\scriptsize,
                    fill=white,
                    fill opacity=0.75,
                    text opacity=1,
                    inner sep=1.0pt
                }
            ]
            \nextgroupplot[title={Padding mask}]
                \addplot[matrix plot*, mesh/cols=5, mesh/rows=5, point meta=explicit] table [meta=z] {
                    x y z
                    0 0 1
                    1 0 1
                    2 0 1
                    3 0 0
                    4 0 0
                    0 1 1
                    1 1 1
                    2 1 1
                    3 1 0
                    4 1 0
                    0 2 1
                    1 2 1
                    2 2 1
                    3 2 0
                    4 2 0
                    0 3 1
                    1 3 1
                    2 3 1
                    3 3 0
                    4 3 0
                    0 4 1
                    1 4 1
                    2 4 1
                    3 4 0
                    4 4 0
                };
                \node[font=\scriptsize, anchor=north west] at (axis cs:-0.4,4.4) {keep};
                \node[font=\scriptsize, anchor=south east] at (axis cs:3.4,0.0) {mask};
            \nextgroupplot[title={Causal mask}]
                \addplot[matrix plot*, mesh/cols=5, mesh/rows=5, point meta=explicit] table [meta=z] {
                    x y z
                    0 0 1
                    1 0 0
                    2 0 0
                    3 0 0
                    4 0 0
                    0 1 1
                    1 1 1
                    2 1 0
                    3 1 0
                    4 1 0
                    0 2 1
                    1 2 1
                    2 2 1
                    3 2 0
                    4 2 0
                    0 3 1
                    1 3 1
                    2 3 1
                    3 3 1
                    4 3 0
                    0 4 1
                    1 4 1
                    2 4 1
                    3 4 1
                    4 4 1
                };
                \node[font=\scriptsize, anchor=north east] at (axis cs:4.4,4.4) {future masked};
        \end{groupplot}
    \end{tikzpicture}
    \caption{Attention masks as heatmaps (queries on rows, keys on columns). Left: padding mask zeroes out attention to padded positions of a shorter sequence in a packed batch. Right: causal mask enforces autoregressive flow by blocking attention to future tokens.}
    \label{fig:lec13_masks}
\end{figure}
\FloatBarrier

The training objective is usually a cross-entropy (CE) loss, i.e., a negative log-likelihood (NLL) of the correct label under the model's predicted distribution. In sequence modeling, you typically sum (or average) that loss across time positions and across batch elements.

We will return to the main Transformer families (encoder-only vs.\ decoder-only vs.\ encoder--decoder) after we write down the block structure in \Cref{sec:transformers_encoder_decoder_stacks_and_stabilizers}. The core idea is simple: families differ mostly by which masks they apply and which prediction problem they are trained on, not by a different attention mechanism.

\subsection{Encoder/Decoder Stacks and Stabilizers}
\label{sec:transformers_encoder_decoder_stacks_and_stabilizers}
Each block uses residual connections and layer normalization:
\begin{align}
\mathbf{H}' &= \operatorname{LayerNorm}(\mathbf{H} + \operatorname{MHA}(\mathbf{H},\mathbf{H},\mathbf{H})),\\
\mathbf{H}_{\text{out}} &= \operatorname{LayerNorm}(\mathbf{H}' + \operatorname{FFN}(\mathbf{H}')).
    \label{eq:auto:lecture_transformers:2}
\end{align}
The feed-forward sublayer (FFN) is position-wise, typically two linear layers with a nonlinearity (e.g., GELU). Dropout and label smoothing are common.

\begin{tcolorbox}[summarybox, breakable, title={Implementation snapshot: block pseudocode, training defaults, and one step}]
\noindent\textbf{Block pseudocode (pre-LN).}
\begin{verbatim}
function Block(H):
    # Pre-normalize inputs (pre-LN stabilizes deep stacks)
    H_norm = LayerNorm(H)
    attn = MHA(H_norm, H_norm, H_norm)
    H = H + Dropout(attn)
    H_norm = LayerNorm(H)
    ff = FFN(H_norm)
    return H + Dropout(ff)
\end{verbatim}
Decoder blocks add causal masks and cross-attention with encoder states. Pre-LN (shown here) is now common because it keeps gradients well behaved for very deep stacks; post-LN (original Transformer) is still used in smaller models.
\medskip
\noindent\textbf{Training defaults (decoder-only, practical baseline).}
AdamW with cosine decay and 1--3\% warmup; LR \(\sim 10^{-3}\) for small models, \(1\text{--}2\times 10^{-4}\) for mid-size. Weight decay \(\approx 0.01\) (exclude biases/LayerNorm gains). Attention/MLP dropout \(\approx 0.1\); clip global norm to 1.0. Mixed precision (FP16/BF16) plus gradient checkpointing for long contexts; tie input embeddings to the LM head; use causal masks for CLM and padding masks for packed batches.

\medskip
\noindent\textbf{One training step (decoder-only, causal mask).}
\begin{verbatim}
x = tokenizer(batch_text)                # [B, T]
mask = causal_mask(x)                    # [B, 1, T, T]
h = embed(x) + pos(x)                    # [B, T, d_model]
for block in blocks:
    h = block(h, mask)                   # pre-LN MHA + FFN
logits = lm_head(h)                      # [B, T, vocab]
loss = CE(logits[:, :-1], x[:, 1:])      # next-token
loss.backward()
clip_grad_norm_(model.parameters(), 1.0)
opt.step(); opt.zero_grad()
\end{verbatim}
At inference, reuse cached K/V states (see \Cref{fig:lec13_micro_figures} and \Cref{sec:transformers_advanced_attention_and_efficiency_notes_snapshot}).
\end{tcolorbox}
\paragraph{Code--math dictionary.} In code blocks, \texttt{x} is the token-index tensor (input IDs), \texttt{h} is the hidden-state array \(\mathbf{H}\), \texttt{mask} is the attention mask, and \texttt{embed(x)} denotes an embedding lookup into the learned table \(\mathbf{W}\) (written algebraically as a row-selection, e.g., \(\mathbf{W}[w_t]\), in \Cref{chap:nlp}).

\subsection{BERT vs.\ GPT vs.\ encoder--decoder}
\label{sec:transformers_model_families}
Once you have attention blocks, most of the ``model family'' distinctions come from two choices: (i) which directions a position is allowed to attend to, and (ii) what prediction problem you train on.
\begin{itemize}
    \item \textbf{Encoder-only (BERT-style):} no causal mask; each token can use information from both left and right. Training commonly hides some input tokens and asks the model to predict the missing pieces (masked language modeling, MLM). A pooled vector (often a prepended \texttt{[CLS]} token) is then used as a sentence representation for classification.
    \item \textbf{Decoder-only (GPT-style):} causal mask; position \(t\) can only attend to positions \(\le t\). Training is next-token prediction (causal language modeling, CLM). Inference is the same loop run forward: predict the next token, append it, and repeat.
    \item \textbf{Encoder--decoder (seq2seq):} the encoder reads the source; the decoder generates the target with causal self-attention plus cross-attention into the encoder outputs. This is the cleanest fit for translation: each output position pulls a weighted average over source-side memory and then commits to one more token.
\end{itemize}
Some BERT variants also used ``next sentence prediction'' (NSP) as an auxiliary task historically; it is not essential for understanding the core encoder-only idea.

\subsection{Long Contexts and Efficient Attention}
\label{sec:transformers_long_contexts_and_efficient_attention}
Memory and compute scale quadratically with sequence length. Practical systems therefore mix several tricks:
\begin{itemize}
    \item \textbf{Sparse or local attention} (e.g., Longformer, BigBird) to limit each query to a sliding or block-sparse neighborhood.
    \item \textbf{Low-rank/kernelized approximations} and recurrent chunking (Performer, Transformer-XL) so that computation/storage grows roughly linearly in context length.
    \item \textbf{I/O-aware kernels} such as FlashAttention that stream tiles through SRAM so the \(O(n^2)\) attention computation remains exact while memory stays manageable.
\end{itemize}
Relative/rotary position schemes and KV caching are summarized in \Cref{sec:transformers_advanced_attention_and_efficiency_notes_snapshot} and \Cref{fig:lec13_micro_figures}.

\subsection{Fine-Tuning and Parameter-Efficient Adaptation}
\label{sec:transformers_fine_tuning_and_parameter_efficient_adaptation}
Pre-training gives you a general-purpose language model; fine-tuning adapts that model to a task, domain, or interaction style. \emph{Full fine-tuning} updates all weights, which can yield the best performance when you have enough high-quality data and a stable objective, but it is also the easiest to overfit or to accidentally ``forget'' useful general behavior.

\emph{Parameter-efficient} methods (LoRA, adapters, prefix/prompt tuning, and related variants) inject small trainable modules while freezing most of the base model, enabling rapid adaptation with lower memory and more predictable changes. Practically, PEFT is attractive when you want many task-specific variants of a shared base model, or when you need to keep the base weights fixed for deployment and auditing.

\paragraph{Audit hooks for adaptation.}
Regardless of whether you update all weights or only a small adapter, treat fine-tuning like any other ERM pipeline: keep a held-out evaluation set that matches the deployment slice, monitor calibration and failure modes (not just loss), and log the exact base checkpoint, tokenizer, and data snapshot so results are reproducible. When fine-tuning for instruction-following or conversational behavior, add explicit tests for regressions (refusals, hallucinations on factual probes, and brittleness to prompt variants) rather than relying on a single aggregate score.

\subsection{Decoding and Evaluation}
\label{sec:transformers_decoding_and_evaluation}
Training produces a distribution over the next token; decoding turns that distribution into an actual sequence. For decoder-only models, decoding runs autoregressively: predict \(p_\theta(x_{t+1}\mid x_{1:t})\), choose a token, append it, and repeat. This is also where the KV cache matters (see \Cref{fig:lec13_micro_figures}): you reuse past keys/values so generating one more token does not require recomputing attention over the entire prefix from scratch.

Greedy decoding takes the argmax at each step; it is fast and often strong for short, factual completions, but it can get stuck in repetitive loops. Beam search keeps multiple partial hypotheses; it can improve likelihood but sometimes harms perceived quality in open-ended generation. Sampling (top-\(k\), top-\(p\) nucleus) trades certainty for diversity; temperature controls how sharp or flat the distribution feels.

For evaluation, perplexity summarizes next-token performance for language modeling (see \Cref{chap:nlp}), but it does not tell you whether generations are useful, safe, or faithful. For downstream classification, prefer metrics that match the deployment slice (e.g., AUPRC for imbalanced problems) and keep decoding settings logged alongside checkpoints so results are reproducible.

\subsection{Audit and failure modes (short list)}
\label{sec:transformers_audit_and_failure_modes}
\begin{tcolorbox}[summarybox, title={Audit and failure modes (engineering view)}]
\begin{itemize}
    \item \textbf{Masking bugs:} missing/incorrect causal masks can leak future tokens; missing padding masks can let padding dominate attention in batched training.
    \item \textbf{Train/inference mismatch:} teacher forcing during training does not automatically tell you how errors compound during decoding; test the decoding strategy you plan to ship.
    \item \textbf{Long-context degradation:} attention enables long-range access, but quality can still decay with length; measure how performance changes as you increase context.
    \item \textbf{Calibration vs.\ correctness:} high probability is not a guarantee of correctness; audit reliability on slices and stress tests, not just average loss.
    \item \textbf{Reproducibility:} tokenizer choices, data filters, and decoding hyperparameters can swing results; log them as part of the experiment.
\end{itemize}
\end{tcolorbox}

\subsection{Alignment (Brief)}
\label{sec:transformers_alignment_brief}
Post-training \emph{alignment} shapes model behavior to match human preferences, safety constraints, and interaction norms. In broad terms, alignment objectives do not change the Transformer mechanics; they change what you reward during optimization (and therefore what behaviors are reinforced).

RLHF optimizes a policy against a learned reward model (with careful regularization to avoid drifting too far from the base model). Preference-based objectives such as DPO, KTO, or ORPO optimize directly from ranked pairs without a full reinforcement-learning loop.

\paragraph{Alignment is not a proof of correctness.}
Alignment can improve helpfulness and reduce obvious failures, but it can also introduce new ones (reward hacking, over-refusal, brittleness to prompt phrasing, or degraded performance off-distribution). Treat it as an engineering stage with explicit test suites and logging: evaluate on held-out tasks, check calibration and refusal behavior, and track changes relative to the pre-alignment model.

\subsection{Advanced attention and efficiency notes (practitioner snapshot)}
\label{sec:transformers_advanced_attention_and_efficiency_notes_snapshot}
\begin{itemize}
    \item \textbf{Relative/rotary positions.} RoPE \citep{Su2021RoPE} and ALiBi \citep{Press2022ALiBi} replace absolute sinusoidal embeddings with rotation/bias terms so extrapolating to longer sequences no longer requires re-fitting positional lookups; the trade-off is that absolute tables keep fixed anchors for classification tokens while rotary/relative schemes favour extrapolation and smoothly sliding windows.
    \item \textbf{KV-cache management.} Decoder-only inference stores per-layer key/value tensors; chunked caching, paged attention, and sliding windows keep memory linear in context length. Speculative decoding and assisted decoding reuse a lightweight draft model to propose tokens that the full model verifies before committing.
    \item \textbf{Efficient kernels.} FlashAttention \citep{Dao2022FlashAttention} computes attention blocks in streaming tiles to keep activations in SRAM. Long-context variants mix windowed attention, recurrent memory, or low-rank adapters; state-space models such as Mamba \citep{Gu2023Mamba} provide linear-time alternatives that back-propagate through implicitly defined kernels.
    \item \textbf{Mixture-of-experts and routing.} Sparse MoE layers \citep{Shazeer2017MoE} add conditional capacity; router z-losses, capacity factors, and load-balancing losses are essential to avoid expert collapse.
    \item \textbf{Test-time scaling.} Curriculum-based sampling (nucleus, temperature annealing), classifier-free guidance, and beam-search variants all tune the accuracy/latency frontier; plan to log decoding hyperparameters alongside checkpoints so experiments are reproducible.
\end{itemize}
Parameter-efficient tuning methods are covered in \Cref{sec:transformers_fine_tuning_and_parameter_efficient_adaptation}.

\subsection{RNNs vs. Transformers: When and Why}
\label{sec:transformers_rnns_vs_transformers_when_and_why}
\begin{center}
\begin{tabular}{@{}>{\raggedright\arraybackslash}p{0.24\linewidth} >{\raggedright\arraybackslash}p{0.34\linewidth} >{\raggedright\arraybackslash}p{0.34\linewidth}@{}}
\toprule
 & \textbf{RNN/LSTM/GRU} & \textbf{Transformer} \\
\midrule
Parallelism & Limited (sequential) & High (tokens in parallel) \\
Long context & Challenging (vanishing) & Natural; quadratic cost \\
Inductive bias & Order, recurrence & Content-based attention \\
Best for & Small data, streaming & Large data, global deps \\
Equivariance & N/A & Permutation-equivariant until positions (cf. conv translation equivariance in \Cref{chap:cnn}) \\
\bottomrule
\end{tabular}
\end{center}

\begin{tcolorbox}[summarybox, title={Key takeaways}]
\noindent\textbf{Terminology.} Masked-LM and next-token LM are \emph{self-supervised} (targets derived from input), not unsupervised.
\par\smallskip
\begin{itemize}
    \item Attention replaces recurrence with content-based mixing, enabling highly parallel training but introducing quadratic cost in sequence length.
    \item Practical stability depends on details (pre-LN vs.\ post-LN, optimizer choices, masking, and careful decoding/evaluation).
    \item Architecture choices (encoder/decoder, positions, caching) are not cosmetic: they determine what the model can reuse at inference time.
\end{itemize}

\medskip
\noindent\textbf{What to be able to do.}
\begin{itemize}
    \item Compute masked attention for a short sequence and explain why the mask enforces causality.
    \item Explain pre-LN vs.\ post-LN and why residual paths influence optimization stability.
    \item Describe KV caching and how it changes inference-time cost compared to training-time cost.
\end{itemize}

\noindent\textbf{Common pitfalls to watch for.}
\begin{itemize}
    \item Incorrect masking (future leakage) or inconsistent tokenization between training and evaluation.
    \item Reporting speed or memory without stating context length, batch size, and caching assumptions.
    \item Treating decoding strategy as an afterthought; greedy, beam, and sampling regimes change observed quality.
\end{itemize}
\end{tcolorbox}

\begin{tcolorbox}[summarybox, title={Exercises and lab ideas}]
\begin{itemize}
    \item Hand-compute a 2-token attention step with masking; verify against a short script.
    \item Implement a single-block decoder-only transformer (embed + pos + pre-LN MHA + FFN) and train on a tiny character corpus; report perplexity.
    \item Compare naive attention vs.\ FlashAttention on \(n \in \{256, 1024, 4096\}\); log peak memory and tokens/s.
    \item Fine-tune a base model with RoPE vs.\ ALiBi and evaluate extrapolation to 2\(\times\) the training context.
    \item Implement DPO on a small preference dataset; report win-rates versus the SFT baseline.
\end{itemize}

\medskip
\noindent\textbf{If you are skipping ahead.} After this chapter, the book pivots away from neural sequence models, so treat this chapter as the last stop for masking discipline and evaluation hygiene. If you need the embedding objectives and bias/deployment checklist, see \Cref{chap:nlp}.
\end{tcolorbox}

\medskip
\paragraph{Where we head next.} \Cref{chap:softcomp} steps away from neural sequence models and re-enters the broader soft-computing toolkit (fuzzy logic and evolutionary ideas) previewed in \Cref{chap:intro}. Read this chapter as the endpoint of the neural sequence thread: representation objectives from \Cref{chap:nlp} plus masking/calibration discipline from \Crefrange{chap:rnn}{chap:transformers}.

\begin{tcolorbox}[summarybox,title={Part II takeaways}]
\begin{itemize}
  \item Representation and training are coupled: expressivity is only useful if gradients can reach the parameters.
  \item Depth and inductive bias trade parameters for structure (MLPs vs.\ CNNs vs.\ recurrence vs.\ attention).
  \item Stability tools recur across architectures: initialization, normalization, regularization, and validation-driven stopping.
  \item Sequence models turn memory into computation: unrolling, masking, and caching define what information can flow.
\end{itemize}
\end{tcolorbox}
% Former Part III (NLP applications) was collapsed into Part II; renumber Parts
% here to keep the TOC continuous and avoid the appearance of a missing Part.
\part*{Part III: Soft computing and fuzzy reasoning}
\addcontentsline{toc}{part}{Part III: Soft computing and fuzzy reasoning}
% Chapter 15
\section{Introduction to Soft Computing}\label{chap:softcomp}
\graphicspath{{assets/lec8/}}

\begin{tcolorbox}[summarybox,title={Learning Outcomes}]
\begin{itemize}
    \item Articulate why soft computing (fuzzy logic, evolutionary search, neural hybrids) complements the statistical strand from earlier chapters.
    \item Define fuzzy sets, linguistic variables, and rule bases at a conceptual level before \Crefrange{chap:fuzzysets}{chap:fuzzyinference} formalize them.
    \item Track a running thermostat/autofocus example that grounds the design choices introduced throughout the fuzzy trilogy.
\end{itemize}
\end{tcolorbox}

\begin{tcolorbox}[summarybox,title={Running example: fuzzy thermostat}]
We revisit a smart thermostat that regulates a room using two linguistic inputs (temperature error and rate of change) and one output (heater power). This compact scenario lets us instantiate membership functions (\Cref{chap:fuzzysets}), transform them between universes (\Cref{chap:fuzzyrelations}), and assemble complete inference systems (\Cref{chap:fuzzyinference}) without inventing new notation each time.
\end{tcolorbox}

After completing the neural strand's sequence toolchain---from recurrence in \Cref{chap:rnn} to embeddings and deployment audits in \Cref{chap:nlp} and attention-based modeling in \Cref{chap:transformers}---we pivot to the behavioral and evolutionary strand previewed in \Cref{chap:intro}: soft computing. The roadmap figure (\Cref{fig:roadmap}) marks this transition explicitly.

\begin{tcolorbox}[summarybox,title={Design motif}]
When precision is costly or ill-defined, make the vagueness explicit and keep the system auditable: represent linguistic concepts with membership functions, reason with operator choices you can explain, and tune those choices with optimization rather than brittle rules.
\end{tcolorbox}

\subsection{Hard Computing: The Classical Paradigm}
\label{sec:softcomp_hard_computing_the_classical_paradigm}

Hard computing refers to the classical approach to computation where the goal is to produce precise, unambiguous, and mathematically exact outputs. This paradigm assumes that the relationships between inputs and outputs can be modeled accurately using well-defined mathematical equations. For example, Einstein's mass-energy equivalence formula,
\begin{equation}
E = mc^2,
\label{eq:auto_softcomp_35ca93a64c}
\end{equation}
is a precise, unambiguous, and exact mathematical expression.

In hard computing, the process typically involves:
\begin{itemize}
    \item Precise inputs,
    \item Deterministic models,
    \item Exact outputs.
\end{itemize}

However, this approach is often inadequate for many real-world problems because:
\begin{enumerate}
    \item The real world is pervasively \emph{imprecise} and \emph{uncertain}.
    \item Achieving precision and certainty is often \emph{costly} and \emph{difficult}.
\end{enumerate}

These limitations motivate the need for alternative computational frameworks that can tolerate and exploit imprecision and uncertainty.

\subsection{Soft Computing: Motivation and Definition}
\label{sec:softcomp_soft_computing_motivation_and_definition}

Soft computing, introduced by \citet{Zadeh1994,Zadeh1997} after his 1965 fuzzy sets paper, is a computational paradigm designed to handle problems where precision and certainty are either impossible or prohibitively expensive to obtain. Unlike hard computing, soft computing tolerates \emph{imprecision}, \emph{uncertainty}, and \emph{approximate reasoning} to achieve solutions that are:
\begin{itemize}
    \item \textbf{Tractable:} Computationally feasible to obtain,
    \item \textbf{Robust:} Insensitive to noise and variations,
    \item \textbf{Low-cost:} Economical in terms of computational resources.
\end{itemize}

Formally, soft computing is not a single homogeneous methodology but rather a \emph{partnership of distinct methods} that conform to these guiding principles. In Zadeh's broad usage, the principal constituents include:
\begin{itemize}
    \item \textbf{Fuzzy Logic:} Handling imprecision and approximate reasoning,
    \item \textbf{Neurocomputing (and neuro-fuzzy hybrids):} Learning from data through neural networks, sometimes combined with fuzzy rule bases (e.g., ANFIS),
    \item \textbf{Genetic Algorithms:} Evolutionary optimization inspired by natural selection.
\end{itemize}

These components often overlap and complement each other in practical applications. In this book, probabilistic modeling is treated in the statistical strand (\Crefrange{chap:supervised}{chap:logistic}); the soft-computing block focuses on fuzzy systems and evolutionary search, with neuro-fuzzy hybrids serving as the bridge back to the neural chapters.

\subsection{Why Soft Computing?}
\label{sec:softcomp_why_soft_computing}

The key insight behind soft computing is to exploit the \emph{tolerance for imprecision and uncertainty} inherent in many real-world problems. Consider the example of handwritten digit recognition using a convolutional neural network (CNN):

\begin{itemize}
    \item The input is a handwritten digit, say the digit "4".
    \item The network extracts features and produces a probability distribution over possible digits.
    \item The output might be:
    \[
    P(\text{digit} = 4) = 0.60, \quad P(\text{digit} = 7) = 0.20, \quad P(\text{digit} = 1) = 0.20.
    \]
\end{itemize}

This output is \emph{not precise} in the classical sense; it expresses uncertainty and partial belief. The system tolerates this imprecision and still makes a decision based on the highest probability, demonstrating robustness and flexibility.

\subsection{Relationship Between Hard and Soft Computing}
\label{sec:softcomp_relationship_between_hard_and_soft_computing}

We can conceptualize the landscape of computing as follows:
\begin{itemize}
    \item \textbf{Hard Computing:} Precise, deterministic, mathematically exact.
    \item \textbf{Soft Computing:} Approximate, tolerant of imprecision and uncertainty, heuristic.
\end{itemize}

There is some overlap, especially in optimization problems, which can be approached via either paradigm depending on the context and requirements.

\subsection{Overview of Soft Computing Constituents}
\label{sec:softcomp_overview_of_soft_computing_constituents}

\begin{description}
    \item[Fuzzy Logic:] Deals with \emph{fuzziness} or vagueness, allowing partial membership in sets and approximate reasoning. It is particularly useful when information is incomplete or linguistic in nature.

    \item[Neurocomputing:] Encompasses various neural network architectures (multilayer perceptrons, convolutional networks, recurrent models, Hopfield networks, and Radial Basis Function (RBF) networks) as well as neuromorphic hardware that learn from data and approximate complex nonlinear mappings.

    \item[Probabilistic Reasoning:] Manages uncertainty using probability theory, belief networks, and Bayesian inference. It assumes known or estimable probability distributions.

    \item[Genetic Algorithms:] Inspired by biological evolution, these algorithms perform heuristic search and optimization by mimicking natural selection and genetic variation.
\end{description}

\subsection{Distinguishing Imprecision, Uncertainty, and Fuzziness}
\label{sec:imprecision-fuzziness}

It is important to clarify the subtle differences among these concepts:
\begin{itemize}
    \item \textbf{Uncertainty} refers to situations where the outcome is unknown but can be described probabilistically. For example, a classifier might assign a 60\% probability to a particular class.

    \item \textbf{Imprecision} refers to limited resolution or vagueness in the available descriptions or measurements. Saying that the outside temperature is ``warm'' rather than specifying $24.5^\circ$C is imprecise because we are unsure about the precise boundary that should separate ``warm'' from ``hot.''

\item \textbf{Fuzziness} captures graded membership in a linguistic category; for instance, the extent to which a day is ``warm.'' Membership values range continuously between 0 and 1 instead of forcing a binary decision.

\end{itemize}
In short, imprecision concerns our knowledge about a precise boundary, whereas fuzziness is a property of the concept itself: even with perfect measurements, ``warm'' transitions smoothly into ``hot.''
For example, reading $24.5^\circ$C from a thermometer with $\pm 1^\circ$C resolution is an \emph{imprecise} observation, whereas deciding whether $24.5^\circ$C should be labelled ``warm'' or ``hot'' is a \emph{fuzzy} membership question that remains even if the thermometer were infinitely precise.
\begin{tcolorbox}[title={Imprecision vs. Fuzziness}, colback=gray!5,colframe=gray!40,boxrule=0.4pt]
\textbf{Imprecision} concerns uncertainty about the exact value or boundary (e.g., measurement error or coarse resolution). \textbf{Fuzziness} concerns graded membership in a concept (e.g., the degree to which a day is ``warm'') even when measurements are exact. Probability quantifies uncertainty about events; fuzziness quantifies degree of truth of linguistic predicates.
\end{tcolorbox}
% Chapter 15 (continued)

\subsection{Soft Computing: Motivation and Overview}
\label{sec:softcomp_soft_computing_motivation_and_overview}

Soft computing is not a monolithic framework but rather a coalition of distinct methods unified by a common goal: to exploit tolerance for imprecision, uncertainty, and partial truth to achieve tractability, robustness, and low solution cost. Unlike traditional hard computing, which demands exact inputs and produces precise outputs, soft computing embraces the inherent vagueness of many real-world problems, particularly those involving human reasoning and perception. The constituents mirror the probabilistic and connectionist tools from \Crefrange{chap:supervised}{chap:transformers} but favour interpretability and rule-based reasoning:
\begin{itemize}
    \item \textbf{Fuzzy Logic:} Captures human knowledge and reasoning expressed in linguistic terms, allowing approximate reasoning with imprecise concepts.
    \item \textbf{Neurocomputing (Neural Networks):} Learning from data and pattern recognition; hybrids such as ANFIS \citep{Jang1993} blend fuzzy rules with trainable neural layers.
    \item \textbf{Probabilistic modeling (already covered):} Bayesian/MAP views from \Crefrange{chap:supervised}{chap:logistic} remain complementary to fuzzy possibility views \citep{Dubois1988}, but the emphasis in this block is on rule-based reasoning rather than probabilistic inference.
    \item \textbf{Genetic/Evolutionary Computation:} \Cref{chap:evo} shows how evolutionary search tunes rule bases and membership parameters \citep{Herrera2008,Ishibuchi2007}.
\end{itemize}

\begin{table}[t]
\centering
\caption{Schematic: Fuzzy vs.\ probabilistic reasoning at a glance. Use this when deciding whether your uncertainty is about randomness (probability) or about graded concepts (fuzziness).}
\label{tab:fuzzy-vs-prob}
\small
\begin{tabular}{@{}p{0.23\linewidth} >{\raggedright\arraybackslash}p{0.34\linewidth} >{\raggedright\arraybackslash}p{0.34\linewidth}@{}}
\toprule
 & \textbf{Fuzzy logic} & \textbf{Probabilistic logic} \\
\midrule
\textbf{Semantics} & Degree of membership (vagueness); e.g., ``temperature is high to degree 0.7.'' & Degree of belief/uncertainty---probability that an event occurs. \\
\textbf{Operators} & t\hyp{}norms / s\hyp{}norms (min, product, max) model AND/OR; implication via fuzzy rules. & Sum / product rules, Bayes' theorem govern AND/OR/conditionals. \\
\textbf{Outputs} & Fuzzy sets defuzzified to crisp actions (e.g., heater power). & Numeric probabilities used for expectation, decision thresholds. \\
\textbf{Typical use} & Rule bases, approximate control, linguistic policies. & Stochastic modeling, hypothesis testing, Bayesian inference. \\
\bottomrule
\end{tabular}
\end{table}

\begin{table}[t]
\centering
\caption{Schematic: Boolean operators vs.\ fuzzy operators at a glance. Use this when translating crisp logic rules into graded operators for fuzzy inference.}
\label{tab:boolean-vs-fuzzy}
\small
\begin{tabular}{@{}p{0.23\linewidth} >{\raggedright\arraybackslash}p{0.34\linewidth} >{\raggedright\arraybackslash}p{0.34\linewidth}@{}}
\toprule
 & \textbf{Boolean logic} & \textbf{Fuzzy logic} \\
\midrule
AND & \(\min(a,b)\) & t\hyp{}norm (e.g., min, product) \\
OR & \(\max(a,b)\) & s\hyp{}norm (e.g., max, prob.\ sum) \\
NOT & \(1-a\) & complement \(1-a\) \\
\bottomrule
\end{tabular}
\end{table}
\FloatBarrier

\subsection{Fuzzy Logic: Capturing Human Knowledge Linguistically}
\label{sec:softcomp_fuzzy_logic_capturing_human_knowledge_linguistically}

One of the most compelling aspects of fuzzy logic is its ability to represent human knowledge and experience in a linguistic form that machines can process. Consider the everyday reasoning:

\begin{quote}
\textit{If you wake up late and the traffic is congested, then you will be late.}
\end{quote}

This statement involves vague concepts such as ``late,'' ``congested,'' and ``will be late,'' which are not crisply defined but are intuitively understood by humans. Fuzzy logic allows us to formalize such rules without requiring precise probabilistic models or extensive training data.

\paragraph{Fuzzy Rules and Approximate Reasoning}

A fuzzy rule typically has the form:
\begin{equation}
    \text{IF } A \text{ AND } B \text{ THEN } C,
\label{eq:auto_softcomp_faadf52146}
\end{equation}
where $A$, $B$, and $C$ are fuzzy propositions characterized by membership functions rather than crisp sets.

For example:
\begin{itemize}
    \item $A$: ``Wake up late'' could be represented by a membership function $\mu_{\text{late}}(t)$ over the waking time $t$.
    \item $B$: ``Traffic is congested'' could be represented by a membership function $\mu_{\text{congested}}(x)$ over traffic density $x$.
    \item $C$: ``You will be late'' is the fuzzy output.
\end{itemize}

Each membership function maps from the relevant universe of discourse to $[0,1]$, i.e., $\mu_{\text{late}}: \mathbb{R} \rightarrow [0,1]$, so that linguistic labels become numeric degrees of support. The fuzzy inference system combines these membership values using \emph{t\hyp{}norm} operators (e.g., $\min$, product) to model logical conjunction and \emph{s\hyp{}norms} (e.g., $\max$) to model disjunction, thereby inferring the degree to which the conclusion $C$ holds. In practical systems the resulting fuzzy set is often \emph{defuzzified} (e.g., via centroid or maximum-membership methods) to obtain a single crisp recommendation.

\paragraph{Advantages over Traditional Systems}

Traditional rule-based systems or statistical models require precise numerical inputs or probability distributions. In contrast, fuzzy logic:
\begin{itemize}
    \item Does not require exact numerical data or probability distributions.
    \item Allows direct encoding of expert knowledge in natural language.
    \item Handles imprecision and vagueness inherent in human concepts.
    \item Provides interpretable models that align with human reasoning.
\end{itemize}

\subsection{Comparison with Other Soft Computing Paradigms}
\label{sec:softcomp_comparison_with_other_soft_computing_paradigms}

\paragraph{Neural Networks}

Neural networks model complex nonlinear relationships by learning from data. They transform input features $\mathbf{x} \in \mathbb{R}^n$ into new feature spaces through weighted sums and nonlinear activations:
\begin{equation}
    \mathbf{h} = \sigma(\mathbf{W}^\top \mathbf{x} + \mathbf{b}),
\label{eq:auto_softcomp_8725b12d97}
\end{equation}
where $\mathbf{W} \in \mathbb{R}^{n \times m}$ maps the $n$-dimensional input into an $m$-dimensional hidden space, $\mathbf{b} \in \mathbb{R}^m$ is the bias vector, and $\sigma(\cdot)$ is a nonlinear activation function applied elementwise.

Unlike fuzzy logic, neural networks require training on large datasets and do not inherently provide interpretable linguistic rules; there is, however, an active line of research on \emph{rule extraction} and network distillation aimed at recovering approximate linguistic descriptions from trained models.

\paragraph{Genetic Algorithms}

Genetic algorithms simulate evolutionary processes to optimize solutions by iteratively selecting, recombining, and mutating candidate solutions. They are useful for derivative-free optimization and problems with complex search spaces.

\paragraph{Probabilistic Reasoning}

Probabilistic methods model uncertainty explicitly using probability distributions and Bayesian inference. They require knowledge or estimation of underlying distributions, which may be difficult in many practical scenarios, but approximate inference schemes (e.g., Monte Carlo sampling, variational methods) can mitigate this requirement when exact distributions are unavailable.

\subsection{Zadeh's Insight and the Birth of Fuzzy Logic}
\label{sec:softcomp_zadeh_s_insight_and_the_birth_of_fuzzy_logic}

Lotfi Zadeh, in the late 1960s, observed that classical statistics and probability theory demand precise knowledge of distributions and exact calculations, which is often unrealistic for human decision-making. Humans rely on approximate, linguistic knowledge rather than exact numerical data.

Zadeh's key insight was to develop a mathematical framework that could:
\begin{itemize}
    \item Represent imprecise concepts using fuzzy sets.
    \item Allow approximate reasoning with these fuzzy sets.
    \item Enable machines to operate based on human-like linguistic rules.
\end{itemize}

This approach revolutionized how we model uncertainty and reasoning in artificial intelligence and control systems.

\subsection{Challenges in Fuzzy Logic Systems}
\label{sec:softcomp_challenges_in_fuzzy_logic_systems}

Despite its advantages, fuzzy logic faces several challenges:
\begin{itemize}
    \item \textbf{Lack of a systematic methodology:} Initially, there was no formal mechanism to construct fuzzy inference systems from human knowledge.
    \item \textbf{Handling imprecision in linguistic terms:} Choosing membership functions and linguistic labels still relies on expert elicitation or data-driven tuning; poor choices can degrade system performance.

\end{itemize}
% Chapter 15 (continued)

\subsection{Mathematical Languages as Foundations for Fuzzy Logic}
\label{sec:softcomp_mathematical_languages_as_foundations_for_fuzzy_logic}

Recall that the motivation behind fuzzy logic was to develop a mathematical and linguistic framework capable of handling imprecision and uncertainty in a principled way. To achieve this, Lotfi Zadeh drew inspiration from several well-established mathematical languages, each with its own syntax, semantics, and rules of inference. Understanding these languages helps us appreciate how fuzzy logic extends and generalizes classical logic to accommodate vagueness.

\subsubsection{Relational Algebra}
\label{sec:softcomp_relational_algebra_sub}

Relational algebra is a formal language used primarily in database theory to manipulate sets and relations. It provides operators such as union ($\cup$), intersection ($\cap$), and set difference ($\setminus$) that operate on sets:

\begin{align}
A \cup B &= \{ x \mid x \in A \text{ or } x \in B \}, \\
A \cap B &= \{ x \mid x \in A \text{ and } x \in B \}.
    \label{eq:auto:lecture_8_part_ii:1}
\end{align}

The third canonical operator is the set difference
\[
A \setminus B = \{ x \mid x \in A \text{ and } x \notin B \},
\]
which removes from \(A\) any elements that also belong to \(B\). For instance, if \(A\) is the set of all graduate students and \(B\) the set of teaching assistants, then \(A \setminus B\) contains graduate students who are not currently TAs.

These operators have well-defined meanings and predictable outputs, making relational algebra a precise language for reasoning about collections of elements. The vocabulary is limited but sufficient for set-theoretic operations.

\subsubsection{Boolean Algebra}
\label{sec:softcomp_boolean_algebra_sub}

Boolean algebra is the algebraic structure underlying classical logic and digital circuits. It operates on binary variables taking values in $\{0,1\}$, with logical operators such as \texttt{AND} ($\wedge$), \texttt{OR} ($\vee$), and \texttt{XOR} ($\oplus$):

\begin{align}
A \vee B &= 1 \quad \text{if } A=1 \text{ or } B=1, \\
A \wedge B &= 1 \quad \text{if } A=1 \text{ and } B=1, \\
A \oplus B &= 1 \quad \text{if } A \neq B.
    \label{eq:auto:lecture_8_part_ii:2}
\end{align}
Conversely, \(A \vee B = 0\) only when both inputs are 0, and \(A \wedge B = 0\) unless both inputs equal 1; the XOR operator returns 0 exactly when both operands share the same truth value.

Boolean algebra provides a crisp, binary framework where propositions are either true or false, with no intermediate values. This crispness is a limitation when modeling real-world phenomena involving gradations of truth.

\subsubsection{Predicate Algebra}
\label{sec:softcomp_predicate_algebra_sub}

Predicate algebra extends Boolean algebra by incorporating quantifiers and variables, allowing statements about properties of elements in a domain. For example, a predicate statement might be:

\[
\forall x \in \mathbb{R}, \quad x^2 \geq 0,
\]

which reads: "For all real numbers $x$, $x^2$ is greater than or equal to zero." This language combines logical connectives with quantifiers such as $\forall$ (for all) and $\exists$ (there exists), enabling more expressive statements about sets and relations.

An example involving two domains could be:

\[
\forall x \in \text{Rabbits}, \quad \forall y \in \text{Tortoises}, \quad \text{Faster}(x,y),
\]

meaning "For any rabbit $x$ and any tortoise $y$, $x$ is faster than $y$."

Predicate algebra thus provides a linguistic and symbolic framework to express complex relationships and properties.

\subsubsection{Propositional Calculus}
\label{sec:softcomp_propositional_calculus_sub}

Propositional calculus (or propositional logic) deals with propositions and their logical connectives. It focuses on the relationships between propositions without internal structure. The basic form involves premises and conclusions, such as:

\begin{align}
P \implies Q, \quad P \quad \Rightarrow \quad Q,
    \label{eq:auto:lecture_8_part_ii:3}
\end{align}

where $P$ and $Q$ are propositions, and $\implies$ denotes implication.

\paragraph{Modus Ponens}

One fundamental rule of inference in propositional calculus is \emph{modus ponens}:

\begin{quote}
If $P \implies Q$ and $P$ is true, then $Q$ must be true.
\end{quote}

Symbolically,

\begin{align}
P \implies Q, \quad P \quad \vdash \quad Q.
    \label{eq:auto:lecture_8_part_ii:4}
\end{align}

This rule affirms the consequent by affirming the antecedent.

\paragraph{Modus Tollens}

Another inference rule is \emph{modus tollens}:

\begin{quote}
If $P \implies Q$ and $Q$ is false, then $P$ must be false.
\end{quote}

Symbolically,

\begin{align}
P \implies Q, \quad \neg Q \quad \vdash \quad \neg P.
    \label{eq:auto:lecture_8_part_ii:5}
\end{align}

This rule denies the antecedent by denying the consequent. However, as noted, this inference can sometimes be risky or invalid in practical scenarios due to exceptions or additional factors.

\paragraph{Hypothetical Syllogism}

A further inference pattern is the \emph{hypothetical syllogism}:

\begin{quote}
If $P \implies Q$ and $Q \implies R$, then $P \implies R$.
\end{quote}

Symbolically,

\begin{align}
P \implies Q, \quad Q \implies R \quad \vdash \quad P \implies R.
    \label{eq:auto:lecture_8_part_ii:6}
\end{align}

This transitive property of implication allows chaining of logical statements.

\subsection{Fuzzy Logic as a New Mathematical Language}
\label{sec:softcomp_fuzzy_logic_as_a_new_mathematical_language}

Zadeh's insight was to synthesize these classical mathematical languages into a new framework that could handle degrees of truth rather than binary true/false values. Fuzzy logic generalizes Boolean algebra by allowing truth values to range continuously over the interval $[0,1]$, representing partial truth

% Chapter 15 (continued)

\subsection{Fuzzy Logic: Motivation and Intuition}
\label{sec:softcomp_fuzzy_logic_motivation_and_intuition}

Recall that classical (crisp) logic deals with binary truth values: a proposition is either true (1) or false (0). For example, the question ``Was the exam easy?'' can be answered crisply as ``Yes'' or ``No.'' However, many real-world situations are not so black-and-white. Often, we want to express uncertainty or partial truth, such as ``The exam was somewhat easy,'' or ``The exam was easy to a certain degree.''

\paragraph{Fuzzy truth values} allow us to express such intermediate degrees of truth. Instead of restricting truth values to \(\{0,1\}\), fuzzy logic permits any value in the continuous interval \([0,1]\). For instance, if the exam was moderately easy, we might assign a truth value of \(0.6\) or \(0.7\), indicating partial truth.

This flexibility captures the inherent vagueness in many human concepts and perceptions. For example, when asked ``Did you enjoy your lunch?'' one might respond ``sort of,'' reflecting a fuzzy assessment rather than a crisp yes/no.

\paragraph{Why fuzzy logic?}
\begin{itemize}
    \item \textbf{Tolerance for imprecision:} Observations and measurements are often noisy or uncertain.
    \item \textbf{Expressiveness:} Allows linguistic hedging such as ``somewhat,'' ``maybe,'' or ``approximately.''
    \item \textbf{Robustness:} Systems can handle ambiguous or incomplete information gracefully.
\end{itemize}

\subsection{From Crisp Sets to Fuzzy Sets}
\label{sec:softcomp_from_crisp_sets_to_fuzzy_sets}

\paragraph{Crisp sets} are classical sets where an element either belongs or does not belong to the set. Formally, for a universe \(X\), a crisp set \(A \subseteq X\) is characterized by its \emph{characteristic function}:
\[
\chi_A(x) = \begin{cases}
1 & \text{if } x \in A, \\
0 & \text{if } x \notin A.
\end{cases}
\]

\paragraph{Example:} Consider two classes:
\[
\text{Class 1} = \{\text{Li}, \text{Rajnish}\}, \quad \text{Class 2} = \{\text{Hamid}, \text{John}, \text{Julia}, \text{Yet}\}.
\]
These are crisp sets since no student belongs to both classes simultaneously.

\paragraph{Fuzzy sets} generalize this notion by allowing partial membership. A fuzzy set \( \tilde{A} \) on \(X\) is characterized by a \emph{membership function}:
\[
\mu_{\tilde{A}} : X \to [0,1],
\]
where \(\mu_{\tilde{A}}(x)\) quantifies the degree to which \(x\) belongs to \(\tilde{A}\).

\subsubsection*{Example: Sizes as fuzzy sets}
Consider the linguistic labels \texttt{Small}, \texttt{Medium}, and \texttt{Large} for weights (in kilograms). A crisp partition such as \([0,10],[11,20],[21,30]\) is disjoint; fuzzy sets allow these labels to overlap smoothly so a weight can belong to both \texttt{Medium} and \texttt{Large} to different degrees. See \Cref{chap:fuzzysets} (especially \Cref{sec:weight-membership}) for the explicit membership formulas and plots; here keep the intuition that fuzzy labels overlap and map a universe of discourse into \([0,1]\).

\paragraph{Thermostat at a glance.} Throughout \Crefrange{chap:fuzzysets}{chap:fuzzyinference} we reuse a fuzzy thermostat: inputs are temperature error and rate (linguistic labels such as \texttt{Cold}, \texttt{Warm}, \texttt{Hot}; \texttt{Cooling}, \texttt{Stable}, \texttt{Heating}); rules map these to heater power; defuzzification turns the fuzzy action into a crisp control signal. Keep this loop in mind as membership functions and operators are introduced.

\begin{tcolorbox}[summarybox,title={Lab prep: fuzzy thermostat starter}]
\begin{itemize}
    \item Install \texttt{scikit-fuzzy} and \texttt{matplotlib}.
    \item Define triangular membership functions for \texttt{Cold}/\texttt{Warm}/\texttt{Hot} over a temperature universe; plot the overlap.
    \item Write one rule: IF error is \texttt{Cold} AND rate is \texttt{Heating} THEN power is \texttt{Low}; preview centroid defuzzification.
\end{itemize}
\end{tcolorbox}

\begin{tcolorbox}[summarybox,title={Exercises (\Cref{chap:softcomp})}]
\begin{itemize}
    \item Classify three scenarios as imprecision vs.\ uncertainty vs.\ fuzziness; justify each.
    \item Write two fuzzy thermostat rules and reason qualitatively about the output for a borderline input.
    \item Compare min vs.\ product t\hyp{}norm on the same antecedent degrees (e.g., 0.4 and 0.7); explain the impact.
    \item Sketch (or code) a triangular membership and a simple IF--THEN rule; describe how defuzzification would proceed.
    \item Identify where probability (\Cref{chap:supervised}) and fuzzy possibility (this chapter) would lead to different interpretations.
\end{itemize}
\end{tcolorbox}

\noindent\textbf{Forward pointer.} \Cref{chap:fuzzysets} builds the membership functions and universes for the thermostat inputs/outputs; \Crefrange{chap:fuzzyrelations}{chap:fuzzyinference} assemble full inference and defuzzification, and \Cref{chap:evo} shows how evolutionary search can tune rule bases and memberships.

% Chapter 15: Conclusion and Closure

\subsection{Wrapping Up Fuzzy Sets and Fuzzy Logic}
\label{sec:softcomp_wrapping_up_fuzzy_sets_and_fuzzy_logic}

In this final part of the chapter, we conclude our introduction to fuzzy sets and fuzzy logic by summarizing key concepts and clarifying the open points from the previous discussion.

\paragraph{Fuzzy Sets Recap}

Recall that a \emph{fuzzy set} \( A \) defined on a universe of discourse \( X \) is characterized by a \emph{membership function}
\[
\mu_A: X \to [0,1],
\]
which assigns to each element \( x \in X \) a degree of membership \(\mu_A(x)\) indicating the extent to which \( x \) belongs to the set \( A \). Unlike classical (crisp) sets where membership is binary (0 or 1), fuzzy sets allow partial membership, capturing the inherent vagueness of many real-world concepts.

\paragraph{Universe of Discourse}

The \emph{universe of discourse} \( X \) is the domain over which fuzzy sets are defined. For example, if \( X \) represents the set of all students, fuzzy subsets could be ``tall students,'' ``medium height students,'' and ``short students,'' each with overlapping membership functions reflecting the subjective nature of these categories.

\paragraph{Fuzziness and Degrees of Truth}

Fuzzy logic extends classical Boolean logic by allowing truth values to range continuously between 0 and 1. This enables reasoning with imprecise or approximate information, such as the statement ``the water is warm,'' which is neither absolutely true nor false but has a degree of truthfulness.

\paragraph{Example: Height Classification}

Consider the linguistic variables ``short,'' ``medium,'' and ``tall.'' In classical logic, a person is either short or not, tall or not, with crisp boundaries. In fuzzy logic, these categories overlap, and a person's height can partially belong to multiple categories simultaneously. This reflects human intuition and natural language better than crisp sets.

\paragraph{Fuzzy Actions and Control}

In intelligent control systems, such as automotive braking, fuzzy logic allows the control actions to be fuzzy themselves. Instead of a binary decision to ``hit the brakes'' or ``not hit the brakes,'' the system can decide to apply the brakes ``somewhat,'' ``moderately,'' or ``strongly,'' based on fuzzy inputs like distance and speed. This leads to smoother, more adaptive control.

\paragraph{Next Steps: Membership Functions and Fuzzy Inference Systems}

\Crefrange{chap:fuzzysets}{chap:fuzzyinference} pick up the thermostat running example and formalize each stage: \Cref{chap:fuzzysets} constructs the membership functions for error/rate labels, \Cref{chap:fuzzyrelations} shows how relations and projections move information between universes, and \Cref{chap:fuzzyinference} assembles the full Mamdani/Sugeno inference pipeline. Keep this soft-computing overview handy as a conceptual map while those chapters work through the algebra.

\begin{tcolorbox}[summarybox,title={Key takeaways}]
\begin{itemize}
    \item Soft computing embraces imprecision via fuzzy logic, evolutionary search, and neural networks.
    \item Fuzzy operators (t\hyp{}norms, implications) enable approximate reasoning under uncertainty.
    \item Choosing operators and membership functions matches problem semantics to inference behavior.
\end{itemize}

\medskip
\noindent\textbf{Minimum viable mastery.}
\begin{itemize}
    \item Define what is being approximated (truth values, search steps, or function classes) in each soft-computing pillar.
    \item Explain why operator choices matter (they encode semantics and shape the resulting decision surfaces).
    \item Connect fuzzy-set primitives to the later inference pipeline (fuzzify, combine, imply, aggregate, defuzzify).
\end{itemize}

\noindent\textbf{Common pitfalls.}
\begin{itemize}
    \item Mixing operator families inconsistently across a pipeline and then debugging symptoms at the end.
    \item Treating ``soft'' as a license to skip validation: soft methods still require measurable objectives and checks.
    \item Ignoring scaling and units when defining universes and membership functions.
\end{itemize}
\end{tcolorbox}

\begin{tcolorbox}[summarybox,title={Exercises and lab ideas}]
\begin{itemize}
    \item Implement a minimal example from this chapter and visualize intermediate quantities (plots or diagnostics) to match the pseudocode.
    \item Stress-test a key hyperparameter or design choice discussed here and report the effect on validation performance or stability.
    \item Re-derive one core equation or update rule by hand and check it numerically against your implementation.
\end{itemize}

\medskip
\noindent\textbf{If you are skipping ahead.} When you reach \Crefrange{chap:fuzzysets}{chap:fuzzyinference}, keep the operator choices explicit and consistent. Many formatting and interpretation problems later come from hidden operator defaults.
\end{tcolorbox}

\paragraph{Where we head next.} \Crefrange{chap:fuzzysets}{chap:fuzzyinference} develop the thermostat running example end-to-end: \Cref{chap:fuzzysets} builds membership functions, \Cref{chap:fuzzyrelations} moves information between universes via relations, and \Cref{chap:fuzzyinference} assembles full inference and defuzzification.

\paragraph{References.} Full citations for works mentioned in this chapter appear in the book-wide bibliography.
\nocite{Zadeh1965,DuboisPrade1980,YenLangari1999}

% Chapter 16
\section{Fuzzy Sets and Membership Functions: Foundations and Representations}\label{chap:fuzzysets}
\graphicspath{{assets/lec9/}{assets/lec16/}}

\noindent Building on \Cref{chap:softcomp}, we formalize the foundations of fuzzy logic: universes of discourse, fuzzy sets, membership functions, and the operators used throughout the fuzzy trilogy. These tools define the thermostat labels used later and prepare the transfer/inference steps developed in \Crefrange{chap:fuzzyrelations}{chap:fuzzyinference}.

\begin{tcolorbox}[summarybox, title={Learning Outcomes}]
After this chapter, you should be able to:
\begin{itemize}
  \item Distinguish imprecision (uncertain value/boundary) from fuzziness (graded membership).
  \item Define and interpret membership functions in discrete and continuous domains.
  \item Apply fuzzy set operations and De Morgan's laws using max/min forms.
  \item Execute an end-to-end Mamdani inference and compute the centroid defuzzification.
\end{itemize}
\end{tcolorbox}

\begin{tcolorbox}[summarybox, title={Design motif}]
Treat vagueness as a first-class modeling choice: write the membership functions down, pick operators explicitly, and then audit how those choices shape the behavior of a rule base.
\end{tcolorbox}

\paragraph{Running example checkpoint.}
For the thermostat scenario introduced in \Cref{chap:softcomp}, the universe of discourse for the inputs is \([-5,5]^\circ\text{C}\) temperature error and \([-2,2]^\circ\text{C}/\text{min}\) rate-of-change. As you study triangular, trapezoidal, and Gaussian membership functions, imagine parameterizing the linguistic labels \textit{Cold}, \textit{Comfy}, and \textit{Hot} for these inputs. We reuse those shapes when composing rules in \Crefrange{chap:fuzzyrelations}{chap:fuzzyinference}.

\subsection{Motivating example: designing a membership function from measurements}
\label{sec:fuzzysets_motivating_example_designing_a_membership_function_from_measurements}

Consider the thermostat's temperature error \(e = T_{\text{room}} - T_{\text{set}}\) in degrees Celsius (so \(e<0\) means ``too cold''). We want a linguistic label \textit{Cold} that is clearly true when the room is far below setpoint, clearly false when the room is at/above setpoint, and graded in between. A simple, auditable choice is a piecewise-linear ``shoulder'' membership:
\begin{equation}
    \mu_{\text{Cold}}(e)=
    \begin{cases}
        1, & e \le -4,\\
        \frac{0-e}{4}, & -4 < e < 0,\\
        0, & e \ge 0.
    \end{cases}
    \label{eq:fuzzysets_cold_membership_example}
\end{equation}
With this design, \(e=-2\) yields \(\mu_{\text{Cold}}(-2)=0.5\): ``cold'' is half true. Later chapters reuse this shoulder \emph{shape} but may shift its breakpoints to reflect different sensors, comfort bands, or operating assumptions. This number is not a probability; it is a degree of truth for a linguistic predicate. The rest of this chapter is about making these mappings explicit (shapes, overlaps, and operators) so they can be inspected and tuned rather than assumed.

\subsection{Fuzzy sets and the universe of discourse}
\label{sec:fuzzysets_recap_fuzzy_sets_and_the_universe_of_discourse}

A \emph{fuzzy set} \( A \) in a universe of discourse \( X \) is characterized by a \emph{membership function} \(\mu_A: X \to [0,1]\). This membership function assigns to each element \( x \in X \) a degree of membership \(\mu_A(x)\), which quantifies the extent to which \( x \) belongs to the fuzzy set \( A \).

\begin{itemize}
    \item If \(\mu_A(x) = 1\), then \( x \) fully belongs to \( A \).
    \item If \(\mu_A(x) = 0\), then \( x \) does not belong to \( A \) at all.
    \item If \(0 < \mu_A(x) < 1\), then \( x \) partially belongs to \( A \) to the degree \(\mu_A(x)\).
\end{itemize}

This contrasts with classical (crisp) sets, where membership is binary (either 0 or 1).

\subsection{Membership Functions: Definition and Interpretation}
\label{sec:fuzzysets_membership_functions_definition_and_interpretation}

A \emph{membership function} \(\mu_A(x)\) maps each element \( x \) in the universe \( X \) to a membership grade in the interval \([0,1]\). The shape and parameters of \(\mu_A\) encode the fuzziness or uncertainty associated with the concept represented by \( A \).

\paragraph{Example:} Consider the fuzzy set \textit{Slow Speed} defined over the universe of speeds \( X \subseteq \mathbb{R} \). The membership function \(\mu_{\text{Slow}}(x)\) might assign high membership values to speeds near 20 km/h and gradually decrease as speed increases, reflecting the gradual transition from "slow" to "not slow."

\paragraph{Mathematical Representation:} For each \( x \in X \),
\begin{equation}
    \mu_A(x) \in [0,1].
\label{eq:auto_fuzzysets_13bb7d7cc1}
\end{equation}

The fuzzy set \( A \) can be represented as the collection of ordered pairs:
\begin{equation}
    A = \{ (x, \mu_A(x)) \mid x \in X \}.
    \label{eq:fuzzy_set_ordered_pairs}
\end{equation}

\subsection{Discrete vs. Continuous Universes of Discourse}
\label{sec:fuzzysets_discrete_vs_continuous_universes_of_discourse}

The universe \( X \) can be either discrete or continuous, which affects how fuzzy sets and membership functions are represented.

\subsubsection{Discrete Universe}
\label{sec:fuzzysets_discrete_universe_sub}

When \( X = \{ x_1, x_2, \ldots, x_n \} \) is finite or countable, the fuzzy set \( A \) is represented as a finite collection of ordered pairs:
\begin{equation}
    A = \{ (x_1, \mu_A(x_1)), (x_2, \mu_A(x_2)), \ldots, (x_n, \mu_A(x_n)) \}.
\label{eq:auto_fuzzysets_c35ca1378f}
\end{equation}

Typically, membership values equal to zero are omitted for brevity, since they indicate no membership.

\paragraph{Example:} Suppose \( X = \{1, 2, 3, 4, 5\} \) and the membership function values are:
\[
\mu_A(1) = 0, \quad \mu_A(2) = 0.1, \quad \mu_A(3) = 0.3, \quad \mu_A(4) = 0.7, \quad \mu_A(5) = 0.
\]
Then,
\[
A = \{ (2, 0.1), (3, 0.3), (4, 0.7) \}.
\]

\subsubsection{Continuous Universe}
\label{sec:fuzzysets_continuous_universe_sub}

When \( X \subseteq \mathbb{R} \) is continuous (e.g., an interval), the fuzzy set \( A \) is described by a membership function \(\mu_A(x)\) defined for all \( x \in X \). The representation is functional rather than enumerative:
\begin{equation}
    A = \int_{x \in X} \mu_A(x) / x,
    \label{eq:continuous_fuzzy_set}
\end{equation}
where the notation \(\int \mu_A(x) / x\) denotes the continuous collection of pairs \((x, \mu_A(x))\).

\paragraph{Interpretation:} The integral sign here is symbolic, indicating a continuous aggregation over \( X \), not a numerical integral in the calculus sense.

\paragraph{Example:} Consider a triangular membership function centered at \( c \) with base width \( w \):
\begin{equation}
    \mu_A(x) = \max\left(0, 1 - \frac{|x - c|}{w}\right).
    \label{eq:triangular_mf}
\end{equation}
This function assigns membership 1 at \( x = c \), decreasing linearly to zero at \( x = c \pm w \).

\subsection{Crisp Sets versus Fuzzy Sets}
\label{sec:fuzzysets_crisp_sets_versus_fuzzy_sets}

Crisp (classical) sets assign membership values in the binary set \(\{0,1\}\), so each element either belongs to the set or it does not. In contrast, fuzzy sets allow intermediate membership values, enabling gradual transitions between full inclusion and full exclusion. Understanding this contrast highlights why membership functions are central to fuzzy logic.
\begin{tcolorbox}[title={Imprecision vs. Fuzziness (recap)}, colback=gray!5, colframe=gray!40, boxrule=0.4pt]
As discussed in \Cref{sec:imprecision-fuzziness}, \textbf{imprecision} concerns uncertainty about the exact value or boundary (e.g., measurement noise or coarse resolution), whereas \textbf{fuzziness} concerns graded membership in a concept (e.g., the degree to which a speed is ``slow'') even when measurements are exact. Probability models uncertainty about events; fuzzy logic models degrees of truth of linguistic predicates.
\end{tcolorbox}

\subsection{Membership Functions in Fuzzy Sets}
\label{sec:fuzzysets_membership_functions_in_fuzzy_sets}

With the universe \(X\) fixed and the concept \(A\) named, the membership function \(\mu_A: X \to [0,1]\) assigns to each element \(x \in X\) a degree of membership \(\mu_A(x)\) indicating the extent to which \(x\) belongs to \(A\).

\paragraph{Triangular Membership Function}

One of the simplest and most intuitive membership functions is the \emph{triangular} membership function. It is defined by three parameters \(a < b < c\) and given by
\begin{equation}
\mu_A(x) = \begin{cases}
0, & x \leq a \\
\frac{x - a}{b - a}, & a < x \leq b \\
\frac{c - x}{c - b}, & b < x < c \\
0, & x \geq c
\end{cases}
\label{eq:triangular-mf}
\end{equation}
This function attains its maximum value 1 at \(x = b\), representing the point of highest confidence that \(x\) belongs to the fuzzy set \(A\). The membership decreases linearly on either side of \(b\), reaching zero at \(a\) and \(c\). This shape expresses a strong belief in membership near \(b\) and uncertainty elsewhere.

\paragraph{Trapezoidal Membership Function}

The trapezoidal membership function generalizes the triangular shape by allowing a flat top, representing a range of values with full membership. It is defined by four parameters \(a < b \leq c < d\):
\begin{equation}
\mu_A(x) = \begin{cases}
0, & x \leq a \\
\frac{x - a}{b - a}, & a < x \leq b \\
1, & b < x \leq c \\
\frac{d - x}{d - c}, & c < x < d \\
0, & x \geq d
\end{cases}
\label{eq:trapezoidal-mf}
\end{equation}
This function models situations where there is full confidence that all values between \(b\) and \(c\) belong to the fuzzy set, with gradual transitions on the edges.

\paragraph{Gaussian Membership Function}

The Gaussian membership function is widely used due to its smoothness and differentiability, which are advantageous in optimization and learning algorithms. It is defined by parameters \(c\) (center) and \(\sigma > 0\) (width):
\begin{equation}
\mu_A(x) = \exp\left(-\frac{(x - c)^2}{2\sigma^2}\right).
\label{eq:gaussian-mf}
\end{equation}
This bell-shaped curve smoothly assigns membership values, with the highest membership at \(x = c\) and decreasing membership as \(x\) moves away from \(c\). The parameter \(\sigma\) controls the spread or fuzziness of the set.

\paragraph{Generalized Bell Membership Function}

Another flexible membership function is the generalized bell function, defined by parameters \(a, b, c\):
\begin{equation}
\mu_A(x) = \frac{1}{1 + \left|\frac{x - c}{a}\right|^{2b}}.
\label{eq:bell-mf}
\end{equation}
This function allows control over the width and slope of the membership curve, interpolating between shapes similar to triangular and Gaussian functions.

\subsection{Comparison of Membership Functions}
\label{sec:fuzzysets_comparison_of_membership_functions}

\begin{itemize}
    \item \textbf{Triangular and Trapezoidal:} These are piecewise linear, computationally inexpensive, and easy to interpret. However, they are not differentiable at the vertices, which can be a limitation in gradient-based learning.
    \item \textbf{Gaussian and Bell:} These are smooth and differentiable, making them suitable for optimization and adaptive systems. They provide more modeling flexibility but are computationally more expensive.
\end{itemize}

\paragraph{Example: Grading System as Fuzzy Sets}

Consider a typical university grading scale as an example of fuzzy sets. Traditional crisp sets assign grades as follows:
\[
\text{F}: [0, 59], \quad \text{D}: [60, 69], \quad \text{C}: [70, 79], \quad \text{B}: [80, 89], \quad \text{A}: [90, 100].
\]
In a crisp set, membership is binary: a score of 75 is fully in \(C\) and not in \(B\).

However, students and instructors may perceive these boundaries differently. For example, some may consider 75 to be a borderline \(B\), or 68 to be a borderline \(C\). This uncertainty can be modeled by fuzzy sets with overlapping membership functions.

For instance, the membership function for grade \(C\) could be trapezoidal:
\[
\mu_C(x) = \begin{cases}
0, & x \leq 65, \\
\dfrac{x - 65}{5}, & 65 < x \leq 70, \\
1, & 70 < x \leq 75, \\
\dfrac{80 - x}{5}, & 75 < x \leq 80, \\
0, & x > 80.
\end{cases}
\]
Similarly, the membership for grade \(B\) could be written as
\[
\mu_B(x) = \begin{cases}
0, & x \leq 75, \\
\dfrac{x - 75}{5}, & 75 < x \leq 80, \\
1, & 80 < x \leq 85, \\
\dfrac{90 - x}{5}, & 85 < x \leq 90, \\
0, & x > 90,
\end{cases}
\]
so a borderline score such as \(x=79\) yields \(\mu_C(79)=(80-79)/5=0.2\) and \(\mu_B(79)=(79-75)/5=0.8\): mostly \(B\) but still partially \(C\).
with analogous expressions for the \(A\) and \(D\) categories. These overlapping trapezoids capture the intuition that a borderline score (e.g., \(79\)) can simultaneously belong to both \(C\) and \(B\) to different degrees, as sketched in \Cref{fig:lec9-grade-trapezoids}.
\begin{figure}[ht]
    \centering
    \vspace{0.4em}
\begin{tikzpicture}
    \begin{axis}[
            width=0.72\linewidth,
            height=4.6cm,
            xmin=60, xmax=95,
            ymin=0, ymax=1.05,
            xlabel={Score}, ylabel={Membership},
            xtick={60,65,70,75,80,85,90},
            ytick={0,0.5,1},
            legend style={at={(0.02,0.98)}, anchor=north west, draw=none, fill=none},
            axis background/.style={fill=white},
            clip=true
        ]
            % Grade C trapezoid
            \addplot[thick, cbBlue, name path=C] coordinates {
                (60,0) (65,0) (70,1) (75,1) (80,0) (95,0)
            };
            \addlegendentry{Grade $C$}
            % Grade B trapezoid
            \addplot[thick, cbOrange, name path=B] coordinates {
                (60,0) (75,0) (80,1) (85,1) (90,0) (95,0)
            };
            \addlegendentry{Grade $B$}
            % Shaded overlap region
            \addplot[fill=cbBlue!20, draw=none] fill between[
                of=C and B,
                soft clip={domain=75:80}
            ];
            \node[cbBlue!60!black] at (axis cs:77.5,0.6) {overlap};
        \end{axis}
    \end{tikzpicture}
    % Avoid inline math in captions; it wraps poorly in some EPUB renderers.
    \caption{Trapezoidal membership functions for grades C and B with the overlapping region shaded. Scores near 78--82 partially satisfy both grade definitions. Use it when designing overlapping grade/linguistic bins so boundary cases behave smoothly.}
    \label{fig:lec9-grade-trapezoids}
\end{figure}


\Cref{fig:lec9-membership-overlap} is the overlap diagnostic used when tuning membership coverage.

\subsection{Example: Overlapping weight labels}\label{sec:weight-membership}

Fuzzy labels often overlap so that borderline values belong to multiple sets. For weights measured in kilograms, one crisp partition is \([0,10],[11,20],[21,30]\) for \texttt{Small}, \texttt{Medium}, \texttt{Large}. A fuzzy partition smooths these transitions:
\[
\mu_{\text{Small}}(x) =
\begin{cases}
    1, & x \leq 10, \\
    1 - \dfrac{x-10}{5}, & 10 < x < 15, \\
    0, & x \geq 15,
\end{cases}
\]
\[
\mu_{\text{Medium}}(x) =
\begin{cases}
    0, & x \leq 10, \\
    \dfrac{x-10}{5}, & 10 < x < 15, \\
    1, & 15 \leq x \leq 20, \\
    \dfrac{25-x}{5}, & 20 < x < 25, \\
    0, & x \geq 25,
\end{cases}
\]
and
\[
\mu_{\text{Large}}(x) =
\begin{cases}
    0, & x \leq 20, \\
    \dfrac{x-20}{5}, & 20 < x < 25, \\
    1, & x \geq 25.
\end{cases}
\]
The overlap reflects the vagueness of the labels: a weight near 22~kg partially satisfies both \texttt{Medium} and \texttt{Large}.
For example, at \(x=22\) we have \(\mu_{\text{Medium}}(22)=(25-22)/5=0.6\) and \(\mu_{\text{Large}}(22)=(22-20)/5=0.4\).

\begin{figure}[t]
    \centering
    \begin{tikzpicture}
        \begin{axis}[
            width=0.78\linewidth,
            height=0.36\linewidth,
            xlabel={Weight (kg)},
            ylabel={Membership degree},
            xmin=0, xmax=30,
            ymin=0, ymax=1.05,
            legend style={at={(0.5,1.05)}, anchor=south, legend columns=3}
        ]
            \addplot[cbBlue, thick] coordinates {(0,1) (10,1) (15,0) (30,0)};
            \addlegendentry{Small}
            \addplot[cbOrange, thick] coordinates {(10,0) (15,1) (20,1) (25,0) (30,0)};
            \addlegendentry{Medium}
            \addplot[cbGreen, thick] coordinates {(20,0) (25,1) (30,1)};
            \addlegendentry{Large}
            \addplot[gray, dashed] coordinates {(10,0) (10,1.05)};
            \addplot[gray, dashed] coordinates {(15,0) (15,1.05)};
            \addplot[gray, dashed] coordinates {(20,0) (20,1.05)};
            \addplot[gray, dashed] coordinates {(25,0) (25,1.05)};
        \end{axis}
    \end{tikzpicture}
    \caption[Overlapping membership functions for Small/Medium/Large labels]{Overlapping membership functions for the ``Small'', ``Medium'', and ``Large'' weight labels. The shaded overlaps capture gradual transitions. Use it when tuning membership overlaps so small numeric changes do not cause abrupt rule changes.}
    \label{fig:lec9-membership-overlap}
\end{figure}


\Cref{fig:tnorm-surfaces} compares operator shapes when choosing conjunction behavior in rule aggregation.

\paragraph{Quick plotting snippet.} With \texttt{scikit-fuzzy} or plain \texttt{matplotlib} you can visualize overlaps to debug label choices:
\begin{verbatim}
import numpy as np, matplotlib.pyplot as plt
x = np.linspace(0, 30, 400)
mu_small  = np.clip(1 - (x-10)/5, 0, 1) * (x <= 15)
mu_med    = np.clip((x-10)/5, 0, 1) * (x < 15)
mu_med   += ((x>=15) & (x<=20))
mu_med   += np.clip((25-x)/5, 0, 1) * (x > 20)
mu_large  = np.clip((x-20)/5, 0, 1)
plt.plot(x, mu_small, label="Small")
plt.plot(x, mu_med, label="Medium")
plt.plot(x, mu_large, label="Large")
plt.legend(); plt.show()
\end{verbatim}

\subsection{Fuzzy Sets: Core Concepts and Terminology}
\label{sec:fuzzysets_fuzzy_sets_core_concepts_and_terminology}

Recall that a \emph{fuzzy set} \( A \) on a universe \( X \) is characterized by a membership function \(\mu_A: X \to [0,1]\), where \(\mu_A(x)\) quantifies the degree to which element \( x \) belongs to \( A \). Unlike crisp sets, where membership is binary (0 or 1), fuzzy sets allow partial membership.

\paragraph{Support Set} The \emph{support} of a fuzzy set \( A \) is the set of all elements with nonzero membership:
\begin{equation}
    \mathrm{supp}(A) = \{ x \in X \mid \mu_A(x) > 0 \}.
\label{eq:auto_fuzzysets_4ef37cdc50}
\end{equation}
This set captures all elements that belong to \( A \) to some degree.

\paragraph{Core Set} The \emph{core} of \( A \) is the set of elements fully belonging to \( A \):
\begin{equation}
    \mathrm{core}(A) = \{ x \in X \mid \mu_A(x) = 1 \}.
\label{eq:auto_fuzzysets_459fd72830}
\end{equation}
The core set generalizes the notion of crisp membership to fuzzy sets.

\paragraph{Normality} A fuzzy set \( A \) is said to be \emph{normal} if there exists at least one element \( x \in X \) such that \(\mu_A(x) = 1\). Otherwise, \( A \) is \emph{subnormal}. Normality ensures the fuzzy set has at least one element fully included.

\paragraph{Crossover Points} For many membership functions, especially triangular or trapezoidal shapes, the \emph{crossover points} \( c^-_A \) and \( c^+_A \) are defined as the points where the membership function crosses the value \( 0.5 \):
\begin{equation}
    \mu_A(c^-_A) = \mu_A(c^+_A) = 0.5.
\label{eq:auto_fuzzysets_17a14afb75}
\end{equation}
These points are useful for interpreting the "core region" and the "fuzzy boundary" of the set.

\paragraph{Open and Closed Fuzzy Sets}
- A \emph{left-shoulder set} reaches membership 1 for sufficiently small $x$ and then decreases smoothly (useful for labels such as ``Very Cold'').
- A \emph{right-shoulder set} mirrors this behavior for large $x$ (e.g., ``Very Hot'').
- A \emph{closed fuzzy set} has a membership function that attains 1 only on a bounded interval, typically forming a trapezoidal or triangular shape.

These distinctions help in modeling asymmetric uncertainties or preferences.

\subsection{Probability vs. Possibility}
\label{sec:fuzzysets_probability_vs_possibility}

It is crucial to distinguish between \emph{probability} and \emph{possibility} when interpreting membership functions:

\begin{itemize}
    \item \textbf{Probability} measures the likelihood of an event occurring based on frequency or relative occurrence in repeated trials. Probabilities of mutually exclusive and exhaustive events sum to 1:
    \[
    \sum_i P(E_i) = 1.
    \]
    For example, the probability that a ball drawn from a bag is red, blue, or black sums to 1.

    \item \textbf{Possibility}, on the other hand, measures the degree of plausibility or evidence supporting an event, without requiring additivity or summation to 1. Possibility values reflect uncertainty or vagueness rather than frequency. For example, a doctor's confidence in a surgery's success might be expressed as a possibility of 0.75, indicating a degree of belief rather than a statistical frequency.
\end{itemize}

Thus, membership functions in fuzzy sets represent \emph{possibility} rather than \emph{probability}. This distinction is fundamental in fuzzy logic and inference (cf.~\Cref{tab:fuzzy-vs-prob} in \Cref{chap:softcomp}). When treating a membership as a possibility distribution \(\pi(x)\), we usually normalize so that \(\sup_x \pi(x)=1\), yielding \(\Pi(A)=\sup_{x\in A}\pi(x)\) and \(N(A)=1-\Pi(A^c)\).

\begin{tcolorbox}[summarybox, title={Alpha-cuts, convexity, and fuzzy numbers}]
\begin{itemize}
    \item \textbf{Alpha-cut:} \(A_\alpha = \{x\in X \mid \mu_A(x)\ge \alpha\}\) for \(\alpha\in(0,1]\); \(A_0\) is the support. Alpha-cuts turn fuzzy sets into nested crisp sets.
    \item \textbf{Convex fuzzy set:} \(A\) is convex if every alpha-cut \(A_\alpha\) is convex. Normal + convex fuzzy sets with bounded support are called \emph{fuzzy numbers}.
    \item \textbf{Why it matters:} Alpha-cuts commute with many continuous/monotone maps, making them a practical tool for the extension principle (\Cref{chap:fuzzyrelations}) and for defuzzification (centroid in \Cref{chap:fuzzyinference}).
\end{itemize}
\end{tcolorbox}

\subsection{Fuzzy Set Operations}
\label{sec:fuzzysets_fuzzy_set_operations}

\begin{tcolorbox}[summarybox, title={Operator defaults used in \Crefrange{chap:fuzzysets}{chap:fuzzyinference}}]
Unless stated otherwise, the fuzzy trilogy uses the \emph{standard} operators:
\begin{itemize}
    \item Complement (negation): \(C(\mu) = 1 - \mu\) (so \(C(C(\mu))=\mu\)).
    \item Intersection and union: \(T_{\min}(a, b)=\min(a, b)\), \(S_{\max}(a, b)=\max(a, b)\).
    \item De Morgan's laws are interpreted with this standard complement.
\end{itemize}
Alternative t-/s\hyp{}norms or complements are called out explicitly when they appear.
\end{tcolorbox}

\begin{tcolorbox}[summarybox, title={Notation handoff}]
In the fuzzy trilogy, \(\mu_A(x)\) always denotes membership grade, \(T\) denotes a t\hyp{}norm when generalized conjunction is needed, and \(S\) denotes an s\hyp{}norm for generalized union. If these symbols clash with other parts, use the local chapter meaning and consult \Cref{app:notation_collisions}.
\end{tcolorbox}

Fuzzy logic introduces operations on fuzzy sets that generalize classical set operations but operate on membership functions. Let \( A \) and \( B \) be fuzzy sets on \( X \) with membership functions \(\mu_A\) and \(\mu_B\).

\paragraph{Union} The union \( A \cup B \) is defined by the membership function:
\begin{equation}
    \mu_{A \cup B}(x) = \max\big(\mu_A(x), \mu_B(x)\big).
    \label{eq:fuzzy_union}
\end{equation}
This generalizes the classical union by taking the maximum membership degree at each element.

\paragraph{Intersection} The intersection \( A \cap B \) is defined by:
\begin{equation}
    \mu_{A \cap B}(x) = \min\big(\mu_A(x), \mu_B(x)\big).
    \label{eq:fuzzy_intersection}
\end{equation}
This corresponds to the minimum membership degree, reflecting the degree to which \( x \) belongs to both sets.

\paragraph{Complement} The complement \( A^c \) is given by:
\begin{equation}
    \mu_{A^c}(x) = 1 - \mu_A(x).
    \label{eq:fuzzy_complement}
\end{equation}
This generalizes the classical complement by inverting the membership degree.
Parameterized complements $C_\lambda$ (e.g., Yager, Sugeno classes) are sometimes used to alter the ``steepness'' of the negation; they rarely satisfy involution ($C(C(x)) = x$). A common Sugeno form is
\[
    C_p(\mu) = \frac{1-\mu}{1+p\,\mu}, \quad p \ge 0,
\]
which preserves $C_p(0)=1$ and $C_p(1)=0$ but is involutive only when $p=0$. Whenever strict involution is required (as in many De Morgan identities), we default to the standard complement $C(\mu)=1-\mu$.

\paragraph{Remarks}
These operations satisfy properties analogous to classical set theory but adapted to fuzzy membership values. For completeness, De Morgan's laws in fuzzy logic can be written either as equivalences between sets or explicitly in max/min form:
\begin{align}
    \mu_{(A \cap B)^c}(x) &= \mu_{A^c \cup B^c}(x) = \max\big(1-\mu_A(x),\,1-\mu_B(x)\big), \\
    \mu_{(A \cup B)^c}(x) &= \mu_{A^c \cap B^c}(x) = \min\big(1-\mu_A(x),\,1-\mu_B(x)\big).
    \label{eq:auto:lecture_9:1}
\end{align}
Throughout the book we adopt $\wedge=\min$ and $\vee=\max$ as the default t-/s\hyp{}norm pair with the standard complement \(1-\mu\) (the De Morgan triple used again in \Cref{chap:fuzzyinference} unless noted otherwise); alternative norms appear later in operator tables.
% Chapter 16 (continued)

\paragraph{Reminder on basic operators}

\Crefrange{eq:fuzzy_union}{eq:fuzzy_complement} already define the max/min/standard-complement pair that we use by default. Rather than restate them, we emphasise their practical role: unions aggregate rule consequents, intersections combine antecedents, and complements capture linguistic negations. The thermostat example later in the chapter uses these defaults unless stated otherwise.

\subsection{Graphical Interpretation}
\label{sec:fuzzysets_graphical_interpretation}

For continuous universes, the union and intersection membership functions can be visualized as the pointwise maximum and minimum of the two membership curves, respectively. The complement is obtained by reflecting the membership function about the horizontal line \(\mu = 0.5\): every membership degree \(m\) is mapped to \(1-m\).

\subsection{Additional Fuzzy Set Operations}
\label{sec:fuzzysets_additional_fuzzy_set_operations}

Beyond the basic operations, several other algebraic operations are defined on fuzzy sets:

\paragraph{Algebraic Product}
The algebraic product of fuzzy sets \(A\) and \(B\) is defined by the product of their membership values:
\begin{equation}
    \mu_{A \cdot B}(x) = \mu_A(x) \cdot \mu_B(x), \quad \forall x \in X.
    \label{eq:algebraic_product}
\end{equation}

\paragraph{Scalar Multiplication}
Given a scalar \(\alpha \in [0,1]\), scalar multiplication of a fuzzy set \(A\) is:
\begin{equation}
    \mu_{\alpha A}(x) = \alpha \cdot \mu_A(x), \quad \forall x \in X.
    \label{eq:scalar_multiplication}
\end{equation}

\paragraph{Algebraic Sum}
The algebraic sum of fuzzy sets \(A\) and \(B\) is given by:
\begin{equation}
    \mu_{A + B}(x) = \mu_A(x) + \mu_B(x) - \mu_A(x) \cdot \mu_B(x), \quad \forall x \in X.
    \label{eq:algebraic_sum}
\end{equation}
This operation ensures the resulting membership values remain within \([0,1]\).

\paragraph{Difference}
The difference between fuzzy sets \(A\) and \(B\), denoted \(A - B\), can be defined as:
\begin{equation}
    \mu_{A - B}(x) = \mu_A(x) \wedge \big(1 - \mu_B(x)\big) = \min\big(\mu_A(x), 1 - \mu_B(x)\big),
    \label{eq:fuzzy_difference}
\end{equation}
where \(\wedge\) denotes the minimum operator.

\paragraph{Bounded Difference}
An alternative definition of difference is the bounded difference:
\begin{equation}
    \mu_{A \ominus B}(x) = \max\big(0, \mu_A(x) - \mu_B(x)\big).
    \label{eq:bounded_difference}
\end{equation}

\paragraph{Remarks:}
\begin{itemize}
    \item The difference operation in \eqref{eq:fuzzy_difference} corresponds to the intersection of \(A\) with the complement of \(B\).
    \item The bounded difference in \eqref{eq:bounded_difference} ensures membership values remain non-negative.
    \item These operations extend classical set difference to fuzzy sets, but their interpretations can vary depending on the application.
\end{itemize}

\subsection{Example: Union and Intersection of Fuzzy Sets}
\label{sec:fuzzysets_example_union_and_intersection_of_fuzzy_sets}

Use the pointwise definitions in \eqref{eq:fuzzy_union}--\eqref{eq:fuzzy_intersection} to compute unions or intersections for any pair of fuzzy sets; the next subsection lifts these operations to relations via Cartesian products.
\paragraph{Example.}
Let \(X=\{x_1,x_2,x_3\}\) and define memberships in the order \((x_1,x_2,x_3)\):
\[
\mu_A=\{0.2,\,0.7,\,0.4\}, \qquad \mu_B=\{0.6,\,0.3,\,0.9\}.
\]
Then
\begin{align*}
\mu_{A\cup B} &= \{\max(0.2,0.6),\; \max(0.7,0.3),\; \max(0.4,0.9)\} = \{0.6,\,0.7,\,0.9\},\\
\mu_{A\cap B} &= \{\min(0.2,0.6),\; \min(0.7,0.3),\; \min(0.4,0.9)\} = \{0.2,\,0.3,\,0.4\}.
\end{align*}
The union keeps the larger membership at each point, while the intersection keeps the smaller.

\subsection{Cartesian Product of Fuzzy Sets}
\label{sec:fuzzysets_cartesian_product_of_fuzzy_sets}

Using the membership-function definition from \Cref{sec:fuzzysets_fuzzy_sets_core_concepts_and_terminology}, the \emph{Cartesian product} of two fuzzy sets $A$ on $X$ and $B$ on $Y$ is a fuzzy relation on the product space $X \times Y$.

\paragraph{Definition:} The membership function of the Cartesian product $A \times B$ is defined as
\begin{equation}
    \mu_{A \times B}(x, y) = \min\big(\mu_A(x), \mu_B(y)\big), \quad \forall x \in X, y \in Y.
    \label{eq:cartesian_product}
\end{equation}

This operation generalizes the classical Cartesian product of crisp sets to fuzzy sets by taking the minimum membership grade of the paired elements.

\paragraph{Example:} Suppose
\[
A = \{(x_1, 1.0), (x_2, 0.8), (x_3, 0.4)\}, \quad B = \{(y_1, 0.6), (y_2, 0.8), (y_3, 1.0)\}.
\]
Then the Cartesian product $A \times B$ is represented by the matrix of membership values:
\[
\begin{array}{c|ccc}
\mu_{A \times B}(x, y) & y_1 & y_2 & y_3 \\ \hline
x_1 & \min(1.0, 0.6) = 0.6 & \min(1.0, 0.8) = 0.8 & \min(1.0, 1.0) = 1.0 \\
x_2 & \min(0.8, 0.6) = 0.6 & \min(0.8, 0.8) = 0.8 & \min(0.8, 1.0) = 0.8 \\
x_3 & \min(0.4, 0.6) = 0.4 & \min(0.4, 0.8) = 0.4 & \min(0.4, 1.0) = 0.4
\end{array}
\]

Note that the Cartesian product lifts the fuzzy sets from one-dimensional membership functions to a two-dimensional fuzzy relation.

\subsection{Properties of Fuzzy Set Operations}
\label{sec:fuzzysets_properties_of_fuzzy_set_operations}

The fuzzy set operations (union, intersection, complement) satisfy several important algebraic properties analogous to classical set theory, but defined in terms of membership functions.

\paragraph{Commutativity:}
\begin{align}
    \mu_{A \cap B}(x) &= \mu_{B \cap A}(x), \\
    \mu_{A \cup B}(x) &= \mu_{B \cup A}(x).
    \label{eq:auto:lecture_9:2}
\end{align}

\paragraph{Associativity:}
\begin{align}
    \mu_{(A \cap B) \cap C}(x) &= \mu_{A \cap (B \cap C)}(x), \\
    \mu_{(A \cup B) \cup C}(x) &= \mu_{A \cup (B \cup C)}(x).
    \label{eq:auto:lecture_9:3}
\end{align}

\paragraph{Distributivity:}
\begin{align}
    \mu_{A \cup (B \cap C)}(x) &= \mu_{(A \cup B) \cap (A \cup C)}(x), \\
    \mu_{A \cap (B \cup C)}(x) &= \mu_{(A \cap B) \cup (A \cap C)}(x).
    \label{eq:auto:lecture_9:4}
\end{align}

\paragraph{Identity Elements:}
\begin{align}
    \mu_{A \cup \emptyset}(x) &= \mu_A(x), \\
    \mu_{A \cap X}(x) &= \mu_A(x),
    \label{eq:auto:lecture_9:5}
\end{align}
where $\emptyset$ is the empty fuzzy set with membership zero everywhere, and $X$ is the universal fuzzy set with membership one everywhere.

\paragraph{Involution:}
\begin{equation}
    \mu_{(A^c)^c}(x) = \mu_A(x),
\label{eq:auto_fuzzysets_39167e7aef}
\end{equation}
In operator notation this reads \(C(C(\mu_A(x))) = \mu_A(x)\): applying the complement twice recovers the original membership degree. For the standard fuzzy complement \(C(\mu_A(x)) = 1 - \mu_A(x)\), involution is just the identity
\[
1 - \bigl(1 - \mu_A(x)\bigr) = \mu_A(x),
\]
so the membership ``returns'' to its original value after two applications.

\paragraph{De Morgan's Laws:}
With the standard complement \(A^c\) and the max/min operators in \Crefrange{eq:fuzzy_union}{eq:fuzzy_complement}, the classical De Morgan identities hold:
\((A \cap B)^c = A^c \cup B^c\) and \((A \cup B)^c = A^c \cap B^c\).

These properties ensure that fuzzy set operations behave in a consistent and algebraically sound manner, enabling the extension of classical set theory to fuzzy logic.

\subsection{Fuzzy Set Operators}
\label{sec:fuzzysets_fuzzy_set_operators}

While operations such as union, intersection, and complement define how to combine or modify fuzzy sets, \emph{operators} formalize the logic or rules by which these combinations occur. Operators are mappings that take one or more fuzzy sets and produce another fuzzy set, often encapsulating specific logical or algebraic behavior.

\paragraph{Examples of Operators:}
\begin{itemize}
    \item \textbf{Equality operator:} Checks if two fuzzy sets are equal by comparing membership functions.

\end{itemize}
% Chapter 16 (continued)

\subsection{Complement Operators in Fuzzy Logic}
\label{sec:fuzzysets_complement_operators_in_fuzzy_logic}

In classical logic, the complement of a proposition \( A \) is simply \( 1 - \mu_A(x) \), where \(\mu_A(x)\) is the membership function of \( A \). However, in fuzzy logic, this complement operation can be generalized to allow more flexible modeling of uncertainty and partial membership.

\paragraph{Standard Complement}
The standard complement operator is defined as:
the standard fuzzy negation \(C(\mu)=1-\mu\), so \(\mu_{A^c}(x)=1-\mu_A(x)\) as in \eqref{eq:fuzzy_complement}. This operator is linear and intuitive but may not capture all nuances of uncertainty.

\paragraph{Parameterized Complement Operators}
To generalize the complement, choose a negation operator \(C_p:[0,1]\to[0,1]\) and apply it pointwise: \(\mu_{C_p(A)}(x)=C_p(\mu_A(x))\). One common (Sugeno-type) family is
\begin{equation}
    C_p(\mu) = \frac{1 - \mu}{1 + p \mu}, \qquad p \ge 0,
\label{eq:auto_fuzzysets_dc1115a9e1}
\end{equation}
which reduces to the standard complement when \(p=0\).

Another simple family is a power-law negation:
\begin{equation}
    C_p(\mu) = (1 - \mu)^{p}, \qquad p > 0,
\label{eq:auto_fuzzysets_c976938164}
\end{equation}
which recovers the standard complement when \(p=1\) and adjusts the steepness for other \(p\).

These operators allow for a nonlinear mapping of the complement, reflecting different degrees of confidence or hesitation in the membership values. Unlike the standard complement, most parameterized families \emph{do not} preserve involution \(C(C(\mu))=\mu\) for arbitrary \(p\); they are typically designed to satisfy boundary conditions and monotonicity instead. When strict involution is required, it is safest to use the standard complement.

\begin{figure}[t]
    \centering
    \includegraphics[width=0.9\linewidth]{lec16_fuzzy_and}
    \ifdefined\HCode
        \caption{Fuzzy AND surfaces comparing minimum versus product t\hyp{}norms; analogous OR surfaces show similar differences. Choices here influence rule aggregation in \Cref{chap:fuzzyinference}. Use it when deciding whether conjunction should behave like a conservative minimum or a multiplicative attenuation.}
    \else
        % Avoid inline math in captions; it wraps poorly in some EPUB renderers.
        \caption{Fuzzy AND surfaces comparing minimum versus product t\hyp{}norms; analogous OR surfaces show similar differences. Choices here influence rule aggregation in \Cref{chap:fuzzyinference}. Use it when deciding whether conjunction should behave like a conservative minimum or a multiplicative attenuation.}
    \fi
    \label{fig:tnorm-surfaces}
\end{figure}


\paragraph{Properties of Complement Operators}
A commonly desired set of properties for a complement operator \( C \) is:
\begin{itemize}
    \item \textbf{Boundary conditions:} \( C(0) = 1 \) and \( C(1) = 0 \).
    \item \textbf{Monotonicity:} \( \mu_A(x) \leq \mu_B(x) \implies C(\mu_A(x)) \geq C(\mu_B(x)) \).
    \item \textbf{Involution (optional):} \( C(C(\mu_A(x))) = \mu_A(x) \).
\end{itemize}

The standard complement satisfies all three. Parameterized complements typically satisfy the first two, while involution may be relaxed to gain extra modeling flexibility; one should check involution explicitly if it is required by a particular application.

\subsection{Triangular norms (t\hyp{}
\label{sec:fuzzysets_triangular_norms_t}norms)}
\label{sec:fuzzysets_triangular_norms_t_sec_fuzzysets_triangular_norms_t_norms}

\paragraph{Motivation}
In fuzzy logic, the logical \texttt{AND} operation is generalized by \emph{triangular norms} (t\hyp{}norms). These are binary operators that combine membership values while preserving certain desirable properties analogous to intersection in classical set theory.

\paragraph{Definition}
A \textbf{t\hyp{}norm} is a binary operator \( T: [0,1]^2 \to [0,1] \) satisfying the following properties for all \( x, y, z \in [0,1] \):

\begin{enumerate}
    \item \textbf{Commutativity:}
    \[
        T(x, y) = T(y, x).
    \]
    \item \textbf{Associativity:}
    \[
        T(x, T(y, z)) = T(T(x, y), z).
    \]
    \item \textbf{Monotonicity:}
    \[
        x \leq x', \quad y \leq y' \implies T(x, y) \leq T(x', y').
    \]
    \item \textbf{Boundary condition (Identity):}
    \[
        T(x,1) = x, \quad T(x,0) = 0.
    \]
\end{enumerate}

These properties ensure that \( T \) behaves like a generalized intersection operator.

\paragraph{Examples of t\hyp{}norms}

\begin{itemize}
    \item \textbf{Minimum t\hyp{}norm:}
    \[
        T_{\min}(x, y) = \min(x, y).
    \]
    This corresponds to the classical intersection in fuzzy sets.

    \item \textbf{Algebraic product t\hyp{}norm:}
    \[
        T_{\text{prod}}(x, y) = x \cdot y.
    \]
    This is a smooth, multiplicative generalization of intersection.

    \item \textbf{{\L}ukasiewicz t\hyp{}norm:}
    \[
        T_{\text{Luk}}(x, y) = \max(0, x + y - 1).
    \]
\end{itemize}

Each t\hyp{}norm captures different semantics of conjunction in fuzzy logic.

\paragraph{Interpretation}
The t\hyp{}norm generalizes the classical intersection operator to fuzzy sets by ensuring the output membership value remains within \([0,1]\) and respects the ordering and boundary conditions expected of an intersection.

\subsection{Triangular conorms (t\hyp{}
\label{sec:fuzzysets_triangular_conorms_t}conorms / s\hyp{}norms)}
\label{sec:fuzzysets_triangular_conorms_t_sec_fuzzysets_triangular_conorms_t_conorms_s_norms}

\paragraph{Definition}
The dual concept to t\hyp{}norms is the \textbf{triangular conorm} (t\hyp{}conorm), also called an \emph{s\hyp{}norm}, which generalizes the logical \texttt{OR} operation. A t\hyp{}conorm \( S: [0,1]^2 \to [0,1] \) satisfies:

\begin{enumerate}
    \item \textbf{Commutativity:}
    \[
        S(x, y) = S(y, x).
    \]
    \item \textbf{Associativity:}
    \[
        S(x, S(y, z)) = S(S(x, y), z).
    \]
    \item \textbf{Monotonicity:}
    If \(x \le x'\) and \(y \le y'\), then
    \[
        S(x, y) \le S(x', y').
    \]
    \item \textbf{Boundary conditions:}
    \[
        S(x,0) = x, \qquad S(x,1) = 1.
    \]
\end{enumerate}

These axioms mirror those of t\hyp{}norms but with \(1\) as the neutral element instead of \(0\). Standard examples include the maximum s\hyp{}norm \(S_{\max}(x, y) = \max(x, y)\), the algebraic sum \(S_{\text{sum}}(x, y) = x + y - xy\), and the bounded sum \(S_{\text{bs}}(x, y) = \min(1, x + y)\); explicit formulas and their dual t\hyp{}norms appear in the next subsection.
Note that the algebraic sum explicitly enforces \(S_{\text{sum}}(x, y)=x+y-xy \leq 1\) for all \(x, y \in [0,1]\).

\subsection{T-Norms and S-Norms: Complementarity and Properties}
\label{sec:fuzzysets_t_norms_and_s_norms_complementarity_and_properties}

We use the t\hyp{}norm and s\hyp{}norm definitions from the previous two subsections; here we focus on their complementarity via negation.

An important relationship between t\hyp{}norms and s\hyp{}norms is their complementarity via a negation operator. Throughout this section we use the \emph{standard} fuzzy negation \(N(x) = 1 - x\), so that the complement of \(\mu_A\) is
\[
\mu_{A^c}(x) = 1 - \mu_A(x).
\]

With this explicit choice of negation, the complementarity between \(T\) and \(S\) reads:
\begin{equation}
T(\mu_A(x), \mu_B(x)) = 1 - S(1 - \mu_A(x), 1 - \mu_B(x)),
\label{eq:tnorm-snorm-complement}
\end{equation}
and equivalently,
\[
S(\mu_A(x), \mu_B(x)) = 1 - T(1 - \mu_A(x), 1 - \mu_B(x)).
\]

This duality ensures that the fuzzy intersection and union are consistent with respect to complementation, generalizing classical De Morgan's laws.

\subsection{Examples of common t\hyp{}
\label{sec:fuzzysets_examples_of_common_t}norm/s\hyp{}norm pairs}
\label{sec:fuzzysets_examples_of_common_t_sec_fuzzysets_examples_of_common_t_norm_s_norm_pairs}

Several standard t\hyp{}norms and their corresponding s\hyp{}norms are widely used:

\begin{itemize}
    \item \textbf{Minimum t\hyp{}norm and maximum s\hyp{}norm:}
    \[
    T_{\min}(x, y) = \min(x, y), \quad S_{\max}(x, y) = \max(x, y).
    \]

    \item \textbf{Algebraic product t\hyp{}norm and algebraic sum s\hyp{}norm:}
    \[
    T_{\text{prod}}(x, y) = x \cdot y, \quad S_{\text{sum}}(x, y) = x + y - xy.
    \]

    \item \textbf{Bounded difference t\hyp{}norm and bounded sum s\hyp{}norm:}
    \[
    T_{\text{bd}}(x, y) = \max(0, x + y - 1), \quad S_{\text{bs}}(x, y) = \min(1, x + y).
    \]
\end{itemize}

Each of these pairs satisfies the complementarity relation \eqref{eq:tnorm-snorm-complement}.

\subsection{Fuzzy Set Inclusion and Subset Relations}
\label{sec:fuzzysets_fuzzy_set_inclusion_and_subset_relations}

In classical set theory, \(A \subseteq B\) means every element of \(A\) is also in \(B\). For fuzzy sets, the notion of subset is generalized via membership functions.

\paragraph{Definition (Fuzzy Subset).}
A fuzzy set \(A\) is a \emph{subset} of fuzzy set \(B\), denoted \(A \subseteq B\), if and only if
\[
\mu_A(x) \leq \mu_B(x), \quad \forall x \in X,
\]
where \(X\) is the universe of discourse.

If the inequality is strict for at least one \(x\), i.e., \(\mu_A(x) < \mu_B(x)\) for some \(x\), then \(A\) is a \emph{proper fuzzy subset} of \(B\).

\paragraph{Interpretation:} Since membership functions represent degrees of belonging, the subset relation is graded rather than binary. This leads naturally to the concept of \emph{degree of inclusion}.

\subsection{Degree of Inclusion}
\label{sec:fuzzysets_degree_of_inclusion}

Because fuzzy membership values lie in \([0,1]\), the subset relation can be quantified by a scalar measure indicating \emph{how much} \(A\) is included in \(B\).

For practical work we often use an \emph{aggregate} measure:
\[
\mathrm{incl}(A, B) = \frac{\sum_{x \in X} \min(\mu_A(x), \mu_B(x))}{\sum_{x \in X} \mu_A(x)}
\]
for discrete universes (integrals for continuous, assuming finite mass). It summarizes how much of the mass of \(A\) lies inside \(B\)'s support. A \emph{pointwise} alternative relies on an implicator \(I\) and defines \(\mathrm{Inc}(A, B)=\inf_x I(\mu_A(x),\mu_B(x))\) (see below); implicator-based grades avoid division by small \(\mu_B\) and behave well when \(B\) has zeros. Both constructions satisfy \(0 \leq \mathrm{incl}(A, B) \leq 1\), where 1 means \(A\) is fully included in \(B\).

\subsection{Set Operations and Inclusion Properties}
\label{sec:fuzzysets_set_operations_and_inclusion_properties}

Given fuzzy sets $A$, $B$, and $C$, the following properties hold for the standard t\hyp{}norm and s\hyp{}norm operations:

\begin{itemize}
    \item If $A \subseteq B$, then $A \cap C \subseteq B \cap C$ and $A \cup C \subseteq B \cup C$.
    Explicitly,
    \[
        \mu_{A\cap C}(x) = \min(\mu_A(x),\mu_C(x)) \leq \min(\mu_B(x),\mu_C(x)) = \mu_{B\cap C}(x),
    \]
    and analogously for the union/max operator.
    \item If $A \subseteq B$, applying any t\hyp{}norm $T$ and its dual s\hyp{}norm $S$ preserves inclusion: $T(A, C) \subseteq T(B, C)$ and $S(A, C) \subseteq S(B, C)$. In terms of memberships,
    \[
        \mu_{T(A, C)}(x) \leq \mu_{T(B, C)}(x) \quad \text{and} \quad \mu_{S(A, C)}(x) \leq \mu_{S(B, C)}(x),\; \forall x.
    \]
    \item Complements reverse inclusion: $A \subseteq B \Rightarrow B^c \subseteq A^c$ because complements flip the ordering of memberships.
    \(\mu_{B^c}(x) = 1-\mu_B(x) \leq 1-\mu_A(x) = \mu_{A^c}(x)\).
\end{itemize}
\subsection{Grades of Inclusion and Equality in Fuzzy Sets}
\label{sec:fuzzysets_grades_of_inclusion_and_equality_in_fuzzy_sets}

Recall that in classical set theory, the notion of subset and equality is crisp: a set \( A \) is a subset of \( B \) if every element of \( A \) is also in \( B \), and \( A = B \) if they contain exactly the same elements. In fuzzy set theory, these notions are generalized via \emph{grades} of inclusion and equality, which quantify the degree to which one fuzzy set is included in or equal to another.

\paragraph{Grade of Inclusion}

Given two fuzzy sets \( A \) and \( B \) defined on the universe \( X \), with membership functions \(\mu_A(x)\) and \(\mu_B(x)\), respectively, the \emph{grade of inclusion} of \( A \) in \( B \), denoted \( \mathrm{Inc}(A, B) \), measures how much \( A \) is a subset of \( B \).

One way to define this grade is:
\begin{equation}
\mathrm{Inc}(A, B) = \inf_{x \in X} I\big(\mu_A(x), \mu_B(x)\big),
\label{eq:grade_inclusion}
\end{equation}
where \( I \) is an \emph{implicator} function, often derived from a chosen t\hyp{}norm \(T\). A common choice is the G\"odel implicator:
\[
I(a, b) = \begin{cases}
1, & \text{if } a \leq b, \\
b, & \text{otherwise}.
\end{cases}
\]

Alternatively, if \( T \) is part of a residuated pair \((T, I)\), one sometimes writes
\[
\mathrm{Inc}(A, B) = \inf_{x \in X} T\big(\mu_A(x), \mu_B(x)\big),
\]
which should be interpreted as computing the tightest lower bound obtainable from the chosen \(T\); this coincides with the implicator-based definition when \(I\) is the residuum of \(T\).

\paragraph{Example}

Suppose \( A \) and \( B \) are fuzzy sets with membership functions such that for some \( x \) we have \(\mu_A(x) \leq \mu_B(x)\), and for others \(\mu_A(x) > \mu_B(x)\). Using the G\"odel implicator,
\[
I_G(\mu_A(x),\mu_B(x)) =
\begin{cases}
1, & \mu_A(x) \leq \mu_B(x),\\
\mu_B(x), & \mu_A(x) > \mu_B(x),
\end{cases}
\]
so the overall grade of inclusion is \(\inf_{x \in X} I_G(\mu_A(x),\mu_B(x))\). This explicitly shows how the implicator returns the smaller membership where \(A\) exceeds \(B\).

\paragraph{Grade of Equality}

Similarly, the \emph{grade of equality} between fuzzy sets \( A \) and \( B \), denoted \( \mathrm{Eq}(A, B) \), measures how close the two sets are to being equal. It can be defined as:
\begin{equation}
\mathrm{Eq}(A, B) = \inf_{x \in X} J\big(\mu_A(x), \mu_B(x)\big),
\label{eq:grade_equality}
\end{equation}
where \( J \) is an equality function. One convenient choice is
\[
J(a, b) = \begin{cases}
1, & \text{if } a = b, \\
T(a, b), & \text{otherwise},
\end{cases}
\]
with \( T \) a \( t \)-norm, so that exact agreement receives unit credit while disagreements are down-weighted via \(T\). Other smooth symmetry measures (e.g., \(J(a, b) = 1 - |a-b|\)) can also be used; the key requirement is that \(J\) be symmetric, bounded in \([0,1]\), and reach 1 only when \(a=b\).

This definition allows for a graded notion of equality, reflecting the fuzzy nature of the sets.

\subsection{Dilation and Contraction of Fuzzy Sets}
\label{sec:fuzzysets_dilation_and_contraction_of_fuzzy_sets}

\paragraph{Motivation}

Constructing fuzzy sets with appropriate membership functions is a challenging task. Often, one starts with an initial fuzzy set \( A \) and wishes to generate related fuzzy sets that represent concepts such as "more or less \( A \)" or "somewhat \( A \)". This leads to the operations of \emph{dilation} and \emph{contraction} of fuzzy sets, which modify the membership function to reflect these linguistic hedges.

\paragraph{Definitions}

Given a fuzzy set \( A \) with membership function \(\mu_A(x)\), we introduce two non-negative shape parameters constrained to \(\alpha\ge 1\) (dilation gain) and \(\beta\ge 1\) (contraction gain) so that the resulting hedges behave monotonically:
\begin{align}
\text{Dilation:} \quad & \mu_{A^{(d)}}(x) = \big(\mu_A(x)\big)^{1/\alpha}, \quad \alpha \geq 1, \label{eq:dilation} \\
\text{Contraction:} \quad & \mu_{A^{(c)}}(x) = \big(\mu_A(x)\big)^{\beta}, \quad \beta \geq 1. \label{eq:contraction}
\end{align}
Using separate symbols \(\alpha\) and \(\beta\) avoids the notational clash that occurs when a single parameter \(k\) is forced to satisfy both \(0<k\leq 1\) (for dilation) and \(k \geq 1\) (for contraction). In some references these two operations are also called \emph{expansion} and \emph{narrowing}; we treat the terms as synonyms.

Note that:
\begin{itemize}
    \item For dilation, \(0 < \mu_A(x) < 1\) implies \(\mu_A(x)^{1/\alpha} \geq \mu_A(x)\) when \(\alpha \geq 1\), so every membership value moves closer to 1, making the fuzzy set "larger" or more inclusive. Setting \(\alpha=1\) leaves the set unchanged.
    \item For contraction, \(0 < \mu_A(x) < 1\) implies \(\mu_A(x)^{\beta} \leq \mu_A(x)\) when \(\beta \geq 1\), so the membership values move toward 0, making the fuzzy set "smaller" or more restrictive. Again, \(\beta=1\) recovers the original set.
\end{itemize}

\paragraph{Properties}

\begin{itemize}
    \item The \emph{core} of the fuzzy set, i.e., the elements with membership 1, remains unchanged under dilation or contraction because \(1^{1/\alpha} = 1^{\beta} = 1\) for all positive \(\alpha,\beta\):
    \[
    \mu_A(x) = 1 \implies \mu_{A^{(d)}}(x) = 1 \text{ and } \mu_{A^{(c)}}(x) = 1.
    \]
\end{itemize}
% Chapter 16: Closure and Final Remarks on Membership Functions and Fuzzy Set Operations

\subsection{Closure of Membership Function Derivations}
\label{sec:fuzzysets_closure_of_membership_function_derivations}

This chapter completes the toolkit for generating new membership functions from existing ones via fuzzy-set operations. Membership functions encode graded belonging over a universe of discourse, and algebraic manipulation of those functions is central to fuzzy logic and fuzzy inference.

\subsubsection{Generating New Membership Functions via Set Operations}
\label{sec:fuzzysets_generating_new_membership_functions_via_set_operations_sub}

Given two membership functions, for example, \(\mu_{\text{young}}(x)\) and \(\mu_{\text{old}}(x)\), defined over the same universe \(X\), we can construct new membership functions by applying the following operations:

\paragraph{Dilation (Expansion)}
Dilation increases the support of a fuzzy set, effectively "loosening" the membership criteria. For instance, dilating the \(\text{old}\) membership function yields a new fuzzy set \(\text{more or less old}\):
\[
\mu_{\text{more or less old}}(x) = \text{dilate}(\mu_{\text{old}}(x))
\]
This operation broadens the range of \(x\) values considered "old" to some degree, reflecting linguistic vagueness.

\paragraph{Contraction (Narrowing)}
Contraction tightens the membership function, focusing on a core subset. For example, contracting \(\mu_{\text{old}}(x)\) produces \(\mu_{\text{too old}}(x)\):
\[
\mu_{\text{too old}}(x) = \text{contract}(\mu_{\text{old}}(x))
\]
This captures a stricter notion of "old."

\paragraph{Complement}
The complement of a fuzzy set reverses membership degrees:
\[
\mu_{\text{not } A}(x) = 1 - \mu_A(x)
\]
For example, \(\mu_{\text{not young}}(x) = 1 - \mu_{\text{young}}(x)\).

\paragraph{Intersection}
The intersection of two fuzzy sets corresponds to the minimum of their membership functions:
\[
\mu_{A \cap B}(x) = \min\{\mu_A(x), \mu_B(x)\}
\]
This operation models the logical AND.

\paragraph{Union}
The union corresponds to the maximum:
\[
\mu_{A \cup B}(x) = \max\{\mu_A(x), \mu_B(x)\}
\]

\subsubsection{Examples of Constructed Membership Functions}
\label{sec:fuzzysets_examples_of_constructed_membership_functions_sub}

Using these operations, we can create nuanced fuzzy sets:

\begin{itemize}
    \item \textbf{Not young and not old:}
    \[
    \mu_{\text{not young and not old}}(x) = \min\big(1 - \mu_{\text{young}}(x),\, 1 - \mu_{\text{old}}(x)\big)
    \]
    This set captures individuals who are neither young nor old, representing a middle-aged group.

    \item \textbf{Young but not too old:}
    First, contract \(\mu_{\text{old}}(x)\) to get \(\mu_{\text{too old}}(x)\), then take its complement, and intersect with \(\mu_{\text{young}}(x)\):
    \[
    \mu_{\text{young but not too old}}(x) = \min\big(\mu_{\text{young}}(x),\, 1 - \mu_{\text{too old}}(x)\big)
    \]
    This set isolates those who are young but excludes those considered "too old," refining the concept of youthfulness.

    \item \textbf{More or less old:}
    Applying dilation to \(\mu_{\text{old}}(x)\) expands the fuzzy set:
    \[
    \mu_{\text{more or less old}}(x) = \text{dilate}(\mu_{\text{old}}(x))
    \]
\end{itemize}

\paragraph{Remark on Normality}
Note that some constructed membership functions may not be \emph{normal}, i.e., their maximum membership degree may be less than 1. This reflects the inherent fuzziness and partial membership in linguistic concepts.

\subsection{Implications for Fuzzy Inference Systems}
\label{sec:fuzzysets_implications_for_fuzzy_inference_systems}

The ability to generate new membership functions from a small set of base functions (e.g., \(\mu_{\text{young}}\) and \(\mu_{\text{old}}\)) is powerful. It allows us to encode complex human knowledge and linguistic nuances into fuzzy sets, which can then be used in fuzzy inference systems.

For example, consider an inference system with inputs:
\[
\text{Age} \quad (x), \quad \text{Exercise Level} \quad (e)
\]
and output:
\[
\text{Health Status} \quad (h)
\]

We can define membership functions for \emph{age} (e.g., young, old) and \emph{exercise level} (e.g., low, high), then use fuzzy operators (intersection, union, complement) to combine these inputs according to rules such as:
\[
\begin{aligned}
\text{IF Age is old AND Exercise is high}\\
\text{THEN Health is good}
\end{aligned}
\]
In a Mamdani-style controller the conjunction ``AND'' is typically modeled by the minimum operator and the implication uses the same t\hyp{}norm (i.e., the consequent is clipped at the firing strength). Other choices include using the product t\hyp{}norm for conjunction, Larsen-style scaling for implication, and max for rule aggregation. Any alternative should be stated explicitly.

The next step is to formalize the \emph{implication} and \emph{aggregation} operators that map these fuzzy inputs to fuzzy outputs, and then perform \emph{defuzzification} to obtain crisp outputs.


\begin{table}[t]
\centering
% Avoid inline math in captions; it wraps poorly in some EPUB renderers.
\caption{Typical operator choices in fuzzy inference and their qualitative effects. Here the t\hyp{}norm implements fuzzy AND, the s\hyp{}norm implements fuzzy OR, and the implication shapes consequents. Use it when choosing default operators for a Mamdani-style pipeline and predicting qualitative behavior.}
\label{tab:fuzzy-operators}
\begin{tabularx}{\linewidth}{@{}>{\raggedright\arraybackslash}X>{\raggedright\arraybackslash}X>{\raggedright\arraybackslash}X>{\raggedright\arraybackslash}X@{}}
\toprule
t\hyp{}norm \(T(a, b)\) & s\hyp{}norm \(S(a, b)\) & Implication \(\Rightarrow\) & Qualitative behavior \\
\midrule
\(\min(a, b)\) & \(\max(a, b)\) & Mamdani (clipping: \(\min(\alpha,\mu_B)\)) & Sharp, piecewise-linear surfaces; conservative. \\
\(a\cdot b\) & \(a + b - ab\) & Larsen (scaling: \(\alpha\,\mu_B\)) & Smoother transitions; products damp small activations. \\
\(\max(0, a+b-1)\) & \(\min(1, a+b)\) & Bounded (e.g., {\L}ukasiewicz) & Bounded sums; useful when saturation is desired. \\
\bottomrule
\end{tabularx}
\end{table}

\Cref{tab:fuzzy-operators} serves as the operator-choice checklist in the Mamdani design discussion.

\subsection{Worked Example: Mamdani Fuzzy Inference (End-to-End)}
\label{sec:fuzzysets_worked_example_mamdani_fuzzy_inference_end_to_end}

We illustrate a complete Mamdani pipeline with one antecedent (temperature) and one consequent (fan speed).

\paragraph{Universes and membership functions}
\begin{itemize}
\item Temperature $T \in [0,40]\,^\circ\mathrm{C}$ with fuzzy sets
    \begin{align*}
        \mu_{\text{Cold}}(t) &= \max\!\Big(0, \; 1 - \tfrac{t-0}{15-0}\Big) && (0,0,15),\\
        \mu_{\text{Warm}}(t) &= \max\!\Big(0, \; 1 - \tfrac{|t-20|}{10}\Big) && (10,20,30),\\
        \mu_{\text{Hot}}(t)  &= \max\!\Big(0, \; \tfrac{t-25}{40-25}\Big) && (25,40,40).
    \end{align*}
    \item Fan speed $S \in [0,1]$ with fuzzy sets
    \begin{align*}
        \mu_{\text{Low}}(s)    &= \max\!\Big(0, \; 1 - \tfrac{s-0}{0.5-0}\Big) && (0,0,0.5),\\
        \mu_{\text{Medium}}(s) &= \max\!\Big(0, \; 1 - \tfrac{|s-0.5|}{0.25}\Big) && (0.25,0.5,0.75),\\
        \mu_{\text{High}}(s)   &= \max\!\Big(0, \; \tfrac{s-0.5}{1-0.5}\Big) && (0.5,1,1).
    \end{align*}
\end{itemize}

\paragraph{Rule base}
\begin{enumerate}
    \item IF $T$ is Cold THEN $S$ is Low.
    \item IF $T$ is Warm THEN $S$ is Medium.
    \item IF $T$ is Hot THEN $S$ is High.
\end{enumerate}

\paragraph{Fuzzify input and compute firing strengths}
For an input temperature $T=27\,^{\circ}\!\mathrm{C}$,
\[
\begin{aligned}
\mu_{\text{Cold}}(27) &= 0,\\
\mu_{\text{Warm}}(27) &= \frac{30-27}{10}=0.3,\\
\mu_{\text{Hot}}(27)  &= \frac{27-25}{15}=\frac{2}{15}\approx 0.133.
\end{aligned}
\]
Using min-implication (clipping), the consequents become
\begin{align*}
    \mu^{'}_{\text{Low}}(s)    &= \min\big(0,\; \mu_{\text{Low}}(s)\big) = 0,\\
    \mu^{'}_{\text{Medium}}(s) &= \min\big(0.3,\; \mu_{\text{Medium}}(s)\big),\\
    \mu^{'}_{\text{High}}(s)   &= \min\big(0.133,\; \mu_{\text{High}}(s)\big).
\end{align*}
Aggregating by max yields the overall output fuzzy set
\[
    \mu_{\text{out}}(s) = \max\big(\mu^{'}_{\text{Low}}(s),\; \mu^{'}_{\text{Medium}}(s),\; \mu^{'}_{\text{High}}(s)\big).
\]

\paragraph{Defuzzification (centroid)}
The crisp fan speed is the centroid
\[
    s^{\star} = \frac{\int_0^1 s\, \mu_{\text{out}}(s)\, ds}{\int_0^1 \mu_{\text{out}}(s)\, ds}.
\]
For symmetric triangles, the centroid of a truncated Medium set remains at $0.5$, and the centroid of High is at $\approx 0.833$. Approximating the centroid of the max\hyp{}aggregated set by a convex combination of these centroids weighted by their peak heights,
\[
    s^{\star} \approx \frac{0.3\cdot 0.5 + 0.133\cdot 0.833}{0.3+0.133} \approx 0.58.
\]
An exact centroid can be computed analytically or numerically by integrating the clipped shapes; the approximation above matches a direct trapezoidal integration on a uniform grid (10k points), which yields $s^\star \approx 0.580$ to three decimals. See \Cref{fig:fuzzy-end-to-end} for the membership functions and clipping levels used in this example.
Practical tip: libraries such as \texttt{scikit-fuzzy} provide a tested \texttt{defuzz} (centroid) routine; when in doubt, compute the centroid numerically rather than relying on heuristic convex combinations.

\begin{figure}[t]
    \centering
    \begin{tikzpicture}
        \begin{groupplot}[
            group style={group size=2 by 1, horizontal sep=1.8cm},
            width=0.38\linewidth, height=0.32\linewidth,
            xmin=0, xmax=1, ymin=0, ymax=1.05,
            xlabel={Fan speed $s$}, ylabel={Membership}
        ]
            % Panel A: base sets + clipping
            \nextgroupplot[
                title={(A) Sets and clipping},
                title style={at={(0.5,1.22)}, anchor=south},
                legend style={font=\scriptsize, at={(0.5,1.08)}, anchor=south, legend columns=3}
            ]
            \addplot[cbBlue, thick] coordinates {(0,1) (0.5,0)}; \addlegendentry{Low}
            \addplot[cbOrange, thick] coordinates {(0.25,0) (0.5,1) (0.75,0)}; \addlegendentry{Medium}
            \addplot[cbGreen, thick] coordinates {(0.5,0) (1,1)}; \addlegendentry{High}
            \addplot[gray, dashed] coordinates {(0,0.3) (1,0.3)};
            \addplot[gray, dashed] coordinates {(0,0.133) (1,0.133)};
            \node[font=\scriptsize, anchor=south east, fill=white, inner sep=1pt] at (axis cs:0.98,0.3){$0.3$};
            \node[font=\scriptsize, anchor=north east, fill=white, inner sep=1pt] at (axis cs:0.98,0.133){$0.133$};
            \node[font=\scriptsize, anchor=south west, fill=white, inner sep=1pt] at (axis cs:0.02,0.3){$0.3$};
            \node[font=\scriptsize, anchor=north west, fill=white, inner sep=1pt] at (axis cs:0.02,0.133){$0.133$};
            % Panel B: aggregated output + centroid
            \nextgroupplot[
                title={(B) Aggregated $\mu_\text{out}$ and centroid},
                title style={at={(0.5,1.22)}, anchor=south}
            ]
            % Baselines for fill
            \addplot[name path=axisLeft, draw=none] coordinates {(0.25,0) (0.75,0)};
            \addplot[name path=axisRight, draw=none] coordinates {(0.5,0) (1,0)};
            % Truncated Medium (peak 0.3)
            \addplot[name path=truncMLeft, cbOrange, thick, domain=0.25:0.5, samples=100] { (x-0.25)/0.25 <= 0.3? (x-0.25)/0.25: 0.3 };
            \addplot[name path=truncMRight, cbOrange, thick, domain=0.5:0.75, samples=100] { (0.75-x)/0.25 <= 0.3? (0.75-x)/0.25: 0.3 };
            % Fill under truncated Medium
            \addplot[cbOrange!30, draw=none] fill between[of=truncMLeft and axisLeft];
            \addplot[cbOrange!30, draw=none] fill between[of=truncMRight and axisLeft];
            % Truncated High (peak 0.133)
            \addplot[name path=truncH, cbGreen, thick, domain=0.5:1, samples=100] { (x-0.5)/0.5 <= 0.133? (x-0.5)/0.5: 0.133 };
            % Fill under truncated High
            \addplot[cbGreen!30, draw=none] fill between[of=truncH and axisRight];
            % Aggregated (max) as envelope (approximate by plotting both truncations)
            % Centroid marker
            \addplot[red, dashed] coordinates {(0.58,0) (0.58,0.35)};
            \node[font=\scriptsize, anchor=south, fill=white, inner sep=1pt] at (axis cs:0.58,0.36){$s^{\star}\approx 0.58$};
        \end{groupplot}
    \end{tikzpicture}
    % Avoid inline math in captions; it wraps poorly in some EPUB renderers.
    \caption{End-to-end fuzzy inference example. (A) Consequent membership functions with clipping levels from firing strengths at T = 27 deg C. (B) Aggregated output set (max of truncated consequents) and a centroid marker near s* approx 0.58. Use it when sanity-checking clipping, aggregation, and centroid defuzzification end to end.}
    \label{fig:fuzzy-end-to-end}
\end{figure}


\begin{tcolorbox}[summarybox, title={Key takeaways}]
\begin{itemize}
    \item Fuzzy sets map elements to degrees in \([0,1]\); membership shapes (triangular, trapezoidal, Gaussian) encode semantics.
    \item Support/core and set operations (intersection/union/complement) generalize crisp logic.
    \item Visualizing membership and operations clarifies design of fuzzy controllers.
\end{itemize}

\medskip
\noindent\textbf{Minimum viable mastery.}
\begin{itemize}
    \item Construct a universe of discourse, define overlapping memberships, and compute degrees for concrete inputs.
    \item Apply basic operations (min/max, product, complements) and predict how the choice changes shape.
    \item Translate a design intent (``smooth'', ``conservative'', ``aggressive'') into membership overlap and operator choices.
\end{itemize}

\noindent\textbf{Common pitfalls.}
\begin{itemize}
    \item Setting memberships without checking units/scales, producing labels that never activate or always saturate.
    \item Using too many overlapping labels without justification (hard to tune, hard to interpret).
    \item Tuning by aesthetics alone instead of checking downstream sensitivity and stability.
\end{itemize}
\end{tcolorbox}

\begin{tcolorbox}[summarybox, title={Exercises and lab ideas}]
\begin{itemize}
    \item Define fuzzy labels for a new universe (e.g., vehicle speed); sketch overlapping memberships and compute degrees for 3 sample points.
    \item Using two different t\hyp{}norm/s\hyp{}norm pairs, compute the union/intersection of two fuzzy sets at specific points; comment on differences.
    \item Write memberships for the thermostat error/rate variables (triangular or trapezoidal) and evaluate them at a few inputs.
    \item Plot overlapping memberships using the provided snippet; adjust parameters to see how overlap changes.
\end{itemize}

\medskip
\noindent\textbf{If you are skipping ahead.} The rest of the fuzzy chapters build on these primitives. If you find later rule outputs unstable, the first place to revisit is the universe scaling and overlap choices here.
\end{tcolorbox}

\paragraph{Where we head next.} \Cref{chap:fuzzyrelations} moves from fuzzification/defuzzification mechanics to system-level design and adaptive fuzzy controllers; relation operators and projections there connect these set-level tools to control and hybrid schemes.

% Chapter 17
\section{Fuzzy Set Transformations Between Related Universes}\label{chap:fuzzyrelations}
\graphicspath{{assets/lec10/}}

Building on \Cref{chap:fuzzysets}, we address a fundamental question: \emph{How do we transfer fuzzy knowledge from one universe of discourse to another related universe?} This question arises whenever the same linguistic label must be reused across units, sensors, or derived variables (Celsius vs.\ Fahrenheit; position vs.\ velocity), each with its own universe and membership functions. This chapter develops that transfer layer so \Cref{chap:fuzzyinference} can assemble full inference pipelines.

\begin{tcolorbox}[summarybox,title={Learning Outcomes}]
\begin{itemize}
    \item Apply the extension principle (single and multi-variable) to transport fuzzy knowledge across domains.
    \item Select appropriate t\hyp{}norms/s\hyp{}norms and understand how those choices affect projection, dilation, and composition.
    \item Tie these transformations to the running thermostat/autofocus example to anticipate how inference rules will behave in \Cref{chap:fuzzyinference}.
\end{itemize}
\end{tcolorbox}

\begin{tcolorbox}[summarybox,title={Design motif}]
Preserve meaning while changing representation: the extension principle is a disciplined way to push fuzzy concepts through transformations so downstream inference remains interpretable (see \Cref{chap:fuzzyinference}).
\end{tcolorbox}

\begin{tcolorbox}[summarybox,title={Running example checkpoint}]
We treat the thermostat's heater power as the target universe while the inputs remain error/rate. When mapping ``Comfortable'' from Celsius to Fahrenheit or translating error/rate pairs into control actions, the extension principle tells us how those fuzzy labels transfer; keep that single example in mind as you work through the upcoming dilation and projection formulas.
\end{tcolorbox}

\subsection{Context and Motivation}
\label{sec:fuzzyrelations_context_and_motivation}

Previously, we studied operations such as \emph{dilation} and \emph{contraction} on fuzzy sets within a single universe of discourse. For example, given a fuzzy set representing the concept \textit{young}, we can generate related fuzzy sets like \textit{less young} or \textit{too old} by applying these operations. By combining these fuzzy sets, we can express nuanced concepts such as \textit{not too young} or \textit{not too old} within the same universe.

However, what if we want to extend this reasoning to a \emph{different} universe of discourse that is related to the original one? For instance, consider the following scenarios:

\begin{itemize}
    \item Mapping temperature from Celsius to Fahrenheit.
    \item Transforming a variable \( x \) to \( y = x^2 \).
    \item Relating speed and acceleration to derive new fuzzy sets.
\end{itemize}

In such cases, the new universe is a function of the original universe, and we want to \emph{preserve} and \emph{transfer} the fuzzy knowledge encoded in the original fuzzy sets to the new universe.

\begin{tcolorbox}[summarybox,title={Operator defaults in this trilogy}]
Unless stated otherwise in \Crefrange{chap:fuzzysets}{chap:fuzzyinference}, we use the standard De Morgan triple: \(\wedge=\min\), \(\vee=\max\), and complement \(C(\mu)=1-\mu\). Parameterized complements (Yager/Sugeno) are noted when used, but they generally lose involution (\(C(C(\mu))\neq \mu\)) unless the parameter is zero.
\end{tcolorbox}

\paragraph{Notation.} Throughout this chapter we use the trilogy defaults stated in the box above: \(\wedge=\min\), \(\vee=\max\), and complement \(1-\mu\). When we introduce a general t\hyp{}norm \(T\), it appears explicitly (e.g., in \Cref{eq:extension-two-var} and \Cref{eq:composition_general}). For symbol overloads when reading across parts, see \Cref{app:notation_collisions}.


\subsection{Problem Statement}
\label{sec:fuzzyrelations_problem_statement}

Let \( X \) and \( Y \) be two universes of discourse, with a known mapping function
\[
y = f(x), \quad x \in X, \quad y \in Y.
\]
Suppose we have a fuzzy set \( A \subseteq X \) with membership function \(\mu_A : X \to [0,1]\). We want to define a fuzzy set \( B \subseteq Y \) with membership function \(\mu_B : Y \to [0,1]\) that corresponds to \( A \) under the transformation \( f \).

The key questions are:
\begin{itemize}
    \item How do we compute \(\mu_B(y)\) for each \( y \in Y \)?
    \item How do we handle the fact that multiple \( x \in X \) may map to the same \( y \in Y \)?
    \item How do we combine membership values \(\mu_A(x)\) for all \( x \) such that \( f(x) = y \)?
\end{itemize}
% Operator defaults are stated once at the top of this chapter (see the "Operator defaults in this trilogy" box).

\subsection{Intuition and Challenges}
\label{sec:fuzzyrelations_intuition_and_challenges}

It is tempting to define \(\mu_B(y) = \mu_A(x)\) where \( y = f(x) \), but this is generally insufficient because:

\begin{itemize}
    \item The mapping \( f \) may not be one-to-one; multiple \( x \) values can map to the same \( y \).
    \item Membership values represent degrees of truth or compatibility, not numerical values to be transformed arithmetically.
    \item Simply applying \( f \) to membership values (e.g., squaring them) does not preserve the semantic meaning of membership.
\end{itemize}

Therefore, we need a principled method to aggregate membership values from all preimages of \( y \) under \( f \).

\subsection{Formal Definition of the Transformed Membership Function}
\label{sec:fuzzyrelations_formal_definition_of_the_transformed_membership_function}

Given the fuzzy set \( A \subseteq X \) with membership function \(\mu_A\), and the mapping \( y = f(x) \), the membership function \(\mu_B\) of the fuzzy set \( B \subseteq Y \) is defined by
\begin{equation}
    \mu_B(y) = \sup_{x \in X : f(x) = y} \mu_A(x)
    \label{eq:membership-transform}
\end{equation}
The strongest pre-image membership determines the membership of \(y\). When the mapping depends on multiple fuzzy variables (e.g., \(f(x_1,x_2)\)), the individual memberships are combined with a chosen t\hyp{}norm before taking the supremum, as shown later in \Cref{eq:extension-two-var}.

\paragraph{Remarks:}
\begin{itemize}
    \item The \(\sup\) (supremum) operator generalizes the maximum operator, capturing the highest membership value among all \( x \) mapping to \( y \); when \(X\) is finite the supremum collapses to an ordinary maximum.
    \item If no \( x \in X \) maps to \( y \), then \(\mu_B(y) = 0\).
    \item For single-input transformations no additional t\hyp{}norm is needed; the aggregation shows up only when several input memberships must be combined before mapping through \(f\).
    \item In continuous settings we assume \(f\) is measurable so that the pre-image sets \(\{x \mid f(x)=y\}\) are well-defined and the supremum exists.
\end{itemize}

\subsection{Interpretation}
\label{sec:fuzzyrelations_interpretation}

\Cref{eq:membership-transform} states that the membership degree of \( y \) in \( B \) is the supremum over all membership degrees of \( x \) in \( A \) such that \( f(x) = y \). For single-input mappings no additional combination is necessary; when multiple fuzzy inputs are involved we first combine their memberships with a chosen T-norm (cf. \Cref{eq:extension-two-var}) and then take the supremum. Intuitively, this means:

\begin{quote}
\emph{The degree to which \( y \) belongs to the transformed fuzzy set \( B \) is determined by the strongest membership degree among all \( x \) values that map to \( y \), appropriately combined.}
\end{quote}

This approach preserves the logical interpretation of membership values and respects the structure of the mapping \( f \).

\subsection{Example Setup}
\label{sec:fuzzyrelations_example_setup}

Consider the universe \( X = \mathbb{R} \) and the fuzzy set \( A \)

\subsection{Transformation of Fuzzy Sets Between Universes}
\label{sec:fuzzyrelations_transformation_of_fuzzy_sets_between_universes}

We continue our discussion on fuzzy set transformations, focusing on mapping fuzzy sets from one universe to another via a function \( y = f(x) \).

\paragraph{Example: Mapping via \( y = x^2 \)}

Consider a fuzzy set \( A \) defined on universe \( X = \{-1, 0, 1, 2\} \) with membership values:
\[
\mu_A(-1) = 0.340, \quad \mu_A(0) = 0.141, \quad \mu_A(1) = 0.242, \quad \mu_A(2) = 0.4.
\]
Note that \( A \) is not \emph{normal} because no element achieves membership 1; a fuzzy set is normal precisely when \(\sup_{x \in X} \mu_A(x) = 1\).

Define the transformation \( y = x^2 \). The image universe \( Y \) consists of:
\[
Y = \{0^2, (-1)^2, 1^2, 2^2\} = \{0, 1, 4\}.
\]

To find the membership function \(\mu_B(y)\) of the transformed fuzzy set \( B \) on \( Y \), we use the extension principle:
\[
\mu_B(y) = \sup_{x \in X: f(x) = y} \mu_A(x).
\]

Calculating explicitly:
\begin{align*}
\mu_B(0) &= \mu_A(0) = 0.141, \\
\mu_B(1) &= \max\{\mu_A(-1), \mu_A(1)\} = \max\{0.340, 0.242\} = 0.340, \\
\mu_B(4) &= \mu_A(2) = 0.4.
\end{align*}

Thus, the transformed fuzzy set \( B \) on \( Y \) is:
\[
B = \{(0, 0.141), (1, 0.340), (4, 0.4)\}.
\]

Even on this very small domain the mapping \(f(x)=x^2\) is \emph{many-to-one}, because \(x=-1\) and \(x=1\) both map to \(y=1\); the example therefore highlights how the supremum handles multiple pre-images.
\begin{tcolorbox}[summarybox,title={Worked example: monotone map (Celsius \(\to\) Fahrenheit)}]
Let \(A=\text{Comfortable}_C\) be triangular on Celsius with breakpoints \((21,23,25)\). For the affine map \(f(x)=1.8x+32\), the image \(B=f(A)\) is triangular with breakpoints \(f(21)=69.8\), \(f(23)=73.4\), \(f(25)=77.0\). Because \(f\) is strictly increasing, \(\mu_B(y)=\mu_A(f^{-1}(y))\) and every \(\alpha\)-cut maps directly: \(B_\alpha = f(A_\alpha)\). This is the fastest way to reuse the same linguistic label across units without recomputing via \eqref{eq:membership-transform}.
\end{tcolorbox}
\paragraph{Visual intuition.} \Cref{fig:lec10-extension} walks through a simple mapping \(y=x^2\), showing how memberships on \(X\) lift to memberships on \(Y\) via the supremum over all pre-images that map to the same point.

\begin{figure}[t]
    \centering
    \begin{tikzpicture}
        \begin{groupplot}[
            group style={group size=2 by 1, horizontal sep=1.4cm},
            width=0.44\linewidth,
            height=0.34\linewidth,
            ymin=0, ymax=0.5,
            ylabel={Membership},
            xlabel style={yshift=4pt}
        ]
            \nextgroupplot[
                xlabel={$x$ in $X$},
                xtick={-1,0,1,2}
            ]
                \addplot[cbBlue, thick, mark=*] coordinates {
                    (-1,0.340) (0,0.141) (1,0.242) (2,0.4)
                };
                \node[anchor=south east, font=\scriptsize] at (axis cs:-0.9,0.34){$\mu_A(x)$};
                \addplot[gray,dashed] coordinates {(-1,0) (-1,0.340)};
                \addplot[gray,dashed] coordinates {(2,0) (2,0.4)};
            \nextgroupplot[
                xlabel={$y=f(x)=x^2$},
                xtick={0,1,4}
            ]
                \addplot[cbOrange, thick, mark=square*] coordinates {
                    (0,0.141) (1,0.340) (4,0.4)
                };
                \node[anchor=south east, font=\scriptsize] at (axis cs:0.1,0.34){$\mu_B(y)$};
                \addplot[gray,dashed] coordinates {(1,0) (1,0.340)};
                \addplot[gray,dashed] coordinates {(4,0) (4,0.4)};
        \end{groupplot}
    \end{tikzpicture}
    % Avoid inline math in captions; it wraps poorly in some EPUB renderers.
    \caption{Mapping a fuzzy set through the function ``y = x-squared''. The membership at an output value y is the supremum over all pre-images x that map to y; shared images such as x = +/-1 map to y = 1 using the maximum membership. Use it when applying the extension principle to a non-invertible function.}
    \label{fig:lec10-extension}
\end{figure}

\paragraph{Extension to Multiple Fuzzy Sets}

Suppose now we have two fuzzy sets \( A_1 \) and \( A_2 \) defined on the same universe \( X = \{-1, 0, 1, 2\} \), with membership functions listed in the order \((-1,0,1,2)\):
\[
\mu_{A_1} = \{0.4, 0.7, 0.5, 0.13\}, \quad \mu_{A_2} = \{0.5, 0.1, 0.4, 0.7\}.
\]
Equivalently, for $A_1$ we have $\mu_{A_1}(-1)=0.4$, $\mu_{A_1}(0)=0.7$, $\mu_{A_1}(1)=0.5$, $\mu_{A_1}(2)=0.13$.
For $A_2$ we have $\mu_{A_2}(-1)=0.5$, $\mu_{A_2}(0)=0.1$, $\mu_{A_2}(1)=0.4$, $\mu_{A_2}(2)=0.7$.

Define a function \( y = f(x_1, x_2) = x_1^2 + x_2^2 \), where \( x_1, x_2 \in X \) and their degrees of membership are taken from \(A_1\) and \(A_2\) respectively.\allowbreak

The universe \( Y \) is the set of all possible sums of squares:
\[
Y = \{x_1^2 + x_2^2 \mid x_1, x_2 \in X\}.
\]

For example, some values in \( Y \) include:
\[
0^2 + 0^2 = 0, \quad (-1)^2 + 0^2 = 1, \quad 1^2 + 1^2 = 2, \quad 2^2 + 2^2 = 8, \quad \ldots
\]

\paragraph{Computing Membership Values in \( Y \)}

The membership function \(\mu_B(y)\) is given by Zadeh's extension principle for two variables:
\begin{equation}
\mu_B(y) = \sup_{(x_1, x_2): f(x_1, x_2) = y} \min\{\mu_{A_1}(x_1), \mu_{A_2}(x_2)\}.
\label{eq:extension-two-var}
\end{equation}
The minimum t\hyp{}norm plays the role of the generic operator \(\otimes\); any other t\hyp{}norm could be substituted so long as the same choice is applied throughout the inference pipeline.

\textbf{Example:} Compute \(\mu_B(0)\).

The pairs \((x_1, x_2)\) such that \(x_1^2 + x_2^2 = 0\) are only \((0,0)\). Then,
\[
\mu_B(0) = \min\{\mu_{A_1}(0), \mu_{A_2}(0)\} = \min\{0.7, 0.1\} = 0.1.
\]

\textbf{Example:} Compute \(\mu_B(1)\).

The pairs \((x_1, x_2)\) such that \(x_1^2 + x_2^2 = 1\) are:
\[
(-1,0), \quad (0,-1), \quad (1,0), \quad (0,1).
\]

Calculate the minimum membership values for each pair:
\begin{align*}
    \min\{\mu_{A_1}(-1), \mu_{A_2}(0)\} &= \min\{0.4, 0.1\} = 0.1, \\
    \min\{\mu_{A_1}(0), \mu_{A_2}(-1)\} &= \min\{0.7, 0.5\} = 0.5, \\
    \min\{\mu_{A_1}(1), \mu_{A_2}(0)\} &= \min\{0.5, 0.1\} = 0.1, \\
    \min\{\mu_{A_1}(0), \mu_{A_2}(1)\} &= \min\{0.7, 0.4\} = 0.4.
\end{align*}
Taking the supremum over all contributing pairs gives
\[
\mu_B(1) = \max\{0.1, 0.5, 0.1, 0.4\} = 0.5.
\]
% Chapter 17 Part I (continued)

\subsection{Extension Principle Recap and Projection Operations}
\label{sec:fuzzyrelations_extension_principle_recap_and_projection_operations}

Recall from the previous discussion that the \emph{extension principle} allows us to extend a fuzzy set defined on one universe to another universe via a known function. For example, if we have a fuzzy set \( A \subseteq X \) and a function \( f: X \to Y \), then the image fuzzy set \( B = f(A) \subseteq Y \) is defined by
\begin{equation}
    \mu_B(y) = \sup_{x \in X: f(x) = y} \mu_A(x).
    \label{eq:extension_principle}
\end{equation}
This corresponds to taking the maximum membership value among all preimages of \( y \) under \( f \). In discrete settings the supremum reduces to a maximum over the (finite) preimage of \(y\); for multi-input maps \(f(x_1,\ldots,x_n)\) we first combine input memberships with a chosen t\hyp{}norm and then take the supremum over all tuples mapping to \(y\).

In the continuous universe, this can become challenging because multiple \( x \) values may map to the same \( y \), requiring careful evaluation of the supremum. The extension principle thus generalizes the image of fuzzy sets under arbitrary mappings.

\begin{tcolorbox}[summarybox,title={Computation and discretisation tips}]
For discrete universes the extension principle costs \(O(|X|)\) per \(y\) (or \(O(|X|^n)\) for \(n\)-ary maps) because we evaluate every preimage tuple. In discrete settings \(\sup\) reduces to a \(\max\). Continuous universes require discretisation: sample each input axis on a uniform or adaptive grid (typical 200--500 points per dimension), apply the t\hyp{}norm/aggregation on that mesh, and approximate the supremum via \(\max\). Sparse grids or Monte Carlo sampling reduce the curse of dimensionality; always report the resolution so readers understand numeric fidelity.
\end{tcolorbox}

\begin{tcolorbox}[summarybox,title={Alpha-cuts as an alternative}]
\begin{itemize}
    \item \textbf{Unary monotone \(f\):} \(B_\alpha = f(A_\alpha)\) for every \(\alpha\in(0,1]\); computationally trivial for affine/monotone maps.
    \item \textbf{Non-monotone \(f\):} split \(X\) into monotone pieces \(D_k\); compute \(B_\alpha = \bigcup_k f(A_\alpha \cap D_k)\). This is the standard route for fuzzy arithmetic on fuzzy numbers.
    \item \textbf{When to use:} alpha-cuts are numerically stable for continuous domains and avoid sampling artifacts when \(f\) is smooth; pointwise \(\sup\) is more convenient on discrete grids.
\end{itemize}
\end{tcolorbox}

\begin{figure}[t]
    \centering
    \begin{tikzpicture}
        \begin{axis}[
            width=0.42\linewidth,
            height=0.32\linewidth,
            xlabel={$x$}, ylabel={$\mu_A(x)$},
            ymin=0, ymax=1.05,
            xmin=-2.5, xmax=2.5,
            grid=both,
            minor grid style={gray!15},
            major grid style={gray!30},
            legend style={at={(0.03,0.97)},anchor=north west}
        ]
            % Triangular fuzzy number on X
            \addplot[cbBlue,thick,domain=-2:0]{0};
            \addplot[cbBlue,thick,domain=-2:0]{0};
            \addplot[cbBlue,thick,domain=-1:0]{x+1};
            \addplot[cbBlue,thick,domain=0:1]{1-x};
            \addlegendentry{$A$ on $X$}
            % Alpha cut level
            \addplot[cbOrange,dashed,domain=-2:2]{0.6};
            \node[cbOrange] at (-2.1,0.6) {$\alpha$};
        \end{axis}
        \begin{axis}[
            at={(0.55\linewidth,0)},
            width=0.42\linewidth,
            height=0.32\linewidth,
            xlabel={$y=x^2$}, ylabel={$\mu_B(y)$},
            ymin=0, ymax=1.05,
            xmin=0, xmax=2.5,
            grid=both,
            minor grid style={gray!15},
            major grid style={gray!30},
            legend style={at={(0.97,0.97)},anchor=north east}
        ]
            % Image of triangle under x^2
            \addplot[cbBlue,thick,domain=0:1]{1 - sqrt(x)};
            \addlegendentry{$B=f(A)$ on $Y$}
            % Alpha-cut interval on Y
            \addplot[cbOrange,dashed,domain=0:1.6]{0.6};
            \addplot[cbOrange,very thick] coordinates {(0.16,0.6) (0.36,0.6)};
            \node[cbOrange,anchor=south] at (0.26,0.65) {$B_\alpha$};
        \end{axis}
    \end{tikzpicture}
    % Avoid inline math in captions; it wraps poorly in some EPUB renderers.
    \caption{Alpha-cuts under the non-monotone map ``y = x-squared''. A symmetric triangular fuzzy set on X maps to a right-skewed fuzzy set on Y. Each alpha-cut on A splits into two intervals whose images union to the output alpha-cut. Use it when propagating a fuzzy set through a non-monotone map via alpha-cuts.}
    \label{fig:alpha-cut-nonmonotone}
\end{figure}

\subsection{Projection of Fuzzy Relations}
\label{sec:fuzzyrelations_projection_of_fuzzy_relations}

Now, consider the case where we have a fuzzy relation \( R \subseteq X \times Y \), where \( X \) and \( Y \) are universes of discourse. The fuzzy relation \( R \) is characterized by a membership function
\[
    \mu_R: X \times Y \to [0,1].
\]
This relation can be viewed as a fuzzy set on the Cartesian product \( X \times Y \).

\paragraph{Cartesian Product of Fuzzy Sets}

Given fuzzy sets \( A \subseteq X \) and \( B \subseteq Y \) with membership functions \( \mu_A \) and \( \mu_B \), their Cartesian product \( R = A \times B \) is defined by
\begin{equation}
    \mu_R(x,y) = T(\mu_A(x), \mu_B(y)),
\label{eq:auto_fuzzyrelations_41c4751ec2}
\end{equation}
where \( T \) is a chosen t\hyp{}norm, commonly the minimum operator:
\[
    T(a,b) = \min(a,b).
\]
A \emph{t\hyp{}norm} is any binary operator \(T : [0,1]^2 \to [0,1]\) that is commutative, associative, monotone in each argument, and has \(1\) as identity, so it faithfully generalizes set intersection to graded memberships. Popular choices include the minimum, the product \(ab\), and the \L{}ukasiewicz t\hyp{}norm \(\max(0, a+b-1)\).

\begin{table}[h]
\centering
\caption{Popular t\hyp{}norms and their typical roles. Use it when choosing a default conjunction operator and understanding its qualitative behavior.}
\label{tab:tnorms}
\begin{tabularx}{0.94\linewidth}{@{}>{\raggedright\arraybackslash}p{0.23\linewidth} >{\raggedright\arraybackslash}p{0.3\linewidth} >{\raggedright\arraybackslash}X@{}}
\toprule
\textbf{T-norm} & \textbf{Dual t\hyp{}conorm / identity} & \textbf{When to use} \\
\midrule
Minimum \(T_{\min}(a,b)=\min(a,b)\) & Dual: $\max(a,b)$; idempotent & Linguistic rules mirroring classical AND; preserves interpretability. \\
Product \(T_{\Pi}(a,b)=ab\) & Dual: probabilistic sum \(a+b-ab\) & Smooth gradients, probabilistic semantics, differentiable control. \\
\L{}ukasiewicz \(T_{\text{Luk}}(a,b)=\max(0,a+b-1)\) & Dual: bounded sum $\min(1,a+b)$ & Allows partial satisfaction to accumulate; useful in preference aggregation and graded constraints; tolerates partial violations. \\
\bottomrule
\end{tabularx}
\end{table}

\paragraph{Example}

Suppose
\[
    \mu_A = \{0.5, 0.9\}, \quad \mu_B = \{0.8, 0.9\}.
\]
Then the Cartesian product membership values are
\[
    \mu_R = \begin{bmatrix}
    \min(0.5,0.8) & \min(0.5,0.9) \\
    \min(0.9,0.8) & \min(0.9,0.9)
    \end{bmatrix} = \begin{bmatrix}
    0.5 & 0.5 \\
    0.8 & 0.9
    \end{bmatrix}.
\]
Here the first row corresponds to \(x_1\), the second row to \(x_2\), and the columns correspond to \(y_1\) and \(y_2\). Keeping that indexing explicit avoids ambiguity when reading off the projected membership values.

\paragraph{Projection of Fuzzy Relations}

Often, we are interested in reducing the dimensionality of a fuzzy relation by projecting it onto one of its component universes. The projection operation extracts a fuzzy set on \( X \) or \( Y \) from the fuzzy relation \( R \).

\paragraph{Definition (Projection onto $X$).}
The projection of \( R \) onto \( X \), denoted \( \pi_X(R) \), is defined by
\begin{equation}
    \mu_{\pi_X(R)}(x) = \sup_{y \in Y} \mu_R(x,y).
    \label{eq:projection_x}
\end{equation}


\paragraph{Definition (Projection onto $Y$).}
Similarly, the projection of \( R \) onto \( Y \), denoted \( \pi_Y(R) \), is defined by
\begin{equation}
    \mu_{\pi_Y(R)}(y) = \sup_{x \in X} \mu_R(x,y).
    \label{eq:projection_y}
\end{equation}


\paragraph{Total Projection}

The \emph{total projection} of \( R \) is the maximum membership value over the entire relation:
\begin{equation}
    \mu_{\pi_{\text{total}}(R)} = \sup_{x \in X, y \in Y} \mu_R(x,y).
    \label{eq:total_projection}
\end{equation}

\paragraph{Interpretation}

- The projection onto \( X \) collapses the \( Y \)-dimension by taking the maximum membership value along each fixed \( x \).
- The projection onto \( Y \) collapses the \( X \)-dimension similarly.
- The total projection gives the single highest membership value in the relation.

\begin{figure}[h]
    \centering
    \begin{tabular}{c|ccc}
        & $y_1$ & $y_2$ & $y_3$ \\
        \hline
        $x_1$ & 0.9 & 0.3 & 0.1 \\
        $x_2$ & 0.4 & 0.8 & 0.2 \\
        $x_3$ & 0.1 & 0.6 & 0.5 \\
    \end{tabular}
    \hspace{1em}
    \begin{tabular}{c}
        $\pi_X(R)$ \\
        \hline
        0.9 \\
        0.8 \\
        0.6 \\
    \end{tabular}
    \hspace{1em}
    \begin{tabular}{c}
        $\pi_Y(R)$ \\
        \hline
        0.9 \\
        0.8 \\
        0.5 \\
    \end{tabular}
    % Avoid inline math in captions; it wraps poorly in some EPUB renderers.
    \caption{Illustrative fuzzy relation table (left) together with its projections onto the error universe (middle) and the rate-of-change universe (right). These are the exact quantities used in the running thermostat example before composing rules. Use it when building rule antecedents from a relation and checking which universes each projection lives on.}
    \label{fig:lec17_projection_matrix}
\end{figure}

\paragraph{Example (continued)}

Using the previous example matrix for \( \mu_R \):
\[
    \mu_R = \begin{bmatrix}
    0.5 & 0.5 \\
    0.8 & 0.9
    \end{bmatrix},
\]
we compute
\[
    \mu_{\pi_X(R)} = \{\max(0.5,0.5), \max(0.8,0.9)\} = \{0.5, 0.9\},
\]
\[
    \mu_{\pi_Y(R)} = \{\max(0.5,0.8), \max(0.5,0.9)\} = \{0.8, 0.9\},
\]
and
\[
    \mu_{\pi_{\text{total}}(R)} = \max\{0.5, 0.8, 0.5, 0.9\} = 0.9.
\]

% Chapter 17 Part I (continued)

\subsection{Dimensional Extension and Projection in Fuzzy Set Operations}
\label{sec:fuzzyrelations_dimensional_extension_and_projection_in_fuzzy_set_operations}

In practical fuzzy set operations, it is common to encounter sets defined over different universes of discourse with differing dimensions. For example, consider the union of two fuzzy sets where one is defined over a one-dimensional universe \(X\), and the other over a two-dimensional universe \(X \times Y\). To perform set operations such as union or intersection, the dimensions must be compatible.

\paragraph{Cylindrical Extension}

The \emph{cylindrical extension} is a technique used to extend a fuzzy set defined on a lower-dimensional universe to a higher-dimensional universe by replicating membership values along the new dimension(s).

Suppose we have a fuzzy set \(A \subseteq X\) with membership function \(\mu_A: X \to [0,1]\). To extend \(A\) to \(X \times Y\), define the cylindrical extension \(A^*\) as:
\begin{equation}
    \mu_{A^*}(x,y) = \mu_A(x), \quad \forall x \in X, y \in Y.
    \label{eq:cylindrical_extension}
\end{equation}
This operation "copies" the membership values of \(A\) uniformly along the \(Y\)-dimension, resulting in a fuzzy set over \(X \times Y\).

\paragraph{Projection}

Conversely, the \emph{projection} operation reduces the dimension of a fuzzy set by aggregating membership values over one or more dimensions. For a fuzzy set \(R \subseteq X \times Y\) with membership function \(\mu_R: X \times Y \to [0,1]\), the projection onto \(X\) is again given by \(\mu_{\pi_X(R)}(x) = \sup_{y \in Y} \mu_R(x,y)\) as in \Cref{eq:projection_x}.
This operation captures the maximum membership value over all \(y \in Y\) for each fixed \(x\), effectively "collapsing" the \(Y\)-dimension.

\paragraph{Example}

Consider a fuzzy set \(A\) on \(X = \{x_1, x_2\}\) with membership values \(\mu_A(x_1) = 0.5\), \(\mu_A(x_2) = 0.7\). Extending \(A\) cylindrically to \(X \times Y\) where \(Y = \{y_1, y_2, y_3\}\) yields:
\[
\mu_{A^*}(x_i, y_j) = \mu_A(x_i), \quad i=1,2; \quad j=1,2,3.
\]
Thus, the membership values are replicated along the \(Y\)-axis. In practice this extension step is often paired with projections to reconcile relation dimensions before composing rules and, later, to marginalize the inferred relation back onto the universe of interest.

\subsection{Fuzzy Inference via Composition of Relations}
\label{sec:fuzzyrelations_fuzzy_inference_via_composition_of_relations}

The ultimate goal of building fuzzy logic systems is to perform \emph{inference}, i.e., to compose fuzzy rules to generate predictions or decisions. This involves combining fuzzy relations that represent knowledge or rules.

\paragraph{Setup}

Suppose we have three universes of discourse \(X, Y, Z\), and two fuzzy relations:
\[
R_1 \subseteq X \times Y, \quad R_2 \subseteq Y \times Z,
\]
with membership functions \(\mu_{R_1}(x,y)\) and \(\mu_{R_2}(y,z)\), respectively.

The question is: can we infer a fuzzy relation \(R \subseteq X \times Z\) that relates \(X\) directly to \(Z\) by composing \(R_1\) and \(R_2\)? This is the essence of fuzzy inference.

\paragraph{Composition of Fuzzy Relations}

The composition \(R = R_1 \circ R_2\) is defined by:
\begin{equation}
    \mu_R(x,z) = \sup_{y \in Y} \min \big( \mu_{R_1}(x,y), \mu_{R_2}(y,z) \big).
    \label{eq:fuzzy_composition}
\end{equation}
This is known as the \emph{sup--min composition} (or max--min composition) of fuzzy relations; replacing \(\min\) with another t\hyp{}norm \(T\) swaps in the chosen operator from \Cref{tab:tnorms}.

\paragraph{Interpretation}

- The \(\min\) operator captures the degree to which \(x\) is related to \(y\) and \(y\) is related to \(z\) simultaneously.
- The \(\sup\) (maximum) over all intermediate \(y\) aggregates all possible "paths" from \(x\) to \(z\) through \(y\).

\paragraph{Dimensional Considerations}

Note that \(R_1\) is defined on \(X \times Y\), and \(R_2\) on \(Y \times Z\). The composition yields \(R\) on \(X \times Z\). If the dimensions of the relations differ or if the universes are not aligned, cylindrical extension or projection can be applied to make the dimensions compatible before composition.

\paragraph{Example}

Let \(X = \{x_1, x_2\}\), \(Y = \{y_1, y_2\}\), and \(Z = \{z_1, z_2\}\). Consider
\[
\mu_{R_1} = \begin{bmatrix}
0.2 & 0.9 \\
0.5 & 0.1
\end{bmatrix}, \qquad
\mu_{R_2} = \begin{bmatrix}
0.7 & 0.3 \\
0.4 & 0.8
\end{bmatrix}.
\]
Using the max--min composition,
\begin{align*}
\mu_R(x_1,z_1) &= \max\{\min(0.2,0.7), \min(0.9,0.4)\} = \max\{0.2, 0.4\} = 0.4, \\
\mu_R(x_1,z_2) &= \max\{\min(0.2,0.3), \min(0.9,0.8)\} = \max\{0.2, 0.8\} = 0.8, \\
\mu_R(x_2,z_1) &= \max\{\min(0.5,0.7), \min(0.1,0.4)\} = \max\{0.5, 0.1\} = 0.5, \\
\mu_R(x_2,z_2) &= \max\{\min(0.5,0.3), \min(0.1,0.8)\} = \max\{0.3, 0.1\} = 0.3.
\end{align*}
Therefore
\[
\mu_R = \begin{bmatrix}
0.4 & 0.8 \\
0.5 & 0.3
\end{bmatrix}.
\]

\begin{tcolorbox}[summarybox,title={Max--min composition as ``fuzzy matrix multiply''}]
Given \(R_1 \in [0,1]^{|X|\times|Y|}\) and \(R_2 \in [0,1]^{|Y|\times|Z|}\),
\begin{verbatim}
for i in range(|X|):
    for k in range(|Z|):
        acc = 0
        for j in range(|Y|):
            acc = max(acc, min(R1[i,j], R2[j,k]))
        R[i,k] = acc
return R  # the composition R1 o R2
\end{verbatim}
Swap \(\min\) for another \(T\) (product, \L{}ukasiewicz) and \(\max\) for the corresponding s\hyp{}norm to instantiate other composition families.
\end{tcolorbox}
% Chapter 17 Part I: Closure of Composition and Fuzzy Relation Operations

\subsection{Recap and Motivation}
\label{sec:fuzzyrelations_recap_and_motivation}

Earlier in this chapter, we introduced fuzzy relations and their compositions, focusing on max--min composition as a fundamental operation. We saw how fuzzy relations can represent uncertain or imprecise mappings between sets, and how compositions allow chaining these relations to infer new relationships.

The goal of this final part is to wrap up the derivations related to fuzzy relation composition, clarify the generalization of these operations, and highlight key properties that enable their effective use in fuzzy inference systems.

\subsection{Generalization of Fuzzy Relation Composition}
\label{sec:fuzzyrelations_generalization_of_fuzzy_relation_composition}

Suppose we have two fuzzy relations:
\[
R_1 \subseteq X \times Y, \quad R_2 \subseteq Y \times Z,
\]
with membership functions \(\mu_{R_1}(x,y)\) and \(\mu_{R_2}(y,z)\), respectively.

The \emph{composition} \(R = R_1 \circ R_2\) is a fuzzy relation from \(X\) to \(Z\) defined by:
\begin{equation}
\mu_R(x,z) = \sup_{y \in Y} T\big(\mu_{R_1}(x,y), \mu_{R_2}(y,z)\big),
\label{eq:composition_general}
\end{equation}
where \(T\) is a chosen t\hyp{}norm (triangular norm) representing fuzzy conjunction (e.g., minimum, product). Recall that a t\hyp{}norm \(T: [0,1]^2 \to [0,1]\) is commutative, associative, monotone in each argument, and satisfies \(T(a,1)=a\); popular choices include the minimum, product, and \L{}ukasiewicz operators.

\paragraph{Max--min Composition:} The most common choice is the max--min composition where
\[
T(a,b) = \min(a,b),
\]
and the supremum is replaced by maximum:
\[
\mu_R(x,z) = \max_{y \in Y} \min\big(\mu_{R_1}(x,y), \mu_{R_2}(y,z)\big).
\]

This operation generalizes the classical composition of crisp relations to fuzzy sets.

\subsection{Example Calculation of Composition}
\label{sec:fuzzyrelations_example_calculation_of_composition}

Consider discrete sets \(X = \{x_1, x_2\}\), \(Y = \{y_1, y_2\}\), and \(Z = \{z_1, z_2\}\), with membership values:

\[
\mu_{R_1} =
\begin{bmatrix}
0.5 & 0.6 \\
0.5 & 0.5
\end{bmatrix}, \quad
\mu_{R_2} =
\begin{bmatrix}
0.5 & 0.1 \\
0.2 & 0.5
\end{bmatrix},
\]
where rows correspond to \(X\) or \(Y\) elements and columns to \(Y\) or \(Z\) elements respectively.

To compute \(\mu_R(x_1, z_1)\), we evaluate:
\[
\mu_R(x_1, z_1) = \max \left\{ \min(0.5, 0.5), \min(0.6, 0.2) \right\} = \max \{0.5, 0.2\} = 0.5.
\]

Similarly, for \(\mu_R(x_1, z_2)\):
\[
\mu_R(x_1, z_2) = \max \left\{ \min(0.5, 0.1), \min(0.6, 0.5) \right\} = \max \{0.1, 0.5\} = 0.5.
\]

This process continues for all pairs \((x_i, z_j)\) to form the composed relation matrix.

\subsection{Properties of Fuzzy Relation Composition}
\label{sec:fuzzyrelations_properties_of_fuzzy_relation_composition}

The composition operation inherits several important algebraic properties, analogous to classical relations:

\begin{itemize}
    \item \textbf{Associativity:} For fuzzy relations \(R_1, R_2, R_3\),
    \[
    (R_1 \circ R_2) \circ R_3 = R_1 \circ (R_2 \circ R_3).
    \]
    This allows chaining multiple relations without ambiguity.

    \item \textbf{Non-commutativity:} Generally,
    \[
    R_1 \circ R_2 \neq R_2 \circ R_1,
    \]
    reflecting the directional nature of relations.

    \item \textbf{Distributivity:} Composition distributes over union:
    \[
    R_1 \circ (R_2 \cup R_3) = (R_1 \circ R_2) \cup (R_1 \circ R_3).
    \]

    \item \textbf{De Morgan's Laws and Inclusion:} These extend naturally to fuzzy relations and their complements, intersections, and unions.

\end{itemize}

\subsection{Alternative Composition Operators}
\label{sec:fuzzyrelations_alternative_composition_operators}

While max--min is standard, other t\hyp{}norms and t\hyp{}conorms can be used to define composition:

\begin{itemize}
    \item \textbf{Max-Product Composition:}
    \[
    \mu_R(x,z) = \max_{y} \left( \mu_{R_1}(x,y) \cdot \mu_{R_2}(y,z) \right).
    \]

    \item \textbf{Max-Average or Other Aggregations:} Depending on application needs, different norms can be used to model conjunction and aggregation.

\end{itemize}

\begin{tcolorbox}[summarybox,title={Author's note: choosing an operator family}]
Start with max--min when safety and monotonicity matter; its outputs stay within the tightest support and preserve ordering. Swap to max--product or algebraic t\hyp{}norms when you need smoother surfaces or when small disagreements should be penalized multiplicatively (e.g., sensor fusion). If the resulting surfaces are too flat, tighten the t\hyp{}norm; if they are too brittle, loosen it. Operator choice is an engineering dial, not an article of faith.
\end{tcolorbox}

The choice of composition operator therefore follows a practical rule: begin with max--min for its interpretability and stability, and reach for the alternatives catalogued in \Cref{tab:tnorms} only when the application demands smoother or more aggressive aggregation.

\begin{tcolorbox}[summarybox,title={Key takeaways}]
\begin{itemize}
    \item The extension principle transfers fuzzy sets across related universes via functions \(y=f(x)\).
    \item Multiple preimages require aggregation (e.g., \(\sup\) over inverse mappings with a chosen t\hyp{}norm).
    \item Clear notation and figures (domains, mappings) prevent ambiguity in fuzzy transformations.
\end{itemize}

\medskip
\noindent\textbf{Minimum viable mastery.}
\begin{itemize}
    \item Given \(y=f(x)\), compute \(\mu_Y(y)\) via inverse mappings and a chosen aggregation rule.
    \item State when the mapping is one-to-one vs.\ many-to-one and how that changes the computation.
    \item Track domain restrictions explicitly so transformed sets respect feasibility (e.g., square-root domains).
\end{itemize}

\noindent\textbf{Common pitfalls.}
\begin{itemize}
    \item Dropping multi-preimage cases and silently producing overconfident outputs.
    \item Hiding domain restrictions, leading to membership mass on invalid regions.
    \item Mixing notations for \(T\), \(\sup\), and complements across examples (hard to audit later).
\end{itemize}
\end{tcolorbox}

\begin{tcolorbox}[summarybox,title={Exercises and lab ideas}]
\begin{itemize}
    \item Implement a minimal example from this chapter and visualize intermediate quantities (plots or diagnostics) to match the pseudocode.
    \item Stress-test a key hyperparameter or design choice discussed here and report the effect on validation performance or stability.
    \item Re-derive one core equation or update rule by hand and check it numerically against your implementation.
\end{itemize}

\medskip
\noindent\textbf{If you are skipping ahead.} The extension principle is the bookkeeping layer for the full inference pipeline. When you reach \Cref{chap:fuzzyinference}, every ``rule output'' is an instance of the same transfer-and-aggregate pattern.
\end{tcolorbox}

\paragraph{Where we head next.} \Cref{chap:fuzzyinference} operationalizes these relation tools: each rule induces a fuzzy relation on input-output space, then sup-$T$ composition and projection produce the implied output set before aggregation and defuzzification. The same defaults (max--min with standard complement) carry over, as summarized in \crefrange{eq:antecedent_min}{eq:output_aggregation}.

\paragraph{References.} Full citations for works mentioned in this chapter appear in the book-wide bibliography.
\nocite{Zadeh1975,BandlerKohout1980,KlirYuan1995,Dubois1988,Klement2000}

% Chapter 18
\section{Fuzzy Inference Systems: Rule Composition and Output Calculation}\label{chap:fuzzyinference}
\markboth{Fuzzy inference}{}

\begin{tcolorbox}[summarybox,title={Learning Outcomes}]
\begin{itemize}
    \item Execute full Mamdani/Larsen style inference: antecedent aggregation, implication, aggregation, and defuzzification.
    \item Compare implication/aggregation choices (product vs.\ min, max vs.\ sum) and their impact on the running thermostat/autofocus example.
    \item Contrast Mamdani systems with Sugeno/Takagi--Sugeno systems to know when weighted-average consequents are preferable.
\end{itemize}
\end{tcolorbox}

\begin{tcolorbox}[summarybox,title={Running example checkpoint}]
For the thermostat, each rule combines the temperature error (\textit{Cold}, \textit{Slightly Warm}, \ldots) and rate-of-change to set heater power. As you work through antecedent aggregation, implication, and defuzzification, keep one concrete rule base (e.g., ``IF error is Cold AND rate is Falling THEN heater power is High'') in mind; the formulas below map directly onto that setup.
\end{tcolorbox}

Throughout this chapter we keep the trilogy defaults from \Crefrange{chap:fuzzysets}{chap:fuzzyrelations}: \(\wedge=\min\), \(\vee=\max\), and the standard complement \(1-\mu\). Aggregation over rule consequents defaults to the max s\hyp{}norm unless stated otherwise; alternatives live in \Cref{tab:tnorms}.

Building on \Cref{chap:fuzzyrelations}, where we developed transfer operators (projection, composition, and the extension principle) to move fuzzy information between related universes, we now assemble complete fuzzy inference systems (FIS): rule composition and output calculation. The roadmap in \Cref{fig:roadmap} shows this as the inference step that turns fuzzy sets into decisions.

\begin{tcolorbox}[summarybox,title={Design motif}]
Local linguistic rules become a global behavior only after you commit to concrete operators (t\hyp{}norm, implication, aggregation, defuzzifier) and then sanity-check the resulting surface.
\end{tcolorbox}

\subsection{Context and Motivation}
\label{sec:fuzzyinference_context_and_motivation}

Recall that a fuzzy inference system maps crisp inputs to fuzzy outputs by applying a set of fuzzy rules. Each rule typically has the form:

\begin{quote}
\textit{If} $x_1$ is $A_1$ \textit{and} $x_2$ is $A_2$ \textit{and} $\cdots$ \textit{then} $y$ is $B$,
\end{quote}

where $A_i$ and $B$ are fuzzy sets defined on the respective universes of discourse. The antecedent (premise) combines multiple fuzzy conditions on inputs, and the consequent (conclusion) specifies the fuzzy output.

\begin{tcolorbox}[summarybox,title={Author's note: rules as lived experience}]
These rules are not immutable physical laws; they are codified experience. We record facts such as ``if it is morning then the sun is in the east,'' yet real observations may arrive at noon. Fuzzy inference exists to bridge that gap: observed memberships are composed with stored rules so that slight deviations in the antecedent produce softened consequents instead of brittle yes/no responses. When you carry out the algebra below, keep that picture of ``experience vs.\ observation'' in mind.
\end{tcolorbox}

The key challenge is to systematically combine the antecedent fuzzy sets and then infer the output fuzzy set for each rule, before aggregating all rules to produce a final output.

\subsection{Rule Antecedent Composition}
\label{sec:fuzzyinference_rule_antecedent_composition}

Given a rule with $n$ antecedents, each associated with a fuzzy set $A_i$ and an input value $x_i$, the degree to which the rule is activated (also called the \emph{firing strength}) is computed by combining the membership values of each antecedent condition.

\paragraph{Membership values of antecedents:} For each input $x_i$, the membership degree in fuzzy set $A_i$ is

\begin{equation}
\mu_{A_i}(x_i) \in [0,1]
\label{eq:auto_fuzzyinference_57eb8c12c9}
\end{equation}

\paragraph{Aggregation operator:} The combined antecedent membership is obtained by applying a fuzzy logical operator, typically the \emph{minimum} (intersection) or the \emph{product} operator:

\begin{align}
\mu_{\text{antecedent}}(x_1, \ldots, x_n) &= \min_{i=1}^n \mu_{A_i}(x_i), \quad \text{(min operator)} \label{eq:antecedent_min} \\
\text{or} \quad \mu_{\text{antecedent}}(x_1, \ldots, x_n) &= \prod_{i=1}^n \mu_{A_i}(x_i). \quad \text{(product operator)} \label{eq:antecedent_prod}
\end{align}

This value quantifies the degree to which the entire antecedent condition is satisfied by the input vector $\mathbf{x} = (x_1, \ldots, x_n)$. More generally, any t\hyp{}norm $T$ can be used in place of the min or product, provided it satisfies the standard properties (commutativity, associativity, monotonicity, and $T(a,1)=a$); the chosen t\hyp{}norm shapes how strictly the rule demands simultaneous satisfaction of all antecedents.

\subsection{Rule Consequent and Output Fuzzy Set}
\label{sec:fuzzyinference_rule_consequent_and_output_fuzzy_set}

Once the antecedent firing strength $\alpha$ is computed, it is used to modify the consequent fuzzy set $B$. The consequent fuzzy set is typically defined by its membership function $\mu_B(y)$ over the output universe.

\paragraph{Implication operator:} The implication step adjusts the consequent membership function based on the firing strength $\alpha$. Commonly used implication methods include:

\begin{itemize}
    \item \textbf{Minimum implication:} Truncate the consequent membership function at level $\alpha$,
    \begin{equation}
    \mu_{B'}(y) = \min \big( \alpha, \mu_B(y) \big).
    \label{eq:min_implication}
    \end{equation}
    \item \textbf{Product implication:} Scale the consequent membership function by $\alpha$,
    \begin{equation}
    \mu_{B'}(y) = \alpha \cdot \mu_B(y).
    \label{eq:prod_implication}
    \end{equation}
\end{itemize}

The resulting fuzzy set $B'$ represents the \emph{output fuzzy set} contributed by this particular rule.
\begin{tcolorbox}[summarybox,title={Implication choices}]
Mamdani uses min-implication (clipping), Larsen uses product-implication (scaling). Other options include residuated and axiomatic implicators paired with their t\hyp{}norms (e.g., G\"odel, Product/Goguen, \L{}ukasiewicz; see \citealp{Klement2000,Dubois1988}). Pick by desired smoothness: clipping preserves shape and interpretability; scaling yields smoother surfaces and is friendlier to gradient-based tuning.
\end{tcolorbox}

\subsection{Aggregation of Multiple Rules}
\label{sec:fuzzyinference_aggregation_of_multiple_rules}

When multiple rules are present, each produces an output fuzzy set $B'_j$ with membership function $\mu_{B'_j}(y)$, where $j$ indexes the rules. These are aggregated to form a combined output fuzzy set:

\begin{equation}
\mu_{B_{\text{agg}}}(y) = \max_j \mu_{B'_j}(y).
\label{eq:output_aggregation}
\end{equation}

The \emph{max} operator corresponds to the fuzzy union of the individual rule outputs, capturing the overall inference result.
\paragraph{Other aggregations} Algebraic sum or bounded sum (\Cref{tab:tnorms}) are used when max is too brittle; they can over-saturate when many rules fire, so start with max unless smooth blending is required.

\subsection{Summary of the Fuzzy Inference Process}
\label{sec:fuzzyinference_summary_of_the_fuzzy_inference_process}

To summarize, the fuzzy inference process for a given input vector $\mathbf{x}$ proceeds as follows:

\begin{enumerate}
    \item For each rule $j$, compute the antecedent membership degree $\alpha_j$ using \eqref{eq:antecedent_min} or \eqref{eq:antecedent_prod}.
\item Modify the consequent fuzzy set $B_j$ by applying the implication operator \eqref{eq:min_implication} or \eqref{eq:prod_implication} to obtain $B'_j$.
\item Aggregate all $B'_j$ using \eqref{eq:output_aggregation} to obtain the overall output fuzzy set $B_{\text{agg}}$.
\end{enumerate}

In sup-$T$ form this is the same compositional rule of inference used in \Cref{chap:fuzzyrelations}; here \(T\) defaults to \(\min\) (or product) and \(\sup\) reduces to a max on discrete grids.

The final step, defuzzification, converts $B_{\text{agg}}$ into a crisp output value. One widely used approach is the centroid (center-of-gravity) method, which computes

\begin{equation}
y^* = \frac{\int_Y y \, \mu_{B_{\text{agg}}}(y) \, dy}{\int_Y \mu_{B_{\text{agg}}}(y) \, dy}.
\label{eq:centroid_defuzz}
\end{equation}

This expression balances all candidate output values $y$ by weighting them according to their membership grade in the aggregated fuzzy set. In discrete implementations, the integral is replaced with a sum over sampled output points.
\paragraph{Other defuzzifiers} Common alternatives are mean/center of maxima (robust to multi-modal sets), smallest/largest of maxima (conservative tie-breaks), and center of sums (less sensitive to overlap than max aggregation). Choose centroid for smoothness, a max-based rule for fast or safety-critical switches, and always handle the zero-mass case explicitly.
\paragraph{Zero-mass fallback} If the denominator in \eqref{eq:centroid_defuzz} is zero (e.g., all consequents clipped to zero), fall back to a max-membership or rule-based tie-breaker to avoid NaNs; log the condition for debugging.
\paragraph{Computation note} With uniform sampling over \(m\) output points, centroid costs \(O(mR)\) per evaluation for \(R\) rules. Non-singleton inputs add a convolution step but reuse the same aggregation/defuzz pipeline; refine the grid near peaks to reduce bias.

\begin{tcolorbox}[pitfallbox,title={Centroid stability and tie-breaking}]
\begin{itemize}
    \item \textbf{Multi-modal sets.} When $B_{\text{agg}}$ has multiple peaks, the centroid may fall between modes. Log numerator and denominator separately and check that $\int \mu_{B_{\text{agg}}}(y)\,dy$ is non-zero; otherwise fall back to max-membership or a rule-based tie-break.
    \item \textbf{Discretisation.} Sampling the universe with too few points biases the centroid. Use uniform grids for smooth consequents and adaptive refinement near peaks for multi-modal sets. Report the step size (e.g., $0.5^\circ$C) to show numeric fidelity.
\end{itemize}
\end{tcolorbox}

\vspace*{1.2\baselineskip}
\noindent
\begin{tcolorbox}[summarybox,title={Pipeline at a glance (Mamdani/Larsen)}]
\begin{verbatim}
for each rule j:
    alpha_j = T( mu_A1(x1), ..., mu_An(xn) )
        # firing strength
    mu_Bj_prime(y) = implication(alpha_j, mu_Bj(y))
        # clip or scale
mu_Bagg(y) = S( mu_B1_prime(y), ..., mu_BR_prime(y) )
    # aggregate
y_star = centroid(mu_Bagg(y))
    # or another defuzzifier
\end{verbatim}
Defaults: \(T=\min\) or product; \(S=\max\); centroid defuzzification. Non-singleton fuzzification (convolving input uncertainty with \(\mu_{A_i}\)) uses the same pipeline once the input blend is computed.
\end{tcolorbox}

\begin{tcolorbox}[summarybox,title={Design checklist}]
\begin{itemize}
    \item Define universes/labels; ensure coverage and reasonable overlap.
    \item Pick \(T/S/\Rightarrow\) via \Cref{tab:tnorms} to get the smoothness/interpretability you need.
    \item Verify rule-base coverage; avoid contradictions/holes.
    \item Choose defuzzifier and sampling resolution; set a minimum-mass fallback.
    \item Test monotonicity/saturation; refine membership widths or rule weights.
\end{itemize}
\end{tcolorbox}

\begin{tcolorbox}[pitfallbox,title={Common pitfalls}]
\begin{itemize}
    \item Max aggregation can mask contributions from several moderate rules; algebraic sum can over-saturate.
    \item Memberships that are too narrow yield sparse firing; too wide produce mushy outputs.
    \item Coarse grids bias centroids; inconsistent units across labels break interpretability.
    \item Neglecting non-singleton inputs: if sensor noise matters, blur inputs before fuzzifying.
\end{itemize}
\end{tcolorbox}

\subsection{Mamdani vs.\ Sugeno/Takagi--Sugeno systems}
\label{sec:fuzzyinference_mamdani_vs_sugeno_takagi_sugeno_systems}

Mamdani-style inference (scaled fuzzy consequents, centroid defuzzification) excels when linguistic interpretability is a priority and when rule consequents must remain human-readable.\\
Sugeno/Takagi--Sugeno (TSK) systems replace fuzzy consequents with crisp functions such as affine models,
\[
\text{IF } e \text{ is } A_i \text{ AND } \dot{e} \text{ is } B_i \text{ THEN } u_i = p_i e + q_i \dot{e} + r_i.
\]
Each rule still produces a firing strength $\lambda_i$ via a t\hyp{}norm, but the final output becomes the weighted average
\[
u^\ast = \frac{\sum_i \lambda_i u_i}{\sum_i \lambda_i},
\]
eliminating the defuzzification integral. The trade-offs are:
{\raggedright
\begin{itemize}
    \item \textbf{Mamdani:} transparent consequents, straightforward incorporation of expert knowledge, but higher computational cost due to aggregation and centroid evaluation.
    \item \textbf{Sugeno/TSK:} faster evaluation (weighted averages), amenable to gradient-based tuning of the consequent parameters, yet less interpretable because consequents are numerical functions rather than linguistic labels.
\end{itemize}
}
For the thermostat example, Mamdani rules (``IF error is Cold AND rate is Falling THEN heater power is High'') are ideal when operators must audit decisions, whereas a Sugeno/TSK variant is preferable when embedding the controller into a high-speed or automatically tuned loop.
\footnote{Classic sources: \citet{Mamdani1975} for clipping implication; \citet{TakagiSugeno1985} for TSK; \citet{Jang1993} for ANFIS; see also \citep{Klement2000,Dubois1988} for operator/implicator theory.}

\begin{tcolorbox}[summarybox,title={Key takeaways}]
\begin{itemize}
    \item Fuzzy inference composes rule antecedents (via a t\hyp{}norm) and modifies consequents by implication.
    \item Aggregation and defuzzification (e.g., centroid) produce crisp outputs from fuzzy rule bases.
    \item Design choices (operators, shapes) trade interpretability and control smoothness.
\end{itemize}

\medskip
\noindent\textbf{Minimum viable mastery.}
\begin{itemize}
    \item For a rule base, compute firing strengths, apply implication, aggregate consequents, and compute a centroid output.
    \item Explain the Mamdani vs.\ Sugeno/TSK tradeoff (interpretability vs.\ efficiency/tunability).
    \item Treat operator choice as part of the specification and keep it consistent across examples and implementations.
\end{itemize}

\noindent\textbf{Common pitfalls.}
\begin{itemize}
    \item Defining memberships on incompatible scales (inputs never trigger, or everything triggers).
    \item Writing many redundant rules and then tuning symptoms rather than consolidating the rule base.
    \item Comparing controllers without stating operators, defuzzification, and evaluation protocol.
\end{itemize}
\end{tcolorbox}

\begin{tcolorbox}[summarybox,title={Exercises and lab ideas}]
\begin{itemize}
    \item Implement a Mamdani thermostat with three error labels and two rate labels; experiment with min/product t\hyp{}norms and report the resulting control surfaces.
    \item Build a Sugeno/TSK variant of the same controller and compare outputs to the Mamdani version under identical test trajectories.
    \item Evaluate different defuzzification methods (centroid, weighted average, max membership) on a toy rule base and quantify the steady-state error they induce.
\end{itemize}

\medskip
\noindent\textbf{If you are skipping ahead.} When you reach \Cref{chap:evo}, keep this chapter's audit instinct: log your design choices. Evolutionary and fuzzy systems both invite silent degrees of freedom that only become visible with disciplined reporting.
\end{tcolorbox}

\paragraph{Where we head next.} \Cref{chap:evo} moves from fuzzy controllers to evolutionary computing, where population-based search and optimization heuristics provide another pillar of soft computing.

\paragraph{References.} Full citations for works mentioned in this chapter appear in the book-wide bibliography.

\begin{tcolorbox}[summarybox,title={Part III takeaways}]
\begin{itemize}
  \item Soft computing makes uncertainty explicit: degrees of membership and rule-based aggregation.
  \item Interpretability is a design variable: rules, operators, and defuzzification encode assumptions.
  \item Debugging is empirical: test edge cases, inspect rule firing, and validate monotonicity and scale.
  \item The same audit mindset applies: define failure modes up front and track them as you tune operators and rules.
\end{itemize}
\end{tcolorbox}
\part*{Part IV: Evolutionary optimization}
\addcontentsline{toc}{part}{Part IV: Evolutionary optimization}
% Chapter 19
\section{Introduction to Evolutionary Computing}\label{chap:evo}
\graphicspath{{assets/lec19/}}

Parts I--III built three complementary toolkits: optimize models against data (ERM), learn representations with gradients, and encode auditable heuristics with fuzzy rules. Each is powerful when you can write a smooth objective or commit to a compact rule base. Many engineering choices, however, live outside that comfort zone: the knobs are discrete or tightly constrained, evaluations are noisy or expensive, and the landscape is rugged enough that local improvement is unreliable.

Evolutionary computing treats that situation as search. Instead of following gradients or hand\hyp{}tuning rules, we maintain a population of candidate designs, score them with a fitness function, and use selection plus variation to explore and refine solutions within a fixed budget. After the fuzzy trilogy (\Crefrange{chap:softcomp}{chap:fuzzyinference}), this chapter develops the evolutionary strand introduced in \Cref{chap:intro} and focuses on population\hyp{}based optimization under constraints, budgets, and stochastic noise.

\paragraph{A distinct paradigm (and what ``nature-inspired'' means here).}
Fuzzy inference systems are a behavioral modeling tool: we write down the reasoning logic directly so a human can audit why a particular decision was made. Evolutionary computing aims at a different target. It encodes \emph{candidate designs} and uses a simplified selection--variation loop---an algorithmic echo of natural evolution---to search for designs that perform well under a fixed evaluation budget (the Karl Sims\hyp{}style intuition: emergence via selection and variation, not hand-coded reasoning). The point is not to be faithful to evolutionary biology; the point is that this process can produce designs that respond well to situations under constraints and noise even when gradients are unavailable and the knobs are mixed discrete/continuous.

\begin{tcolorbox}[summarybox, title={Learning Outcomes}]
\begin{itemize}
    \item Explain the evolutionary-computation toolbox (GAs, GP, CMA-ES, DE) and when each is well-suited.
    \item Implement population-based optimization loops with selection, crossover/mutation, and constraint handling.
    \item Diagnose convergence, premature stagnation, and feasibility trade-offs on examples such as controller tuning.
\end{itemize}
\end{tcolorbox}

\begin{tcolorbox}[summarybox, title={Design motif}]
When gradients are unavailable or the landscape is rugged, treat design as search: maintain a diverse population, score candidates with a fitness function, and let selection plus variation drive improvement under constraints.
\end{tcolorbox}

\subsection{Context and Motivation}
\label{sec:evo_context_and_motivation}

Evolutionary computing is the strand you reach for when the objective is a black box (simulation, experiment, or expensive audit) or when the design space is a mix of continuous and discrete choices. The modeling contract stays simple: pick an encoding, define a fitness function, state the constraints, and commit to an evaluation budget. The algorithm then trades exploration (diversity) against exploitation (selecting what works) as it searches a rugged landscape.

In the fuzzy trilogy, once you commit to membership shapes and a rule base, a practical question remains: how do you tune those choices (and their trade-offs) when hand tweaks do not scale? Evolutionary search provides a disciplined answer: score candidate controllers, select the better ones, perturb/recombine them, and keep iterating until the improvements flatten or the budget runs out.

\begin{tcolorbox}[summarybox, title={Vignette: budgeted search in the wild (locomotion, ML/LLMs, and physical experiments)}]
\textbf{Robot gait / locomotion.} Suppose we want a legged robot to move forward robustly across slightly different terrains. The design has mixed knobs: continuous parameters (gains, trajectory coefficients, penalty weights) and discrete choices (controller mode, safety logic, constraint-handling policy). A single trial is expensive and noisy (contact dynamics, initial conditions, sensor noise), so it is natural to treat performance as a black-box fitness, average over multiple seeds, and search under a strict evaluation budget.\par
\textbf{Hyperparameter and pipeline tuning (including LLM systems).} Many modern learning pipelines have the same structure: discrete design choices (model family, prompt/toolchain variant, data filters) plus continuous knobs (learning rates, thresholds, temperatures), measured by an empirical score with variance across runs. Evolutionary search is attractive when gradients are unavailable, the evaluation surface is rugged, or the objective invites silent failure modes (metric gaming, constraint violations).\par
\textbf{Physical-world analogy (jet nozzle).} Classic evolution-strategy examples describe optimizing a jet-nozzle shape via random mutations plus selection: you do not differentiate the wind tunnel; you spend a limited trial budget to evolve better designs. The point of the analogy is not biology, but budgeted optimization under constraints.
\end{tcolorbox}

\subsection{Philosophical and Historical Background}
\label{sec:evo_philosophical_and_historical_background}

Evolutionary computing traces its roots back to the 1950s and 1960s, contemporaneous with early developments in neural networks. It is important to recognize that evolutionary algorithms are not direct scientific models of biological evolution; rather, they are inspired by a simplified, abstracted view of evolutionary principles such as selection, mutation, and reproduction.

\paragraph{Key Insight:} These algorithms are \emph{heuristics}; they provide practical methods to find \emph{good enough} solutions to problems that are otherwise computationally intractable, rather than guaranteed optimal solutions. Consequently, convergence proofs typically ensure improvement in expectation or under restrictive assumptions, but not attainment of the true global optimum.

\begin{tcolorbox}[summarybox, title={Author's note: a pragmatic take on evolution}]
Evolutionary algorithms borrow the language of biology to provide a disciplined way to search rugged landscapes, not to recreate population genetics. The design mandate is pragmatic: deliver a respectable solution within the computational budget, even if it is only approximately optimal. Keep that lens in mind when evaluating selection, mutation, or recombination operators; they are tuned because they help optimization, not because they are biologically faithful.
\end{tcolorbox}

\subsection{Problem Setting: Optimization}
\label{sec:evo_problem_setting_optimization}

\begin{tcolorbox}[summarybox, title={Notation handoff}]
Vectors of candidate solutions are written as \(\mathbf{x}\), objective values as \(J(\mathbf{x})\), and population members as individuals/chromosomes depending on encoding. If symbol reuse from earlier chapters becomes ambiguous, default to local definitions and consult \Cref{app:notation_collisions}.
\end{tcolorbox}

Consider an optimization problem where the goal is to find an input vector \( \mathbf{x} \in \mathbb{R}^n \) that minimizes (or maximizes) a given objective function \( J: \mathbb{R}^n \to \mathbb{R} \). Formally, we want to solve

\begin{align}
    \mathbf{x}^* = \arg \min_{\mathbf{x} \in \mathcal{D}} J(\mathbf{x}), \label{eq:optimization_problem}
\end{align}

where \(\mathcal{D} \subseteq \mathbb{R}^n\) is the feasible domain incorporating any bound, equality, or inequality constraints required by the application.

\paragraph{Challenges:}

\begin{itemize}
    \item The function \( J \) may be \emph{non-convex}, exhibiting multiple local minima and maxima.
    \item The objective or constraints may be undefined or discontinuous in parts of the domain, breaking gradient assumptions.
    \item The feasible set may include combinatorial or integer constraints that resist continuous methods.
    \item There may be no closed-form or deterministic method to find the global optimum, especially for NP-hard variants.
    \item The search space \(\mathcal{D}\) can be large or complex, making exhaustive search (brute force) computationally prohibitive.
    \item Real-time or practical constraints often require solutions within limited time frames.
\end{itemize}

\subsection{Illustrative Example}
\label{sec:evo_illustrative_example}

Imagine a function \( J \) with multiple peaks and valleys (local maxima and minima). The global minimum is the lowest valley, but many local minima exist that can trap naive optimization methods.

\paragraph{Goal:} Instead of guaranteeing the global optimum, evolutionary computing aims to find a \emph{good enough} solution---one that is sufficiently close to optimal and found within a reasonable computational budget.

\subsection{Why Not Brute Force?}
\label{sec:evo_why_not_brute_force}

While brute force search guarantees finding the global optimum by evaluating all possible candidates, it is often infeasible due to:

\begin{itemize}
    \item \textbf{Computational complexity:} The number of candidate solutions can be astronomically large.
    \item \textbf{Time constraints:} Real-world applications often require timely decisions, making exhaustive search impractical.
\end{itemize}

For example, in control systems, one might want to tune parameters to regulate temperature or pressure optimally. Waiting for a brute force search to complete could be unacceptable, whereas a near-optimal solution found quickly is valuable.

\subsection{Summary}
\label{sec:evo_summary}

Checkpoint: you now have the \emph{why} for evolutionary search---rugged landscapes, constraints, discontinuities, and budgets that make local improvement unreliable. The rest of this chapter builds one canonical template in three passes:
(i) a big-picture story (population + evaluation + variation),
(ii) operator-level definitions (selection/crossover/mutation and constraint handling),
and (iii) an implementation-facing sanity-check (flowchart, pseudocode, and a one-generation numeric trace).
Read repeated mentions of ``the GA loop'' as \emph{preview \(\rightarrow\) formalization \(\rightarrow\) checklist}, not as duplication.

Keep the ``rugged landscape'' picture in mind when you later interpret convergence traces, premature stagnation, and diversity metrics in population-based search.

\subsection{Challenges in Continuous Optimization and Motivation for Evolutionary Computing}
\label{sec:evo_challenges_in_continuous_optimization_and_motivation_for_evolutionary_computing}

In many continuous optimization problems, the objective function may be undefined or discontinuous in certain regions of the domain. For example, consider a function with singularities or points where the function value is not defined (akin to division by zero). Such characteristics pose significant challenges for classical optimization methods such as gradient descent or hill climbing, which rely on smoothness and continuity to navigate the search space effectively.

These issues compound the nonconvexity and constraint challenges above: discontinuities and combinatorial feasibility break gradient assumptions, and global guarantees are out of reach for many NP-hard variants. Given these challenges, deterministic approaches may be infeasible or computationally expensive. Instead, we can tolerate approximate solutions and employ heuristic or metaheuristic methods that explore the search space more flexibly. This motivates the use of \emph{evolutionary computing} methods.

\subsection{Evolutionary Computing at a Glance}
\label{sec:evo_introduction_to_evolutionary_computing}

Evolutionary computing (EC) is best viewed as a \emph{budgeted optimizer}: maintain a population of candidates, evaluate them with a fitness function, bias sampling toward better candidates (selection), generate variation (crossover/mutation), handle constraints, and replace the population. Repeat until a stopping rule triggers (budget, target quality, or no-improvement window).

\paragraph{Key Idea}
We evolve a \emph{set} of candidate solutions over successive generations. Unlike gradient-based methods, evolutionary algorithms do not require differentiability, and unlike brute force, they trade guarantees for tractable performance under noise, discontinuities, and combinatorial feasibility.

\paragraph{Genetic Algorithms (GAs)}
One of the most well-known evolutionary algorithms is the Genetic Algorithm (GA). GAs \emph{loosely} mimic evolutionary ideas using a simplified, abstracted model of mechanisms such as selection, crossover, and mutation.

\subsection{Biological Inspiration: Evolutionary Concepts}
\label{sec:evo_biological_inspiration_evolutionary_concepts}

This section is terminology and motivation. If you already think of a GA as ``selection + variation + replacement under a fitness budget,'' feel free to skim; the algorithmic details begin in \Cref{sec:evo_genetic_algorithms_modeling_chromosomes}.

To understand GAs, we briefly review relevant biological concepts:

\paragraph{Chromosomes and Genes}
In biology, an organism's genetic information is encoded in chromosomes, which are long sequences of DNA. Each chromosome contains many genes, which determine specific traits.

\paragraph{Cell Division: Mitosis vs. Meiosis}
\begin{itemize}
    \item \textbf{Mitosis:} A process where a cell divides to produce two genetically identical daughter cells, each containing the full chromosome set (e.g., 46 chromosomes, i.e., 23 pairs in humans). This process is responsible for growth and tissue repair.
    \item \textbf{Meiosis:} A specialized form of cell division that produces gametes (sperm or egg cells) with half the number of chromosomes (haploid). When two gametes combine during fertilization, they form a new cell with a full set of chromosomes (diploid), mixing genetic material from both parents.
\end{itemize}

\paragraph{Genetic Recombination and Variation}
\begin{sloppypar}
During meiosis, chromosomes undergo \emph{crossover} events. Segments of genetic material are exchanged between paired chromosomes. Recombination increases genetic diversity.
\end{sloppypar}

\paragraph{Inheritance and Heredity}
The offspring's chromosomes are a mixture of the parents' genetic material, but not a simple half-and-half split. Instead, genes from multiple previous generations contribute to the genetic makeup, introducing variability and enabling adaptation over time.

\subsection{Implications for Genetic Algorithms}
\label{sec:evo_implications_for_genetic_algorithms}

The biological processes suggest several principles that GAs incorporate:

\begin{itemize}
    \item \textbf{Population-based search:} Maintain a population of candidate solutions (analogous to organisms).
    \item \textbf{Selection:} Preferentially choose better solutions to reproduce, mimicking survival of the fittest.
    \item \textbf{Replacement/elitism:} Form the next generation by replacing some individuals with offspring; optionally preserve the best solutions explicitly.
    \item \textbf{Crossover (Recombination):} Combine parts of two or more parent solutions to create offspring solutions, promoting exploration of new regions in the search space.
    \item \textbf{Mutation:} Introduce random changes to offspring to maintain diversity and avoid premature convergence.
    \item \textbf{Generational evolution:} Repeat the process over multiple generations, gradually improving solution quality.
\end{itemize}

The stochastic nature of these operations allows GAs to explore complex, multimodal landscapes and handle problems where deterministic methods struggle.

\subsection{Summary of Biological Mechanisms Modeled in GAs}
\label{sec:evo_summary_of_biological_mechanisms_modeled_in_gas}

\begin{center}
\begin{tabularx}{0.86\linewidth}{@{}l >{\raggedright\arraybackslash}X@{}}
\toprule
\textbf{Biological Process} & \textbf{GA Analog} \\
\midrule
Chromosomes and genes & Encoding of candidate solutions (chromosomes) \\
Meiosis and fertilization & Crossover of parent chromosomes to produce offspring \\
Genetic recombination & Mixing of solution components \\
Mutation & Random perturbations in offspring \\
Selection & Fitness-based selection of parents and survivors \\
Generations & Iterative improvement over time \\
\bottomrule
\end{tabularx}
\end{center}

The remainder of this section formalizes how candidate solutions are encoded and how genetic operators manipulate those encodings during evolution.

\begin{tcolorbox}[summarybox, title={GA hyperparameters at a glance}]
As a starting point for binary encodings of length \(L\), choose population sizes between \(5L\) and \(10L\), tournament selection with small tournaments (2--4 individuals), one-point or uniform crossover, and mutation probability near \(1/L\) per bit. For real-coded GAs, replace bit-flips with Gaussian mutations on parameters and use simulated binary crossover (SBX) or blend-style crossover to mix parents smoothly \citep{DebAgrawal1995}; then tune population size and mutation scale empirically based on convergence speed and population diversity.
\end{tcolorbox}

\begin{tcolorbox}[summarybox, title={Author's note: population and mutation heuristics}]
Budget dictates population size and diversity mechanisms. If evaluations are cheap, spend them on larger populations to cover the search space; if evaluations are expensive, keep populations modest but invest in diversity-preserving steps (mutation, niching, restarts) to avoid premature convergence. Use the defaults in the box above as a first pass, then tune based on convergence traces and diversity diagnostics.
\end{tcolorbox}

\subsection{Genetic Algorithms: Modeling Chromosomes}
\label{sec:evo_genetic_algorithms_modeling_chromosomes}

In the previous discussion, we introduced the concept of diversity in genetic algorithms (GAs) and the probabilistic nature of evolutionary processes. We now delve deeper into modeling chromosomes and the mechanisms of genetic inheritance, crossover, and mutation, drawing parallels to optimization problems.

\begin{tcolorbox}[summarybox, title={Genetic algorithm at a glance}]
\textbf{Objective:} Optimize an objective \(J(\mathbf{x})\) over a discrete or continuous search space by evolving a population of encoded candidate solutions according to selection, crossover, and mutation.\par
\textbf{Key hyperparameters:} Population size, selection pressure (tournament size or selection temperature), crossover and mutation rates, encoding scheme, and stopping criteria (max generations, no-improvement window, target fitness).\par
\textbf{Common pitfalls:} Premature convergence due to excessive selection pressure or low mutation, deceptive fitness landscapes, constraint violations when mutation or crossover produce infeasible solutions, and overinterpreting stochastic runs without multiple seeds.
\end{tcolorbox}

\paragraph{Chromosomes as Information Carriers}

Recall that chromosomes in GAs represent candidate solutions encoded as strings of data. For modeling purposes, we consider each chromosome as a sequence of bits or symbols, each encoding a piece of information relevant to the problem domain. Formally, let a chromosome be represented as
\[
\mathbf{c} = (c_1, c_2, \ldots, c_L),
\]
where each gene \( c_i \) encodes a particular trait or parameter, and \( L \) is the chromosome length.

\paragraph{Inheritance and Crossover}

During reproduction, offspring inherit genes from parents via crossover and mutation (with some genes passing through unchanged). The next figure gives an intuitive operator-level picture; full selection and crossover details follow in \Cref{sec:evo_selection_in_genetic_algorithms} and \Cref{sec:evo_crossover_operator}.


\begin{figure}[t]
    \centering
    \begin{tikzpicture}
        \begin{groupplot}[
            group style={group size=3 by 1, horizontal sep=0.9cm},
            width=0.28\linewidth,
            height=0.28\linewidth,
            axis lines=none,
            xmin=0, xmax=1,
            ymin=0, ymax=1
        ]
            % Selection
            \nextgroupplot[title={Selection}]
                \node[draw, rounded corners, fill=cbBlue!10, minimum width=1.6cm, minimum height=0.5cm] (p1) {fit};
                \node[draw, rounded corners, fill=cbBlue!5, minimum width=1.6cm, minimum height=0.5cm, below=0.35cm of p1] (p2) {mid};
                \node[draw, rounded corners, fill=cbBlue!5, minimum width=1.6cm, minimum height=0.5cm, below=0.35cm of p2] (p3) {low};
                \draw[->, thick] (p1.east) -- ++(1.0cm,0) node[anchor=west, font=\scriptsize]{higher selection prob.};
                \draw[->, thick] (p2.east) -- ++(0.8cm,0);
                \draw[->, thick] (p3.east) -- ++(0.6cm,0);
            % Crossover
            \nextgroupplot[title={Crossover}]
                \node[draw, rounded corners, fill=cbOrange!15, minimum width=2.1cm, minimum height=0.5cm] (c1) {1011|001};
                \node[draw, rounded corners, fill=cbOrange!10, minimum width=2.1cm, minimum height=0.5cm, below=0.35cm of c1] (c2) {0100|111};
                \draw[->, thick] (c1.east) -- ++(0.6cm,-0.4cm);
                \draw[->, thick] (c2.east) -- ++(0.6cm,0.4cm);
                \node[draw, rounded corners, fill=cbGreen!20, minimum width=2.1cm, minimum height=0.5cm, right=1.8cm of c1] (o1) {1011|111};
                \node[draw, rounded corners, fill=cbGreen!20, minimum width=2.1cm, minimum height=0.5cm, right=1.8cm of c2] (o2) {0100|001};
                \draw[dashed, gray!70] (c1.east) ++(0.05cm,-0.25cm) -- ++(2.2cm,0);
            % Constraints
            \nextgroupplot[title={Constraints}]
                \node[draw, rounded corners, fill=cbBlue!10, minimum width=1.4cm, minimum height=0.5cm] (cand) {candidate};
                \node[draw, rounded corners, fill=cbOrange!20, minimum width=1.5cm, minimum height=0.5cm, right=1.1cm of cand] (repair) {repair};
                \node[draw, rounded corners, fill=cbGreen!15, minimum width=1.6cm, minimum height=0.5cm, right=1.1cm of repair] (eval) {evaluate};
                \draw[->, thick] (cand.east) -- (repair.west);
                \draw[->, thick] (repair.east) -- (eval.west);
                \node[font=\scriptsize, below=0.15cm of repair] {or penalty/feasible-first};
        \end{groupplot}
    \end{tikzpicture}
    \caption{Evolutionary micro-operators. Left: fitter individuals get sampled more often (roulette/tournament). Middle: crossover splices parents by a mask (one-point shown). Right: constraint handling routes offspring through repair/penalty/feasibility before evaluation. Use this to map an implementation to the canonical operator loop.}
    \label{fig:lec11-ea-micro}
\end{figure}

\Cref{fig:lec11-ea-micro} summarizes selection, crossover, and constraint handling at the operator level.

\paragraph{Modeling the Genetic Operations}

Let \(\mathbf{p}_1\) and \(\mathbf{p}_2\) be parent chromosomes. The offspring chromosome \(\mathbf{o}\) is formed by combining segments from \(\mathbf{p}_1\) and \(\mathbf{p}_2\) according to a crossover pattern, and then applying mutation:
\begin{align}
    \mathbf{o} = \text{Mutate}\big(\text{Crossover}(\mathbf{p}_1, \mathbf{p}_2, \mathbf{u})\big).
    \label{eq:evo_offspring_from_parents}
\end{align}

The crossover operator selects which genes come from which parent, often modeled by a binary mask \(\mathbf{u}\in\{0,1\}^L\), where
\[
o_i = \begin{cases}
(p_1)_i, & \text{if } u_i = 0, \\
(p_2)_i, & \text{if } u_i = 1.
\end{cases}
\]

Mutation (see \Cref{sec:evo_mutation_operator}) perturbs genes with a small probability \(p_m\).

\paragraph{Fitness and selection (preview)}
Chromosomes encode candidate solutions (phenotypes), and fitness scores quantify how well each candidate meets the objective. For example, consider chromosomes representing facial feature variants with fitness values
\[
f = \{80, 75, 60, 65, 40, 20\}.
\]
Higher-fitness chromosomes should be sampled more often, but not deterministically: occasional selection of weaker candidates preserves diversity and reduces premature convergence. A simple baseline is roulette selection: normalize nonnegative fitness values into probabilities and sample with replacement. The point is the \emph{bias} toward what works (not biological fidelity); we formalize selection and work a roulette example in \Cref{sec:evo_selection_in_genetic_algorithms}.

\subsection{Mapping Genetic Algorithms to Optimization Problems}
\label{sec:evo_mapping_genetic_algorithms_to_optimization_problems}

Genetic algorithms can be viewed as heuristic optimization methods. To formalize this analogy, consider the components of an optimization problem:

\begin{itemize}
    \item \textbf{Objective function:} \( J(\mathbf{x}) \), which we seek to maximize or minimize.
    \item \textbf{Constraints:} Conditions restricting the feasible set of solutions.
    \item \textbf{Input parameters:} Decision variables \( \mathbf{x} \).
\end{itemize}

In GAs, the chromosome encodes the input parameters \( \mathbf{x} \), and the fitness function corresponds to the objective function \( J(\mathbf{x}) \).

\paragraph{Key GA Components in Optimization Terms}

\begin{itemize}
    \item \textbf{Encoding:} The method of representing \( \mathbf{x} \) as chromosomes.
    \item \textbf{Initial population:} The starting set of candidate solutions.
    \item \textbf{Fitness evaluation:} Computing \( f(\mathbf{c}) = J(\mathbf{x}) \) for each chromosome.
    \item \textbf{Selection:} Choosing chromosomes for reproduction based on fitness.
    \item \textbf{Crossover and mutation:} Generating new candidate solutions by recombining and perturbing chromosomes.
    \item \textbf{Convergence criteria:} Determining when the algorithm has sufficiently optimized the objective.
\end{itemize}

\begin{tcolorbox}[summarybox, title={Constraint handling in GAs}]
Realistic optimization problems often involve constraints \(g_j(\mathbf{x})\le 0\) or \(h_k(\mathbf{x})=0\). Common strategies include (i) \emph{repair}, which projects infeasible offspring back into the feasible set; (ii) \emph{penalties}, which modify the fitness as \(\tilde{f}(\mathbf{x}) = f(\mathbf{x}) - \rho \sum_j \max(0, g_j(\mathbf{x}))\) for some \(\rho>0\); and (iii) \emph{feasibility rules}, which prefer feasible individuals over infeasible ones at equal objective value. Penalty methods are simple and mesh well with existing GA code, while repair and feasibility rules are preferable when constraints encode hard physical or safety limits.
\end{tcolorbox}

\paragraph{Fitness as Objective Function Proxy}

The fitness function guides the search towards optimal solutions. The closer a chromosome's phenotype is to the desired optimum, the higher its fitness:
\[
f(\mathbf{c}) \propto \text{closeness to optimum}.
\]

This relationship allows GAs to explore the solution space stochastically, balancing exploitation of high-fitness regions and exploration via mutation and recombination.

\subsection{Encoding in Genetic Algorithms}
\label{sec:evo_encoding_in_genetic_algorithms}

We use the chromosome representation from \Cref{sec:evo_genetic_algorithms_modeling_chromosomes}; the focus here is how different encodings affect operators, constraints, and search efficiency.

\paragraph{Genotype and Phenotype}
\begin{itemize}
    \item \textbf{Genotype:} The encoded representation of a solution, e.g., a binary string.
    \item \textbf{Phenotype:} The decoded solution in the problem domain, e.g., real-valued parameters.
\end{itemize}

The goal is to design an encoding scheme that allows efficient exploration of the search space while respecting constraints and enabling effective genetic operations.

\subsubsection{Common Encoding Schemes}
\label{sec:evo_common_encoding_schemes_sub}

\paragraph{1. Binary Encoding}
Each parameter is represented as a binary string of fixed length. For example, if a parameter \( x_i \) is to be represented with precision \( p \), the length of the binary string is chosen accordingly.

\begin{itemize}
    \item Advantages: Simple, well-studied, easy to implement crossover and mutation.
    \item Disadvantages: May suffer from Hamming cliffs (large phenotypic changes from small genotypic changes).
\end{itemize}

\paragraph{2. Floating-Point Encoding}
Parameters are represented directly as floating-point numbers.

\begin{itemize}
    \item Advantages: No decoding needed, natural representation for real-valued parameters.
    \item Genetic operators can be adapted, e.g., crossover by averaging.
    \item Disadvantages: More complex mutation and crossover operators; may require specialized operators to maintain diversity.
\end{itemize}

\paragraph{3. Gray Coding}
A binary encoding where consecutive numbers differ by only one bit, reducing the Hamming distance between adjacent values.

\begin{itemize}
    \item Useful to reduce large jumps in phenotype space due to small genotypic changes.
    \item Decoding involves mapping Gray code to decimal values.
\end{itemize}

\subsubsection{Example: Binary Encoding of Parameters}
\label{sec:evo_example_binary_encoding_of_parameters_sub}

Suppose we want to encode four parameters \( x_1, x_2, x_3, x_4 \) each represented by a binary string of length \( l_i \). The genotype is the concatenation of these binary strings:

\[
\underbrace{b_{1,1} b_{1,2} \cdots b_{1, l_1}}_{x_1} \quad
\underbrace{b_{2,1} b_{2,2} \cdots b_{2, l_2}}_{x_2} \quad
\underbrace{b_{3,1} b_{3,2} \cdots b_{3, l_3}}_{x_3} \quad
\underbrace{b_{4,1} b_{4,2} \cdots b_{4, l_4}}_{x_4}
\]

For example, a genotype might look like:
\[
011 \quad 00100 \quad 0101 \quad 011110
\]

Each substring is decoded to a decimal or real value according to the encoding scheme.

\subsubsection{Example Problem: Minimization with Constraints}
\label{sec:evo_example_problem_minimization_with_constraints_sub}

Consider the problem:
\[
\min_{x} \quad f(x) = \frac{x}{2} + \frac{125}{x}
\]
subject to
\[
0 < x \leq 15
\]
For example, \(x=5\) gives \(f(x)=5/2+125/5=27.5\).

\paragraph{Encoding Strategy}
\begin{itemize}
    \item Since \( x \) is bounded between 0 and 15, we can encode \( x \) as a binary string representing integers in \([1, 15]\).
    \item For example, 4 bits can represent integers from 0 to 15, so we can use 4 bits and exclude zero.
    \item Each genotype corresponds to a candidate solution \( x \).
\end{itemize}

\paragraph{Decoding}
\[
x = \text{decimal value of binary string}
\]

If the decoded value is zero, it is invalid due to division by zero, so such genotypes are discarded or penalized.

\paragraph{Fitness Evaluation}
For minimization problems, selection still expects ``higher is better,'' so define fitness as a monotone transform of the cost (e.g., \(F(x)=-f(x)\) or \(F(x)=1/(1+f(x))\)), and then incorporate penalties or feasibility rules for constraint violations.

\subsection{Population Initialization and Size}
\label{sec:evo_population_initialization_and_size}

Once encoding is decided, the initial population is generated by randomly sampling genotypes within the feasible space.

\paragraph{Population Size}
\begin{itemize}
    \item Larger populations provide better coverage of the search space but increase computational cost.
    \item Smaller populations may converge prematurely.
    \item Typical sizes range from 20 to several hundreds depending on problem complexity.
\end{itemize}

\paragraph{Example}
For the problem above, a population of 50 individuals with 4-bit genotypes representing \( x \in [1,15] \) can be initialized by randomly generating 50 binary strings of length 4.

\subsection{Genetic Operators}
\label{sec:evo_genetic_operators}

After initialization, genetic operators are applied to evolve the population.

Selection is detailed in \Cref{sec:evo_selection_in_genetic_algorithms}, crossover in \Cref{sec:evo_crossover_operator}, and mutation in \Cref{sec:evo_mutation_operator}.

% Chapter 19 (continued)

\subsection{Selection in Genetic Algorithms}
\label{sec:evo_selection_in_genetic_algorithms}

We now formalize selection: how fitness scores translate into parent choice and survival probabilities so the search favors better solutions without collapsing diversity.

\subsubsection{Fitness and Selection Probability}
\label{sec:evo_fitness_and_selection_probability_sub}

Given a population of $N$ chromosomes, each chromosome $i$ has an associated fitness value $f_i$. The fitness function quantifies the quality of the solution represented by the chromosome.

A common approach to selection is to assign each chromosome a probability of being chosen proportional to its fitness. This can be expressed as:
\begin{align}
    p_i = \frac{f_i}{\sum_{j=1}^N f_j}, \quad i=1,2,\ldots, N,
    \label{eq:selection_probability}
\end{align}
where $p_i$ is the probability that chromosome $i$ is selected.

\paragraph{Practical note (fitness scaling).}
Proportional (roulette) selection assumes \emph{nonnegative} fitness values. If your objective can be negative (as in \Cref{tab:ga-toy}) or you are minimizing a cost, define a shifted/scaled fitness, e.g.,
\(\tilde f_i = f_i - \min_j f_j + \varepsilon\) with \(\varepsilon>0\), or use rank/tournament selection, which only relies on order comparisons.

\paragraph{Roulette Wheel Selection}

This proportional selection method is often called \emph{roulette wheel selection}. Imagine a wheel divided into $N$ slices, each slice corresponding to a chromosome and sized proportionally to $p_i$. To select a chromosome, a random number is generated to "spin" the wheel, and the chromosome corresponding to the slice where the wheel stops is chosen.

Key properties:
\begin{itemize}
    \item Chromosomes with higher fitness have a larger slice and thus a higher chance of being selected.
    \item The same chromosome can be selected multiple times, reflecting its relative superiority.
    \item This stochastic process maintains diversity but can be sensitive to fitness scaling.
\end{itemize}

\paragraph{Example}

Suppose we have 5 chromosomes with fitness values:
\[
f = [10, 20, 5, 15, 50].
\]
The total fitness is $100$, so the selection probabilities are:
\[
p = [0.10, 0.20, 0.05, 0.15, 0.50].
\]
For instance, \(p_2 = 20/100 = 0.20\) and \(p_5 = 50/100 = 0.50\).
Chromosome 5 has a 50\% chance of selection, making it likely to be chosen multiple times.

\subsubsection{Ranking Selection}
\label{sec:evo_ranking_selection_sub}

When fitness values are close or vary widely, roulette wheel selection may not perform well. For example, if fitness values are very close, selection probabilities become nearly uniform, reducing selection pressure. Conversely, if one chromosome dominates, diversity may be lost prematurely.

\emph{Ranking selection} addresses this by assigning selection probabilities based on the rank of chromosomes rather than raw fitness values.

\paragraph{Procedure}

\begin{enumerate}
    \item Sort chromosomes by fitness in descending order.
    \item Assign ranks $r_i$ such that the best chromosome has rank 1, the second best rank 2, and so forth.
    \item Define a selection probability function $p(r_i)$ decreasing with rank.
\end{enumerate}

A simple linear ranking scheme is:
\begin{align}
    p(r_i) = \frac{2 - s}{N} + \frac{2(r_i - 1)(s - 1)}{N(N - 1)},
    \label{eq:linear_ranking}
\end{align}
where $s \in [1,2]$ controls selection pressure. When $s=1$, all chromosomes have equal probability; when $s=2$, the best chromosome has twice the average probability.

\paragraph{Elitism}

Ranking selection can be combined with \emph{elitism}, where the best chromosome(s) are guaranteed to survive to the next generation. This ensures that the highest-quality solutions are preserved.

\paragraph{Advantages}

\begin{itemize}
    \item Controls selection pressure explicitly.
    \item Prevents premature convergence by maintaining diversity.
    \item Avoids issues with scaling fitness values.
\end{itemize}

\begin{tcolorbox}[summarybox, title={Selection pressure and exploration/exploitation}]
\begin{itemize}
    \item \textbf{Knobs:} Tournament size, rank-selection parameter $s$, crossover/mutation rates, and elitism all modulate pressure.
    \item \textbf{High pressure} (large tournaments, strong elitism, low mutation) speeds exploitation but risks premature convergence.
    \item \textbf{Low pressure} (small tournaments, higher mutation) preserves diversity but slows progress.
    \item \textbf{Practical default:} start with tournament size 2--3, modest elitism (top 1--5\%), \(p_c \approx 0.8\text{--}0.9\), and mutation tuned so 1--5\% of bits/genes change per generation.
\end{itemize}
\end{tcolorbox}

\subsection{Crossover Operator}
\label{sec:evo_crossover_operator}

After selection, the \emph{crossover} operator generates new offspring chromosomes by recombining parts of two parent chromosomes; this formalizes the intuition previewed above and supports controlled exploration.

\subsubsection{One-Point Crossover}
\label{sec:evo_one_point_crossover_sub}

Consider two parent chromosomes represented as binary strings of length $L$:
\[
\text{Parent 1: } \mathbf{c}^{(1)} = (c^{(1)}_1, c^{(1)}_2, \ldots, c^{(1)}_L)
\]
\[
\text{Parent 2: } \mathbf{c}^{(2)} = (c^{(2)}_1, c^{(2)}_2, \ldots, c^{(2)}_L)
\]

One-point crossover proceeds as follows:

\begin{enumerate}
    \item Choose a crossover point $k$ uniformly at random from $\{1, 2, \ldots, L-1\}$.
    \item Create two offspring by exchanging the tails of the parents at point $k$:
    \begin{align*}
        \text{Offspring 1} &= (c^{(1)}_1, \ldots, c^{(1)}_k, c^{(2)}_{k+1}, \ldots, c^{(2)}_L), \\
        \text{Offspring 2} &= (c^{(2)}_1, \ldots, c^{(2)}_k, c^{(1)}_{k+1}, \ldots, c^{(1)}_L).
    \end{align*}
\end{enumerate}

This operator allows mixing of genetic material between parents to create novel combinations.

\paragraph{Multi-point and uniform crossover.}
Two-point or multi-point crossover swaps multiple segments between parents; uniform crossover swaps each gene independently with a fixed probability. These variants increase mixing but can disrupt building blocks if overused.

\paragraph{Crossover probability.}
Apply crossover with probability \(p_c\) (typically 0.6--0.9). When crossover is skipped, offspring are usually copied forward and then mutated, which helps preserve good structures while maintaining exploration.

\subsection{Mutation Operator}
\label{sec:evo_mutation_operator}

Mutation introduces random alterations to individual chromosomes, mimicking biological mutations. It serves to maintain genetic diversity within the population and helps the algorithm escape local optima.

\paragraph{Biological motivation} Mutation is a rare event in nature but crucial for evolution. For example, the white coloration of polar bears is a mutation that provided an adaptive advantage in snowy environments. Similarly, environmental pressures can select for mutations, such as female elephants in Africa evolving to lack ivory tusks to avoid poaching.

\paragraph{Role in optimization} Mutation allows the algorithm to explore new regions of the search space that are not reachable by crossover alone. Consider a fitness landscape with multiple local maxima and minima. Mutation can randomly perturb a solution, potentially moving it from a local minimum to a region near a global maximum.

\paragraph{Implementation of mutation} In binary-encoded chromosomes, mutation typically involves flipping a bit:
\[
0 \to 1, \quad 1 \to 0
\]
The mutation is applied with a small \emph{mutation probability} \(p_m\), often on the order of \(10^{-3}\) to \(10^{-1}\).

\paragraph{Mutation operator formalization}
Given a binary chromosome \(\mathbf{c}\in\{0,1\}^L\), mutation produces \(\mathbf{c}'\) by mutating each bit \(c_i\) independently with probability \(p_m\):
\[
c_i' = \begin{cases}
1 - c_i, & \text{with probability } p_m, \\
c_i, & \text{with probability } 1 - p_m.
\end{cases}
\]

\subsection{Summary of Genetic Operators and Their Probabilities}
\label{sec:evo_summary_of_genetic_operators_and_their_probabilities}

You have now seen each operator in isolation. When you implement a GA, you typically configure the loop with a small set of \emph{operator knobs}:
\begin{itemize}
    \item \textbf{Selection scheme + pressure:} roulette/ranking/tournament selection, plus an explicit pressure parameter (e.g., fitness scaling, rank parameter \(s\), tournament size) and often an elitism rate (top \(k\) or top \(e\%\) copied forward).
    \item \textbf{Crossover probability \(p_c\):} with probability \(p_c\), recombine two parents; otherwise copy parents forward (and still allow mutation).
    \item \textbf{Mutation probability \(p_m\):} for binary encodings, mutate each bit independently with probability \(p_m\) (a common default is \(p_m \approx 1/L\) for length \(L\)); for real encodings, \(p_m\) usually corresponds to a mutation \emph{scale} (e.g., Gaussian noise standard deviation) rather than bit flips.
\end{itemize}

To keep notation compact later, we write these scheme-dependent settings abstractly as
\[
\begin{aligned}
p_s &\equiv \text{selection pressure parameter(s) (scheme-dependent)},\\
p_c &= \text{probability of applying crossover},\\
p_m &= \text{mutation probability (per gene/bit, unless stated otherwise)}.
\end{aligned}
\]

Tuning these settings controls exploration versus exploitation and strongly affects whether the algorithm stagnates early or continues to make progress.

% Chapter 19 (continued)

\subsection{Known Issues in Genetic Algorithms}
\label{sec:evo_known_issues_in_genetic_algorithms}

\begin{tcolorbox}[summarybox, title={Risk \& audit}]
\begin{itemize}
    \item \textbf{Premature convergence:} aggressive selection or weak mutation collapses diversity before good regions are explored.
    \item \textbf{Budget mismatch:} comparisons are invalid unless algorithms share evaluation budgets and stopping rules.
    \item \textbf{Constraint leakage:} hidden infeasibility can inflate scores; use explicit feasibility checks in logs and plots.
    \item \textbf{Single-run overclaim:} stochastic algorithms require multi-seed reporting with spread, not one best trajectory.
    \item \textbf{Objective gaming:} fitness proxies can drift from deployment goals; audit secondary metrics and failure cases.
\end{itemize}
\end{tcolorbox}

While genetic algorithms (GAs) provide a powerful heuristic framework for optimization, several well-known issues can affect their performance and reliability:

\paragraph{Premature Convergence}
Because GAs rely on heuristic search without a global optimality guarantee, they often converge prematurely to local minima rather than the global minimum. This is especially common if the initial population is not diverse or if the selection pressure is too high, causing loss of genetic diversity early on.

\begin{tcolorbox}[summarybox, title={Diversity maintenance}]
\begin{itemize}
    \item \textbf{Crowding/sharing:} penalize overly similar individuals to keep multiple niches in multi-modal landscapes.
    \item \textbf{Restarts/islands:} run multiple subpopulations (often in parallel) with occasional migration; robust against stagnation.
    \item \textbf{Adaptive mutation:} increase mutation or inject random individuals when diversity drops.
\end{itemize}
\end{tcolorbox}

\paragraph{Mutation Interference}
Mutation is intended to introduce diversity and help escape local minima by randomly altering genes. However, excessive or poorly controlled mutation can cause oscillations, where beneficial mutations are undone by subsequent mutations. This back-and-forth effect can prevent convergence and degrade solution quality.

\paragraph{Deception}
Deception refers to situations where the encoding or representation of solutions misleads the GA's fitness evaluation. Low-order schemata with high observed fitness may actually guide the search away from the global optimum, so that combining ``good'' building blocks produces worse offspring. There is no single formal definition, but a deceptive fitness landscape is one in which local improvements inferred from schemata systematically lead the GA to suboptimal basins of attraction.

\paragraph{Fitness Misinterpretation}
Since selection is driven by fitness values, any inaccuracies or misleading fitness evaluations can cause the GA to make poor decisions about which individuals to propagate. This can arise from noisy fitness functions, poorly designed objective functions, or deceptive encodings.

\subsection{Convergence Criteria}
\label{sec:evo_convergence_criteria}

Determining when to stop the GA is a critical practical consideration. Common convergence criteria include:

\begin{itemize}
    \item \textbf{Fixed number of generations:} Run the GA for a predetermined number of iterations.
    \item \textbf{Time limit:} Stop after a fixed amount of computational time.
    \item \textbf{No improvement:} Terminate if the best fitness value has not improved over a specified number of generations.
    \item \textbf{Manual inspection:} Periodically inspect the population to decide if the solutions are satisfactory.
\end{itemize}

In practice, a combination of these criteria is often used. For example, one might stop if \emph{either} (a) no improvement in the best fitness is observed for 10 consecutive generations, \emph{or} (b) the run reaches 100 generations in total, whichever condition is met first. \Cref{fig:lec11-ga-progress} visualizes such a run, making it easy to spot plateaus and the persistent gap between best and mean fitness.
\begin{figure}[t]
    \centering
    \begin{tikzpicture}
        \begin{axis}[
            set layers,
            width=0.82\linewidth,
            height=0.42\linewidth,
            xlabel={Generation},
            ylabel={Normalized fitness},
            xmin=0, xmax=50,
            ymin=0.4, ymax=1.02,
            legend style={at={(0.5,1.03)}, anchor=south, legend columns=2}
        ]
            % no-improvement window shading (draw first so it stays behind the curves)
            \addplot[draw=none, fill=gray!15, on layer=axis background] coordinates {(32,0.4) (45,0.4) (45,1.0) (32,1.0)};
            \addplot[cbBlue, thick, mark=*, mark size=1.8pt] table[row sep=\\] {
                0 0.55 \\
                5 0.60 \\
                10 0.68 \\
                15 0.74 \\
                20 0.79 \\
                25 0.83 \\
                30 0.87 \\
                35 0.90 \\
                40 0.93 \\
                45 0.95 \\
                50 0.96 \\
            };
            \addlegendentry{Best fitness}
            \addplot[cbOrange, thick, dashed, mark=x, mark size=1.8pt] table[row sep=\\] {
                0 0.52 \\
                5 0.55 \\
                10 0.58 \\
                15 0.63 \\
                20 0.67 \\
                25 0.71 \\
                30 0.75 \\
                35 0.78 \\
                40 0.80 \\
                45 0.81 \\
                50 0.82 \\
            };
            \addlegendentry{Mean fitness}
            \addplot[gray, dash dot] coordinates {(32,0.4) (32,1.0)};
            \node[anchor=south east, font=\scriptsize] at (axis cs:31.5,0.82){Flat region $\Rightarrow$ consider stopping};
        \end{axis}
    \end{tikzpicture}
    \caption{Illustrative GA run showing the best and mean normalized fitness over 50 generations. Flat regions motivate ``no improvement'' stopping rules, while steady separation between best and mean indicates ongoing selection pressure. Helpful for diagnosing premature convergence versus ongoing exploration.}
    \label{fig:lec11-ga-progress}
\end{figure}


\subsection{Summary of Genetic Algorithm Workflow}
\label{sec:evo_summary_of_genetic_algorithm_workflow}

This is the implementation-facing assembly of the GA loop. Earlier sections motivated the need for population-based search and defined each operator; here we put them back together and give three complementary sanity-check views: \Cref{fig:lec11-ga-flow} for control flow, \Cref{tab:ga-toy} for a one-generation numeric trace, and \Cref{sec:evo_pseudocode_representation} for a minimal implementation template. All three describe the same loop at different levels of detail.

\begin{figure}[!ht]
\centering
\begin{adjustbox}{max width=\linewidth, center}
\begin{tikzpicture}[
  font=\small,
  >=Stealth,
  node distance=18mm and 26mm,
  line width=0.9pt,
  block/.style={
    draw, rounded corners=2pt,
    minimum width=38mm, minimum height=12mm,
    align=center
  },
  decision/.style={
    draw, diamond, aspect=2.2,
    inner sep=1.5pt, align=center,
    minimum width=24mm, minimum height=14mm
  },
  flow/.style={->},
  loop/.style={->, dashed},
  lab/.style={font=\footnotesize, inner sep=1pt}
]

% --- Nodes (stable layout) ---
\node[block]   (init) {Initialization\\random population};
\node[block, right=of init] (fit) {Fitness evaluation};
\node[decision, right=of fit] (term) {Termination?};

\node[block, below=18mm of fit] (sel) {Selection};
\node[block, right=of sel] (cross) {Crossover};
\node[block, right=of cross] (mut) {Mutation};
\node[block, below=18mm of cross] (rep) {Replacement};

% --- Main flow ---
\draw[flow] (init) -- (fit);
\draw[flow] (fit) -- (term);

\draw[flow] (term.south) -- node[right, lab, xshift=2pt]{no} (sel.north);
\draw[flow] (sel) -- (cross);
\draw[flow] (cross) -- (mut);

% mutation down then to replacement (right-angle, no wandering)
\draw[flow] (mut.south) -- ++(0,-10mm) -| (rep.east);

% replacement feeds next generation evaluation (clean dashed loop)
\draw[loop] (rep.north) -- node[left, lab, pos=0.35, yshift=1pt]{next generation} (fit.south);

% optional ``inner loop'' annotation (slightly above for readability)
\draw[<->, dashed] ([yshift=4mm]sel.east) -- node[above, lab, yshift=2pt]{inner loop} ([yshift=4mm]cross.west);

% yes/stop arrow drawn as overlay so it does not blow up the bounding box
\begin{scope}[overlay]
  \draw[flow] (term.north) -- ++(0,12mm) node[above, lab, yshift=1pt]{yes / stop};
\end{scope}

\end{tikzpicture}
\end{adjustbox}
\caption{GA flowchart showing the iterative process: initialization leads to fitness evaluation and a termination check. If not terminated, the algorithm proceeds through selection, crossover, mutation, and replacement, which then feeds the next generation's fitness evaluation. Reference when verifying that your implementation preserves the intended control flow.}
\label{fig:lec11-ga-flow}
\end{figure}

For the toy generation in \Cref{tab:ga-toy}, fitness values are computed using \(f(x)=\cos(5\pi x)\exp(-x^2)\) from \Cref{sec:evo_example_ga_for_a_constrained_optimization_problem}.

\begin{table}[h]
\centering
\small
\begin{tabularx}{\linewidth}{@{}l >{\raggedright\arraybackslash}X >{\raggedleft\arraybackslash}p{0.18\linewidth} >{\raggedleft\arraybackslash}p{0.16\linewidth}@{}}
\toprule
Step & Example bitstrings & Decoded \(x\) & Fitness \(f(x)\) \\
\midrule
Initial (subset) & \(\,0001_2\), \(\,0010_2\), \(\,0101_2\), \(\,0111_2\) & 0.0625, 0.125, 0.3125, 0.4375 & 0.553, -0.377, 0.177, 0.687 \\
Select parents (example) & \(\,0010_2\), \(\,0101_2\) & 0.125, 0.3125 & -0.377, 0.177 \\
Crossover (one-point) & Parents: \(\,00|10\), \(\,01|01\) \(\to\) offspring: \(\,00|01\), \(\,01|10\) & 0.0625, 0.3750 & 0.553, 0.803 \\
Mutation (flip one bit) & \(\,0110_2 \to 0111_2\) & 0.4375 & 0.687 \\
\bottomrule
\end{tabularx}
% Avoid inline math in captions; it wraps poorly in some EPUB renderers.
\caption{Toy GA generation on a bounded interval. One crossover and mutation illustrate how the fitness function guides selection before the next generation. Use this to explain how variation operators interact with selection pressure.}
\label{tab:ga-toy}
\end{table}

\Cref{tab:ga-toy} is the step-by-step toy generation trace used to ground the operator sequence.

\begin{tcolorbox}[summarybox, title={Defaults that work (DE/CMA-ES)}]
\textbf{Differential Evolution (DE):} start with population \(10\!-\!20\times D\) (dimension \(D\)), mutation scale \(F\in[0.5,0.8]\), crossover \(C_r\in[0.7,0.9]\).
\textbf{CMA-ES:} population \(\lambda\approx 4+\lfloor 3\ln D\rfloor\), initial step-size \(\sigma_0 \approx 0.3\) of the variable range. These defaults are robust first tries before tuning.
\end{tcolorbox}

\subsection{Pseudocode Representation}
\label{sec:evo_pseudocode_representation}

The GA can be expressed in pseudocode as follows. Treat this as the ``minimum viable'' skeleton; a production implementation additionally records random seeds, evaluation budgets, constraint handling decisions, and per-generation diagnostics (best/mean fitness, diversity) so results are reproducible and interpretable.

\begin{verbatim}
Initialize population P with N chromosomes
Evaluate fitness of each chromosome in P

while termination criteria not met do
    Select parents from P based on fitness
    Apply crossover to parents to create offspring
    Apply mutation to offspring
    Evaluate fitness of offspring
    Replace some or all of P with offspring
end while

Return best chromosome found
\end{verbatim}

\subsection{Example: GA for a Constrained Optimization Problem}
\label{sec:evo_example_ga_for_a_constrained_optimization_problem}

Consider the problem of \emph{maximizing} the function:
\[
f(x) = \cos(5 \pi x) \cdot \exp(-x^2)
\]
subject to the constraint:
\[
0 \leq x \leq 0.5
\]
with a precision of three decimal places.

\paragraph{GA Parameters:}
\begin{itemize}
    \item Population size: 10 chromosomes
    \item Encoding: Fixed-point with resolution 0.001: store an integer \(n\in\{0,1,\ldots,500\}\) and decode via \(x=n/1000\) (a 9-bit representation covers 0--511)
    \item Crossover probability: 25\%
    \item Mutation probability: 10\%
    \item Selection: Truncation selection (example): select the top five chromosomes by fitness as parents each generation
\end{itemize}

\paragraph{Initialization:}
Generate 10 random values of \(x\) uniformly distributed in \([0, 0.5]\), each rounded to three decimal places. When prior designs or surrogate models exist, \emph{warm start} a few chromosomes with those known-good solutions before filling the rest randomly; seeding accelerates convergence without losing diversity if you keep most of the population stochastic.

\paragraph{Fitness Evaluation:}
Calculate \(f(x)\) for each chromosome and treat it as a fitness score (higher is better). If instead you are minimizing a cost, convert it to fitness via a monotone transform (e.g., \(-J(x)\), a shifted score, or a rank-based scheme) so that selection still prefers better candidates.

\paragraph{Evolutionary Cycle:}
Apply selection, crossover, and mutation to produce new offspring, then evaluate their fitness. Repeat this process for multiple generations until convergence criteria are met.

\paragraph{Remarks:}
In practice, some initial chromosomes may fall outside the constraint bounds due to rounding or mutation; these should be clipped or repaired to maintain feasibility.

As an illustration, if the initial population contains
\[
\begin{aligned}
x &\in \{0.04,\;0.09,\;0.13,\;0.18,\;0.22,\;0.27,\\
&\qquad 0.31,\;0.36,\;0.42,\;0.48\},
\end{aligned}
\]
then the corresponding objective values are
\[
\begin{aligned}
f(x) &\in \{0.808,\;0.155,\;-0.446,\;-0.921,\;-0.906,\;-0.422,\\
&\qquad 0.142,\;0.711,\;0.797,\;0.245\},
\end{aligned}
\]
rounded to three decimals.
Each chromosome uses a 9\text{-}bit fixed-point code (3 fractional digits), decoded by interpreting the bits as an integer \(n\) and scaling via \(x=n/1000\). Any decoded values outside \([0,\,0.5]\) are repaired (e.g., clipped to bounds); note that \(x=0\) is allowed in this example.

A single generation could proceed as follows:
\begin{itemize}
    \item Select the top five chromosomes by fitness.
    \item Apply one-point crossover on the 9-bit fixed-point codes (MSB-first). For example, \(0.203 \mapsto 011001011_2\) and \(0.359 \mapsto 101100111_2\); cutting after 5 bits and swapping tails yields offspring \(011000111_2 \mapsto 0.199\) and \(101101011_2 \mapsto 0.363\).
    \item Mutate each bit with probability 0.1, ensuring all decoded values remain within \([0,\,0.5]\).
    \item Re-evaluate fitness and retain the best ten individuals for the next generation.
\end{itemize}

\begin{tcolorbox}[summarybox, title={Constraint handling playbook}]
\begin{itemize}
    \item \textbf{Penalty methods} soften constraints by augmenting the fitness with a violation term, e.g., \(F(\mathbf{x}) = f(\mathbf{x})+\lambda \sum_k \max\{0, g_k(\mathbf{x})\}^2\); increase \(\lambda\) when infeasible individuals survive too often.
    \item \textbf{Repair operators} project infeasible chromosomes back into the feasible region (clip bound violations, renormalize equality constraints, or rerun a problem-specific solver) before evaluation.
    \item \textbf{Feasibility-first selection} ranks feasible candidates ahead of infeasible ones, then compares raw fitness only within each group; among infeasible solutions, select those with the smallest violation.
    \item \textbf{Deb's feasibility rules} \citep{Deb2001Book}: (i) if one solution is feasible and the other is not, pick the feasible one; (ii) if both are feasible, pick the better objective; (iii) if both are infeasible, pick the one with smaller total constraint violation. Adaptive penalties can be layered on top when violations persist.
\end{itemize}
\paragraph{Reproducibility and fair comparison} Fix random seeds when debugging, run many seeds (e.g., 20+) for reporting, match evaluation budgets across algorithms, and report mean/median best-so-far with variability bands. Log all hyperparameters and share code/configs to make comparisons fair, following \Cref{app:repro_standards} as the default reporting template.
Penalty terms are easy to implement, repair operators exploit domain knowledge, and feasibility-first policies are useful in safety-critical controllers where violating constraints is unacceptable even temporarily.
\end{tcolorbox}

% End of Chapter 19 (continued)

\subsection{Genetic Algorithms: Iterative Process and Convergence}
\label{sec:evo_genetic_algorithms_iterative_process_and_convergence}

This section focuses on convergence behavior; the GA cycle itself is summarized in \Cref{sec:evo_summary_of_genetic_algorithm_workflow} and the control flow in \Cref{fig:lec11-ga-flow}. Over generations, populations tend to cluster around better regions of the fitness landscape; flat best\hyp{}fitness curves signal stagnation, while a persistent gap between best and mean indicates ongoing selection pressure (see \Cref{fig:lec11-ga-progress}). Convergence typically occurs after a problem\hyp{}dependent number of generations, and the best\hyp{}so\hyp{}far solution should be treated as a high\hyp{}quality approximation rather than a guaranteed global optimum.

\subsection{Beyond canonical GAs: real-coded strategies}
\label{sec:evo_beyond_canonical_gas_real_coded_strategies}

Bit-string encodings are ideal for combinatorial search, yet most engineering problems have continuous decision variables. Two mature real-coded families are now standard tools:

\begin{itemize}
    \item \textbf{Covariance Matrix Adaptation Evolution Strategy (CMA-ES)} \citep{Hansen2001} maintains a multivariate Gaussian search distribution. Successful steps update the mean, adapt the global step size, and rotate the covariance to align with the landscape's principal directions. CMA-ES shines on smooth, ill-conditioned black-box functions where gradients are unavailable but the objective rewards second-order adaptation.
    \item \textbf{Differential Evolution (DE)} \citep{Storn1997} perturbs a target vector with scaled differences of two other individuals, \(\mathbf{v} = \mathbf{x}_r + F(\mathbf{x}_p - \mathbf{x}_q)\), then mixes \(\mathbf{v}\) with the original via binomial or exponential crossover. This simple mechanism balances exploration/exploitation with only three hyperparameters \((F,\, C_r,\, N)\) and handles noisy, non-smooth objectives well.
\end{itemize}

\begin{tcolorbox}[summarybox, title={Evolution strategies (ES): step-size adaptation and self-adaptation}]
Evolution strategies are a real-coded evolutionary family that makes one engineering idea explicit: \emph{mutation step size is part of the algorithm's state, not a fixed constant.} In a simple isotropic form,
\[
\mathbf{x}'=\mathbf{x}+\boldsymbol{\epsilon},\qquad \boldsymbol{\epsilon}\sim\mathcal{N}(\mathbf{0},\,\sigma^2\mathbf{I}),
\]
where \(\sigma\) controls exploration (large \(\sigma\)) versus local refinement (small \(\sigma\)). A classic adaptive heuristic is the ``1/5 success rule'': if more than 20\% of recent mutations improve fitness, increase \(\sigma\); if fewer than 20\% succeed, decrease \(\sigma\). Self-adaptive variants go one step further by embedding \(\sigma\) (and sometimes covariance structure) into the chromosome itself so selection pressure co-evolves the search behavior alongside the solution parameters.
\end{tcolorbox}

Both algorithms plug into the same evaluation loop shown earlier and can reuse the constraint-handling policies in the preceding box. In practice many teams prototype with DE (fast, few knobs) and switch to CMA-ES when the problem demands higher precision or adaptive covariance modeling.

\subsection{Genetic Programming (GP)}
\label{sec:evo_genetic_programming_gp}

Genetic programming extends the principles of genetic algorithms to the evolution of computer programs or symbolic expressions rather than fixed-length parameter vectors.

\paragraph{Problem Setup}

Consider a problem where the relationship between input variables $x_1, x_2, \ldots, x_n$ and output $y$ is unknown. Unlike traditional parameter estimation, we do not assume a fixed functional form. Instead, we want to discover the function $f$ such that
\[
y = f(x_1, x_2, \ldots, x_n).
\]

\paragraph{Representation of Programs}

In GP, candidate solutions are represented as tree-like structures encoding mathematical expressions or programs composed of:

\begin{itemize}
    \item \textbf{Terminals:} Input variables ($x_1, x_2, \ldots$) and constants.
    \item \textbf{Functions:} Arithmetic operations (addition, subtraction, multiplication, division), logical operations, or other domain-specific functions.
\end{itemize}

For example, a candidate program might represent the expression
\[
(x_1 \times x_3) + (x_1 + x_4).
\]

\paragraph{Genetic Operators in GP}

\begin{itemize}
    \item \textbf{Crossover:} Subtrees from two parent programs are exchanged to create offspring programs.
    \item \textbf{Mutation:} Random modifications are made to nodes in the program tree, such as changing an operator or replacing a subtree.
\end{itemize}

These operations allow the evolution of increasingly complex and effective programs.

\paragraph{Fitness Evaluation}

A candidate program is evaluated by executing it on a training set and comparing its outputs with the desired targets. Fitness functions often measure mean squared error, classification accuracy, or accumulated reward, and penalize programs that raise runtime exceptions or exceed resource limits. Individuals with higher fitness are more likely to be selected for reproduction.

\paragraph{Example}

Suppose we have the following initial program trees:

\begin{center}
\begin{tabular}{c c}
Parent 1: & $f_1 = (x_1 \times x_3) + (x_1 + x_4)$ \\
Parent 2: & $f_2 = (x_2 - 5) \times (x_4 + 1)$
\end{tabular}
\end{center}

Suppose we exchange the right subtree of $f_1$ (the addition node $x_1 + x_4$) with the left subtree of $f_2$ (the subtraction node $x_2 - 5$). The resulting offspring are
\[
f'_1 = (x_1 \times x_3) + (x_2 - 5), \qquad
f'_2 = (x_1 + x_4) \times (x_4 + 1).
\]
Mutation might then replace the terminal $x_4$ in $f'_1$ with a constant (e.g., $5$) or switch the addition operator to multiplication, thereby exploring nearby program structures while keeping the tree depth bounded.

\paragraph{Recursive and Modular Programs}

GP can evolve recursive functions and modular code blocks (subroutines), enabling the discovery of complex behaviors and algorithms. In practice this is achieved by allowing trees to reference automatically defined functions (ADFs) or macros that are evolved alongside the main program. The depth of the program trees and the number of reusable modules are usually constrained to prevent uncontrolled growth and to keep execution cost manageable.

\paragraph{Applications}

Genetic programming is particularly useful for:

\begin{itemize}
    \item Symbolic regression: discovering analytical expressions fitting data.
    \item Automated program synthesis: generating code for control, decision-making, or data processing.
    \item Robotics: evolving control programs for navigation, obstacle avoidance, or manipulation.
\end{itemize}

\paragraph{Example: Robot Obstacle Avoidance}

Consider evolving a program that controls a robot's movement based on sensor inputs indicating obstacles. The function set might include commands like \texttt{move\_forward}, \texttt{turn\_left}, \texttt{turn\_right}, and conditional statements. The GP evolves sequences and combinations of these commands to maximize the robot's ability to navigate without collisions.

\paragraph{Summary}

Genetic programming generalizes genetic algorithms by evolving program structures (often trees) rather than fixed-length chromosomes.

\subsection{Wrapping Up Genetic Algorithms and Genetic Programming}
\label{sec:evo_wrapping_up_genetic_algorithms_and_genetic_programming}

In this final segment of the chapter, we conclude our discussion on genetic algorithms (GAs) and genetic programming (GP), emphasizing their conceptual foundations, practical implications, and the distinctions between them.

\paragraph{Recap of Genetic Algorithms}

Genetic algorithms are population-based metaheuristics; the earlier sections already define selection, crossover, mutation, and the loop in detail. For recap, keep the design levers in view and revisit the worked traces when needed:
\begin{itemize}
    \item \textbf{Representation:} How solutions are encoded (bit strings, real vectors, trees) and what constraints must be preserved.
    \item \textbf{Fitness and evaluation:} What you reward, how you handle noisy measurements, and how constraints enter the score.
    \item \textbf{Operator/loop settings:} Selection pressure, crossover/mutation rates, and stopping criteria.
\end{itemize}
See \Cref{sec:evo_summary_of_genetic_algorithm_workflow} for control flow, \Cref{tab:ga-toy} for a numeric generation trace, and \Crefrange{sec:evo_selection_in_genetic_algorithms}{sec:evo_mutation_operator} for operator details.

\paragraph{Genetic Programming: Structure over Parameters}

Genetic programming extends the GA paradigm by evolving computer programs or symbolic expressions rather than fixed-length parameter vectors. The fundamental difference is that GP searches over the space of program structures (trees of functions and terminals) instead of numeric parameter values.

Key points about GP include:

\begin{itemize}
    \item \textbf{Representation:} Programs are represented as hierarchical trees, where internal nodes are functions (e.g., arithmetic operators, logical functions) and leaves are terminals (input variables, constants).
    \item \textbf{Evolution of Programs:} Genetic operators manipulate program trees:
    \begin{itemize}
        \item \emph{Crossover} exchanges subtrees between parent programs.
        \item \emph{Mutation} randomly modifies nodes or subtrees.
    \end{itemize}
    \item \textbf{Fitness Evaluation:} Programs are executed on input data, and their outputs are compared against desired outputs to compute fitness.
    \item \textbf{Emergent Solutions:} GP can discover novel program structures that model complex phenomena without explicit programming, often yielding surprising and insightful results.
\end{itemize}

\paragraph{Applications and Insights}

Genetic programming is particularly powerful for modeling complex systems where the underlying relationships are unknown or difficult to specify explicitly. For example, given inputs such as wind speed, humidity, and temperature, GP can evolve models that predict environmental phenomena without prior assumptions about the functional form.

This capability highlights the strength of GP as a tool for automated model discovery and symbolic regression.

\paragraph{Further Topics and Extensions}

While this chapter provided a concise overview, the field of evolutionary computation encompasses many advanced topics, including:

\begin{itemize}
    \item \textbf{Multi-objective Genetic Algorithms:} Handling optimization problems with multiple conflicting objectives.
    \item \textbf{Constraint Handling:} Incorporating problem-specific constraints into the evolutionary process.
    \item \textbf{Hybrid Methods:} Combining GAs/GP with other optimization or machine learning techniques.
    \item \textbf{Scalability and Parallelization:} Efficiently implementing evolutionary algorithms for large-scale problems.
\end{itemize}

Readers are encouraged to explore these topics through further reading and research; the short primer below highlights the most widely used multi-objective GA.

\subsection{Multi-objective search and NSGA-II}
\label{sec:evo_multi_objective_search_and_nsga_ii}
When two or more objectives conflict, we seek a set of Pareto-optimal solutions rather than a single best point. The Non-dominated Sorting Genetic Algorithm II (NSGA-II) sorts each generation into Pareto fronts: rank-1 individuals are non-dominated, rank-2 are dominated only by rank-1, etc. Replacement preserves all members of the best fronts and uses crowding distance to maintain diversity along the trade-off curve. NSGA-II's combination of elitist survival and \(O(N \log N)\) non-dominated sorting makes it the default baseline for multi-objective evolutionary optimization \citep{Deb2002}.
\paragraph{Metrics and variants} Hypervolume (area/volume dominated by the front with respect to a reference point) is a common scalar indicator; report it alongside the spread of solutions \citep{Zitzler2002}. MOEA/D decomposes objectives into weighted subproblems; SPEA2/IBEA are popular alternatives. Always plot the Pareto set and budget-matched hypervolume traces when comparing algorithms.


\begin{figure}[t]
    \centering
    \includegraphics[width=0.58\linewidth]{lec19_pareto}
\caption{Sample Pareto front for two objectives. NSGA-II keeps all non-dominated points (blue) while pushing dominated solutions (orange) toward the front via selection, yielding a spread of trade-offs in one run. Use this when interpreting multi-objective results as trade-offs, not a single optimum.}
\label{fig:lec11-pareto}
\end{figure}

\Cref{fig:lec11-pareto} is the trade-off view used to interpret multi-objective runs.

\clearpage

\begin{tcolorbox}[summarybox, title={Key takeaways}]
\begin{itemize}
    \item Evolutionary algorithms optimize without gradients by iterating selection, variation, and replacement under a fitness evaluation loop.
    \item Constraint handling is part of the design: penalties, repair, and feasibility-first selection each encode a different notion of ``acceptable'' search.
    \item Multi-objective search replaces a single optimum with a Pareto front; NSGA-II is a standard baseline for producing diverse trade-offs.
\end{itemize}

\medskip
\noindent\textbf{Minimum viable mastery.}
\begin{itemize}
    \item Specify genotype, variation operators, selection scheme, and termination criteria in a way that is reproducible.
    \item Distinguish constraint handling strategies and justify the one used (penalty vs.\ repair vs.\ feasibility rules).
    \item Report multi-objective results as trade-offs (Pareto fronts) rather than as a single scalar score.
\end{itemize}

\noindent\textbf{Common pitfalls.}
\begin{itemize}
    \item Over-interpreting a single stochastic run (report multiple seeds and dispersion).
    \item Using aggressive selection and low mutation, collapsing diversity and getting stuck in deceptive basins.
    \item Comparing algorithms without matching evaluation budget (fitness calls), constraint handling, and stopping rules.
\end{itemize}
\end{tcolorbox}

\begin{tcolorbox}[summarybox, title={Exercises and lab ideas}]
\begin{itemize}
    \item Implement a simple GA for $f(x)=\cos(5\pi x)\exp(-x^2)$, experimenting with penalty vs.\ repair strategies for the $[0,0.5]$ constraint.
    \item Prototype a CMA-ES or Differential Evolution solver on a noisy Rosenbrock function and compare convergence traces against the canonical GA.
    \item Build a tiny GP to rediscover a closed-form expression (e.g., $y = x^3 + x$) from samples; report how often crossover/mutation produce valid programs.
\end{itemize}

\medskip
\noindent\textbf{If you are skipping ahead.} This chapter is largely self-contained. If you revisit earlier model chapters, keep the common thread in mind: every method still requires a clear objective, a diagnostic that detects failure, and a reporting protocol that survives replication.
\end{tcolorbox}

\paragraph{Where we head next.} This chapter closes the soft-computing thread. For targeted refreshers, \Cref{app:linear_systems} summarizes linear-systems prerequisites and \Cref{app:kernels} consolidates kernel-method context used earlier in the book.

\nocite{Holland1975, Koza1992, Goldberg1989, Deb2001Book, Mitchell1998}

\begin{tcolorbox}[summarybox,title={Part IV takeaways}]
\begin{itemize}
  \item Evolutionary search is an optimization tool when gradients are unavailable or unreliable.
  \item Operators (selection, crossover, mutation) encode exploration vs.\ exploitation; tune them against variance across runs.
  \item Constraints and multi-objectives are first-class: define feasibility and trade-offs before optimizing.
  \item Report reproducibly: seeds, multiple runs, and distributions matter more than a single best trajectory.
\end{itemize}
\end{tcolorbox}
